%% Hello altogether,
%%
%% this is a release of the English AS manual.  I haven't
%% done the entire translation myself, large parts of it are the work of some
%% other people around the net who deserve my deep appreciation for this job.
%% My parts of the translation are the results of a brute-force attempt,
%% so there are surely tons of spelling errors and passages that will
%% make people with English as their mother tongue either laugh or cry...

%% Alfred Arnold

%%         translation by: Oliver Sellke (OSIP, D-65199 Wiesbaden)
%%                           (proof-read in parts by Stefan Hilse, Wiesbaden)
%%                         Alfred Arnold
%%                         Stephan Kanthak
%%                         Vittorio De Tomasi
%%
%%         thanks to the authors of:
%%                         FB-translator
%%                         GNU-ispell
%%
%% ------------------------------------------------------------------------------

%%TITLE User's Manual for Macro Assembler AS
\documentclass[12pt,twoside]{report}
\usepackage{makeidx}
\usepackage{hyperref}
\usepackage{longtable}
\pagestyle{headings}
\sloppy
%%\textwidth 15cm
%%\evensidemargin 0.5cm
%%\oddsidemargin 0.5cm
\topsep 1mm
\parskip 0.3cm plus0.25cm minus0.25cm
\parindent 0cm

\newcommand{\ii}[1]{{\it #1}}
\newcommand{\bb}[1]{{\bf #1}}
\newcommand{\tty}[1]{{\tt #1}}
\newcommand{\tin}[1]{{\scriptsize #1}}
\newcommand{\ttindex}[1]{\index{#1@{\tt #1}}}

\font\mengft=cmss9 scaled \magstep1
\def \rz{\hbox{\mengft{I \hskip -1.7mm R}}}

\makeindex

%%===========================================================================

\begin{document}

\thispagestyle{empty}

\
\vspace{7cm}\par

\begin{raggedright}
{\large Alfred Arnold, Stefan Hilse, Stephan Kanthak, Oliver
 Sellke, Vittorio De Tomasi}
\vspace{1cm}\par
{\huge Macro Assembler AS V1.42}\\
\rule{9.5cm}{0.3mm}\\
\vspace{2mm}\par
{\huge User's Manual}

\vspace{1cm}\par

{\large Edition Januar 2022}
\end{raggedright}

\clearpage
\thispagestyle{empty}

\ \vspace{5cm}

{\em IBM, PPC403Gx, OS/2, and PowerPC} are registered trademarks of IBM
Corporation.

{\em Intel, MCS-48, MCS-51, MCS-251, MCS-96, MCS-196 und MCS-296} are
registered trademarks of Intel Corp. .

{\em Motorola and ColdFire} are registered trademarks of Motorola Inc. .

{\em MagniV} is a registered trademark of Freescale Semiconductor.

{\em PicoBlaze} is a registered trademark of Xilinx Inc.

{\em UNIX} is a registered trademark of the The Open Group.

{\em Linux} is a registered trademark of Linus Thorvalds.

{\em Microsoft, Windows, and MS-DOS} are registered trademarks of
Microsoft Corporation.

All other trademarks not explicitly mentioned in this section and used in
this manual are properties of their respective owners.

\vspace{5cm}

This document has been processed with the LaTeX typesetting system, using
the Linux operating system.

\clearpage

%%===========================================================================

{\parskip 0cm plus0.1cm \tableofcontents}

%%===========================================================================

\cleardoublepage
\chapter{Introduction}

This instruction is meant for programmers who are already very familiar
with Assembler and who like to know how to work with AS.  It is rather a
reference than a user's manual and so it neither tries to explain the
''language assembler'' nor the processors.  I have listed further
literature in the bibliography which was substantial in the implementation
of the different code generators.  There is no book I know where you can
learn Assembler from the start, so I generally learned this by ''trial and
error''.

%%---------------------------------------------------------------------------

\section{License Agreement}
\label{SectLicense}

Before we can go ''in medias res'', first of all the inevitable prologue:

As in the present version is licensed according to the Gnu General Public
License (GPL); the details of the GPL may be read in the file COPYING
bundled with this distribution.  If you did not get it with AS, complain
to the one you got AS from!

Shortly said, the GPL covers the following points:
\begin{itemize}
\item{Programs based upon AS must also be licensed according to the GPL;}
\item{distribution is explicitly allowed;}
\item{explicit disclaiming of all warranties for damages resulting from
      usage of this program.}
\end{itemize}
...however, I really urge you to read the file COPYING for the details!

To accelerate the error diagnose and correction, please add the
following details to the bug report:
\begin{itemize}
\item{Operating system (DOS, Windows, Linuc) and its version}
\item{Version of AS used, resp. dates of the \tty{EXE}-files}
\item{If you compiled the assembler yourself, the compiler used and its version}
\item{If possible, the source file that triggered the bug}
\end{itemize}
You can contact me as follows:
\begin{itemize}
\item{by Surface Mail: \begin{description}
               \item{Alfred Arnold}
               \item{Hirschgraben 29}
               \item{D-52062 Aachen}
               \item{Germany}
               \end{description}}
\item{by E-Mail: \tty{alfred@ccac.rwth-aachen.de}}
\end{itemize}
If someone likes to meet me personally to ask questions and lives
near Aachen (= Aix-la-Chapelle), you will be able to meet me there.
You can do this most probably on thursdays from 8pm to 9pm at the
RWTH Aachen Computer Club (Elisabethstrasse 16, first floor, corridor
on the right).

Please don't call me by phone.  First, complex relations are
extremely hard to discuss at phone.  Secondly, the telephone
companies are already rich enough...

The latest version of AS (DPMI, Win32, C) is available from
the following Server:
\begin{verbatim}
 http://john.ccac.rwth-aachen.de:8000/as
\end{verbatim}
or shortly
\begin{verbatim}
 http://www.alfsembler.de
\end{verbatim}

Whoever has no access to an FTP-Server can ask me to send the assembler
by mail.  Only requests containing a blank CD-R and a self-addressed,
(correctly) stamped envelope will be answered.  Don't send any money!

Now, after this inevitable introduction we can turn to the actual
documentation:

%%---------------------------------------------------------------------------

\section{General Capabilities of the Assembler}

In contrast to ordinary assemblers, AS offers the possibility to
generate code for totally different processors.  At the moment, the
following processor families have been implemented:
\begin{itemize}
\item{Motorola 68000..68040,, 683xx, and Coldfire incl. coprocessor and MMU}
\item{Motorola ColdFire}
\item{Motorola DSP5600x,DSP56300}\item{Motorola M-Core}
\item{Motorola/IBM MPC601/MPC505/PPC403/MPC821}
\item{Motorola 6800, 6801, 68(HC)11(K4) and Hitachi 6301}
\item{Motorola/Freescale 6805, 68HC(S)08}
\item{Motorola 6809 / Hitachi 6309}
\item{Motorola/Freescale 68HC12(X) including XGATE}
\item{Freescale/NXP S12Z (''MagniV'')}
\item{Motorola 68HC16}
\item{Freescale 68RS08}
\item{Hitachi H8/300(H)}
\item{Hitachi H8/500}
\item{Hitachi SH7000/7600/7700}
\item{Hitachi HMCS400}
\item{Hitachi H16}
\item{Rockwell 6502, 65(S)C02, Commodore 65CE02, WDC W65C02S, Rockwell 65C19, and
      Hudson HuC6280}
\item{CMD 65816}
\item{Mitsubishi MELPS-740}
\item{Mitsubishi MELPS-7700}
\item{Mitsubishi MELPS-4500}
\item{Mitsubishi M16}
\item{Mitsubishi M16C}
\item{Intel 4004/4040}
\item{Intel MCS-48/41, including Siemens SAB80C382, and the OKI
      variants}
\item{Intel MCS-51/251, Dallas DS80C390}
\item{Intel MCS-96/196(Nx)/296}
\item{Intel 8080/8085}
\item{Intel i960}
\item{Signetics 8X30x}
\item{Signetics 2650}
\item{Philips XA}
\item{Atmel (Mega-)AVR}
\item{AMD 29K}
\item{Siemens 80C166/167}
\item{Zilog Z80 (including undocumented instructions), Z180, Z380}
\item{Sharp LR35902 (,,Gameboy Z80'')}
\item{Zilog Z8, Super8, Z8 Encore}
\item{Zilog Z8000}
\item{Xilinx KCPSM/KCPSM3 ('PicoBlaze')}
\item{LatticeMico8}
\item{Toshiba TLCS-900(L)}
\item{Toshiba TLCS-90}
\item{Toshiba TLCS-870(/C)}
\item{Toshiba TLCS-47}
\item{Toshiba TLCS-9000}
\item{Toshiba TC9331}
\item{Microchip PIC16C54..16C57}
\item{Microchip PIC16C84/PIC16C64}
\item{Microchip PIC17C42}
\item{Parallax SX20/28}
\item{SGS-Thomson ST6}
\item{SGS-Thomson ST7/STM8}
\item{SGS-Thomson ST9}
\item{SGS-Thomson 6804}
\item{Texas Instruments TMS32010/32015}
\item{Texas Instruments TMS3202x}
\item{Texas Instruments TMS320C3x/TMS320C4x}
\item{Texas Instruments TMS320C20x/TMS320C5x}
\item{Texas Instruments TMS320C54x}
\item{Texas Instruments TMS320C6x}
\item{Texas Instruments TMS99xx/99xxx}
\item{Texas Instruments TMS7000}
\item{Texas Instruments TMS1000}
\item{Texas Instruments TMS370xxx}
\item{Texas Instruments MSP430(X)}
\item{National Semiconductor SC/MP}
\item{National Semiconductor INS807x}
\item{National Semiconductor COP4}
\item{National Semiconductor COP8}
\item{National Semiconductor SC144xx}
\item{National Semiconductor NS32xxx}
\item{Fairchild ACE}
\item{Fairchild F8}
\item{NEC $\mu$PD78(C)0x/$\mu$PD 78(C)1x}
\item{NEC $\mu$PD75xx}
\item{NEC $\mu$PD 75xxx (alias 75K0)}
\item{NEC 78K0}
\item{NEC 78K2}
\item{NEC 78K3}
\item{NEC 78K4}
\item{NEC $\mu$PD7720/7725}
\item{NEC $\mu$PD77230}
\item{Fujitsu F$^2$MC8L}
\item{Fujitsu F$^2$MC16L}
\item{OKI OLMS-40}
\item{OKI OLMS-50}
\item{Panafacom MN1610/MN1613}
\item{Padauk PMS/PMC/PFSxxx}
\item{Symbios Logic SYM53C8xx (yes, they are programmable!)}
\item{Intersil CDP1802/1804/1805(A)}
\item{XMOS XS1}
\item{MIL STD 1750}
\item{KENBAK-1}
\item{GI CP-1600}
\end{itemize}
under work / planned / in consideration :
\begin{itemize}
\item{Analog Devices ADSP21xx}
\item{SGS-Thomson ST20}
\item{Texas Instruments TMS320C8x}
\end{itemize}
Unloved, but now, however, present :
\begin{itemize}
\item{Intel 80x86, 80186, Nec V30\&V35 incl. coprocessor 8087}
\end{itemize}
The switch to a different code generator is allowed even within one
file, and as often as one wants!

The reason for this flexibility is that AS has a history, which may also
be recognized by looking at the version number. AS was created as an
extension of a macro assembler for the 68000 family. On special request, I
extended the original assembler so that it was able to translate 8051
mnemonics.  On this way (decline ?!) from the 68000 to 8051, some other
processors were created as by-products.  All others were added over time
due to user requests.  So At least for the processor-independent core of
AS, one may assume that it is well-tested and free of obvious bugs.
However, I often do not have the chance to test a new code generator in
practice (due to lack of appropriate hardware), so surprises are not
impossible when working with new features.  You see, the things stated in
section \ref{SectLicense} have a reason...

This flexibility implies a somewhat exotic code format, therefore I
added some tools to work with it. Their description can be found in
chapter \ref{ChapTools}.

AS is a macro assembler, which means that the programmer has the
possibility to define new ''commands'' by means of macros.
Additionally it masters conditional assembling.  Labels inside macros
are automatically processed as being local.

For the assembler, symbols may have either integer, string or floating
point values.  These will be stored - like interim values in formulas -
with a width of 32 bits for integer values, 80 or 64 bits for floating
point values, and 255 characters for strings.  For a couple of micro
controllers, there is the possibility to classify symbols by segmentation.
So the assembler has a (limited) possibility to recognize accesses to
wrong address spaces.

The assembler does not know explicit limits in the nesting depth of
include files or macros; a limit is only given by the program stack
restricting the recursion depth.  Nor is there a limit for the
symbol length, which is only restricted by the maximum line length.

From version 1.38 on, AS is a multipass-assembler.  This pompous term
means no more than the fact that the number of passes through the
source code need not be exactly two. If the source code does not
contain any forward references, AS needs only one pass.  In case AS
recognizes in the second pass that it must use a shorter or longer
instruction coding, it needs a third (fourth, fifth...) pass to
process all symbol references correctly. There is nothing more behind
the term ''multipass'', so it will not be used further more in this
documentation.

After so much praise a bitter pill: AS cannot generate linkable code.
An extension with a linker needs considerable effort and is not planned
at the moment.

Those who want to take a look at the sources of AS can simply get the
Unix version of AS, which comes as source for self-compiling.  The sources
are definitely not in a format that is targeted at easy understanding -
the original Pascal version still raises its head at a couple of places,
and I do not share a couple of common opinions about 'good' C coding.

%%---------------------------------------------------------------------------

\section{Supported Platforms}

Though AS started as a pure DOS \marginpar{{\em DOS}} program, there are a
couple of versions available that are able to exploit a bit more than the
Real Mode of an Intel CPU.  Their usage is kept as compatible to the DOS
version as possible, but there are of course differences concerning
installation and embedding into the operating system in question.
Sections in this manual that are only valid for a specific version of AS
are marked with a corresponding sidemark (at this paragraph for the DOS
version) aheaded to the paragraph.  In detail, the following further
versions exist (distributed as separate packages):

In case you run \marginpar{{\em DPMI}}into memory problems when assembling
large and complex programs under DOS, there is a DOS version that runs in
protected mode via a DOS extender and can therefore make use of the whole
extended memory of an AT.  The assembly becomes significantly slower by
the extender, but at least it works...

There is a native OS/2 \marginpar{{\em OS/2}} version of AS for friends of
IBM's OS/2 operating system.  Since version 1.41r8, this is a full 32-bit
OS/2 application, which of course means that OS/2 2.x and at least an
80386 CPU are mandatory.

You can leave \marginpar{{\em UNIX}} the area of PCs-only with the C
version of AS that was designed to be compilable on a large number of UNIX
systems (this includes OS/2 with the emx compiler) without too much of
tweaking.  In contrast to the previously mentioned versions, the C version
is delivered in source code, i.e. one has to create the binaries by
oneself using a C compiler.  This is by far the simpler way (for me) than
providing a dozen of precompiled binaries for machines I sometimes only
have limited access to...

%%===========================================================================

\cleardoublepage
\chapter{Assembler Usage}

\begin{quote}\begin{raggedright}{\it
Scotty: Captain, we din\verb!'! can reference it! \\
Kirk:   Analysis, Mr. Spock? \\
Spock:  Captain, it doesn\verb!'!t appear in the symbol table. \\
Kirk:   Then it\verb!'!s of external origin? \\
Spock:  Affirmative. \\
Kirk:   Mr. Sulu, go to pass two. \\
Sulu:   Aye aye, sir, going to pass two. \\
}\end{raggedright}\end{quote}

%%---------------------------------------------------------------------------

\section{Hardware Requirements}

The hardware requirements of AS vary substantially from version to
version:

The DOS version \marginpar{{\em DOS}} will principally run on any
IBM-compatible PC, ranging from a PC/XT with 4-dot-little megahertz up to
a Pentium.  However, similar to other programs, the fun using AS increases
the better your hardware is.  An XT user without a hard drive will
probably have significant trouble placing the overlay file on a floppy
because it is larger than 500 Kbytes...the PC should therefore have at
least a hard drive, allowing acceptable loading times.  AS is not very
advanced in its main memory needs: the program itself allocates less than
300 Kbytes main memory, AS should therefore work on machines with at least
512 Kbytes of memory.

The version of AS \marginpar{{\em DPMI}} compiled for the DOS Protected
Mode Interface (DPMI) requires at least 1 Mbyte of free extended memory.
A total memory capacity of at least 2 Mbytes is therefore the absolute
minimum given one does not have other tools in the XMS (like disk caches,
RAM disks, or a hi-loaded DOS); the needs will rise then appropriately.
If one uses the DPMI version in a DOS box of OS/2, one has to assure that
DPMI has been enabled via the box's DOS settings (set to \tty{on} or
\tty{auto}) and that a sufficient amount of XMS memory has been assigned
to the box.  The virtual memory management of OS/2 will free you
from thinking about the amount of free real memory.

The hardware requirements of the OS/2 \marginpar{{\em OS/2}} version
mainly result from the needs of the underlying operating system, i.e. at
minimum an 80386SX processor, 8 Mbytes of RAM (resp. 4 Mbytes without the
graphical user interface) and 100..150 Mbytes of hard disk space.  AS2 is
only a 16-bit application and therefore it should also work on older OS/2
versions (thereby reducing the processor needs to at least an 80286
processor); I had however no chance to test this.

The C version of AS \marginpar{{\em UNIX}} is delivered as source code and
therefore requires a UNIX or OS/2 system equipped with a C compiler.  The
compiler has to fulfill the ANSI standard (GNU-C for example is
ANSI-compliant).  You can look up in the \tty{README} file whether your
UNIX system has already been tested so that the necessary definitions have
been made.  You should reserve about 15 Mbytes of free hard disk space for
compilation; this value (and the amount needed after compilation to store
the compiled programs) strongly differs from system to system, so you
should take this value only as a rough approximation.

%%---------------------------------------------------------------------------

\section{Delivery}

Principally, you can obtain AS in one of two forms: as a {\em binary} or a
{\em source} distribution.  In case of a binary distribution, one gets AS,
the accomanying tools and auxiliary files readily compiled, so you can
immediately start to use it after unpacking the archive to the desired
destination on your hard drive.
Binary distibutions are made for widespread platforms, where either the
majority of users does not have a compiler or the compilation is tricky
(currently, this includes DOS and OS/2).  A source distribution in
contrast contains the complete set of C sources to generate AS; it is
ultimately a snapshot of the source tree I use for development on AS.  The
generation of AS from the sources and their structure is described in
detail in appendix \ref{ChapSource}, which is why at this place, only the
contents and installation of a binary distribution will be described:

The contents of the archive is separated into several subdirectories,
therefore you get a directory subtree immediately after unpacking without
having to sort out things manually.  The individual directories contain
the following groups of files:
\begin{itemize}
\item{{\tt BIN}: executable programs, text resources;}
\item{{\tt INCLUDE}: include files for assembler programs, e.g. register
      definitions or standard macros;}
\item{{\tt MAN}: quick references for the individual programs in Unix
      'man' format.}
\end{itemize}
A list of the files found in every binary distribution is given in table
\ref{TabCommonPackageList}.  In case a file listed in one of these (or the
following) tables is missing, someone took a nap during copying (probably
me)...

\begin{center}\begin{longtable}{|l|l|}
\hline
File              & Function \\
\hline
\hline
\endhead
{\bf Directory BIN} & \\
\hline
AS.EXE            & executable of assembler \\
PLIST.EXE         & lists contents of code files \\
PBIND.EXE         & merges code files \\
P2HEX.EXE         & converts code files to hex files \\
P2BIN.EXE         & converts code files to binary files \\
AS.MSG            & text resources for AS \\
PLIST.MSG         & text resources for PLIST \\
PBIND.MSG         & text resources for PBIND \\
P2HEX.MSG         & text resources for P2HEX \\
P2BIN.MSG         & text resources for P2BIN \\
TOOLS.MSG         & common text resources for all tools \\
CMDARG.MSG        & common text resources for all programs \\
IOERRS.MSG        & \\
\hline
{\bf Directory DOC} & \\
\hline
AS\_DE.DOC        & german documentation, ASCII format \\
AS\_DE.HTML       & german documentation, HTML format \\
AS\_DE.TEX        & german documentation, LaTeX format \\
AS\_EN.DOC        & english documentation, ASCII format \\
AS\_EN.HTML       & english documentation, HTML format \\
AS\_EN.TEX        & english documentation, LaTeX format \\
\hline
{\bf Directory INCLUDE} & \\
\hline
BITFUNCS.INC      & functions for bit manipulation \\
CTYPE.INC         & functions for classification of \\
                  & characters \\
80C50X.INC        & register addresses SAB C50x \\
80C552.INC        & register addresses 80C552 \\
H8\_3048.INC      & register addresses H8/3048 \\
KENBAK.INC        & register addressed Kenbak-1 \\
REG166.INC        & addresses and instruction macros 80C166/167 \\
REG251.INC        & addresses and bits 80C251 \\
REG29K.INC        & peripheral addresses AMD 2924x \\
REG53X.INC        & register addresses H8/53x \\
REG6303.INC       & register addresses 6303 \\
REG683XX.INC      & register addresses 68332/68340/68360 \\
REG7000.INC       & register addresses TMS70Cxx \\
REG78310.INC      & register addresses \& vectors 78K3 \\
REG78K0.INC       & register addresses 78K0 \\
REG96.INC         & register addresses MCS-96 \\
REGACE.INC        & register addresses ACE \\
REGF8.INC         & register and memory addresses F8 \\
REGAVROLD.INC     & register and bit addresses AVR family (old)\\
REGAVR.INC        & register and bit addresses AVR family \\
REGCOLD.INC       & register and bit addresses Coldfire family \\
REGCOP8.INC       & register addresses COP8 \\
REGGP32.INC       & register addresses 68HC908GP32 \\
REGH16.INC        & register addresses H16\\
REGHC12.INC       & register addresses 68HC12 \\
REGM16C.INC       & register addresses Mitsubishi M16C \\
REGMSP.INC        & register addresses TI MSP430 \\
REGPDK.INC        & register and bit addresses PMC/PMS/PFSxxx \\
REGS12Z.INC       & register and bit addresses S12Z family \\
REGST6.INC        & register and macro definitions ST6 \\
REGST7.INC        & register and macro definitions ST7 \\
REGSTM8.INC       & register and macro definitions STM8 \\
REGST9.INC        & register and macro definitions ST9 \\
REGZ380.INC       & register addresses Z380 \\
STDDEF04.INC      & register addresses 6804 \\
STDDEF16.INC      & instruction macros and register addresses \\
                  & PIC16C5x \\
STDDEF17.INC      & register addresses PIC17C4x \\
STDDEF18.INC      & register addresses PIC16C8x \\
STDDEF2X.INC      & register addresses TMS3202x \\
STDDEF37.INC      & register and bit addresses TMS370xxx \\
STDDEF3X.INC      & peripheral addresses TMS320C3x \\
STDDEF4X.INC      & peripheral addresses TMS320C4x \\
STDDEF47.INC      & instruction macros TLCS-47 \\
STDDEF51.INC      & definition of SFRs and bits for \\
                  & 8051/8052/80515 \\
STDDEF56K.INC     & register addresses DSP56000 \\
STDDEF5X.INC      & peripheral addresses TMS320C5x \\
STDDEF60.INC      & instruction macros and register addresses \\
                  & PowerPC \\
REGSX20.INC       & register and bit addresses Parallax SX20/28 \\
AVR/\*.INC        & register and bit addresses AVR family \\
                  & (do not include directly, use REGAVR.INC) \\
COLDFIRE/\*.INC   & register and bit addresses ColdFire family \\
                  & (do not include directly, use REGCOLD.INC) \\
PDK/\*.INC        & register and bit addresses PMC/PMS/PFSxxx \\
                  & (do not include directly, use REGPDK.INC) \\
S12Z/\*.INC       & register and bit addresses S12Z family \\
                  & (do not include directly, use REGS12Z.INC) \\
ST6/\*.INC        & register and bit addresses ST6 family \\
                  & (do not include directly, use REGST6.INC) \\
ST7/\*.INC        & register and bit addresses ST7 family \\
                  & (do not include directly, use REGST7.INC) \\
STM8/\*.INC       & register and bit addresses STM8 family \\
                  & (do not include directly, use REGSTM8.INC) \\
STDDEF62.INC      & register addresses and macros ST6 (old)\\
STDDEF75.INC      & register addresses 75K0 \\
STDDEF87.INC      & register and memory addresses TLCS-870 \\
STDDEF90.INC      & register and memory addresses TLCS-90 \\
STDDEF96.INC      & register and memory addresses TLCS-900 \\
STDDEFXA.INC      & SFR and bit addresses Philips XA \\
STDDEFZ8.INC      & register addresses Z8 family (old) \\
REGZ8.INC         & register addresses Z8 family (new) \\
Z8/\*.INC         & register and bit addresses Z8 family \\
                  & (do not include directly, use REGZ8.INC) \\
\hline
{\bf Directory LIB} & \\
\hline
{\bf Directory MAN} & \\
\hline
ASL.1             & Short Reference for AS \\
PLIST.1           & Short Reference for PLIST \\
PBIND.1           & Short Reference for PBIND \\
P2HEX.1           & Short Reference for P2HEX \\
P2BIN.1           & Short Reference for P2BIN \\
\hline
\caption{Standard Contents of a Binary Distribution
         \label{TabCommonPackageList}}
\end{longtable}\end{center}


Depending on the platform, a binary distribution however may contain more
files to allow operation, like files necessary for DOS extenders. In case
of the DOS DPMI version \marginpar{{\em DPMI}}, the extensions listed in
table \ref{TabDPMIPackageList} result.  Just to mention it: it is
perfectly O.K. to replace the tools with their counterparts from a DOS
binary distribution; on the on hand, they execute significantly faster
without the extender's overhead, and on the other hand, they do not need
the extended memory provided by the extender.

\begin{table*}[htp]
\begin{center}\begin{tabular}{|l|l|}
\hline
File              & Function \\
\hline
\hline
{\bf Directory MAN} & \\
\hline
ASL.1             & quick reference for AS \\
PLIST.1           & quick reference for PLIST \\
PBIND.1           & quick reference for PBIND \\
P2HEX.1           & quick reference for P2HEX \\
P2BIN.1           & quick reference for P2BIN \\
\hline
\hline
{\bf Directory BIN} & \\
\hline
DPMI16BI.OVL   & DPMI server for the assembler \\
RTM.EXE        & runtime module of the extender \\
\hline
\end{tabular}\end{center}
\caption{Additional Files in a DPMI Binary Distribution
         \label{TabDPMIPackageList}}
\end{table*}

An OS/2 binary distribution \marginpar{{\em OS/2}} contains in addition to
the base files a set of DLLs belonging to the runtime environment of the
emx compiler used to build AS (table \ref{TabOS2PackageList}).  In case
you already have these DLLs (or newer versions of them), you may delete
these and use your ones insted.

\begin{table*}[htp]
\begin{center}\begin{tabular}{|l|l|}
\hline
File              & function \\
\hline
\hline
{\bf Directory BIN} & \\
\hline
EMX.DLL           & runtime libraries for AS and \\
EMXIO.DLL         & its tools \\
EMXLIBC.DLL       & \\
EMXWRAP.DLL       & \\
\hline
\end{tabular}\end{center}
\caption{Additional Files in an OS/2 binary distribution
         \label{TabOS2PackageList}}
\end{table*}

%%---------------------------------------------------------------------------

\section{Installation}

There is no need for a \marginpar{{\em DOS}} special installation prior to
usage of AS.  It is sufficient to unpack the archive in a fitting place
and to add a few minor settings.  For example, this is an installation a
user used to UNIX-like operating systems might choose:

Create a directory \verb!c:\as! an (I will assume in the following that
you are going to install AS on drive C), change to this directory and
unpack the archiv, keeping the path names stored in the archive (when
using PKUNZIP, the command line option \verb!-d! is necessary for that).
You now should have the following directory tree:
\begin{verbatim}
c:\as
c:\as\bin
c:\as\include
c:\as\lib
c:\as\man
c:\as\doc
c:\as\demos
\end{verbatim}
Now, append the directory \verb!c:\as\bin! to the \tty{PATH} statement in
your \tty{AUTOEXEC.BAT}, which allows the system to find AS and its tools.
With your favourite text editor, create a file named \tty{AS.RC} in the
\tty{lib} directory with the following contents:
\begin{verbatim}
-i c:\as\include
\end{verbatim}
This so-called {\em key file} tells AS where to search for its include
files.  The following statement must be added to your \tty{AUTOEXEC.BAT}
to tell AS to read this file:
\begin{verbatim}
set ASCMD=@c:\as\lib\as.rc
\end{verbatim}
There are many more things you can preset via the key file; they are
listed in the following section.

The installation of the DPMI version \marginpar{{\em DPMI}} should
principally take the same course as for the pure DOS version; as soon as
the PATH contains the {\tt bin} directory, the DOS extender's files will
be found automatically and you should not notice anything of this
mechanism (except for the longer startup time...).  When working on an
80286-based computer, it is theoretically possible tha you get confronted
with the following message upon the first start:
\begin{verbatim}
  machine not in database (run DPMIINST)
\end{verbatim}
Since the DPMIINST tool ins not any more included in newer versions of
Borland's DOS extender, I suppose that this is not an item any more...in
case you run into this, contact me!

The installation of the OS/2 version \marginpar{{\em OS/2}} can generally
be done just like for the DOS version, with the addition that the DLLs
have to be made visible for the operating system. In case you do not want
to extend the {\tt LIBPATH} entry in your {\tt CONFIG.SYS}, it is of
course also valid to move the DLLs into a directory already listed in {\tt
LIBPATH}.

As already mentioned, the installation instructions in this section limit
themselves to binary distributions.  Since an installation under Unix
\marginpar{{\em UNIX}} is currently alway a source-based installation, the
only hint I can give here is a reference to appendix \ref{ChapSource}.

%%---------------------------------------------------------------------------

\section{Start-Up Command, Parameters}
\label{SectCallConvention}

AS is a command line driven program, i.e. all parameters and file
options are to be given in the command line.

A couple of message files belongs to AS (recognizable by their suffix {\tt
MSG}) AS accesses to dynamically load the messages appropriate for the
national language.  AS searches the following directories for these files:
\begin{itemize}
\item{the current directory;}
\item{the EXE-file's directory;}
\item{the directory named in the {\tt AS\_MSGPATH} environment variable,
      or alternitavely the directories listed in the {\tt PATH} environment
      variable;}
\item{the directory compiled into AS via the {\tt LIBDIR} macro.}
\end{itemize}
These files are {\em indispensable} for a proper operation of AS, i.e. AS
will terminate immediately if these files are not found.

The language selection (currently only German and English) is based on the
{\tt COUNTRY} setting under DOS and OS/2 respectively on the {\tt LANG}
environment variable under Unix.

In order to fulfill \marginpar{{\em DOS}} AS's memory requirements under
DOS, the various code generator modules of the DOS version were moved into
an overlay which is part of the EXE file.  A separate OVR file like in
earlier versions of AS therefore dose not exist any more, AS will however
still attempt to reduce the overlaying delays by using eventually
available EMS or XMS memory.  In case this results in
trouble, you may suppress usage of EMS or XMS by setting the environment
variable \tty{USEXMS} or \tty{USEEMS} to \tty{n}.  E.g., it is possible to
suppress the using of XMS by the command:
\begin{verbatim}
   SET USEXMS=n
\end{verbatim}
Since AS performs all in- and output via the operating system (and
therefore it should run also on not 100\% compatible DOS-PC's) and
needs some basic display control, it emits ANSI control sequences
during the assembly.
In case you \marginpar{{\em DOS/}} should see strange characters in the
messages displayed by AS, your \tty{CONFIG.SYS} is obviously lacking a
line like this:
\begin{verbatim}
   device=ansi.sys
\end{verbatim}
but the further \marginpar{{\em DPMI}} functions of AS will not be
influenced hereby.  Alternatively you are able to suppress the output of
ANSI sequences completely by setting the environment variable
\tty{USEANSI} to \tty{n}.

The DOS extender of the DPMI version \marginpar{{\em DPMI}} can be
influenced in its memory allocation strategies by a couple of environment
variables; if you need to know their settings, you may look up them in the
file \tty{DPMIUSER.DOC}.  ASX is additionally able to extend the available
memory by a swap file.  To do this, set up an environment variable
\tty{ASXSWAP} in the following way:
\begin{verbatim}
  SET ASXSWAP=<size>[,file name]
\end{verbatim}
The size specification has to be done in megabytes and \bb{has} to be done.
The file name in contrast is optional; if it is missing, the file is
named \tty{ASX.TMP} and placed in the current directory.  In any case, the
swap file is deleted after program end.

The command line parameters can roughly be divided into three categories:
switches, key file references (see below) and file specifications.
Parameters of these two categories may be arbitrarily mixed in the command
line.  The assembler evaluates at first all parameters and then assembles
the specified files.  From this follow two things:
\begin{itemize}
\item{the specified switches affect all specified source files. If
      several source files shall be assembled with different switches,
      this has to be done in separate runs.}
\item{it is possible to assemble more than one file in one shot and to
      bring it to the top, it is allowed that the file specs contain
      wildcards.}
\end{itemize}
Parameter switches are recognized by AS by starting with
a slash (/) or hyphen (-).  There are switches that are only one
character long and additionally switches composed of a whole word.
Whenever AS cannot interpret a switch as a whole word, it tries to
interprete every letter as an individual switch.  For example, if you
write
\begin{verbatim}
 -queit
\end{verbatim}
instead of
\begin{verbatim}
 -quiet
\end{verbatim}
AS will take the letters \tty{q, u, e, i}, and \tty{t} as individual
switches.  Multiple-letter switches additionally have the difference to
single-letter switches that AS will accept an arbitrary mixture of upper
and lower casing, whereas single-letter switches may have a different
meaning depending on whether upper or lower case is used.

At the moment, the following switches are defined:
\ttindex{SHARED}
\begin{itemize}
\item{\tty{l}: sends assembler listing to console terminal (mostly screen).
      In case several passes have to be done, the listing of all
      passes will be send to the console (in opposite to the next
      option).}
\item{\tty{L}: writes assembler listing into a file. The list file will get
      the same name as the source file, only the extension is
      replaced by \tty{LST}.  Except one uses... }
\item{\tty{OLIST}: with a fiel name as argument allows to redirect the
      listing to a different file or a different path.  This option may
      be used multiple times in case multiple files are assembled with
      one execution.}
\item{\tty{LISTRADIX}: By default, all numeric output in the listing
      (addresses, generated code, symbol values) is written in hexadecimal
      notation.  This switch requests usage of a different number system in the
      range of 2 to 36.  For instance, '-listradix 8' requests octal output.}
\item{\tty{SPLITBYTE [character]}: Display numbers in the listing in byte groups,
      separated by the given character.  A period is used as separator if
      no explicit character is given.  This option is usually used in conjunction
      with the \tty{LISTRADIX} option.  For instance, list radix 8 with a
      period as character results in the so-called 'split octal' notation.}
\item{\tty{o}: Sets the new name of the code file generated by AS.  If this
      option is used multiple times, the names will be assigned, one
      after the other, to the source files which have to be
      assembled.  A negation (see below) of this option in
      connection with a name erases this name from the list.  A
      negation without a name erases the whole list.}
\item{\tty{SHAREOUT}:ditto for a SHARE file eventually to be created.}
\item{\tty{c}: SHARED-variables will be written in a format which permits
      an easy integration into a C-source file.  The extension of
      the file is \tty{H}.}
\item{\tty{p}: SHARED-variables will be written in a format which permits
      easy integration into the CONST-block of a Pascal program.
      The extension of the file is \tty{INC}.}
\item{\tty{a}: SHARED-variables will be written in a format which permits
      easy integration into an assembler source file. The
      extension of the file is \tty{INC}.}
\end{itemize}
Concerning effect and function of the SHARED-symbols please see
chapters \ref{ChapShareMain} resp. \ref{ChapShareOrder}.
\begin{itemize}
\item{\tty{g [format]}: This switch instructs AS to create an additional
      file that contains debug information for the program.  Allowed
      formats are the AS-specific MAP format ({\tt format=MAP}), a
      NoICE-compatible command file ({\tt format=NOICE}), and the Atmel
      format used by the AVR tools ({\tt format=ATMEL}). The information
      stored in the MAP format is comprised of a symbol table and a table
      describing the assignment of source lines to machine addresses.  A
      more detailed description of the MAP format can be found in section
      \ref{SectDebugFormat}  The file's extension is \tty{MAP}, \tty{NOI},
      resp. \tty{OBJ}, depending on the chosen format.  If no explicit
      format specification is done, the MAP format is chosen.}
\item{\tty{noicemask [value]}: By default, AS lists only symbols from the
      CODE segment in NoICE debug info files.  With this option and an
      integer value interpreted as a bit mask, symbols fom other segments
      may be added.  The assignment of segments to bit positions may
      be taken from table \ref{TabSegmentNums}.}
\item{\tty{w}: suppress issue of warnings;}
\item{\tty{E [file]}: error messages and warnings produced by AS will be
      redirected to a file. Instead of a file, the 5 standard
      handles (STDIN..STDPRN) can also be specified as
      \tty{!0} to \tty{!4} . Default is \tty{!2}, meaning STDERR.  If the
      file option is left out, the name of the error file
      is the same as of the source file, but with the
      extension \tty{LOG}.}
\item{\tty{q}: This switch suppresses all messages of AS, the exceptions are
      error messages and outputs which are are forced from the
      source file.  The time needed for assembly is slightly reduced
      hereby and if you call AS from a shell there is no redirection
      required.  The disadvantage is that you may ''stay in the dark''
      for several minutes ... It is valid to write \tty{quiet} instead
      of \tty{q}.}
\item{\tty{h}: write hexadecimal numbers in lowercase instead of capital
      letters. This option is primarily a question of personal
      taste.}
\item{\tty{i $<$path list$>$}: issues a list of directories where the
      assembler shall automatically search for include
      files, in case it didn't find a file in the
      current directory.  The different directories
      have to be separated by semicolons.}
\item{\tty{u}: calculate a list of areas which are occupied in the segments.
      This option is effective only in case a listing is
      produced. This option requires considerable additional
      memory and computing performance. In normal operation it
      should be switched off.}
\item{\tty{C}: generates a list of cross references.  It lists which (global)
      symbols are used in files and lines.  This list will also be
      generated only in case a listing is produced.  This option
      occupies, too, additional memory capacity during assembly.}
\item{\tty{s}: issues a list of all sections (see chapter
      \ref{ChapLocSyms}).  The nesting is indicated  by indentations
      (Pascal like).}
\item{\tty{t}: by means of this switch it is possible to separate single
      components of the standard issued assembler-listing.  The assignment
      of bits to parts can be found in the next section, where the exact
      format of the assembly listing is explained.}
\item{\tty{D}: defines symbols.  The symbols which are specified behind this
      option and separated by commas are written to the
      global symbol table before starting the assembly.  As default
      these symbols are written as integer numbers with the
      value TRUE, by means of an appended equal sign, however, you
      can select other values.  The expression following the equals
      sign may include operators or internal functions, but \bb{not}
      any further symbols, even if these should have been defined
      before in the list!  Together with the commands for
      conditional assembly (see there) you may produce different
      program versions out of one source file by command line
      inputs. {\bf CAUTION!} If the case-sensitive mode is used, this has
      to be specified in the command line {\em before} any symbol
      definitions, otherwise symbol names will be converted to upper
      case at this place!}
\item{\tty{A}: stores the list of global symbols in another, more compact
      form.  Use this option if the assembler crashes with a stack
      overflow because of too long symbol tables.  Sometimes this
      option can increase the processing speed of the assembler, but
      this depends on the sources.}
\item{\tty{x}: Sets the level of detail for error messages.  The level
      is increased resp. decreased by one each time this option is given.
      While on level 0 (default) only the error message itself is printed,
      an extended message is added beginning at level 1 that should
      simplify the identification of the error's cause.  Appendix
      \ref{ChapErrMess} lists which error messages carry which extended
      messages.  At level 2 (maximum), the source line containing the
      error is additionally printed.}
\item{\tty{n}: If this option is set, the error messages will be issued
      additionally with their error number (see appendix
      \ref{ChapErrMess}).  This is primarily intended for use with shells
      or IDE's to make the identification of errors easier by those
      numbers.}
\item{\tty{U}: This option switches AS to the case-sensitive mode, i.e.
      upper and lower case in the names of symbols, sections, macros,
      character sets, and user-defined functions will be distinguished.
      This is not the case by default.}
\item{\tty{P}: Instructs AS to write the source text processed by macro
      processor and conditional assembly into a file.  Additional
      blank and pure comment lines are missing in this file.  The
      extension of this file is \tty{I}.}
\item{\tty{M}: If this switch is given, AS generates a file, that contains
      definitions of macros defined in the source file that did not
      use the \tty{NOEXPORT} option.  This new file has the same name as
      the source file, only the extension is modified into \tty{MAC}.}
\item{\tty{G}: this switch defines whether AS should produce code or not.
      If switched off, the processing will be stopped after the macro
      processor. This switch is activated by default (logically,
      otherwise you would not get a code file). This switch can be
      used in conjunction with the \tty{P} switch, if only the macro
      processor of AS shall be used.}
\item{\tty{r [n]}: issue warnings if situations occur that force a further
      pass. This information can be used to reduce the number of
      passes.  You may optionally specify the number of the
      first pass where issuing of such messages shall start.
      Without this argument, warnings will come starting with
      the first pass.  Be prepared for a bunch of messages!!}
\item{\tty{relaxed}: this switch enables the RELAXED mode right from the
      beginning of the program, which otherwise has to be enabled by the
      pseudo instruction of sane name(see section \ref{SectRELAXED}).}
\item{\tty{supmode}: this switch enables right from the beginning of the
      program usage of machine instructions that may only be used in the
      processor's supervisor mode (see section \ref{SectSUPMODE}).}
\item{\tty{Y}: This switch instructs AS to to suppress all messages about
      out-of-branch conditions, once the necessity for another pass is given.
      See section \ref{ForwRefs} for the (rare) situations that might make
      use of this switch necessary.}
\item{\tty{cpu $<$name$>$}: this switch allows to set the target processor
      AS shall generate code for, in case the source file does not contain
      a {\tt CPU} instruction and is not 68008 code.  If the selected target
      supports CPU arguments (see section \ref{SectCPU}), they may be used
      on the command line as well.}
\item{\tty{alias $<$new$>$=$<$old$>$}:
      defines the processor type \tty{$<$new$>$} to be an alias for the
      type \tty{$<$old$>$}.  See section \ref{SectAlias} for the sense of
      processor aliases.}
\item{{\tt gnuerrors}: display messages about errors resp. warnings not
      in the AS standard format, but instead in a format similar to the
      GNU C compiler.  This simplifies the integration of AS into
      environments tuned for this format, however also suppresses the
      display of precise error positions in macro bodies!}
\item{{\tt maxerrors [n]}: instructs the assembler to terminate
      assembly after the given number of errors.}
\item{{\tt maxerrors [n]}: instructs the assembler to terminate
      assembly if the include nesting level exceeds the given limit
      (default is 200).}
\item{{\tt Werror}: instructs the assembler to treat warnings as errors.}
\item{\tty{compmode}: This switch instructs the assembler to operate by
      default in compatibility mode.  See section \ref{SectCompMode} for
      more information about this mode.}
\end{itemize}
As long as switches require no arguments and their concatenation does
not result in a multi-letter switch, it is possible to specify several
switches at one time, as in the following example :
\begin{verbatim}
 as test*.asm firstprog -cl /i c:\as\8051\include
\end{verbatim}
All files \tty{TEST*.ASM} as well as the file \tty{FIRSTPROG.ASM} will be
assembled, whereby listings of all files are displayed on the
console terminal.  Additional sharefiles will be generated in the C-
format.  The assembler should search for additional include files
in the directory \verb!C:\AS\8051\INCLUDE!.

This example shows that the assembler assumes \tty{ASM} as the default
extension for source files.

A bit of caution should be applied when using switches that have
optional arguments: if a file specification immediately follows such
aswitch without the optional argument, AS will try to interprete the
file specification as argument - what of course fails:
\begin{verbatim}
 as -g test.asm
\end{verbatim}
The solution in this case would either be to move the -g option the
end or to specify an explicit MAP argument.


Beside from specifying options in the command line, permanently
needed options may be placed in the environment variable \tty{ASCMD}.  For
example, if someone always wants to have assembly listings and has a
fixed directory for include files, he can save a lot of typing with
the following command:
\begin{verbatim}
 set ascmd=-L -i c:\as\8051\include
\end{verbatim}
The environment options are processed before the command line,
so options in the command line can override contradicting ones in the
environment variable.

In the case of very long path names, space in the \tty{ASCMD} variable may
become a problem.  For such cases a key file may be the alternative,
in which the options can be written in the same way as in the command
line or the \tty{ASCMD}-variable.  But this file may contain several lines
each with a maximum length of 255 characters.  In a key file it is
important, that for options which require an argument, switches and
argument have to be written in the \bb{same} line.  AS gets informed of
the name of the key file by a \tty{@} aheaded in the \tty{ASCMD} variable,
e.g.
\begin{verbatim}
set ASCMD=@c:\as\as.key
\end{verbatim}
In order to neutralize options in the \tty{ASCMD} variable (or in the
key file), prefix the option with a plus sign.  For example, if you
do not want to generate an assembly listing in an individual case,
the option can be retracted in this way:
\begin{verbatim}
as +L <file>
\end{verbatim}
Naturally it is not consequently logical to deny an option by a
plus sign....  UNIX soit qui mal y pense.

References to key files may not only come from the {\tt ASCMD} variable,
but also directly from the command line.  Similarly to the {\tt ASCMD}
variable, prepend the file's name with a \@ character:
\begin{verbatim}
 as @<file> ....
\end{verbatim}
The options read from a key file in this situation are processed as if
they had been written out in the command line in place of the reference,
{\em not} like the key file referenced by the {\tt ASCMD} variable that is
processed prior to the command line options.

Referencing a key file from a key file itself is not allowed and will be
answered wit an error message by AS.

In case that you like to start AS from another program or a shell and
this shell hands over only lower-case or capital letters in the
command line, the following workaround exists: if a tilde (\verb!~!) is put
in front of an option letter, the following letter is always
interpreted as a lower-case letter.  Similarly a \tty{\#} demands the
interpretation as a capital letter.  For example, the following
transformations result for:
\begin{verbatim}
 /~I ---> /i
 -#u ---> -U
\end{verbatim}
In dependence of the assembly's outcome, the assembler ends with
the following return codes:
\begin{description}
\item[0]{error free run, at maximum warnings occurred}
\item[1]{The assembler displayed only its command-line parameters and
         terminated immediately afterwards.}
\item[2]{Errors occurred during assembly, no code file has been produced.}
\item[3]{A fatal error occurred what led to immediate termination of the run.}
\item[4]{An error occurred already while starting the assembler.
         This may be a parameter error or a faulty overlay file.}
\item[255]{An internal error occurred during initialization that should not
         occur in any case...reboot, try again, and contact me if the
         problem is reproducible!}
\end{description}
Similar to UNIX, OS/2 \marginpar{{\em OS/2}} extends an application's data
segment on demand when the application really needs the memory.
Therefore, an output like
\begin{verbatim}
  511 KByte available memory
\end{verbatim}
does not indicate a shortly to come system crash due to memory lack,
it simply shows the distance to the limit when OS/2 will push up the
data segment's size again...

As there is no compatible way in C \marginpar{{\em UNIX}} under different
operating systens to find out the amount of available memory resp. stack,
both lines are missing completely from the statistics the C version prints.

%%---------------------------------------------------------------------------

\section{Format of the Input Files}
\label{AttrTypes}

Like most assemblers, AS expects exactly one instruction per line
(blank lines are naturally allowed as well).  The lines must not be
longer than 255 characters, additional characters are discarded.

A single line has following format:
\begin{verbatim}
[label[:]] <mnemonic>[.attr] [param[,param..]] [;comment]
\end{verbatim}
A line may also be split over several lines in the source file,
continuation characters chain these parts together to a single line.  One
must however consider that, due to the internal buffer structure, the
total line must not be longer than 256 characters.  Line references in
error messages always relate to the last line of such a composed source
line.
\par
The colon for the label is optional, in case the label starts in the
first column (the consequence is that a machine or pseudo
instruction must not start in column 1).  It is necessary to set the
colon in case the label does not start in the first column so that AS
is able to distinguish it from a mnemonic.  In the latter case, there
must be at least one space between colon and mnemonic if the processor
belongs to a family that supports an attribute that denotes an
instruction format and is separated from the mnemonic by a colon.  This
restriction is necessary to avoid ambiguities: a distinction between a
mnemonic with format and a label with mnemonic would otherwise be
impossible.

Some signal processor families from Texas Instruments optionally use
a double line (\verb!||!) in place of the label to signify the
prallel execution with the previous instruction(s).  If these two
assembler instructions become a single instruction word at machine
level (C3x/C4x), an additional label in front of the second
instruction of course does not make sense and is not allowed.  The
situation is different for the C6x with its instruction packets of
variable length: If someone wants to jump into the middle of an
instruction packet (bad style, if you ask me...), he has to place the
necessary label {\em before} into a separate line.  The same is valid
for conditions, which however may be combined with the double line in
a single source line.

The attribute is used by a couple of processors to specify variations or
different codings of a certain instruction.  The most prominent usage of
the attibute is is the specification of the operand size, for example in
the case of the 680x0 family (table \ref{TabAttrs}).
\begin{table*}[htb]
\begin{center}\begin{tabular}{|l|l|l|}
\hline
attribute & arithmetic-logic instruction & jump instruction\\
\hline
\hline
B     & byte (8 bits)                       &  8-bit-displacement \\
W     & word (16 bits)                      &  16-bit-displacement \\
L     & long word (32 bits)                 &  16-bit-displacement \\
Q     & quad word (64 bits)                 &  --------- \\
C     & half precision (16 bits)            &  --------- \\
S     & single precision (32 bits)          &  8-bit-displacement \\
D     & double precision (64 bits)          &  --------- \\
X     & extended precision (80/96 bits)     &  32-bit-displacement \\
P     & decimal floating point (80/96 bits) &  --------- \\
\hline
\end{tabular}\end{center}
\caption{Allowed Attributes (Example 680x0) \label{TabAttrs}}
\end{table*}
\par
Since this manual is not also meant as a user's manual for the processor
families supported by AS, this is unfortunately not the place to enumerate
all possible attributes for all families.  It should however be mentioned
that in general, not all instructions of a given instruction set allow all
attributes and that the omission of an attribute generally leads to the
usage of the ''natural'' operand size of a processor family. For more
thorough studies, consult a reasonable programmer's manual, e.g.
\cite{Williams} for the 68K's.

In the case of TLCS-9000, H8/500, and M16(C), the attribute serves
both as an operand size specifier (if it is not obvious from the
operands) and as a description of the instruction format to be used.
A colon has to be used to separate the format from the operand size,
e.g. like this:
\begin{verbatim}
    add.w:g   rw10,rw8
\end{verbatim}
This example does not show that there may be a format specification
without an operand size.  In contrast, if an operand size is used
without a format specification, AS will automatically use the
shortest possible format.  The allowed formats and operand sizes
again depend on the machine instruction and may be looked up e.g. in
\cite{Tosh900}, \cite{HitH8_5}, \cite{MitM16}, resp. \cite{MitM16C}.

The number of instruction parameters depends on the mnemonic and is
principally located between 0 and 20.  The separation of the parameters
from each other is to be performed only by commas (exception: DSP56xxx,
its parallel data transfers are separated with blanks).  Commas that
are included in brackets or quotes, of course, are not taken into
consideration.

Instead of a comment at the end, the whole line can consist of
comment if it starts in the first column with a semicolon.

To separate the individual components you may also use tabulators
instead of spaces.

%%---------------------------------------------------------------------------

\section{Format of the Listing}

The listing produced by AS using the command line options i or I is
roughly divisible into the following parts :
\begin{enumerate}
\item{issue of the source code assembled;}
\item{symbol list;}
\item{usage list;}
\item{cross reference list.}
\end{enumerate}
The two last ones are only generated if they have been demanded by
additional command line options.

In the first part, AS lists the complete contents of all source files
including the produced code.  A line of this listing has the following
form:
\begin{verbatim}
[<n>] <line>/<address> <code> <source>
\end{verbatim}
In the field \tty{n}, AS displays the include nesting level.  The main file
(the file where assembly was started) has the depth 0, an included
file from there has depth 1 etc..  Depth 0 is not displayed.

In the field \tty{line}, the source line number of the referenced file is
issued. The first line of a file has the number 1.  The address at
which the code generated from this line is written follows after the
slash in the field \tty{address}.

The code produced is written behind \tty{address} in the field \tty{code},
in hexadecimal notation. Depending on the processor type and actual
segment the values are formatted either as bytes or 16/32-bit-words.
If more code is generated than the field can take, additional lines
will be  generated, in which case only this field is used.

Finally, in the field \tty{source}, the line of the source file is issued in
its original form.

The symbol table was designed in a way that it can be displayed on an
80-column display whenever possible. For symbols of ''normal length'',
a double column output is used.  If symbols exceed (with their name
and value) the limit of 40 columns (characters), they will be issued
in a separate line. The output is done in alphabetical order.
Symbols that have been defined but were never used are marked with a
star (*) as prefix.

The parts mentioned so far as well as the list of all macros/functions
defined can be selectively masked out from the listing.
This can be done by the already mentioned command line switch \tty{-t}.
There is an internal byte inside AS whose bits represent which parts
are to be written.  The assignment of bits to parts of the listing is
listed in table \ref{TabTBits}.
\par
\begin{table*}[htb]
\begin{center}\begin{tabular}{|l|l|}
\hline
bit &  part \\
\hline
\hline
0   & source file(s) + produced code \\
1   & symbol table \\
2   & macro list \\
3   & function list \\
4   & line numbering \\
5   & register symbol list \\
7   & character set table \\
\hline
\end{tabular}\end{center}
\caption{Assignment of Bits to Listing Components\label{TabTBits}}
\end{table*}
All bits are set to 1 by default, when using the switch
\begin{verbatim}
-t <mask>
\end{verbatim}
Bits set in \tty{$<$mask$>$} are cleared, so that the respective listing
parts are suppressed.  Accordingly it is possible to switch on single
parts again with a plus sign, in case you had switched off too much
with the \tty{ASCMD} variable... If someone wants to have, for example,
only the symbol table, it is enough to write:
\begin{verbatim}
-t 2
\end{verbatim}
The usage list issues the occupied areas hexadecimally for every
single segment.  If the area has only one address, only this is written,
otherwise the first and last address.

The cross reference list issues any defined symbol in alphabetical
order and has the following form:
\begin{verbatim}
 symbol <symbol name> (=<value>,<file>/<line>):
  file <file 1>:
  <n1>[(m1)]  ..... <nk>[(mk)]
  .
  .
  file <file l>:
  <n1>[(m1)]  ..... <nk>[(mk)]
\end{verbatim}
The cross reference list lists for every symbol in which files and lines
it has been used.  If a symbol was used several times in the same line,
this would be indicated by a number in brackets behind the line number.
If a symbol was never used, it would not appear in the list; The same is
true for a file that does not contain any references for the symbol in
question.

\bb{CAUTION!}  AS can only print the listing correctly if it was
previously informed about the output media's page length and width!
This has to be done with the \tty{PAGE} instruction (see there).  The
preset default is a length of 60 lines and an unlimited line width.

%%---------------------------------------------------------------------------

\section{Symbol Conventions}
\label{SectSymConv}

Symbols are allowed to be up to 255 characters long (as hinted already
in the introduction) and are being distinguished on the whole
length, but the symbol names have to meet some conventions:

Symbol names are allowed to consist of a random combination of
letters, digits, underlines and dots, whereby the first character must
not be a digit. The dot is only allowed to meet the MCS-51 notation of
register bits and should - as far as possible - not be used in own symbol
names.  To separate symbol names in any case the underline (\tty{\_}) and not
the dot (\tty{.}) should be used .

AS is by default not case-sensitive, i.e. it does not matter whether
one uses upper or lower case characters.  The command line switch \tty{U}
however allows to switch AS into a mode where upper and lower case
makes a difference.  The predefined symbol \tty{CASESENSITIVE} signifies
whether AS has been switched to this mode: TRUE means case-sensitiveness,
and FALSE its absence.

Table \ref{TabPredefined} shows the most important symbols which are
predefined by AS.
\begin{table*}[htb]
\begin{center}\begin{tabular}{|l|l|}
\hline
name          & meaning \\
\hline
\hline
TRUE          & logically ''true'' \\
FALSE         & logically ''false'' \\
CONSTPI       & Pi (3.1415.....) \\
VERSION       & version of AS in BCD-coding, \\
              & e.g. 1331 hex for version 1.33p1 \\
ARCHITECTURE  & target platform AS was compiled for, in \\
              & the style processor-manufacturer-operating \\
              & system \\
DATE          & date and \\
TIME          & time of the assembly (start) \\
MOMCPU        & current target CPU \\
              & (see the CPU instruction) \\
MOMFILE       & current source file \\
MOMLINE       & line number in source file \\
MOMPASS       & number of the currently running pass \\
MOMSECTION    & name of the current section \\
              & or an empty string \\
\verb!*!, \$ resp. PC & current value of program counter \\
\hline
\end{tabular}\end{center}
\caption{Predefined Symbols\label{TabPredefined}}
\end{table*}
\bb{CAUTION!}  While it does not matter in case-sensitive mode which
combination of upper and lower case to use to reference predefined
symbols, one has to use exactly the version given above (only upper
case) when AS is in case-sensitive mode!

Additionally some pseudo instructions define symbols that reflect the
value that has been set  with these instructions.  Their descriptions
are explained at the individual commands belonging to them.

A hidden feature (that has to be used with care) is that symbol names
may be assembled from the contents of string symbols.  This can be
achieved by framing the string symbol's name with curly braces and
inserting it into the new symbol's name.  This allows for example to
define a symbol's name based on the value of another symbol:
\begin{verbatim}
cnt             set     cnt+1
temp            equ     "\{CNT}"
                jnz     skip{temp}
                .
                .
skip{temp}:     nop
\end{verbatim}
\bb{CAUTION:} The programmer has to assure that only valid symbol names
are generated!

A complete list of all symbols predefined by AS can be found in
appendix \ref{AppInternSyms}.

Apart from its value, every symbol also owns a marker which signifies to
which {\em segment} it belongs.  Such a distinction is mainly needed for
processors that have more than one address space.  The additional
information allows AS to issue a warning when a wrong instruction is used
to access a symbol from a certain address space.  A segment attribute is
automatically added to a symbol when is gets defined via a label or a
special instruction like \tty{BIT}; a symbol defined via the ''allround
instructions'' \tty{SET} resp. \tty{EQU} is however ''typeless'', i.e. its
usage will never trigger warnings.  A symbol's segment attribute may be
queried via the buit-in function \tty{SYMTYPE}, e.g.:
\begin{verbatim}
Label:
        .
        .
Attr    equ     symtype(Label)  ; results in 1
\end{verbatim}
The individual segment types have the assigned numbers listed in table
\ref{TabSegNums}.  Register symbols which do not really fit into the order
of normal symbols are explained in section \ref{SectRegSyms}.  The
\tty{SYMTYPE} function delivers -1 as result when called with an undefined
symbol as argument.  However, if all you want to know is whether a symbol
is defined or not, you may as well use the \tty{DEFINED} function.
\begin{table}[htb]
\begin{center}
\begin{tabular}{|l|c|}
\hline
segment & return value \\
\hline
$<$none$>$ & 0 \\
CODE & 1 \\
DATA & 2 \\
IDATA & 3 \\
XDATA & 4 \\
YDATA & 5 \\
BITDATA & 6 \\
IO & 7 \\
REG & 8 \\
ROMDATA & 9 \\
EEDATA & 10 \\
$<$Register Symbol$>$ & 128 \\
\hline
\end{tabular}
\end{center}
\caption{return values of the \tty{SYMTYPE} function\label{TabSegNums}}
\end{table}

%%---------------------------------------------------------------------------

\section{Temporary Symbols}

Especially when dealing with programs that contain sequences of loops of
if-like statements, one is continuously faced with the problem of
inventing new names for labels - labels of which you know exactly that you
will never need to reference them again afterwards and you really would
like to get 'rid' of them somehow.  A simple solution if you don't want to
swing the large hammer of sections (see chapter \ref{ChapLocSyms}) are
{\em temporary} symbols which remain valid as long as a new,
non-temporary symbol gets defined.  Other assemblers offer a similar
mechanism which is commonly referred as 'local symbols'; however, for the
sake of a better distinction, I want to stay with the term 'temporary
symbols'.  AS knows three different types of temporary symbols, in the
hope to offer everyone 'switching' to AS a solution that makes conversion
as easy as possible.  However, practically every assembler has its own
interpretation of this feature, so there will be only few cases where a
1:1 solution for existing code:

\section{Named Temporary Symbols}

A symbol whose name starts with two dollar signs (something that is
neither allowed for non-temporary symbols nor for constants) is a named
temporary symbol.  AS keeps an internal counter which is reset to 0 before
assembly begins and which gets incremented upon every definition of a
non-temporary symbol.  When a temporary symbol is defined or referenced,
both leading dollar signs are discarded and the counter's current value is
appended.  This way, one regains the used symbol names with every
definition of a non-temporary symbol - but you also cannot reach the
previously symbols any more! Temporary symbols are therefore especially
suited for usage in small instruction blocks, typically a dozen of machine
instructions, definitely not more than one screen.  Otherwise, one easily
gets confused...

Here is a small example:
\begin{verbatim}
$$loop: nop
        dbra    d0,$$loop

split:

$$loop: nop
        dbra    d0,$$loop
\end{verbatim}
Without the non-temporary label between the loops, of course an error
message about a double-defined symbol would be the result.

\subsection{Nameless Temporary Symbols}

For all those who regard named temporary symbols still as too complicated,
there is an even simpler variant: If one places a single puls or minus
sign as a label, this is converted to symbol names of {\tt \_\_forwnn}
respectively {\tt \_\_backmm}, with {\tt nn} respectively {\tt mm} being
counters that start counting at zero.  Those symbols are referenced via
the special names {\tt - -- ---} respectively {\tt + ++ +++}, which refer
to the three last 'minus symbols' and the next three 'plus symbols'.
Therefore, the selection between these two variants depends on whether one
wants to forward- or backward-reference a symbol.

Apart from plus and minus, {\em defining} nameless temporary symbols also
exists in a third variant, namely a slash (/).  A temporary symbol defined
in this way may be referenced both backward and forward, i.e. it is
treated either as a plus or a minus, depending on the way it is being
referenced.

Nameless temporary symbols are usually used in constructs that fit on one
screen page, like skipping a few machine instructions or tight loops -
things would becone to puzzling otherwise (this only a good advice,
however...).  An example for this is the following piece of code, this
time as 65xx code:
\begin{verbatim}
        cpu     6502

-       ldx     #00
-       dex
        bne     -           ; branch to 'dex'
        lda     RealSymbol
        beq     +           ; branch to 'bne --'
        jsr     SomeRtn
        iny
+       bne     --          ; branch to 'ldx #00'

SomeRtn:
        rts

RealSymbol:
        dfs     1

  	inc	ptr
   	bne 	+      	    ; branch to 'tax'
   	inc 	ptr+1
+ 	tax

 	bpl 	++     	    ; branch to 'dex'
   	beq 	+      	    ; branch forward to 'rts'
   	lda 	#0
/  	rts            	    ; slash used as wildcard.
+ 	dex
   	beq 	-           ; branch backward to 'rts'

ptr:	dfs	2
\end{verbatim}

\subsection{Composed Temporary Symbols}

This is maybe the type of temporary symbols that is nearest to the concept
of local symbols and sections.  Whenever a symbol's name begins with a dot
(.), the symbol is not directly stored with this name in the symbol table.
Instead, the name of the most recently-defined symbol not beginning with a
dot is prepended to the symbols name.  This way, 'non-dotted' symbols take
the role of section separators and 'dotted' symbol names may be reused
after a 'non-dotted' symbol has been defined.  Take a look at the
following little example:
\begin{verbatim}
proc1:				; non-temporary symbol 'proc1'

.loop	moveq	#20,d0		; actually defines 'proc1.loop'
	dbra	d0,.loop
	rts

proc2:				; non-temporary symbol 'proc2'

.loop	moveq	#10,d1		; actually defines 'proc2.loop'
	jsr	proc1
	dbra	d1,.loop
	rts
\end{verbatim}
Note that it is still possible to access all temporary symbols, even
without being in the same 'area', by simply using the composed name (like
'proc2.loop' in the previous example).

It is principally possible to combine composed temporary symbols with
sections, which makes them also to local symbols.  Take however into
account that the most recent non-temporary symbol is not stored
per-section, but simply globally.  This may change however in a future
version, so one shouldn't rely on the current behaviour.

%%---------------------------------------------------------------------------

\section{Formula Expressions}

In most places where the assembler expects numeric inputs, it is
possible to specify not only simple symbols or constants, but also
complete formula expressions.  The components of these formula
expressions can be either single symbols and constants.  Constants may be
either integer, floating point, or string constants.

\subsection{Integer Constants}
\label{SectIntConsts}

Integer constants describe non-fractional numbers.  They are witten as a
sequence of digits.  This may be done in different numbering systems (see
table \ref{TabSystems}).
\par
\begin{table*}[htb]
\begin{center}\begin{tabular}{|l|c|c|c|c|}
\hline
          & Intel Mode    & Motorola Mode & C Mode    & IBM Mode \\
\hline
\hline
Decimal   & Direct        & Direct      & Direct      & Direct \\
Hex       & Suffix H      & Prefix \$   & Prefix 0x   & X'..' or H'..' \\
\ii{Ident}& \tty{hexh}    & \tty{\$hex} & \tty{0xhex} & \tty{x'hex'} \\
          &               &             &             & \tty{h'hex'} \\
Binary    & Suffix B      & Prefix \%   & Prefix 0b   & O'..' \\
\ii{Ident}& \tty{binb}    & \tty{\%bin} & \tty{0bbin} & \tty{b'bin'} \\
Octal     & Suffix O or Q & Prefix @    & Prefix 0    & B'..' \\
\ii{Ident}& \tty{octo}    & \tty{@oct}  & \tty{0oct}  & \tty{o'oct'} \\  
          & \tty{octq}    &             &             & \\
\hline
\end{tabular}\end{center}
\caption{Defined Numbering Systems and Notations\label{TabSystems}}
\end{table*}
In case the numbering system has not been explicitly stated by adding the
special control characters listed in the table, AS assumes the base given
with the {\tt RADIX} statement (which has itself 10 as default).  This
statement allows to set up 'unusual' numbering systems, i.e. others than
2, 8, 10, or 16.

Valid digits are numbers from 0 to 9 and letters from A to Z (value 10 to
35) up to the numbering system's base minus one. The usage of letters in
integer constants however brings along some ambiguities since symbol names
also are sequences of numbers and letters: a symbol name must not start
with a character from 0 to 9.  This means that an integer constant which
is not clearly marked a such with a special prefix character never may
begin with a letter.  One has to add an additional, otherwise superfluous
zero in front in such cases.  The most prominent case is the writing of
hexadecimal constants in Intel mode:  If the leftmost digit is between A
and F, the trailing H doesn't help anything, an additional 0 has to be
prefixed (e.g. 0F0H instead of F0H).  The Motorola and C syntaxes whis
both mark the numbering system at the front of a constant do not have this
problem.

Quite tricky is furthermore that the higher the default numbering system
set via {\tt RADIX} becomes, the more letters used to denote numbering
systems in Intel and C syntax become 'eaten'.  For example, you cannot
write binary constants anymore after a {\tt RADIX 16}, and starting at
{\tt RADIX 18}, the Intel syntax even doesn't allow to write hexadecimal
constants any more.  Therefore {\bf CAUTION!}

Appendix \ref{SectPseudoInst} lists which syntax is used by which target
by default.  Independent of this default, there is always the option to
add or delete individual syntax variants via the \tty{INTSYNTAX} instruction
(see section \ref{SectINTSYNTAX}).  The names listed as \ii{Ident},
prefixed with a plus or minus sign, serve as arguments to this instruction.

The \tty{RELAXED} instruction (see section \ref{SectRELAXED}) serves as
a sort 'global enable switch': in relaxed mode, all notations may be used,
independent of the selected target processor.  The result is that an arbitrary
syntax may be used (possibly loosing compatibility to standard assemblers).

Both \tty{INTSYNTAX} and \tty{RELAXED} specifically enable usage of the
'IBM syntax' for all targets, which is sometimes found on other assemblers:

This notation puts the actual value into apostrophes and prepends
the numbering system ('x' or 'h' for hexadecimal, 'o' for octal and 'b'
for binary).  So, the integer constant 305419896 can be written in the
following ways:
\begin{verbatim}
 x'12345678'
 h'12345678'
 o'2215053170'
 b'00010010001101000101011001111000'
\end{verbatim}
Another variant of this notation for some targets is to leave away the
closing apostrophe, to allow simpler porting of existing code.  It is
not recommended for new programs.

\subsection{Floating Point Constants}

Floating point constants are to be written in the usual scientific
notation, which is known in the most general form:
\begin{verbatim}
 [-]<integer digits>[.post decimal positions][E[-]exponent]
\end{verbatim}
\bb{CAUTION!} The assembler first tries to interprete a constant as an
integer constant and makes a floating-point format try only in case
the first one failed.  If someone wants to enforce the evaluation as
a floating point number, this can be done by dummy post decimal
positions, e.g.  \tty{2.0} instead of \tty{2}.

\subsection{String Constants}
\label{SectStringConsts}

String constants have to be enclosed in single or double quotation marks.
In order to make it possible to include quotation marks or special characters
in string constants, an ''escape mechanism'' has been implemented, which
should sound familiar for C programmers:

The assembler understands a backslash (\verb!\!) with a following decimal
number of three digits maximum in the string as a character with the
according decimal ASCII value.  The numerical value may alternitavely be
written in hexadecimal or octal notation if it is prefixed with an x resp.
a 0.  In case of hexadecimal notation, the maximum number of digits is
limited to 2.  For example, it is possible to include an ETC character by
writing {\tt\verb!\!3}.  But be careful with the definition of NUL
characters!  The C \marginpar{{\em UNIX}} version currently uses C strings
to store strings internally.  As C strings use a NUL character for
termination, the usage of NUL characters in strings is currently not
portable!

Some frequently used control characters can also be reached with the
following abbreviations:
\begin{verbatim}
\b : Backspace           \a : Bell         \e : Escape
\t : Tabulator           \n : Linefeed     \r : Carriage Return
\\ : Backslash           \' or \H : Apostrophe
\" or \I : Quotation marks
\end{verbatim}
Both upper and lower case characters may be used for the
identification letters.

By means of this escape character, you can even work formula
expressions into a string, if they are enclosed by curly braces: e.g.
\begin{verbatim}
     message "root of 81 : \{sqrt(81)}"
\end{verbatim}
results in
\begin{verbatim}
              root of 81 : 9
\end{verbatim}
AS chooses with the help of the formula result type the correct
output format, further string constants, however, are to be avoided
in the expression.  Otherwise the assembler will get mixed up at the
transformation of capitals into lower case letters.  Integer results will
by default be written in hexadecimal notation, which may be changed via
the \tty{OUTRADIX} instruction.

Except for the insertion of formula expressions, you can use this
''escape-mechanism'' as well in ASCII defined integer constants,
like this:
\begin{verbatim}
     move.b   #'\n',d0
\end{verbatim}
However, everything has its limits, because the parser with higher
priority, which disassembles a line into op-code and parameters, does
not know what it is actually working with, e.g. here:
\begin{verbatim}
     move.l   #'\'abc',d0
\end{verbatim}
After the third apostrophe, it will not find the comma any more,
because it presumes that it is the start of a further character
constant. An error message about a wrong parameter number is the result.
A workaround would be to write e.g., \verb!\i! instead of \verb!\'!.

\subsection{String to Integer Conversion and Character Constants}

Earlier versions of AS strictly distinguished between character
strings and so-called ''character constants'': At first glance, a
character constant looks like a string, the characters are however
enclosed in single instead of double quotation marks.  Such an object had
the data type 'Integer', i.e. it represented a number with the value
given by the (ASCII) code of the character, and it was something
completely different:

\begin{verbatim}
   move.b   #65,d0
   move.b   #'A',d0      ; equal to first instruction
   move.b   #"A",d0      ; not allowed in older versions of AS!
\end{verbatim}

This strict differentiation {\em no longer exists}, so it is irrelevant
whether single or double quotes are used.  If an integer value
is expected as argument, and a string is used, the conversion via the
character's (ASCII) value is done ''on the fly'' at this place. This
means that in the example given, {\em all three lines} result in the
same machine code.

Such an implicit conversion to integer values also take place for strings
consisting of multiple constancs, which are sometimes called ''multi
character constants'':
\begin{verbatim}
'A'    ==$41
'AB'   ==$4142
'ABCD' ==$41424344
\end{verbatim}
Multi character constants are the only case where using single or double
quotes still makes a difference.  Many targets define pseudo instructions
to dispose constants in memory, and which accept different data types.
In such a case, it is still necessary to use double quotes if a character
string shall be placed in memory:
\begin{verbatim}
    dc.w    "ab"  ; disposes two words (0x0041,0x0042)
    dc.w    'ab'  ; disposes one word (0x4142)
\end{verbatim}
Important: using the correct quotation is not necessary if the character
string is longer than the used operand size, which is two characters or
16 bits in this example.

\subsection{Evaluation}

The calculation of intermediary results within formula expressions is
always done with the highest available resolution, i.e. 32 or 64 bits for
integer numbers, 80 bit for floating point numbers and 255 characters
for strings.  An possible test of value range overflows is done only
on the final result.

The portable C version \marginpar{{\em UNIX}} only supports floating
point values up to 64 bits (resulting in a maximum value of roughly
$10^{308}$), but in turn features integer lengths of 64 bits on some
platforms.

\subsection{Operators}

The assembler provides the operands listed in table \ref{TabOps} for
combination.
\begin{table*}[htbp]
\begin{center}\begin{tabular}{|c|l|c|c|c|c|c|}
\hline
operand & function            & \#operands & integer & float & string & rank \\
\hline
\hline
$<>$        & inequality       & 2 & yes   & yes & yes & 14 \\
$>=$        & greater or equal & 2 & yes   & yes & yes & 14 \\
$<=$        & less or equal    & 2 & yes   & yes & yes & 14 \\
$<$         & truly smaller    & 2 & yes   & yes & yes & 14 \\
$>$         & truly greater    & 2 & yes   & yes & yes & 14 \\
$=$         & equality         & 2 & yes   & yes & yes & 14 \\
$==$        & alias for $=$    &   &       &     &     &    \\
            & & & & & &  \\
$!!$        & log. XOR         & 2 & yes   & no  & no  & 13 \\
$||$        & log. OR          & 2 & yes   & no  & no  & 12 \\
\&\&        & log. AND         & 2 & yes   & no  & no  & 11 \\
\verb! ~~ ! & log. NOT         & 1 & yes   & no  & no  & 2 \\
            & & & & & &  \\
-           & difference       & 2 & yes   & yes & no  & 10 \\
+           & sum              & 2 & yes   & yes & yes & 10 \\
\#          & modulo division  & 2 & yes   & no  & no  & 9 \\
/           & quotient         & 2 & yes*) & yes & no  & 9 \\
\verb! * !  & product          & 2 & yes   & yes & no  & 9 \\
\verb! ^ !  & power            & 2 & yes   & yes & no  & 8 \\
            & & & & & &  \\
$!$         & binary XOR       & 2 & yes   & no  & no  & 7 \\
$|$         & binary OR        & 2 & yes   & no  & no  & 6 \\
\&          & binary AND       & 2 & yes   & no  & no  & 5 \\
$><$        & mirror of bits   & 2 & yes   & no  & no  & 4 \\
$>>$        & log. shift right & 2 & yes   & no  & no  & 3 \\
$<<$        & log. shift left  & 2 & yes   & no  & no  & 3 \\
\verb! ~ !  & binary NOT       & 1 & yes   & no  & no  & 1 \\
\hline
\multicolumn{7}{|l|}{*) remainder will be discarded} \\
\hline
\end{tabular}\end{center}
\caption{Operators Predefined by AS\label{TabOps}}
\end{table*}
''Rank'' is the priority of an operator at the separation of expressions
into subexpressions.  The operator with the highest rank will be
evaluated at the very end.  The order of evaluation can be defined by
new bracketing.

The compare operators deliver TRUE in case the condition fits,
and FALSE in case it doesn't.  For the logical operators an expression
is TRUE in case it is not 0, otherwise it is FALSE.

The mirroring of bits probably needs a little bit of explanation: the
operator mirrors the lowest bits in the first operand and leaves the
higher priority bits unchanged.  The number of bits which is to be
mirrored is given by the right operand and may be between 1 and 32 .

A small pitfall is hidden in the binary complement: As the
computation is always done with 32 resp. 64 bits, its application on
e.g. 8-bit masks usually results in values taht do not fit into 8-bit
numbers any more due to the leading ones.  A binary AND with a
fitting mask is therefore unavoidable!

\subsection{Functions}

In addition to the operators, the assembler defines another line of
primarily transcendental functions with floating point arguments which are
listed in tables \ref{TabFuncs1} and \ref{TabFuncs2}.
\begin{table*}[htbp]
\begin{center}\begin{tabular}{|l|l|l|l|}
\hline
name     & meaning              & argument             & result \\
\hline
\hline
SQRT     & square root          & $arg \geq 0$         & floating point \\
         &                      &                      & \\
SIN      & sine                 & $arg \in \rz$        & floating point \\
COS      & cosine               & $arg \in \rz$        & floating point \\
TAN      & tangent              & $arg \neq (2n+1)*\frac{\pi}{2}$ & floating point \\
COT      & cotangent            & $arg \neq n*\pi$     & floating point \\
         &                      &                      & \\
ASIN     & inverse sine         & $\mid arg \mid \leq 1$ & floating point \\
ACOS     & inverse cosine       & $\mid arg \mid \leq 1$ & floating point \\
ATAN     & inverse tangent      & $arg \in \rz$        & floating point \\
ACOT     & inverse cotangent    & $arg \in \rz$        & floating point \\
         &                      &                      & \\
EXP      & exponential function & $arg \in \rz$        & floating point \\
ALOG     & 10 power of argument & $arg \in \rz$        & floating point \\
ALD      & 2 power of argument  & $arg \in \rz$        & floating point \\
SINH     & hyp. sine            & $arg \in \rz$        & floating point \\
COSH     & hyp. cosine          & $arg \in \rz$        & floating point \\
TANH     & hyp. tangent         & $arg \in \rz$        & floating point \\
COTH     & hyp. cotangent       & $arg \neq 0$         & floating point \\
         &                      &                      & \\
LN       & nat. logarithm       & $arg > 0$            & floating point \\
LOG      & dec. logarithm       & $arg > 0$            & floating point \\
LD       & bin. logarithm       & $arg > 0$            & floating point \\
ASINH    & inv. hyp. Sine       & $arg \in \rz$        & floating point \\
ACOSH    & inv. hyp. Cosine     & $arg \geq 1$         & floating point \\
ATANH    & inv. hyp. Tangent    & $arg < 1$            & floating point \\
ACOTH    & inv. hyp. Cotangent  & $arg > 1$            & floating point \\
         &                      &                      & \\
INT      & integer part         & $arg \in \rz$        & floating point \\
\hline
BITCNT   & number of one's      & integer              & integer \\
FIRSTBIT & lowest 1-bit         & integer              & integer \\
\hline
\end{tabular}\end{center}
\caption{Functions Predefined by AS - Part 1 (Integer and
         Floating Point Functions \label{TabFuncs1}}
\end{table*}
\begin{table*}[htbp]
\begin{center}\begin{tabular}{|l|l|l|l|}
\hline
name        & meaning              & argument & result \\
\hline
\hline
LASTBIT     & highest 1-bit        & integer              & integer \\
BITPOS      & unique 1-bit         & integer              & integer \\
            &                      &                      & \\
SGN         & sign (0/1/-1)        & floating point       & integer \\
            &                      & or integer           & \\
ABS         & absolute value       & integer or           & integer or \\
            &                      & floating point       & floating point \\
TOUPPER     & matching capital     & integer              & integer \\
TOLOWER     & matching lower case  & integer              & integer \\
            &                      &                      & \\
UPSTRING    & changes all          & string               & string \\
            & characters           &                      & \\
            & into capitals        &                      & \\
            &                      &                      & \\
LOWSTRING   & changes all          & string               & string \\
            & characters           &                      & \\
            & into to lower case   &                      & \\
            &                      &                      & \\
STRLEN      & returns the length   & string               & integer \\
            & of a string          &                      & \\
            &                      &                      & \\
SUBSTR      & extracts parts of a  & string,              & string \\
            & string               & integer,             & \\
            &                      & integer              & \\
CHARFROMSTR & extracts a character & string,              & integer \\
            & from a string        & integer              & \\
STRSTR      & searches a substring & string,              & integer \\
            & in a string          & string               & \\
VAL         & evaluates contents   & string               & depends on \\
            & as expression        &                      & argument \\
EXPRTYPE    & delivers type of     & integer,             & 0 \\
            & argument             & float,               & 1 \\
            &                      & string               & 2 \\
\hline
\end{tabular}\end{center}
\caption{Functions Predefined by AS - Part 2 (Integer and
         String Functions \label{TabFuncs2}}
\end{table*}
The functions \tty{FIRSTBIT}, \tty{LASTBIT}, and \tty{BITPOS} return -1 as
result if no resp. not exactly one bit is set.  \tty{BITPOS} additionally
issues an error message in such a case.

The string function \tty{SUBSTR} expects the source string as first
parameter, the start position as second and the number of characters to be
extracted as third parameter (a 0 means to extract all characters up to
the end).  Similarly, \tty{CHARFROMSTR} expects the source string as
first argument and the character position as second argument.  In case the
position argument is larger or equal to the source string's length,
\tty{SUBSTR} returns an empty string while \tty{CHARFROMSTR} returns -1.
A position argument smaller than zero is treated as zero by \tty{SUBSTR},
while \tty{CHARFROMSTR} will return -1 also in this case.

Here is an example how to use these both functions.  The task is to put a
string into memory, with the string end being signified by a set MSB in
the last character:

\begin{verbatim}
dbstr   macro   arg
        if      strlen(arg) > 1
         db     substr(arg, 0, strlen(arg) - 1)
        endif
        if      strlen(arg) > 0
         db     charfromstr(arg, strlen(arg) - 1) | 80h
        endif
        endm
\end{verbatim}

\tty{STRSTR} returns the first occurence of the second string
within the first one resp. -1 if the search pattern was not found.
Similarly to \tty{SUBSTR} and  \tty{CHARFROMSTR}, the first character
has the position 0.

If a function expects floating point arguments, this does not mean it
is impossible to write e.g.
\begin{verbatim}
    sqr2 equ sqrt(2)
\end{verbatim}
In such cases an automatic type conversion is engaged. In the reverse
case the \tty{INT}-function has to be applied to convert a floating point
number to an integer.  When using this function, you have to pay
attention that the result produced always is a signed integer and
therefore has a value range of approximately +/-2.0E9.

When AS is switched to case-sensitive mode, predefined functions may be
accessed with an arbitrary combination of upper and lower case (in
contrast to predefined symbols).  However, in the case of user-defined
functions (see section \ref{SectFUNCTION}), a distinction between upper
and lower case is made.  This has e.g. the result that if one defines a
function \tty{Sin}, one can afterwards access this function via \tty{Sin}, but all
other combinations of upper and lower case will lead to the predefined
function.

For a correct conversion \marginpar{{\em DOS/DPMI}} of lower case letters
into capital letters a DOS version $\geq$ 3.30 is required.

%%---------------------------------------------------------------------------

\section{Forward References and Other Disasters}
\label{ForwRefs}

This section is the result of a significant amount of hate on the
(legal) way some people program.  This way can lead to trouble in
conjunction with AS in some cases.  The section will deal with
so-called 'forward references'.  What makes a forward reference
different from a usual reference?  To understand the difference, take
a look at the following programming example (please excuse my bias
for the 68000 family that is also present in the rest of this
manual):
\begin{verbatim}
        move.l  #10,d0
loop:   move.l  (a1),d1
        beq     skip
        neg.l   d1
skip:   move.l  d1,(a1+)
        dbra    d0,loop
\end{verbatim}
If one overlooks the loop body with its branch statement, a program
remains that is extremely simple to assemble: the only reference is
the branch back to the body's beginning, and as an assembler
processes a program from the beginning to the end, the symbol's value
is already known before it is needed the first time.  If one has a
program that only contains such backward references, one has the nice
situation that only one pass through the source code is needed to
generate a correct and optimal machine code.  Some high level
languages like Pascal with their strict rule that everything has to
be defined before it is used exploit exactly this property to speed
up the compilation.

Unfortunately, things are not that simple in the case of assembler,
because one sometimes has to jump forward in the code or there are
reasons why one has to move variable definitions behind the code.
For our example, this is the case for the conditional branch that is
used to skip over another instruction.  When the assembler hits the
branch instruction in the first pass, it is confronted with the
situation of either leaving blank all instruction fields related to
the target address or offering a value that ''hurts noone'' via the
formula parser (which has to evaluate the address argument).  In case
of a ''simple'' assembler that supports only one target architecture
with a relatively small number of instructions to treat, one will
surely prefer the first solution, but the effort for AS with its
dozens of target architectures would have become extremely high.
Only the second way was possible: If an unknown symbol is detected in
the first pass, the formula parser delivers the program counter's
current value as result!  This is the only value suitable to offer an
address to a branch instruction with unknown distance length that
will not lead to errors.  This answers also a frequently asked
question why a first-pass listing (it will not be erased e.g. when AS
does not start a second pass due to additional errors) partially
shows wrong addresses in the generated binary code - they are the
result of unresolved forward references.

The example listed above however uncovers an additional difficulty of
forward references: Depending on the distance of branch instruction
and target in the source code, the branch may be either long or
short.  The decision however about the code length - and therefore
about the addresses of following labels - cannot be made in the first
pass due to missing knowledge about the target address.  In case the
programmer did not explicitly mark whether a long or short branch
shall be used, genuine 2-pass assemblers like older versions of MASM
from Microsoft ''solve'' the problem by reserving space for the longest
version in the first pass (all label addresses have to be fixed after
the first pass) and filling the remaining space with \tty{NOP}s in the
second pass.  AS versions up to 1.37 did the same before I switched
to the multipass principle that removes the strict separation into
two passes and allows an arbitrary number of passes.  Said in detail,
the optimal code for the assumed values is generated in the first
pass.  In case AS detects that values of symbols changed in the second
pass due to changes in code lengths, simply a third pass is done, and
as the second pass'es new symbol values might again shorten or
lengthen the code, a further pass is not impossible.  I have seen
8086 programs that needed 12 passes to get everything correct and
optimal.  Unfortunately, this mechanism does not allow to specify a
maximum number passes; I can only advise that the number of passes
goes down when one makes more use of explicit length specifications.

Especially for large programs, another situation might arise: the
position of a forward directed branch has moved so much in the second
pass relative to the first pass that the old label value still valid
is out of the allowed branch distance.  AS knows of such situations
and suppresses all error messages about too long branches when it is
clear that another pass is needed.  This works for 99\% of all cases,
but there are also constructs where the first critical instruction
appears so early that AS had no chance up to now to recognize that
another pass is needed.  The following example constructs such a
situation with the help of a forward reference (and was the reason
for this section's heading...):
\begin{verbatim}
        cpu   6811

        org     $8000
        beq     skip
        rept    60
         ldd    Var
        endm
skip:   nop

Var     equ     $10
\end{verbatim}
Due to the address position, AS assumes long addresses in the first
pass for the \tty{LDD} instructions, what results in a code length of 180
bytes and an out of branch error message in the second pass (at the
point of the \tty{BEQ} instruction, the old value of \tty{skip} is still valid,
i.e. AS does not know at this point that the code is only 120 bytes
long in reality) is the result.  The error can be avoided in three
different ways:
\begin{enumerate}
\item{Explicitly tell AS to use short addressing for the \tty{LDD}
      instructions (\tty{ldd <Var})}
\item{Remove this damned, rotten forward reference and place the \tty{EQU}
      statement at the beginning where it has to be (all right, I'm
      already calming down...)}
\item{For real die-hards: use the \tty{-Y} command line option.  This
      option tells AS to forget the error message when the address
      change has been detected.  Not pretty, but...}
\end{enumerate}
Another tip regarding the \tty{EQU} instruction: AS cannot know in which
context a symbol defined with \tty{EQU} will be used, so an \tty{EQU} containing
forward references will not be done at all in the first pass.  Thus,
if the symbol defined with \tty{EQU} gets forward-referenced in the second
pass:
\begin{verbatim}
        move.l  #sym2,d0
sym2    equ     sym1+5
sym1    equ     0
\end{verbatim}
one gets an error message due to an undefined symbol in the second
pass...but why on earth do people do such things?

Admittedly, this was quite a lengthy excursion, but I thought it was
necessary.  Which is the essence you should learn from this section?
\begin{enumerate}
\item{AS always tries to generate the shortest code possible.  A
      finite number of passes is needed for this.  If you do not tweak
      AS extremely, AS will know no mercy...}
\item{Whenever sensible and possible, explicitly specify branch and
      address lengths.  There is a chance of significantly reducing the
      number of passes by this.}
\item{Limit forward references to what is absolutely needed.  You make
      your and AS's live much easier this way!}
\end{enumerate}

%%---------------------------------------------------------------------------

\section{Register Symbols}
\label{SectRegSyms} \ttindex{register symbols}

{\em valid for: PowerPC, M-Core, XGate, 4004/4040, MCS-48/(2)51, 80C16x,
     AVR, XS1, Z8, KCPSM, Mico8, MSP430(X), ST9, M16, M16C, H8/300,
     H8/500, SH7x00, H16, i960, XA, 29K, TLCS-9000, KENBAK}

Sometimes it is desirable not only to assign symbolic names to memory
addresses or constants, but also to a register, to emphasize its function
in a certain program section.  This is no problem for processors that
treat registers simply as another address space, as this allows to use
numeric expressions and one can use simple \tty{EQU}s to define such
symbols.  (e.g. for the MCS-96 or TMS70000).  However, for most
processors, register identifiers are fixed literals which are seperately
treated by AS for speed reasons.  Therefore, registers symbols (sometime
also called 'register aliases') are also a separate type of symbols in
the symbol table.  Just like other symbols, they may be defined or re-defined
with \tty{EQU} or \tty{SET}, and there is a specialized \tty{REG} instruction
which accepts only symbols and expressions of this type.

On the other hand, register symbols are subject of a couple of restrictions: the
number of literals is limited and depends on the selected target processor, and
arithmetic operations are not possibl eon registers A construct like tihs:
\begin{verbatim}
myreg   reg     r17         ; definition of register symbol
        addi    myreg+1,3   ; does not work!
\end{verbatim}
is {\em not} valid.  Simple assignments are however possible:
\begin{verbatim}
myreg   reg     r17         ; definition of register symbol
myreg2  reg     myreg       ; myreg2 -> r17
\end{verbatim}
Furthermore, forward references are even more critical than for other
types of symbols.  If a symbol is not (yet) defined, AS does not know
which type it is going to have,a nd will decide for a plain integer number.
For most target processors, a number is the equivalent of absolute
memory addressing, and on most processors, usage of memory operands
is more limited than of registers.  Depending on situation, one will
get an error message about a non-allowed addressing mode, and no second
pass will be started...

Analogous to ordinary symbols, register symbols are local to sections and
it is possible to access a register symbol from a specific section by
appending the section's name enclosed in brackets.

%%---------------------------------------------------------------------------

\section{Share File}
\label{ChapShareMain}
\ttindex{SHARED}

This function is a by-product from the old pure-68000 predecessors of
AS, I have kept them in case someone really needs it.  The basic
problem is to access certain symbols produced during assembly,
because possibly someone would like to access the memory of the
target system via this address information.  The assembler allows to
export symbol values by means of \tty{SHARED} pseudo commands (see there).
For this purpose, the assembler produces a text file with the required
symbols and its values in the second pass.  This file may be included
into a higher-level language or another assembler program.  The
format of the text file (C, Pascal or Assembler) can be set by the
command line switches \tty{p, c} or, \tty{a}.

\bb{CAUTION!} If none of the switches is given, no file will be
generated and it makes no difference if \tty{SHARED}-commands are in the
source text or not!

When creating a Sharefile, AS does not check if a file with the
same name already exists, such a file  will be simply overwritten.
In my opinion a request does not make sense, because AS would
ask at each run if it should overwrite the old version of the
Sharefile, and that would be really annoying...

%%---------------------------------------------------------------------------

\section{Processor Aliases}
\label{SectAlias}

Common microcontroller families are like rabbits: They become more at
a higher speed than you can provide support for them.  Especially the
development of processor cores as building blocks for ASICs and of
microcontroller families with user-definable peripherals has led to a
steeply rising number of controllers that only deviate from a
well-known type by a slightly modified peripheral set.  But the
distinction among them is still important, e.g. for the design of
include files that only define the appropriate subset of peripherals.
I have struggled up to now to integrate the most important
reperesentatives of a processor family into AS (and I will continue
to do this), but sometimes I just cannot keep pace with the
development...there was an urgent need for a mechanism to extend the
list of processors by the user.

The result are processor aliases: the alias command line option allows to
define a new processor type, whose instruction set is equal to another
processor built into AS.  After switching to this processor via the
\tty{CPU} instruction, AS behaves exactly as if the original processor had
been used, with a single difference: the variables \tty{MOMCPU} resp.
\tty{MOMCPUNAME} are set to the alias name, which allows to use the new
name for differentiation, e.g. in include files.

There were two reasons to realize the definition of aliases by the
command line and not by pseudo instructions: first, it would anyway be
difficult to put the alias definitions together with register definitions
into a single include file, because a program that wants to use such a
file would have to include it before and after the CPU instruction - an
imagination that lies somewhere between inelegant and impossible.  Second,
the definition in the command line allows to put the definitions in a key
file that is executed automatically at startup via the \tty{ASCMD}
variable, without a need for the program to take any further care about
this.

%%===========================================================================

\cleardoublepage
\chapter{Pseudo Instructions}

Not all pseudo instructions are defined for all processors.  A note
that shows the range of validity is therefore prepended to every
individual description.

%%---------------------------------------------------------------------------

\section{Definitions}

%%- - - - - - - - - - - - - - - - - - - - - - - - - - - - - - - - - - - - - -

\subsection{SET, EQU, and CONSTANT}
\ttindex{SET}\ttindex{EQU}\ttindex{CONSTANT}

{\em valid for: all processors, {\tt CONSTANT} only for KCPSM(3)}

\tty{SET} and \tty{EQU} allow the definition of typeless constants, i.e.  they
will not be assigned to a segment and their usage will not generate warnings
because of segment mixing.  \tty{EQU} defines constants which can not be
modified (by \tty{EQU}) again, but \tty{SET} permits the definition of
variables, which can be modified during the assembly.  This is useful e.g.
for the allocation of resources like interrupt vectors, as shown in the
following example:
\begin{verbatim}
VecCnt  set     0       ; somewhere at the beginning
        .
        .
        .
DefVec  macro   Name    ; allocate a new vector
Name    equ     VecCnt
VecCnt  set     VecCnt+4
        endm
        .
        .
        .
        DefVec  Vec1    ; results in Vec1=0
        DefVec  Vec2    ; results in Vec2=4
\end{verbatim}
constants and variables are internally stored in the same way, the only
difference is that they are marked as unchangeable if defined via \tty{EQU}.
Trying to change a constant with \tty{SET} will result in an error
message.

\tty{EQU/SET} allow to define constants of all possible types, e.g.
\begin{verbatim}
IntTwo   equ    2
FloatTwo equ    2.0
\end{verbatim}
Some processors unfortunately have already a \tty{SET} instruction. For
these targets, \tty{EVAL} must be used instead of \tty{SET} if no
differentiation via the argument count is possible.

A simple equation sign may be used instead of \tty{EQU}.  Similarly, one
may simply write \tty{:=} instead of \tty{SET} resp. \tty{EVAL}.
Furthermore, there is an 'alternate' syntax that does not take
the symbol's name from the label field, but instead from the
first argument.  So for instance, it is valid to write:
\begin{verbatim}
          EQU   IntTwo,2
          EQU   FloatTwo,2.0
\end{verbatim}

For compatibility reasons to the original assembler, the KCPSM target also
knows the {\tt CONSTANT} statement, which - in contrast to \tty{EQU} -
takes both name and value as argument.  For example:
\begin{verbatim}
      CONSTANT  const1, 2
\end{verbatim}
{\tt CONSTANT} is however limited to integer constants.

Symbols defined with \tty{SET} or \tty{EQU} are typeless by default, but
optionally a segment name (\tty{CODE, DATA, IDATA, XDATA, YDATA, BITDATA,
IO}, or \tty{REG}) or \tty{MOMSEGMENT} for the currently active segment
may be given as a second or thirsd parameter, allowing to assign the symbol to a
specific address space.  AS does not check at this point if the used
address space exists on the currently active target processor!

A little hidden extra feature allows to set the program counter via
\tty{SET} or \tty{EQU}, something one would ordinarily do via \tty{ORG}.
To accomplish this, use the special value as symbol name that may also
be used to query the current program counter's value.  Depending on the
selected target architecture, this is either an asterisk, a dollar sign
or \tty{PC}.

In case the target architecture supports instruction attributes to define
the operand size (e.g. on 680x0), those are also allowed for \tty{SET} and
\tty{EQU}.  The operand size will be stored along with the symbol's value in
the symbol table.  Its use is architecture-dependant.

%%- - - - - - - - - - - - - - - - - - - - - - - - - - - - - - - - - - - - - -

\subsection{SFR and SFRB}
\ttindex{SFR}\ttindex{SFRB}

{\em valid for: various, \tty{SFRB} only MCS-51}

These instructions act like \tty{EQU}, but symbols defined with them are
assigned to the directly addressable data resp. I/O segment, i.e. they are
preferrably used for the definition of (as the name lets guess) hardware
registers mapped into the data res. I/O area.  The allowed range of values
is equal to the range allowed for \tty{ORG} in the data segment (see
section \ref{SectORG}).  The difference between \tty{SFR} and \tty{SFRB}
is that \tty{SFRB} marks the register as bit addressable, which is why AS
generates 8 additional symbols which will be assigned to the bit segment
and carry the names xx.0 to xx.7, e.g.
\begin{verbatim}
PSW     sfr     0d0h    ; results in PSW = D0H (data segment)

PSW     sfrb    0d0h    ; results in extra PSW.0 = D0H (bit)
                        ;               to PSW.7 = D7H (bit)
\end{verbatim}
The \tty{SFRB} instruction is not any more defined for the 80C251 as it
allows direct bit access to all SFRs without special bit symbols; bits
like \tty{PSW.0} to \tty{PSW.7} are automatically present.

Whenever a bit-addressable register is defined via \tty{SFRB}, AS checks
if the memory address is bit addressable (range 20h..3fh resp. 80h, 88h,
90h, 98h...0f8h).  If it is not bit-addressable, a warning is issued and
the generated bit symbols are undefined.

%%- - - - - - - - - - - - - - - - - - - - - - - - - - - - - - - - - - - - - -

\subsection{XSFR and YSFR}
\ttindex{XSFR}\ttindex{YSFR}

{\em valid for: DSP56xxx}

Also the DSP56000 has a few peripheral registers memory-mapped to the RAM,
but the affair becomes complicated because there are two data areas, the
X- and Y-area.  This architecture allows on the one hand a higher
parallelism, but forces on the other hand to divide the normal \tty{SFR}
instruction into the two above mentioned variations.  They works
identically to \tty{SFR}, just that \tty{XSFR} defines a symbol in the X-
addressing space and YSFR a corresponding one in the Y-addressing space.
The allowed value range is 0..\$ffff.

%%- - - - - - - - - - - - - - - - - - - - - - - - - - - - - - - - - - - - - -

\subsection{LABEL}
\ttindex{LABEL}

{\em valid for: all processors}

The function of the \tty{LABEL} instruction is identical to \tty{EQU}, but
the symbol does not become typeless, it gets the attribute ''code''.
\tty{LABEL} is needed exactly for one purpose: Labels are normally local
in macros, that means they are not accessible outside of a macro.  With an
\tty{EQU} instruction you could get out of it nicely, but the phrasing
\begin{verbatim}
<name>  label   $
\end{verbatim}
generates a symbol with correct attributes.

%%- - - - - - - - - - - - - - - - - - - - - - - - - - - - - - - - - - - - - -

\subsection{BIT}
\ttindex{BIT}

{\em valid for: MCS/(2)51, XA, 80C166, 75K0, ST9, AVR, S12Z, SX20/28, H16,
                H8/300, H8/500, KENBAK, Padauk}

\tty{BIT} serves to equate a single bit of a memory cell with a symbolic
name.  This instruction varies from target platform to target platform due
to the different ways in which processors handle bit manipulation and
addressing:

The MCS/51 family has an own address space for bit operands.  The function
of \tty{BIT} is therefore quite similar to \tty{SFR}, i.e. a simple integer
symbol with the specified value is generated and assigned to the
\tty{BDATA} segment.  For all other processors, bit addressing is done in
a two-dimensional fashion with address and bit position.  In these cases,
AS packs both parts into an integer symbol in a way that depends on the
currently active target processor and separates both parts again when the
symbol is used.  The latter is is also valid for the 80C251: While an
instruction like
\begin{verbatim}
My_Carry bit    PSW.7
\end{verbatim}
would assign the value 0d7h to \tty{My\_Carry} on an 8051, a value of
070000d0h would be generated on an 80C251, i.e. the address is located in
bits 0..7 and the bit position in bits 24..26.  This procedure is equal to
the way the \tty{DBIT} instruction handles things on a TMS370 and is also
used on the 80C166, with the only difference that bit positions may range
from 0..15:
\begin{verbatim}
MSB     BIT     r5.15
\end{verbatim}
On a Philips XA, the bit's address is located in bits 0..9 just with
the same coding as used in machine instructions, and the 64K bank of
bits in RAM memory is placed in bits 16..23.

The \tty{BIT} instruction of the 75K0 family even goes further: As bit
expressions may not only use absolute base addresses, even expressions
like
\begin{verbatim}
bit1    BIT     @h+5.2
\end{verbatim}
are allowed.

The ST9 in turn allows to invert bits, what is also allowed in the
\tty{BIT} instruction:
\begin{verbatim}
invbit  BIT     r6.!3
\end{verbatim}
More about the ST9's \tty{BIT} instruction can be found in the processor
specific hints.

In case of H16, note that the address and bit position arguments are swapped.
This was done to make the syntax of BIT consistent with the machine instructions
that maipulate individual bits.

%%- - - - - - - - - - - - - - - - - - - - - - - - - - - - - - - - - - - - - -

\subsection{DBIT}
\ttindex{DBIT}

{\em valid for: TMS 370xxx}

Though the TMS370 series does not have an explicit bit segment, single bit
symbols may be simulated with this instruction.  \tty{DBIT} requires two
operands, the address of the memory cell that contains the bit and the
exact position of the bit in the byte.  For example,
\begin{verbatim}
INT3        EQU  P019
INT3_ENABLE DBIT 0,INT3
\end{verbatim}
defines the bit that enables interrupts via the INT3 pin.  Bits defined
this way may be used in the instructions \tty{SBIT0, SBIT1, CMPBIT,
JBIT0}, and \tty{JBIT}.

%%- - - - - - - - - - - - - - - - - - - - - - - - - - - - - - - - - - - - - -

\subsection{DEFBIT and DEFBITB}
\ttindex{DEFBIT}
\ttindex{DEFBITB}

\subsubsection{S12Z}

The S12Z family's processor core provides instructions to manipulate
individual bits in registers or memory cells.  To conveniently
address bits in the CPU's I/O area (first 4 Kbytes of the address
space), a bit may be given a symbolic name.  The bit is defined by
its memory address and the bit position:
\begin{verbatim}
<name>         defbit[.size]   <address>,<position>
\end{verbatim}
The \tty{address} must be located within the first 4 Kbytes, and the
operand size may be 8, 16, or 32 bits (\tty{size}=b/w/l).
Consequently, the \tty{position} may at most be 7, 15 or 31.  If no
operand size is given, byte size (.b) is assumed.  A bit defined
this way may be used as argument for the instructions {\tt BCLR,
BSET, BTGL, BRSET,} and {\tt BRCLR}:
\begin{verbatim}
mybit   defbit.b  $200,4
        bclr.b    $200,#4
        bclr      mybit
\end{verbatim}
Both uses of {\tt bclr} in this example generate identical code.
Since a bit defined this way ''knows'' its size, the size attribute
may be omitted when using it.

It is also possible to define bits that are located within a
structure's element:
\begin{verbatim}
mystruct struct    dots
reg      ds.w      1
flag     defbit    reg,4
         ends

         org       $100
data     mystruct

         bset      data.flag  ; same as bset.w $100,#4
\end{verbatim}

\subsubsection{Super8}

Opposed to the 'classic' Z8, the Super8 core supports instructions to operate
on bits in working or general registers.  ONe however has to to regard that
some of them can only operate on bits in one of the 16 working registers.
The \tty{DEFBIT} instruction allows to define bits of either type:
\begin{verbatim}
workbit defbit  r3,#4
slow    defbit  emt,#6
\end{verbatim}
Bits that have been defined this way may be used just like a argument duple of
register and bit position:
\begin{verbatim}
        ldb     r3,emt,#6
        ldb     r3,slo          ; same result

        bitc    r3,#4
        bitc    workbit         ; same result
\end{verbatim}

\subsubsection{Z8000}

The Z8000 features instructions to set and clear bits, however they cannot access
addresses in I/O space.  For this reason, both {\tt DEFBIT} and {\tt DEFBITB}
only allow to define bit objects in memory space.  The differentiation in operand
size is important because the Z8000 is a big endian processor: bit {\em n} of a
16 bit word at address {\em m} corresponds to bit {\em n} of an 8-bit byte at
address {\em m+1}.

%%- - - - - - - - - - - - - - - - - - - - - - - - - - - - - - - - - - - - - -

\subsection{DEFBITFIELD}
\ttindex{DEFBITFIELD}

{\em valid for: S12Z}

The S12Z family's CPU core not only deals with individual bits, it
is also able to extract a field of consecutive bits from an
8/16/24/32 value or to insert a bit field into such a value. Similar
to \tty{DEFBIT}, a bit field may be defined symbolically:
\begin{verbatim}
<Name>     defbitfield[.size] <address>,<width>:<position>
\end{verbatim}
Opposed to individual bits, an operand size of 24 bits (.p) is also
alloweed.  The range of \tty{position} and \tty{width} is accordingly 
0 to 23 resp. 1 to 24.  It is also allowed to define bit fields as
parts of structures:
\begin{verbatim}
mystruct struct      dots
reg      ds.w        1
clksel   defbitfield reg,4:8
         ends

         org       $100
data     mystruct

         bfext     d2,data.clksel ; fetch $100.w bits 4..11
                                  ; to D2 bits 0..7
         bfins     data.clksel,d2 ; insert D2 bits 0..7 into
                                  ; $100.w bits 4..11
\end{verbatim}
The internal representation of bits defined via \tty{DEFBIT} is
equivalent to bit fields with a width of one.  Therefore, a
symbolically defined bit may also be used as argument for
\tty{BFINS} and \tty{BFEXT}.

%%- - - - - - - - - - - - - - - - - - - - - - - - - - - - - - - - - - - - - -

\subsection{PORT}
\ttindex{PORT}

{\em valid for: 8008/8080/8085/8086, XA, Z80, Z8000, 320C2x/5x, TLCS-47, AVR, F8}

\tty{PORT} works similar to \tty{EQU}, just the symbol becomes assigned to the
I/O-address range.  Allowed values are 0..7 for the 3201x and 8008, 0..15 for the
320C2x, 0..65535 for the 8086, Z8000, and 320C5x, 0..63 for the AVR, and 0..255
for the rest.

Example : an 8255 PIO is located at address 20H:
\begin{verbatim}
PIO_port_A port 20h
PIO_port_B port PIO_port_A+1
PIO_port_C port PIO_port_A+2
PIO_ctrl   port PIO_port_A+3
\end{verbatim}

%%- - - - - - - - - - - - - - - - - - - - - - - - - - - - - - - - - - - - - -

\subsection{REG and NAMEREG}
\ttindex{REG}\ttindex{NAMEREG}

{\em valid for: 680x0, AVR, M*Core, ST9, 80C16x, Z8000, KCPSM \\
     ({\tt NAMEREG} valid only for KCPSM(3)), LatticeMico8, MSP430(X)}

Though it always has the same syntax, this instruction has a slightly
different meaning from processor to processor:  If the processor uses a
separate addressing space for registers, \tty{REG} has the same effect as
a simple \tty{EQU} for this address space (e.g. for the ST9).  \tty{REG}
defines register symbols for all other processors whose function is
described in section \ref{SectRegSyms}.

{\tt NAMEREG} exists for compatibility reasons to the original KCPSM
assembler.  It has an identical function, however both register and
symbolic name are given as arguments, for example:
\begin{verbatim}
     NAMEREG  s08, treg
\end{verbatim}

%%- - - - - - - - - - - - - - - - - - - - - - - - - - - - - - - - - - - - - -

\subsection{LIV and RIV}
\ttindex{LIV}\ttindex{RIV}

{\em valid for: 8X30x}

\tty{LIV} and \tty{RIV} allow to define so-called ''IV bus objects''.
These are
groups of bits located in a peripheral memory cell with a length of 1
up to 8 bits, which can afterwards be referenced symbolically.  The
result is that one does not anymore have to specify address,
position, and length separately for instructions that can refer to
peripheral bit groups.  As the 8X30x processors feature two
peripheral address spaces (a ''left'' and a ''right'' one), there are two
separate pseudo instructions.  The parameters of these instructions
are however equal: three parameters have to be given that specify
address, start position and length.  Further hints for the usage of
bus objects can be found in section \ref{8X30xSpec} .

%%- - - - - - - - - - - - - - - - - - - - - - - - - - - - - - - - - - - - - -

\subsection{CHARSET}
\ttindex{CHARSET}

{\em valid for: all processors}

Single board systems, especially when driving LCDs, frequently use
character sets different to ASCII.  So it is probably purely coincidental
that the umlaut coding corresponds with the one used by the PC.  To avoid
error-prone manual encoding, the assembler contains a translation table
for characters which assigns a target character to each source-code.  To
modify this table (which initial translates 1:1), one has to use the
\tty{CHARSET} instruction.  \tty{CHARSET} may be used with different
numbers and types of parameters.  If there is only a single parameter, it
has to be a string expression which is interpreted as a file name by AS.
AS reads the first 256 bytes from this table and copies them into the
translation table.  This allows to activate complex, externally generated
tables with a single statement.  For all other variants, the first
parameter has to be an integer in the range of 0 to 255 which designates
the start index of the entries to be modified in the translation table.
One or two parameters follow, giving the type of modification:

A single additional integer modies exactly one entry.  For example,
\begin{quote}{\tt
       CHARSET  '\"a',128
}\end{quote}
means that the target system codes the '\"a'  into the number 128
(80H).  If however two more integers are given, the first one describes
the last entry to be modified, and the second the new value of the first
table entry.  All entries up to the index end are loaded sequentially.
For example, in case that the target system does not support lower-case
characters, a simple
\begin{verbatim}
        CHARSET 'a','z','A'
\end{verbatim}
translates all lower-case characters  automatically into the
matching capital letters.

For the last variant, a string follows the start index and contains the
characters to be placed in the table.  The last example therefore may also
be written as
\begin{verbatim}
        CHARSET 'a',"ABCDEFGHIJKLMNOPQRSTUVWXYZ"
\end{verbatim}

\tty{CHARSET} may also be called without any parameters, which however has
a drastical effect: the translation table is reinitialized to its initial
state, i.e. all character translations are removed.

\bb{CAUTION!} \tty{CHARSET} not only affects string constants stored in
memory, but also integer constants written as ''ASCII''. This means that
an already modified translation table can lead to other results in the
above mentioned examples!

%%- - - - - - - - - - - - - - - - - - - - - - - - - - - - - - - - - - - - - -

\subsection{CODEPAGE}
\ttindex{CODEPAGE}

{\em valid for: all processors}

Though the \tty{CHARSET} statement gives unlimited freedom in the
character assignment between host and target platform, switching among
different character {\em sets} can become quite tedious if several
character sets have to be supported on the target platform.  The
\tty{CODEPAGE} instruction however allows to define and keep different
character sets and to switch with a single statement among them.
\tty{CODEPAGE} expects one or two arguments: the name of the set to be
used hereafter and optionally the name of another table that defines its
initial contents (the second parameter therefore only has a meaning for
the first switch to the table when AS automatically creates it).  If the
second parameter is missing, the initial contents of the new table are
copied from the previously active set.  All subsequent \tty{CHARSET}
statements {\em only} modify the new set.

At the beginning of a pass, AS automatically creates a single character
set with the name \tty{STANDARD} with a one-to-one translation.  If no
\tty{CODEPAGE} instructions are used, all settings made via \tty{CHARSET}
refer to this table.

%%- - - - - - - - - - - - - - - - - - - - - - - - - - - - - - - - - - - - - -

\subsection{ENUM, NEXTENUM, and ENUMCONF}
\ttindex{ENUM}
\ttindex{NEXTENUM}
\ttindex{ENUMCONF}

{\em valid for: all processors}

Similar to the same-named instruction known from C, \tty{ENUM} is used to
define enumeration types, i.e. a sequence of integer constants that
are assigned sequential values starting at 0.  The parameters are the
names of the symbols, like in the following example:
\begin{verbatim}
        ENUM    SymA,SymB,SymC
\end{verbatim}
This instruction will assign the values 0, 1, and 2 to the symbols
\tty{SymA, SymB,} and \tty{SymC}.

If you want to split an enumeration over more than one line, use
\tty{NEXTENUM} instead of \tty{ENUM} for the second and all
following lines.  The internal counter that assigns sequential
values to alls symbols will then not be reset to zero, like in the
following case:
\begin{verbatim}
        ENUM     January=1,February,March,April,May,June
        NEXTENUM July,August,September,October
        NEXTENUM November,December
\end{verbatim}
This example also demonstrates that it is possible to assign
explicit values to individual symbols.  The internal counter will
be updated accordingly if this feature is used.

A definition of a symbol with \tty{ENUM} is equal to a definition with
\tty{EQU}, i.e. it is not possible to assign a new value to a symbol that
already exists.

The \tty{ENUMCONF} statement allows to influence the behaviour of
\tty{ENUM}.  \tty{ENUMCONF} accepts one or two arguments.  The
first argument is always the value the internal counter is
incremented for every symbol in an enumeration.  For instance,
the statement
\begin{verbatim}
      ENUMCONF 2
\end{verbatim}
has the effect that symbols get the values 0,2,4,6... instead of
0,1,2,3...

The second (optional) argument of \tty{ENUMCONF} rules which
address space the defined symbols are assigned to.  By default,
symbols defined by \tty{ENUM} are typeless.  For instance, the
statement
\begin{verbatim}
      ENUMCONF 1,CODE
\end{verbatim}
defines that they should be assigned to the instruction address
space.  The names of the address spaces are the same as for the
\tty{SEGMENT} instruction (\ref{SEGMENT}), with the addition of
\tty{NOTHING} to generate typeless symbols again.

%%- - - - - - - - - - - - - - - - - - - - - - - - - - - - - - - - - - - - - -

\subsection{PUSHV and POPV}
\ttindex{PUSHV}\ttindex{POPV}

{\em valid for: all processors}

\tty{PUSHV} and \tty{POPV} allow to temporarily save the value of a symbol
(that is not macro-local) and to restore it at a later point of time.  The
storage is done on stacks, i.e. Last-In-First-Out memory structures.  A
stack has a name that has to fulfill the general rules for symbol names
and it exists as long as it contains at least one element: a stack that
did not exist before is automatically created upon \tty{PUSHV}, and a
stack becoming empty upon a \tty{POPV} is deleted automatically.  The name
of the stack that shall be used to save or restore symbols is the first
parameter of \tty{PUSH} resp. \tty{POPV}, followed by a list of symbols as
further parameters.  All symbols referenced in the list already have to
exist, it is therefore \bb{not} possible to implicitly define symbols with
a \tty{POPV} instruction.

Stacks are a global resource, i.e. their names are not local to
sections.

It is important to note that symbol lists are \bb{always} processed from
left to right.  Someone who wants to pop several variables from a stack
with a \tty{POPV} therefore has to use the exact reverse order used in the
corresponding \tty{PUSHV}!

The name of the stack may be left blank, like this:
\begin{verbatim}
        pushv   ,var1,var2,var3
        .
        .
        popv    ,var3,var2,var1
\end{verbatim}
AS will then use a predefined internal default stack.

AS checks at the end of a pass if there are stacks that are not empty and
issues their names together with their ''filling level''.  This allows to
find out if there are any unpaired \tty{PUSHVs} or \tty{POPVs}.  However,
it is in no case possible to save values in a stack beyond the end of a
pass: all stacks are cleared at the beginning of a pass!

%%---------------------------------------------------------------------------

\section{Code Modification}

%%- - - - - - - - - - - - - - - - - - - - - - - - - - - - - - - - - - - - - -

\subsection{ORG}
\label{SectORG}
\ttindex{ORG}

{\em valid for: all processors}

\tty{ORG} allows to load the internal address counter (of the assembler)
with a new value. The value range depends on the currently selected
segment and on the processor type (table \ref{TabORG}).
The lower bound is always zero, and the upper bound is the given value
minus 1.
\par
{\bf CAUTION}: If the \tty{PHASE} instruction is also used, one
has to keep in mind that the argument of \tty{ORG} always is the
{\em load address} of the code.  Expressions using the \$ or \*
symbol to refer to the current program counter however deliver
the {\em execution address} of the code and do not yield the
desired result when used as argument for \tty{ORG}.  The
\tty{RORG} statement (\ref{SectRORG}) should be used in such cases.

\hfuzz=60pt
\small
\begin{longtable}{|l|c|c|c|c|c|c|c|c|c|c|}
\hline
\tin{Target} & \tin{CODE} & \tin{DATA} & \tin{I-}   & \tin{X-}   & \tin{Y-}   & \tin{BIT-} & \tin{IO} & \tin{REG} & \tin{ROM-}  & \tin{EE-}  \\
             &            &            & \tin{DATA} & \tin{DATA} & \tin{DATA} & \tin{DATA} &          &           & \tin{DATA}  & \tin{DATA} \\
\hline
\hline
\endhead
\tin{68xxx/}    & 4G      &   ---   &   ---   &   ---   &   ---   &   ---   & ---  &   ---  & ---  & --- \\
\tin{MCF}       &         &         &         &         &         &         &      &        &      &     \\
\hline
\tin{DSP56000}  & 64K/    &   ---   &   ---   & 64K/    & 64K/    &   ---   & ---  &   ---  & ---  & --- \\
\tin{DSP56300}  & 16M     &         &         & 16M     & 16M     &         &      &        &      &     \\
\hline
\tin{PowerPC}   & 4G      &   ---   &   ---   &   ---   &   ---   &   ---   & ---  &   ---  & ---  & --- \\
\hline
\tin{PALM}      & 64K     &   ---   &   ---   &   ---   &   ---   &   ---   & 16   &   ---  & ---  & --- \\
\hline
\tin{M*Core}    & 4G      &   ---   &   ---   &   ---   &   ---   &   ---   & ---  &   ---  & ---  & --- \\
\hline
\tin{6800,6301,}& 64K     &   ---   &   ---   &   ---   &   ---   &   ---   & ---  &   ---  & ---  & --- \\
\tin{6811}      &         &         &         &         &         &         &      &        &      &     \\
\hline
\tin{6805/}     & 8K/     &   ---   &   ---   &   ---   &   ---   &   ---   & ---  &   ---  & ---  & --- \\
\tin{HC08}      & 64K     &   ---   &   ---   &   ---   &   ---   &   ---   & ---  &   ---  & ---  & --- \\
\hline
\tin{6809,}     & 64K     &   ---   &   ---   &   ---   &   ---   &   ---   & ---  &   ---  & ---  & --- \\
\tin{6309}      &         &         &         &         &         &         &      &        &      &     \\
\hline
\tin{68HC12,}   & 64K     &   ---   &   ---   &   ---   &   ---   &   ---   & ---  &   ---  & ---  & --- \\
\tin{68HC12X,}  &         &         &         &         &         &         &      &        &      &     \\
\tin{XGATE}     &         &         &         &         &         &         &      &        &      &     \\
\hline
\tin{S12Z}      & 16M     &   ---   &   ---   &   ---   &   ---   &   ---   & ---  &   ---  & ---  & --- \\
\hline
\tin{68HC16}    & 1M      &   ---   &   ---   &   ---   &   ---   &   ---   & ---  &   ---  & ---  & --- \\
\hline
\tin{68RS08}    & 16K     &   ---   &   ---   &   ---   &   ---   &   ---   & ---  &   ---  & ---  & --- \\
\hline
\tin{H8/300}    & 64K     &   ---   &   ---   &   ---   &   ---   &   ---   & ---  &   ---  & ---  & --- \\
\tin{H8/300H}   & 16M     &         &         &         &         &         &      &        &      &     \\
\hline
\tin{H8/500}    & 64K     &   ---   &   ---   &   ---   &   ---   &   ---   & ---  &   ---  & ---  & --- \\
\tin{(Min)}     &         &         &         &         &         &         &      &        &      &     \\

\tin{H8/500}    & 16M     &   ---   &   ---   &   ---   &   ---   &   ---   & ---  &   ---  & ---  & --- \\
\tin{(Max)}     &         &         &         &         &         &         &      &        &      &     \\
\hline
\tin{SH7000/}   & 4G      &   ---   &   ---   &   ---   &   ---   &   ---   & ---  &   ---  & ---  & --- \\
\tin{7600/7700} &         &         &         &         &         &         &      &        &      &     \\
\hline
\tin{HD614023}  & 2K      & 160     &   ---   &   ---   &   ---   &   ---   & 16   &   ---  & ---  & --- \\
\tin{HD614043}  & 4K      & 256     &   ---   &   ---   &   ---   &   ---   & 16   &   ---  & ---  & --- \\
\tin{HD614081}  & 8K      & 512     &   ---   &   ---   &   ---   &   ---   & 16   &   ---  & ---  & --- \\
\hline
\tin{HD641016}  & 16M     &   ---   &   ---   &   ---   &   ---   &   ---   & ---  &   ---  & ---  & --- \\
\hline
\tin{6502,}     & 64K     &   ---   &   ---   &   ---   &   ---   &   ---   & ---  &   ---  & ---  & --- \\
\tin{MELPS-}    &         &         &         &         &         &         &      &        &      &     \\
\tin{740}       &         &         &         &         &         &         &      &        &      &     \\
\hline
\tin{HUC6280}   & 2M      &   ---   &   ---   &   ---   &   ---   &   ---   & ---  &   ---  & ---  & --- \\
\hline
\tin{65816,}    & 16M     &   ---   &   ---   &   ---   &   ---   &   ---   & ---  &   ---  & ---  & --- \\
\tin{MELPS-}    &         &         &         &         &         &         &      &        &      &     \\
\tin{7700}      &         &         &         &         &         &         &      &        &      &     \\
\hline
PPS-4           & 4K      & 4K      &   ---   &   ---   &   ---   &   ---   & 16   &   ---  & ---  & --- \\
\hline
\tin{MELPS-}    & 8K      & 416     &   ---   &   ---   &   ---   &   ---   & ---  &   ---  & ---  & --- \\
\tin{4500}      &         &         &         &         &         &         &      &        &      &     \\
\hline
\tin{M16}       & 4G      &   ---   &   ---   &   ---   &   ---   &   ---   & ---  &   ---  & ---  & --- \\
\hline
\tin{M16C}      & 1M      &   ---   &   ---   &   ---   &   ---   &   ---   & ---  &   ---  & ---  & --- \\
\hline
\tin{PDP-11}    & 64K     &   ---   &   ---   &   ---   &   ---   &   ---   & ---  &   ---  & ---  & --- \\
                & 256K    &   ---   &   ---   &   ---   &   ---   &   ---   & ---  &   ---  & ---  & --- \\
                & 4M$^{10}$ & ---   &   ---   &   ---   &   ---   &   ---   & ---  &   ---  & ---  & --- \\
\hline
\tin{4004}      & 4K      & 256     &   ---   &   ---   &   ---   &   ---   & ---  &   ---  & ---  & --- \\
\hline
\tin{8008}      & 16K     & 8       &   ---   &   ---   &   ---   &   ---   & ---  &   ---  & ---  & --- \\
\hline
\tin{MCS-48,}   & 1/2/4/ &  ---   & 256     & 256$^{8}$     &   ---   &   ---   & ---  &   ---  & ---  & --- \\
\tin{MCS-41}    & 6/8K$^{6}$ &    &         &         &         &         &      &        &      &     \\
\hline
\tin{MCS-51}    & 64K     & 256     & 256$^{1}$ & 64K     &   ---   & 256     &   ---   &   ---  & ---  & --- \\
\hline
\tin{80C390}    & 16M     & 256     & 256$^{1}$ & 16M     &   ---   & 256     &   ---   &   ---  & ---  & --- \\
\hline
\tin{MCS-251}   & 16M     &   ---   &   ---   &   ---   &   ---   &   ---   & 512     &   ---  & ---  & --- \\
\hline
\tin{MCS-96 }   & 64K     &   ---   &   ---   &   ---   &   ---   &   ---   &   ---   &   ---  & ---  & --- \\
\tin{196(N)/}   & 16M     &         &         &         &         &         &         &        &      &     \\
\tin{296}       &         &         &         &         &         &         &         &        &      &     \\
\hline
\tin{8080,}     & 64K     &   ---   &   ---   &   ---   &   ---   &   ---   & 256     &   ---  & ---  & --- \\
\tin{8085}      &         &         &         &         &         &         &         &        &      &     \\
\hline
\tin{80x86,}    & 64K     & 64K     &   ---   & 64K     &   ---   &   ---   & 64K     &   ---  & ---  & --- \\
\hline
\tin{68xx0}     & 4G      &   ---   &   ---   &   ---   &   ---   &   ---   & ---     &   ---  & ---  & --- \\
\hline
\tin{8X30x}     & 8K      &   ---   &   ---   &   ---   &   ---   &   ---   & ---     &   ---  & ---  & --- \\
\hline
\tin{2650}      & 32K     &   ---   &   ---   &   ---   &   ---   &   ---   & ---     &   ---  & ---  & --- \\
\hline
\tin{XA}        & 16M     & 16M     &   ---   &   ---   &   ---   &   ---   & 2K$^{3}$ &   ---  & ---  & --- \\
\hline
\tin{AVR}       & 128K$^{6}$ & 32K$^{6}$ &   ---   &   ---   &   ---   &   ---   & 64      &   ---  & ---  & 8K$^{7}$ \\
\hline
\tin{29XXX}     & 4G      &   ---   &   ---   &   ---   &   ---   &   ---   &   ---   &   ---  & ---  & --- \\
\hline
\tin{80C166,}   & 256K    &   ---   &   ---   &   ---   &   ---   &   ---   &   ---   &   ---  & ---  & --- \\
\tin{80C167}    & 16M     &         &         &         &         &         &         &        &      &     \\
\hline
\tin{GBZ80}     & 64K     &   ---   &   ---   &   ---   &   ---   &   ---   &  ----   &   ---  & ---  & --- \\
\tin{Z80,}      & 64K     &   ---   &   ---   &   ---   &   ---   &   ---   & 256     &   ---  & ---  & --- \\
\tin{Z180,}     & 512K$^{2}$ &      &         &         &         &         & 256     &        &      & --- \\
\tin{Z380}      & 4G      &         &         &         &         &         & 4G      &        &      &     \\
\hline
\tin{Z8}        & 64K     & 256     &   ---   &   ---   &   ---   &   ---   &   ---   &   ---  & ---  & --- \\
\hline
\tin{eZ8}       & 64K     & 256     &   ---   &   64K   &   ---   &   ---   &   ---   &   ---  & ---  & --- \\
\hline
\tin{Z8001,}    & 8M      & ---     &   ---   &   ---   &   ---   &   ---   & 64K     &   ---  & ---  & --- \\
\tin{Z8003}     &         &         &         &         &         &         &         &        &      &     \\
\hline
\tin{Z8002,}    & 64K     & ---     &   ---   &   ---   &   ---   &   ---   & 64K     &   ---  & ---  & --- \\
\tin{Z8004}     &         &         &         &         &         &         &         &        &      &     \\
\hline
\tin{KCPSM}     & 256     & 256     &   ---   &   ---   &   ---   &   ---   &   ---   &   ---  & ---  & --- \\
\hline
\tin{KCPSM3}    & 256     & 64      &   ---   &   ---   &   ---   &   ---   &  256    &   ---  & ---  & --- \\
\hline
\tin{Mico8 }    & 4096    & 256     &   ---   &   ---   &   ---   &   ---   &  256    &   ---  & ---  & --- \\
\hline
\tin{TLCS-}     & 16M     &   ---   &   ---   &   ---   &   ---   &   ---   &   ---   &   ---  & ---  & --- \\
\tin{900(L)}    &         &         &         &         &         &         &         &        &      &     \\
\hline
\tin{TLCS-90}   & 64K     &   ---   &   ---   &   ---   &   ---   &   ---   &   ---   &   ---  & ---  & --- \\
\hline
\tin{TLCS-}     & 64K     &   ---   &   ---   &   ---   &   ---   &   ---   &   ---   &   ---  & ---  & --- \\
\tin{870(/C)}   &         &         &         &         &         &         &         &        &      &     \\
\hline
\tin{TLCS-47}   & 64K     & 1K      &   ---   &   ---   &   ---   &   ---   & 16      &   ---  & ---  & --- \\
\hline
\tin{TLCS-}     & 16M     &   ---   &   ---   &   ---   &   ---   &   ---   &   ---   &   ---  & ---  & --- \\
\tin{9000}      &         &         &         &         &         &         &         &        &      &     \\
\hline
\tin{TC9331}    & 320     &   ---   &   ---   &   ---   &   ---   &   ---   &   ---   &   ---  & ---  & --- \\
\hline
\tin{PIC}       & 2K      & 32      &   ---   &   ---   &   ---   &   ---   &   ---   &   ---  & ---  & --- \\
\tin{16C5x}     &         &         &         &         &         &         &         &        &      &     \\
\hline
\tin{PIC}       & 2K      & 32      &   ---   &   ---   &   ---   &   ---   &   ---   &   ---  & ---  & --- \\
\tin{16C5x}     &         &         &         &         &         &         &         &        &      &     \\
\hline
\tin{PIC}       &         &         &         &         &         &         &         &        &      &     \\
\tin{16C64,}    & 8K      & 512     &   ---   &   ---   &   ---   &   ---   &   ---   &   ---  & ---  & 256$^{6}$ \\
\tin{16C86}     &         &         &         &         &         &         &         &        &      &     \\
\hline
\tin{PIC}       & 64K     & 256     &   ---   &   ---   &   ---   &   ---   &   ---   &   ---  & ---  & --- \\
\tin{17C42}     &         &         &         &         &         &         &         &        &      &     \\
\hline
\tin{SX20}      & 2K      & 256     &   ---   &   ---   &   ---   &   ---   &   ---   &   ---  & ---  & --- \\
\hline
\tin{ST6}       & 4K      & 256     &   ---   &   ---   &   ---   &   ---   &   ---   &   ---  & ---  & --- \\
\hline
\tin{ST7}       & 64K     &  ---    &   ---   &   ---   &   ---   &   ---   &   ---   &   ---  & ---  & --- \\
\hline
\tin{STM8}      & 16M     &  ---    &   ---   &   ---   &   ---   &   ---   &   ---   &   ---  & ---  & --- \\
\hline
\tin{ST9}       & 64K     & 64K     &   ---   &   ---   &   ---   &   ---   &   ---   & 256    & ---  & --- \\
\hline
\tin{6804}      & 4K      & 256     &   ---   &   ---   &   ---   &   ---   &   ---   &   ---  & ---  & --- \\
\hline
\tin{32010}     & 4K      & 144     &   ---   &   ---   &   ---   &   ---   & 8       &   ---  & ---  & --- \\
\tin{32015}     & 4K      & 256     &         &         &         &         & 8       &        &      &     \\
\hline
\tin{320C2x}    & 64K     & 64K     &   ---   &   ---   &   ---   &   ---   & 16      &   ---  & ---  & --- \\
\hline
\tin{320C3x}    & 16M     &   ---   &   ---   &   ---   &   ---   &   ---   &   ---   &   ---  & ---  & --- \\
\hline
\tin{320C40}    & 4G      &   ---   &   ---   &   ---   &   ---   &   ---   &   ---   &   ---  & ---  & --- \\
\hline
\tin{320C44}    & 32M     &   ---   &   ---   &   ---   &   ---   &   ---   &   ---   &   ---  & ---  & --- \\
\hline
\tin{320C5x/}   & 64K     & 64K     &   ---   &   ---   &   ---   &   ---   & 64K     &   ---  & ---  & --- \\
\tin{320C20x/}  &         &         &         &         &         &         &         &        &      &     \\
\tin{320C54x}   &         &         &         &         &         &         &         &        &      &     \\
\hline
\tin{TMS}       & 64K     &   ---   &   ---   &   ---   &   ---   &   ---   &   ---   &   ---  & ---  & --- \\
\tin{9900}      &         &         &         &         &         &         &         &        &      &     \\
\hline
\tin{TMS}       & 64K     &   ---   &   ---   &   ---   &   ---   &   ---   &   ---   &   ---  & ---  & --- \\
\tin{70Cxx}     &         &         &         &         &         &         &         &        &      &     \\
\hline
\tin{370xxx}    & 64K     &   ---   &   ---   &   ---   &   ---   &   ---   &   ---   &   ---  & ---  & --- \\
                &         &         &         &         &         &         &         &        &      &     \\
\hline
\tin{MSP430}    & 64K     &   ---   &   ---   &   ---   &   ---   &   ---   &   ---   &   ---  & ---  & --- \\
                &         &         &         &         &         &         &         &        &      &     \\
\hline
\tin{TMS1000}   & 1K      & 64      &   ---   &   ---   &   ---   &   ---   &   ---   &   ---  & ---  & --- \\
\tin{TMS1200}   &         &         &         &         &         &         &         &        &      &     \\
\hline
\tin{TMS1100}   & 2K      & 128     &   ---   &   ---   &   ---   &   ---   &   ---   &   ---  & ---  & --- \\
\tin{TMS1300}   &         &         &         &         &         &         &         &        &      &     \\
\hline
\tin{SC/MP}     & 64K     &   ---   &   ---   &   ---   &   ---   &   ---   &   ---   &   ---  & ---  & --- \\
\hline
\tin{807x}      & 64K     &   ---   &   ---   &   ---   &   ---   &   ---   &   ---   &   ---  & ---  & --- \\
\hline
\tin{COP4}      & 512     &   ---   &   ---   &   ---   &   ---   &   ---   &   ---   &   ---  & ---  & --- \\
\hline
\tin{COP8}      & 8K      & 256     &   ---   &   ---   &   ---   &   ---   &   ---   &   ---  & ---  & --- \\
\hline
\tin{SC144xx}   & 256     &   ---   &   ---   &   ---   &   ---   &   ---   &   ---   &   ---  & ---  & --- \\
\hline
\tin{NS16008/}  & 16M     &   ---   &   ---   &   ---   &   ---   &   ---   &   ---   &   ---  & ---  & --- \\
\tin{NS32008/}  &         &         &         &         &         &         &         &        &      &     \\
\tin{NS08032/}  &         &         &         &         &         &         &         &        &      &     \\
\tin{NS16032/}  &         &         &         &         &         &         &         &        &      &     \\
\tin{NS32016/}  &         &         &         &         &         &         &         &        &      &     \\
\tin{NS32032/}  &         &         &         &         &         &         &         &        &      &     \\
\tin{NS32CG16}  &         &         &         &         &         &         &         &        &      &     \\
\hline
\tin{NS32332/}  & 4G      &   ---   &   ---   &   ---   &   ---   &   ---   &   ---   &   ---  & ---  & --- \\
\tin{NS32532}   &         &         &         &         &         &         &         &        &      &     \\
\hline
\tin{ACE}       & 4K$^{4}$ &   ---   &   ---   &   ---   &   ---   &   ---   &   ---   &   ---  & --- & --- \\
\hline
\tin{CP-3F/}    & 16K     &  48     &   ---   &   ---   &   ---   &   ---   &     8   &        &      &     \\
\tin{M380/}     &         &         &         &         &         &         &         &        &      &     \\
\tin{LP8000}    &         &         &         &         &         &         &         &        &      &     \\
\hline
\tin{F3850}     & 64K    &   64    &   ---   &   ---   &   ---   &   ---   &   256   &   ---  & ---  & --- \\
\tin{F8}        & 4K     &   64    &   ---   &   ---   &   ---   &   ---   &   256   &   ---  & ---  & --- \\
\hline
\tin{$\mu$PD}   & 64K     &   ---   &   ---   &   ---   &   ---   &   ---   &   ---   &   ---  & ---  & --- \\
\tin{78(C)xx}   &         &         &         &         &         &         &         &        &      &     \\
\hline
\tin{7566}      & 1K      & 64      &   ---   &   ---   &   ---   &   ---   &   ---   &   ---  & ---  & --- \\
\hline
\tin{7508}      & 4K      & 256     &   ---   &   ---   &   ---   &   ---   &    16   &   ---  & ---  & --- \\
\hline
\tin{75K0}      & 16K     & 4K      &   ---   &   ---   &   ---   &   ---   &   ---   &   ---  & ---  & --- \\
\hline
\tin{78K0}      & 64K     &   ---   &   ---   &   ---   &   ---   &   ---   &   ---   &   ---  & ---  & --- \\
\hline
\tin{78K2}      & 1M      &   ---   &   ---   &   ---   &   ---   &   ---   &   ---   &   ---  & ---  & --- \\
\hline
\tin{78K3}      & 64K     &   ---   &   ---   &   ---   &   ---   &   ---   &   ---   &   ---  & ---  & --- \\
\hline
\tin{78K4}      & 16M$^{5}$ &   ---   &   ---   &   ---   &   ---   &   ---   &   ---   &   ---  & ---  & --- \\
\hline
\tin{7720}      & 512     & 128     &   ---   &   ---   &   ---   &   ---   &   ---   &   ---  & 512  & --- \\
                &         &         &         &         &         &         &         &        &      &     \\
\hline
\tin{7725}      & 2K      & 256     &   ---   &   ---   &   ---   &   ---   &   ---   &   ---  & 1024  & --- \\
                &         &         &         &         &         &         &         &        &       &     \\
\hline
\tin{77230}     & 8K      & ---     &   ---   & 512     & 512     &   ---   &   ---   &   ---  & 1K    & --- \\
\hline
\tin{70616}     & 4G      & ---     &   ---   &   ---   &   ---   &   ---   &   16M   &   ---  & ---  & --- \\
\hline
\tin{53C8XX}    & 4G      & ---     &   ---   &   ---   &   ---   &   ---   &   ---   &   ---  & ---  & --- \\
\hline
\tin{F$^{2}$MC8L} & 64K   & ---     &   ---   &   ---   &   ---   &   ---   &   ---   &   ---  & ---  & --- \\
\hline
\tin{F$^{2}$MC16L} & 16M  & ---     &   ---   &   ---   &   ---   &   ---   &   ---   &   ---  & ---  & --- \\
\hline
\tin{MSM5840}   & 2K      & 128     &   ---   &   ---   &   ---   &   ---   &   ---   &   ---  & ---  & --- \\
\hline
\tin{MSM5842}   & 768     & 32      &   ---   &   ---   &   ---   &   ---   &   ---   &   ---  & ---  & --- \\
\hline
\tin{MSM58421}  & 1.5K    & 40      &   ---   &   ---   &   ---   &   ---   &   ---   &   ---  & ---  & --- \\
\tin{MSM58422}  &         &         &         &         &         &         &         &        &      &     \\
\hline
\tin{MSM5847}   & 1.5K    & 96      &   ---   &   ---   &   ---   &   ---   &   ---   &   ---  & ---  & --- \\
\hline
\tin{MSM5054}   & 1K      & 62      &   ---   &   ---   &   ---   &   ---   &   ---   &   ---  & ---  & --- \\
\hline
\tin{MSM5055}   & 1.75K   & 96      &   ---   &   ---   &   ---   &   ---   &   ---   &   ---  & ---  & --- \\
\hline
\tin{MSM5056}   & 1.75K   & 90      &   ---   &   ---   &   ---   &   ---   &   ---   &   ---  & ---  & --- \\
\hline
\tin{MSM6051}   & 2.5K    & 119     &   ---   &   ---   &   ---   &   ---   &   ---   &   ---  & ---  & --- \\
\hline
\tin{MN1610}    & 64K     & ---     &   ---   &   ---   &   ---   &   ---   &   64K   &   ---  & ---  & --- \\
\hline
\tin{MN1613}    & 256K    & ---     &   ---   &   ---   &   ---   &   ---   &   64K   &   ---  & ---  & --- \\
\hline
\tin{PMCxxx/}   & 1..     & 64..    &  ---    &   ---   &   ---   &   ---   & 32..    &   ---  & ---  & --- \\
\tin{PMSxxx/}   & 4K$^{9}$ & 256$^{9}$ &      &         &         &         & 128$^{9}$ &      &      &     \\
\tin{PFSxxx}    &         &         &         &         &         &         &         &        &      &     \\
\hline
\tin{180x}      & 64K     & ---     &   ---   &   ---   &   ---   &   ---   &   8     &   ---  & ---  & --- \\
\hline
\tin{XS1}       & 4G      & ---     &   ---   &   ---   &   ---   &   ---   &   ---   &   ---  & ---  & --- \\
\hline
\tin{1750}      & 64K     & ---     &   ---   &   ---   &   ---   &   ---   &   ---   &   ---  & ---  & --- \\
\hline
\tin{KENBAK}    & 256     & ---     &   ---   &   ---   &   ---   &   ---   &   ---   &   ---  & ---  & --- \\
\hline
\tin{CP1600}    & 64K     & ---     &   ---   &   ---   &   ---   &   ---   &   ---   &   ---  & ---  & --- \\
\hline
\tin{NANO}      & 2K      & ---     &   ---   &   ---   &   ---   &   ---   &   ---   &   ---  & ---  & --- \\
\hline
\tin{IM6100}    & 4K      & ---     &   ---   &   ---   &   ---   &   ---   &   ---   &   ---  & ---  & --- \\
\hline
\tin{IM6120}    & 32K     & ---     &   ---   &   ---   &   ---   &   ---   &   ---   &   ---  & ---  & --- \\

\\ \hline
\multicolumn{11}{|l|}{$^{1}$ Initial value 80h.} \\
\multicolumn{11}{|l|}{   As the 8051 does not have any RAM beyond 80h, this value has to be} \\
\multicolumn{11}{|l|}{   adapted with ORG for the 8051 as target processor!}\\
\hline
\multicolumn{11}{|l|}{$^{2}$ As the Z180 still can address only 64K logically, the whole}\\
\multicolumn{11}{|l|}{   address space can only be reached via \tty{PHASE} instructions!}\\
\hline
\multicolumn{11}{|l|}{$^{3}$ initial value 400h.}\\
\hline
\multicolumn{11}{|l|}{$^{4}$ initial value 800h resp. 0C00h} \\
\hline
\multicolumn{11}{|l|}{$^{5}$ area for program code is limited to 1 MByte} \\
\hline
\multicolumn{11}{|l|}{$^{6}$ size depends on target processor} \\
\hline
\multicolumn{11}{|l|}{$^{7}$ size and availibility depend on target processor} \\
\hline
\multicolumn{11}{|l|}{$^{8}$ only on variants supporting the \tty{MOVX} instruction} \\
\hline
\multicolumn{11}{|l|}{$^{9}$ device dependant} \\
\hline
\caption{Address Ranges for \tty{ORG}}
\label{TabORG}
\end{longtable}
\normalsize
\hfuzz=0pt

In case that different variations in a processor family have address
spaces of different size, the maximum range is listed for each.

\tty{ORG} is mostly needed to give the code a new starting address or to
put different, non-continuous code parts into one source file.  In case
there is no explicit other value listet in a table entry, the initial
address for this segment (i.e. the start address used without {\tt ORG})
is 0.

%%- - - - - - - - - - - - - - - - - - - - - - - - - - - - - - - - - - - - - -

\subsection{RORG}
\label{SectRORG}
\ttindex{RORG}

{\em valid for: all processors}

\tty{RORG} modifies the program counter just like \tty{ORG},
however it does not expect an absolute address as argument.
Instead, it expects a relative value (positive or negative) that
is added to the current program counter.  A possible application
of this statement is the reservation of a certain amount of
address space, or the use in code parts that are included
multiple times (e.g. via macros or includes) and that shall be
position-independent.  Another application is the use in code
that has an execution address different from the load address
(i.e. the \tty{PHASE} statement is used).  There is no symbol to
refer to the current {\em load address}, but it can be referred
to indirectly via the \tty{RORG} statement.

%%- - - - - - - - - - - - - - - - - - - - - - - - - - - - - - - - - - - - - -

\subsection{CPU}
\label{SectCPU}
\ttindex{CPU}

{\em valid for: all processors}

This command rules for which processor the further code shall be
generated.  Instructions of other processor families are not
accessible afterwards and will produce error messages!

The processors can roughly be distinguished in families, inside the
families different types additionally serve for a detailed
distinction:
%%-----------
\begin{quote}
\begin{tabbing}
\hspace{0.7cm} \= \kill
a) \> 68008 $\rightarrow$ 68000 $\rightarrow$ 68010 $\rightarrow$ 68012 $\rightarrow$ \\
   \> MCF5202 $\rightarrow$ MCF5204 $\rightarrow$ MCF5206 $\rightarrow$ MCF5208$\rightarrow$ \\
   \> MCF52274 $\rightarrow$ MCF52277 $\rightarrow$ MCF5307 $\rightarrow$ MCF5329 $\rightarrow$ \\
   \> MCF5373 $\rightarrow$ MCF5407 $\rightarrow$ MCF5470 $\rightarrow$ MCF5471 $\rightarrow$ \\
   \> MCF5472 $\rightarrow$ MCF5473 $\rightarrow$ MCF5474 $\rightarrow$ MCF5475 $\rightarrow$ \\
   \> MCF51QM $\rightarrow$ \\
   \> 68332 $\rightarrow$ 68340 $\rightarrow$ 68360 $\rightarrow$ \\
   \> 68020 $\rightarrow$ 68030 $\rightarrow$ 68040
\end{tabbing}
\end{quote}
The differences in this family are additional instructions and
addressing modes (starting from the 68020).  A small exception is the step
to the 68030 that misses two instructions: \tty{CALLM} and \tty{RTM}.  The
three representatives of the 683xx family have the same processor core (a
slightly reduced 68020 CPU), however completely different peripherals.
MCF5xxx represents various ColdFire variants from Motorola/Freescale/NXP,
RISC processors downwardly binary compatible to the 680x0.  For the 68040,
additional control registers (reachable via \tty{MOVEC}) and instructions
for control of the on-chip MMU and caches were added.
%%-----------
\begin{quote}
b) 56000 $\longrightarrow$ 56002 $\longrightarrow$ 56300
\end{quote}
While the 56002 only adds instructions for incrementing and decrementing
the accumulators, the 56300 core is almost a new processor: all address
spaces are enlarged from 64K words to 16M and the number of instructions
almost has been doubled.
%%-----------
\begin{quote}
c) PPC403 $\rightarrow$ MPPC403 $\rightarrow$ MPC505 $\rightarrow$ MPC601 $\rightarrow$ MPC821 $\rightarrow$ RS6000
\end{quote}
The PPC403 is a reduced version of the PowerPC line without a floating
point unit, which is why all floating point instructions are disabled for
him; in turn, some microcontroller-specific instructions have been added
which are unique in this family.  The GC variant of the PPC403
incorporates an additional MMU and has therefore some additional
instructions for its control.  The MPC505 (a microcontroller variant
without a FPU) only differ in its peripheral registers from the 601 as
long as I do not know it better - \cite{Mot505} is a bit reluctant in this
respect...  The RS6000 line knows a few instructions more (that are
emulated on many 601-based systems), IBM additionally uses different
mnemonics for their pure workstation processors, as a reminiscence of 370
mainframes...
%%-----------
\begin{quote}
d) MCORE
\end{quote}
%%-----------
\begin{quote}
e) XGATE
\end{quote}
%%-----------
\begin{quote}
f) 6800 $\rightarrow$ 6801 $\rightarrow$ 6301 $\rightarrow$ 6811
\end{quote}
While the 6801 only offers a few additional instructions (and the
6301 even a few more), the 6811 provides a second index register and
much more instructions.
%%-----------
\begin{quote}
g) 6809/6309 and 6805/68HC(S)08
\end{quote}
These processors are partially source-code compatible to the other
68xx processors, but they have a different binary code format and a
significantly reduced (6805) resp. enhanced (6809) instruction set.
The 6309 is a CMOS version of the 6809 which is officially only
compatible to the 6809, but inofficially offers more registers and a
lot of new instructions (see \cite{Kaku}).
%%-----------
\begin{quote}
h) 68HC12 $\longrightarrow$ 68HC12X
\end{quote}
The 12X core offers a couple of new instructions, and existing
instructions were were enriched with new addressing modes.
%%-----------
\begin{quote}
i) S912ZVC19F0MKH, S912ZVC19F0MLF,\\
   S912ZVCA19F0MKH, S912ZVCA19F0MLF,\\
   S912ZVCA19F0WKH, S912ZVH128F2CLQ,\\
   S912ZVH128F2CLL, S912ZVH64F2CLQ,\\
   S912ZVHY64F1CLQ, S912ZVHY32F1CLQ,\\
   S912ZVHY64F1CLL, S912ZVHY32F1CLL,\\
   S912ZVHL64F1CLQ, S912ZVHL32F1CLQ,\\
   S912ZVHL64F1CLL, S912ZVHL32F1CLL,\\
   S912ZVFP64F1CLQ, S912ZVFP64F1CLL,\\
   S912ZVH128F2VLQ, S912ZVH128F2VLL,\\
   S912ZVH64F2VLQ, S912ZVHY64F1VLQ,\\
   S912ZVHY32F1VLQ, S912ZVHY64F1VL,\\
   S912ZVHY32F1VLL, S912ZVHL64F1VLQ 
\end{quote}
All variants contain the same processor core and the same
instruction set, only the on-chip peripherals and the
amount of built-in memory (RAM, Flash-ROM, EEPROM)
vary from device to device.
%%-----------
\begin{quote}
j) 68HC16
\end{quote}
%%-----------
\begin{quote}
k) HD6413308 $\rightarrow$ HD6413309
\end{quote}
These both names represent the 300 and 300H variants of the H8
family; the H version owns a larger address space (16Mbytes instead
of 64Kbytes), double-width registers (32 bits), and knows a few more
instructions and addressing modes.  It is still binary upward
compatible.
%%-----------
\begin{quote}
l) HD6475328 $\rightarrow$ HD6475348 $\rightarrow$ HD6475368 $\rightarrow$ HD6475388
\end{quote}
These processors all share the same CPU core; the different types are
only needed to include the correct subset of registers in the file
\tty{REG53X.INC}.
%%-----------
\begin{quote}
m) SH7000 $\rightarrow$ SH7600 $\longrightarrow$ SH7700
\end{quote}
The processor core of the 7600 offers a few more instructions that
close gaps in the 7000's instruction set (delayed conditional and
relative and indirect jumps, multiplications with 32-bit operands and
multiply/add instructions).  The 7700 series (also known as SH3)
furthermore offers a second register bank, better shift instructions, and
instructions to control the cache.
%%-----------
\begin{quote}
n)HD614023 $\longrightarrow$ HD614043 $\longrightarrow$ HD614081
\end{quote}
These three variants of the HMCS400 series differ by the size of
the internal ROM and RAM.
%%-----------
\begin{quote}
o) HD641016
\end{quote}
This is currently the only target with H16 core.
%%-----------
\begin{quote}
p) 6502 $\rightarrow$ 65(S)C02 \\
   $\rightarrow$ 65CE02 / W65C02S / 65C19 / MELPS740 / HUC6280 / 6502UNDOC
\end{quote}
The CMOS version defines some additional instructions, as well as a number of
some instruction/addressing mode combinations were added which were not
possible on the 6502.  The W65C02S adds two opcodes to the 65C02 instruction
set to give more fine-grained control over how to stop the CPU for low power
modes.  The 65SC02 lacks the bit manipulation instructions of the
65C02.  The 65CE02 adds branch instructions with 16-bit displacement, a Z
register, a 16 bit stack pointer, a programmable base page, and a couple of
new instructions.

The 65C19 is {\em not} binary upward compatible to the original
6502! Some addressing modes have been replaced by others.
Furthermore, this processor contains instruction set extensions
that facilitate digital signal processing.

The Mitsubishi micro controllers in opposite expand
the 6502 instruction set primarily to bit operations and multiplication /
division instructions.  Except for the unconditional jump and instructions
to increment/decrement the accumulator, the instruction extensions
have nothing in common.

For the HuC 6280, the feature that sticks out most is the larger
address space of 2 MByte instead of 64 KBytes.  This is achieved
with a buil-tin banking mechanism.  Furthermore, it features some
special instructions to communicate with a video processor (this
chip was used in video games) and to copy memory areas.

The 6502UNDOC processor type enables access to the "undocumented"
6502 instructions, i.e. the operations that result from the usage of bit
combinations in the opcode that are not defined as instructions.  The
variants supported by AS are listed in the appendix containing processor-specific
hints.
%%-----------
\begin{quote}
q) MELPS7700, 65816
\end{quote}
Apart from a '16-bit-version' of the 6502's instruction set, these
processors both offer some instruction set extensions.  These are
however orthogonal as they are oriented along their 8-bit
predecessors (65C02 resp. MELPS-740).  Partially, different
mnemonics are used for the same operations.
%%-----------
\begin{quote}
r) MELPS4500
\end{quote}
%%-----------
\begin{quote}
s) M16
\end{quote}
%%-----------
\begin{quote}
t) M16C
\end{quote}
%%-----------
\begin{quote}
u) 4004 $\rightarrow$ 4040
\end{quote}
Opposed to its predecessor, the 4040 features about a dozen additional
machine instructions.
%%-----------
\begin{quote}
v) 8008 $\rightarrow$ 8008NEW
Intel redefined the mnemonics around 1975, the second variant reflects
this new instruction set.  A simultaneous support of both sets was not
possible due to mnemonic conflicts.
\end{quote}
%%-----------
\begin{quote}
w) 8021, 8022, \\
   8401, 8411, 8421, 8461, \\
   8039, (MSM)80C39, 8048, (MSM)80C48, 8041, 8042, 80C382
\end{quote}
For the ROM-less versions 8039 and 80C39, the commands which are
using the BUS (port 0) are forbidden.  The 8021 and 8022 are special
versions with a strongly shrinked instruction set, for which the 8022
has two A/D- converters and the necessary control-commands.  The 
instruction set of the MAB8401 to 8461 (designed by Philips) is
somewhere in between the 8021/8022 and a ''complete'' MC-48 instruction
set.  On the other hand, they provide serial ports and up to 8 KBytes
of program memory.

It is possible to transfer the CMOS-versions with the \tty{IDL} resp.
\tty{HALT} command into a stop mode with lower current consumption.
The 8041 and 8042 have some additional instructions for controlling the
bus interface, but in turn a few other commands were omitted.
The code address space of 8041, 8042, 84x1, 8021, and 8022 is not externally
extendable, and so AS limits the code segment of these processors to
the size of the internal ROM.  The (SAB)80C382 is a variant especially
designed by Siemens for usage in telephones.  It also knows a
\tty{HALT} instruction, plus ist supports indirect addressing for
\tty{DJNZ} and \tty{DEC}.  In turn, several instructions of the
'generic' 8048 were left out.  The OKI variants (MSM...) also
feature indirect addressing for \tty{DJNZ} and \tty{DEC}, plus
enhanced control of power-down modes, plus the full basic MCS-48
instrucion set.
%%-----------
\begin{quote}
\begin{tabbing}
\hspace{0.7cm} \= \kill
x) \> 87C750 $\rightarrow$ 8051, 8052, 80C320, 80C501, 80C502, \\
   \> 80C504, 80515, and 80517 \\
   \> $\rightarrow$ 80C390 \\
   \> $\rightarrow$ 80C251
\end{tabbing}
\end{quote}
The 87C750 can only access a maximum of 2 Kbytes program memory which is
why it lacks the \tty{LCALL} and \tty{LJMP} instructions.  AS does not
make any distinction among the processors in the middle, instead it only
stores the different names in the \tty{MOMCPU} variable (see below), which
allows to query the setting with \tty{IF} instructions.  An exception is
the 80C504 that has a mask flaw in its current versions.  This flaw shows
up when an \tty{AJMP} or \tty{ACALL} instruction starts at the second last
address of a 2K page.  AS will automatically use long instructions or
issues an error message in such situations.  The 80C251 in contrast
represents a drastic progress in the the direction 16/32 bits, larger
address spaces, and a more orthogonal instruction set.  One might call the
80C390 the 'small solution': Dallas Semiconductor modified instruction set
and architecture only as far as it was necessary for the 16 Mbytes large
address spaces.
%%-----------
\begin{quote}
y) 8096 $\rightarrow$ 80196 $\rightarrow$ 80196N $\rightarrow$ 80296
\end{quote}
Apart from a different set of SFRs (which however strongly vary from
version to version), the 80196 knows several new instructions and
supports a 'windowing' mechanism to access the larger internal RAM.
The 80196N family extends the address space to 16 Mbytes and
introduces a set of instructions to access addresses beyond 64Kbytes.
The 80296 extends the CPU core by instructions for signal processing
and a second windowing register, however removes the Peripheral
Transaction Server (PTS) and therefore looses again two machine
instructions.
%%-----------
\begin{quote}
z) 8080 $\rightarrow$ 8085 $\rightarrow$ 8085UNDOC
\end{quote}
The 8085 knows the additional commands \tty{RIM} and \tty{SIM} for
controlling the interrupt mask and the two I/O-pins.  The type {\tt
8085UNDOC} enables additional instructions that are not documented
by Intel.  These instructions are documented in section \ref{8085Spec}.
%%-----------
\begin{quote}
aa) 8086 $\rightarrow$ 80186 $\rightarrow$ V30 $\rightarrow$ V35
\end{quote}
Only new instructions are added in this family.  The corresponding
8-bit versions are not mentioned due to their instruction
compatibility, so one e.g. has to choose 8086 for an 8088-based
system.
%%-----------
\begin{quote}
ab) 80960
\end{quote}
%%-----------
\begin{quote}
ac) 8X300 $\rightarrow$ 8X305
\end{quote}
The 8X305 features a couple of additional registers that miss on the
8X300.  Additionally, it can do new operations with these registers
(like direct writing of 8 bit values to peripheral addresses).
%%-----------
\begin{quote}
ad) XAG1, XAG2, XAG3
\end{quote}
These processors only differ in the size of their internal ROM which
is defined in \tty{STDDEFXA.INC}.
%%-----------
\begin{quote}
ae) AT90S1200, AT90S2313, AT90S2323, AT90S233, AT90S2343,\\
    AT90S4414, AT90S4433, AT90S4434, AT90S8515,\\
    AT90C8534, AT90S8535, ATTINY4, ATTINY5, ATTINY9,\\
    ATTINY10, ATTINY11, ATTINY12, ATTINY13, ATTINY13A,\\
    ATTINY15, ATTINY20, ATTINY24(A), ATTINY25,\\
    ATTINY26, ATTINY28, ATTINY40, ATTINY44(A),\\
    ATTINY45, ATTINY48, ATTINY84(A), ATTINY85,\\
    ATTINY87, ATTINY88, ATTINY102, ATTINY104,\\
    ATTINY167, ATTINY261, ATTINY261A, ATTINY43U,\\
    ATTINY441, ATTINY461, ATTINY461A, ATTINY828,\\
    ATTINY841, ATTINY861, ATTINY861A, ATTINY1634,\\
    ATTINY2313, ATTINY2313A, ATTINY4313, ATMEGA48,\\
    ATMEGA8, ATMEGA8515, ATMEGA8535, ATMEGA88,\\
    ATMEGA8U2, ATMEGA16U2, ATMEGA32U2,\\
    ATMEGA16U4, ATMEGA32U4, ATMEGA32U6, AT90USB646,\\
    AT90USB647, AT90USB1286, AT90USB1287, AT43USB355,\\
    ATMEGA16, ATMEGA161, ATMEGA162, ATMEGA163,\\
    ATMEGA164, ATMEGA165, ATMEGA168, ATMEGA169,\\
    ATMEGA32, ATMEGA323, ATMEGA324, ATMEGA325,\\
    ATMEGA3250, ATMEGA328, ATMEGA329, ATMEGA3290,\\
    ATMEGA406, ATMEGA64, ATMEGA640, ATMEGA644,\\
    ATMEGA644RFR2, ATMEGA645, ATMEGA6450,\\
    ATMEGA649, ATMEGA6490, ATMEGA103, ATMEGA128,\\
    ATMEGA1280, ATMEGA1281, ATMEGA1284,\\
    ATMEGA1284RFR2, ATMEGA2560, ATMEGA2561
\end{quote}
The various AVR chip variants mainly differ in the amount of
on-chip memory (flash, SRAM, EEPROM) an the set of built-in
peripherals (GPIO, timers, UART, A/D converter...).  Compared to
the AT90... predecessors, the ATmega chip also provide additional
instructions, while the ATtinys do not support the multiplication
instructions.
%%-----------
\begin{quote}
af) AM29245 $\rightarrow$ AM29243 $\rightarrow$ AM29240 $\rightarrow$ AM29000
\end{quote}
The further one moves to the right in this list, the fewer the
instructions become that have to be emulated in software.  While e.g.
the 29245 not even owns a hardware multiplier, the two representors in
the middle only lack the floating point instructions.  The 29000
serves as a 'generic' type that understands all instructions in
hardware.
%%-----------
\begin{quote}
ag) 80C166 $\rightarrow$ 80C167,80C165,80C163
\end{quote}
80C167 and 80C165/163 have an address space of 16 Mbytes instead of 256
Kbytes, and furthermore they know some additional instructions for
extended addressing modes and atomic instruction sequences.  They are
'second generation' processors and differ from each other only in the
amount of on-chip peripherals.
%%-----------
\begin{quote}
ah) LR35902/GBZ80 $\rightarrow$ Z80 $\rightarrow$ Z80UNDOC \\
    $\rightarrow$ Z180 $\rightarrow$ Z380
\end{quote}
While there are only a few additional instructions for the Z180, the
Z380 owns 32-bit registers, a linear address space of 4 Gbytes, a
couple of instruction set extensions that make the overall
instruction set considerably more orthogonal, and new addressing
modes (referring to index register halves, stack relative).  These
extensions partially already exist on the Z80 as undocumented
extensions and may be switched on via the Z80UNDOC variant.  A list
with the additional instructions can be found in the chapter
with processor specific hints.

The processor built into the Gameboy (official designation LR35092,
commonly referred to as ''Gameboy Z80'') is a mixture of an 8080
and Z80.  It lacks the IX/IY registers, the I/O address space,
the second register bank and a couple of 16 bit instructions.
%%-----------
\begin{quote}
ai) Z8601, Z8603, z86C03, z86E03, Z86C06, Z86E06, \\
    Z86C08, Z86C21, Z86E21, Z86C30, Z86C31, Z86C32 Z86C40 \\
    $\rightarrow$ Z88C00, Z88C01 \\
    $\rightarrow$ eZ8, Z8F0113, Z8F011A, Z8F0123, Z8F012A, \\
    Z8F0130, Z8F0131, Z8F0213, Z8F021A, Z8F0223, Z8F022A, \\
    Z8F0230, Z8F0231, Z8F0411, Z8F0412, Z8F0413, Z8F041A, \\
    Z8F0421, Z8F0422, Z8F0423, Z8F042A, Z8F0430, Z8F0431, \\
    Z8F0811, Z8F0812, Z8F0813, Z8F081A, Z8F0821, Z8F0822, \\
    Z8F0823, Z8F082A, Z8F0830, Z8F0831, Z8F0880, Z8F1232, \\
    Z8F1233, Z8F1621, Z8F1622, Z8F1680, Z8F1681, Z8F1682, \\
    Z8F2421, Z8F2422, Z8F2480, Z8F3221, Z8F3222, Z8F3281, \\
    Z8F3282, Z8F4821, Z8F4822, Z8F4823, Z8F6081, Z8F6082, \\
    Z8F6421, Z8F6422, Z8F6423, Z8F6481, Z8F6482
\end{quote}
The variants with Z8 core only differ in internal memory size and
on-chip peripherals, i.e. the choice does not have an effect on the
supported instruction set.  Super8 and eZ8 are substantially different,
each with an instruction set that was vastly extended (into different
directions), and they are not fully upward-compatible on source code
level as well.
%%-----------
\begin{quote}
aj) Z8001, Z8002, Z8003, Z8004
\end{quote}
The operation mode (segmented for Z8001 and Z8003, non-segmented for
Z8002 and Z8004) is selected via the processor type.  There is currently
no further differentiation between Z8001/8002 and Z8003/8004.
%%-----------
\begin{quote}
ak) KCPSM
\end{quote}
Both processor cores are not available as standalone components, they are
provided as logic cores for gate arrays made by Xilinx The -3 variant
offers a larger address space and some additional instructions.  Note that
it is not binary upward-compatible!
%%-----------
\begin{quote}
al) MICO8\_05, MICO8\_V3, MICO8\_V31
\end{quote}
Lattice unfortunately changed the machine instructions more than once, so
different targets became necessary to provide continued support for older
projects.  The first variant is the one described in the 2005 manual, the
two other ones represent versions 3.0 resp. 3.1.
%%-----------
\begin{quote}
am) 96C141, 93C141
\end{quote}
These two processors represent the two variations of the processor
family: TLCS-900 and TLCS-900L.  The differences of these two variations
will be discussed in detail in section \ref{TLCS900Spec}.
%%-----------
\begin{quote}
an) 90C141
\end{quote}
%%-----------
\begin{quote}
ao) 87C00, 87C20, 87C40, 87C70
\end{quote}
The processors of the TLCS-870 series have an identical CPU core, but
different peripherals depending on the type.  In part registers with
the same name are located at different addresses.  The file
\tty{STDDEF87.INC} uses, similar to the MCS-51-family, the distinction
possible by different types to provide the correct symbol set
automatically.
%%-----------
ap) TLCS-870/C
Currently, only the processor core of the TLCS-870/C family is
implemented.
%%-----------
\begin{quote}
aq) 47C00 $\rightarrow$ 470C00 $\rightarrow$ 470AC00
\end{quote}
These three variations of the TLCS-47-family have on-chip RAM and ROM
of different size, which leads to several bank switching instructions
being added or suppressed.
%%-----------
\begin{quote}
ar) 97C241
\end{quote}
%%-----------
\begin{quote}
as) TC9331
\end{quote}
%%-----------
\begin{quote}
at) 16C54 $\rightarrow$ 16C55 $\rightarrow$ 16C56 $\rightarrow$ 16C57
\end{quote}
These processors differ by the available code area, i.e. by the address
limit after which AS reports overruns.
%%-----------
\begin{quote}
au) 16C84, 16C64
\end{quote}
Analog to the MCS-51 family, no distinction is made in the code generator,
the different numbers only serve to include the correct SFRs in
\tty{STDDEF18.INC}.
%%-----------
\begin{quote}
av) 17C42
\end{quote}
%%-----------
\begin{quote}
aw) SX20, SX28
\end{quote}
The SX20 uses a smaller housing and lacks port C.
%%-----------
\begin{quote}
ax) ST6200, ST6201, ST6203, ST6208, ST6209,\\
    ST6210, ST6215, ST6218, ST6220, ST6225,\\
    ST6228, ST6230, ST6232, ST6235, ST6240,\\
    ST6242, ST6245, ST6246, ST6252, ST6253,\\
    ST6255, ST6260, ST6262, ST6263, ST6265,\\
    ST6280, ST6285
\end{quote}
The various ST6 derivates differ in the amount of
on-chip peripherals and built-in memory.
%%-----------
\begin{quote}
ay) ST7 \\
    ST72251G1, ST72251G2, ST72311J2, ST72311J4, \\
    ST72321BR6, ST72321BR7, ST72321BR9, ST72325S4, \\
    ST72325S6, ST72325J7, ST72325R9, ST72324J6, \\
    ST72324K6, ST72324J4, ST72324K4, ST72324J2, \\
    ST72324JK21, ST72325S4, ST72325J7, ST72325R9, \\
    ST72521BR6, ST72521BM9, ST7232AK1, ST7232AK2, \\
    ST7232AJ1, ST7232AJ2, ST72361AR4, ST72361AR6, \\
    ST72361AR7, ST72361AR9, ST7FOXK1, ST7FOXK2, \\
    ST7LITES2Y0, ST7LITES5Y0, ST7LITE02Y0, \\
    ST7LITE05Y0, ST7LITE09Y0 \\
    ST7LITE10F1, ST7LITE15F1, ST7LITE19F1, \\
    ST7LITE10F0, ST7LITE15F0, ST7LITE15F1, \\
    ST7LITE19F0, ST7LITE19F1, \\
    ST7LITE20F2, ST7LITE25F2, ST7LITE29F2, \\
    ST7LITE30F2, ST7LITE35F2, ST7LITE39F2, \\
    ST7LITE49K2, \\
    ST7MC1K2, ST7MC1K4, ST7MC2N6, ST7MC2S4, \\
    ST7MC2S6, ST7MC2S7, ST7MC2S9, ST7MC2R6, \\
    ST7MC2R7, ST7MC2R9, ST7MC2M9, \\
    STM8 \\
    STM8S001J3, STM8S003F3, STM8S003K3, STM8S005C6,\\
    STM8S005K6, STM8S007C8, STM8S103F2, STM8S103F3,\\
    STM8S103K3, STM8S105C4, STM8S105C6, STM8S105K4,\\
    STM8S105K6, STM8S105S4, STM8S105S6, STM8S207MB,\\
    STM8S207M8, STM8S207RB, STM8S207R8, STM8S207R6,\\
    STM8S207CB, STM8S207C8, STM8S207C6, STM8S207SB,\\
    STM8S207S8, STM8S207S6, STM8S207K8, STM8S207K6,\\
    STM8S208MB, STM8S208RB, STM8S208R8, STM8S208R6,\\
    STM8S208CB, STM8S208C8, STM8S208C6, STM8S208SB,\\
    STM8S208S8, STM8S208S6, STM8S903K3, STM8S903F3,\\
    STM8L050J3, STM8L051F3, STM8L052C6, STM8L052R8,\\
    STM8L001J3, STM8L101F1, STM8L101F2, STM8L101G2,\\
    STM8L101F3, STM8L101G3, STM8L101K3, STM8L151C2,\\
    STM8L151K2, STM8L151G2, STM8L151F2, STM8L151C3,\\
    STM8L151K3, STM8L151G3, STM8L151F3, STM8L151C4,\\
    STM8L151C6, STM8L151K4, STM8L151K6, STM8L151G4,\\
    STM8L151G6, STM8L152C4, STM8L152C6, STM8L152K4,\\
    STM8L152K6, STM8L151R6, STM8L151C8, STM8L151M8,\\
    STM8L151R8, STM8L152R6, STM8L152C8, STM8L152K8,\\
    STM8L152M8, STM8L152R8, STM8L162M8, STM8L162R8,\\
    STM8AF6366, STM8AF6388, STM8AF6213, STM8AF6223,\\
    STM8AF6226, STM8AF6246, STM8AF6248, STM8AF6266,\\
    STM8AF6268, STM8AF6269, STM8AF6286, STM8AF6288,\\
    STM8AF6289, STM8AF628A, STM8AF62A6, STM8AF62A8,\\
    STM8AF62A9, STM8AF62AA, STM8AF5268, STM8AF5269,\\
    STM8AF5286, STM8AF5288, STM8AF5289, STM8AF528A,\\
    STM8AF52A6, STM8AF52A8, STM8AF52A9, STM8AF52AA,\\
    STM8AL3136, STM8AL3138, STM8AL3146, STM8AL3148,\\
    STM8AL3166, STM8AL3168, STM8AL3L46, STM8AL3L48,\\
    STM8AL3L66, STM8AL3L68, STM8AL3188, STM8AL3189,\\
    STM8AL318A, STM8AL3L88, STM8AL3L89, STM8AL3L8A,\\
    STM8TL52F4, STM8TL52G4, STM8TL53C4, STM8TL53F4,\\
    STM8TL53G4
\end{quote}
The STM8 core extends the address space to 16 Mbytes and introduces
a couple of new instructions.  Though many instructions have the same
machine code as for ST7, it is not binary upward compatible.
%%-----------
\begin{quote}
az) ST9020, ST9030, ST9040, ST9050
\end{quote}
These 4 names represent the four ''sub-families'' of the ST9 family,
which only differ in their on-chip peripherals.  Their processor
cores are identical, which is why this distinction is again only used
in the include file containing the peripheral addresses.
%%-----------
\begin{quote}
ba) 6804
\end{quote}
%%-----------
\begin{quote}
bb) 32010$\rightarrow$32015
\end{quote}
The TMS32010 owns just 144 bytes of internal RAM, and so AS limits
addresses in the data segment just up to this amount.  This restriction
does not apply for the 32015, the full range from 0..255 can be used.
%%-----------
\begin{quote}
bc) 320C25 $\rightarrow$ 320C26 $\rightarrow$ 320C28
\end{quote}
These processors only differ slightly in their on-chip peripherals
and in their configuration instructions.
%%-----------
\begin{quote}
bd) 320C30, 320C31 $\rightarrow$ 320C40, 320C44
\end{quote}
The 320C31 is a reduced version with the same instruction set,
however fewer peripherals.  The distinction is exploited in
\tty{STDDEF3X.INC}.  The C4x variants are sourcecode upward
compatible, the machine codes of some instructions are however
slightly different.  Once again, the C44 is a stripped-down
version of the C40, with less peripherals and a smaller address
space.
%%-----------
\begin{quote}
be) 320C203 $\rightarrow$ 320C50, 320C51, 320C53
\end{quote}
The first one represents the C20x family of signal processors which
implement a subset of the C5x instruction set.  The distinction among the
C5x processors is currently not used by AS.
%%-----------
\begin{quote}
bf) 320C541
\end{quote}
This one at the moment represents the TMS320C54x family...
%%-----------
\begin{quote}
bg) TI990/4, TI990/10, TI990/12 \\
    TMS9900, TMS9940, TMS9995, TMS99105, TMS99110
\end{quote}
The TMS99xx/99xxx processors are basically single chip implementations
of the TI990 minicomputers.  Some TI990 models are even based on such
a processor instead of a discrete CPU.  The individual models differ in their
instruction set (the TI990/12 has the largest one) and the presence of a
privileged mode.
%%-----------
\begin{quote}
\begin{tabbing}
\hspace{0.7cm} \= \kill
bh) \> TMS70C00, TMS70C20, TMS70C40,\\
    \> TMS70CT20, TMS70CT40,\\
    \> TMS70C02, TMS70C42, TMS70C82,\\
    \> TMS70C08, TMS70C48\\
\end{tabbing}
\end{quote}
All members of this family share the same CPU core, they therefore do not
differ in their instruction set.  The differences manifest only in the
file \tty{REG7000.INC} where address ranges and peripheral addresses are
defined.  Types listed in the same row have the same amount of internal
RAM and the same on-chip peripherals, they differ only in the amount of
integrated ROM.
%%-----------
\begin{quote}
bi) 370C010, 370C020, 370C030, 370C040 and 370C050
\end{quote}
Similar to the MCS-51 family, the different types are only used to
differentiate the peripheral equipment in \tty{STDDEF37.INC}; the
instruction set is always the same.
%%-----------
\begin{quote}
bj) MSP430 $\rightarrow$ MSP430X
The X variant of the CPU core extends the address space from 64
KiBytes to 1 MiByte and augments the instruction set, e.g. by
prefixed to repeat instructions.
\end{quote}
%%-----------
\begin{quote}
bk) TMS1000, TMS1100, TMS1200, TMS1300
\end{quote}
TMS1000 and TMS1200 each provide 1 KByte of ROM and 64 nibbles of
RAM, while TMS1100 and TMS1300 provide twice the amount of RAM
and ROM.  Furthermore, TI has defined a significantly different
default instruction set fot TMS1100 and TMS1300(AS only knows the
default instruction sets!)
%%-----------
\begin{quote}
bl) SC/MP
\end{quote}
%%-----------
\begin{quote}
bm) 8070
\end{quote}
This processor represents the whole 807x family (which consists at least
of the 8070, 8072, and 8073), which however shares identical CPU cores.
%%-----------
\begin{quote}
bn) COP87L84
\end{quote}
This is the only member of National Semiconductor's COP8 family that
is currently supported.  I know that the family is substantially
larger and that there are representors with differently large
instruction sets which will be added when a need occurs.  It is a
beginning, and National's documentation is quite extensive...
%%-----------
\begin{quote}
bo) COP410 $\rightarrow$ COP420 $\rightarrow$ COP440 $\rightarrow$ COP444
The COP42x derivates offer some additional instructions, plus other
instructions have an extended operand range.
\end{quote}
%%-----------
\begin{quote}
\begin{tabbing}
\hspace{0.7cm} \= \kill
bp) \> SC14400, SC14401, SC14402, SC14404, SC14405, \\
    \> SC14420, SC14421, SC14422, SC14424 \\
\end{tabbing}
\end{quote}
This series of DECT controllers differentiates itself by the amount of
instructions, since each of them supports different B field formats and
their architecture has been optimized over time.
%%-----------
\begin{quote}
bq) NS16008, NS32008, NS08032, NS16032, NS32016, NS32032, \\
    NS32332, NS32CG16, NS32532
\end{quote}
National renamed the first-generation CPUs several times in the early
years, NS16008/NS32008/NS08032 resp. NS16032/NS32016 are the same chips.
NS32332 and NS32532 support an address space of  4 GBytes instead of 16
MBytes, and the NS32CG16 is an embedded variant with additional instructions
for bit block transfers.
%%-----------
\begin{quote}
br) ACE1101, ACE1202
\end{quote}
%%-----------
\begin{quote}
bs) F3850, MK3850, \\
    MK3870, MK3870/10, MK3870/12, "MK3870/20, MK3870/22, \\
    MK3870/30, MK3870/32, MK3870/40, MK3870/42, \\
    MK3872, MK3873, MK3873/10, MK3873/12, MK3873/20, \\
    MK3873/22, MK3874, MK3875, MK3875/22, MK3875/42, \\
    MK3876, MK38P70/02, MK38C70, MK38C70/10, \\
    MK38C70/20, MK97400, MK97410, MK97500, MK97501, \\
    MK97503
\end{quote}
This huge amount of variants partially results from the fact that
Mostek renamed some variants in the early 80s.  The new naming scheme
allows to deduce the amount of internal ROM (0 to 4 for 0 to 4 Kbytes) 
and executable RAM (0 or 2 for 0 or 64 bytes) from the suffix.  3850
and MK975xx support a 64K address space, which is only 4 Kbytes for all
other variants.  P variants have an EEPROM piggyback socket for prototyping,
C variants are fabricated in CMOS technology and feature two new machine
instructions (HET and HAL). The MK3873's feature is a built-in serial port,
while the MK3875 offers a second supply voltage pin to buffer the internal
memory in standby mode.
%%-----------
\begin{quote}
bt) 7800, 7801, 7802 \\
78C05, 78C06 \\
7810$\rightarrow$78C10, 78C11, 78C12, 78C14, 78C17, 78C18
\end{quote}
$\mu$PD7800 to $\mu$PD7802 represent the ''first generation'' of the
uCOM87 family from NEC.  $\mu$PD78C05 and $\mu$PD78C06 are reduced
variants that implement only a subset of the instruction set. All $\mu$PD781x
variants belong to the uCOM87AD series, which - aside from an A/D converter -
also supports additional registers and machine instructions. \bb{NOTE:} The
instruction set is only partially binary upward compatible!
The NMOS version $\mu$PD7810 has no stop-mode; the respective command and the ZCM
register are omitted.  \bb{CAUTION!}  NMOS and CMOS version partially
differ in the reset values of some registers!
%%-----------
\begin{quote}
bu) 7500 $\leftrightarrow$ 7508
\end{quote}
There are two different types of CPU cores in the $\mu$PD75xx
family: the 7566 represents the the 'instruction set B', which
provides less instructions, less registers and smaller address
spaces.  The 7508 represents the 'full' instruction set A.  {\bf
CAUTION!} These instruction sets are not 100\% binary compatible!
%%-----------
\begin{quote}
\begin{tabbing}
\hspace{0.7cm} \= \kill
bv) \> 75402,\\
    \> 75004, 75006, 75008,\\
    \> 75268,\\
    \> 75304, 75306, 75308, 75312, 75316,\\
    \> 75328,\\
    \> 75104, 75106, 75108, 75112, 75116,\\
    \> 75206, 75208, 75212, 75216,\\
    \> 75512, 75516\\
\end{tabbing}
\end{quote}
This 'cornucopia' of processors differs only by the RAM size in one
group; the groups themselves again differ by their on-chip
peripherals on the one hand and by their instruction set's power on
the other hand.
%%-----------
\begin{quote}
bw) 78070
\end{quote}
This is currently the only member of NEC's 78K0 family I am familiar
with.  Similar remarks like for the COP8 family apply!
%%-----------
\begin{quote}
bx) 78214
\end{quote}
This is currently the representor of NEC's 78K2 family.
%%-----------
\begin{quote}
by) 78310
\end{quote}
This is currently the representor of NEC's 78K3 family.
%%-----------
\begin{quote}
bz) 784026
\end{quote}
This is currently the representor of NEC's 78K4 family.
%%-----------
\begin{quote}
ca) 7720 $\rightarrow$ 7725
\end{quote}
The $\mu$PD7725 offers larger address spaces and som more instructions
compared to his predecessor. {\bf CAUTION!}  The processors are not binary
compatible to each other!
%%-----------
\begin{quote}
cb) 77230
\end{quote}
%%-----------
\begin{quote}
\begin{tabbing}
cc) \= SYM53C810, SYM53C860, SYM53C815, SYM53C825, \\
    \> SYM53C875, SYM53C895
\end{tabbing}
\end{quote}
The simpler members of this family of SCSI processors lack some
instruction variants, furthermore they are different in their set of
internal registers.
%%-----------
\begin{quote}
cd) MB89190
\end{quote}
This processor type represents Fujitsu's F$^{2}$MC8L series...
%%-----------
\begin{quote}
ce) MB9500
\end{quote}
...just like this one does it currently for the 16-bit variants from
Fujitsu!
%%-----------
\begin{quote}
cf) MSM5840, MSM5842, MSM58421, MSM58422, MSM5847
\end{quote}
These variants of the OLMS-40 family differ in their instruction
set and in the amount of internal program and data memory.
%%-----------
\begin{quote}
cg) MSM5054, MSM5055, MSM5056, MSM6051, MSM6052
\end{quote}
The as for the OLMS-40 family: differences in instruction
set and the amount of internal program and data memory.
%%-----------
\begin{quote}
ch) MN1610[ALT] $\rightarrow$ MN1613[ALT]
\end{quote}
In addition to its predecessor's features, the MN1613 offers a larger
address space, a floating point unit and a couple of new machine
instructions.
%%-----------
\begin{quote}
ci) PMC150, PMS150, PFS154, PMC131, PMS130, PMS131 \\
PMS132, PMS132B, PMS152, PMS154B, PMS154C, PFS173 \\
PMS133, PMS134, DF69, MCS11, PMC232, PMC234, PMC251 \\
PMC271,PMC884, PMS232, PMS234, PMS271
\end{quote}
The Padauk controllers differ in the size of the internal
(ROM/RAM) memory, the type of internal ROM (erasable or OTP),
the built-in peripherals, and their instruction set (both
extent and binary coding).
%%-----------
\begin{quote}
cj) 1802 $\rightarrow$ 1804, 1805, 1806 $\rightarrow$ 1804A,
1805A, 1806A
\end{quote}
1804, 1805, and 1806 feature an instruction set that is slightly
enhanced, compared to the 'original' 1802, plus on-chip RAM and
an integrated timer.  The A variants extend the instruction set by
\tty{DSAV}, \tty{DBNZ}, and instructions for addition and
subtraction in BCD format.
%%-----------
\begin{quote}
ck) XS1
\end{quote}
This type represents the XCore-"family".
%%-----------
\begin{quote}
cl) 1750
\end{quote}
MIL STD 1750 is a standard, therefore there is only one
(standard) variant...
%%-----------
\begin{quote}
cm) KENBAK
\end{quote}
Since there has never been a KENBAK-2, the target is simply KENBAK...
%%-----------
\begin{quote}
cn) CP-1600
\end{quote}

The \tty{CPU} instruction needs the processor type as a simple literal, a
calculation like:
\begin{verbatim}
        CPU     68010+10
\end{verbatim}
is not allowed.  Valid calls are e.g.
\begin{verbatim}
        CPU     8051
\end{verbatim}
or
\begin{verbatim}
        CPU     6800
\end{verbatim}
Regardless of the processor type currently set, the integer variable
\tty{MOMCPU} contains the current status as a hexadecimal number.  For
example, \tty{MOMCPU}=\$68010 for the 68010 or \tty{MOMCPU}=80C48H for the
80C48.  As one cannot express all letters as hexadecimal digits (only A..F
are possible), all other letters must must be omitted in the hex notation;
for example, \tty{MOMCPU}=80H for the Z80.

You can take advantage of this feature to generate different code
depending on the processor type.  For example, the 68000 does not have a
machine instruction for a subroutine return with stack correction.  With
the variable \tty{MOMCPU} you can define a macro that uses the machine
instruction or emulates it depending on the processor type:
\begin{verbatim}
myrtd   macro   disp
        if      MOMCPU<$68010 ; emulate for 68008 & 68000
         move.l (sp),disp(sp)
         lea    disp(sp),sp
         rts
        elseif
         rtd    #disp         ; direct use on >=68010
        endif
        endm


        cpu     68010
        myrtd   12            ; results in RTD #12

        cpu     68000
        myrtd   12            ; results in MOVE../LEA../RTS
\end{verbatim}
As not all processor names are built only out of numbers and letters
from A..F, the full name is additionally stored in the string
variable named \tty{MOMCPUNAME}.

The assembler implicitly switches back to the \tty{CODE} segment when a
\tty{CPU} instruction is executed.  This is done because \tty{CODE} is the
only segment all processors support.

The default processor type is 68008, unless it has been changed via the
command line option with same name.

Some targets define options or variants that are so fundamental for
operation, that they have to be selected with the \tty{CPU} instruction.
Such options are appended to the argument, separated by double
colons:
\begin{verbatim}
  CPU <CPU Name>:<var1>=<val1>:<var2>=<val2>:...
\end{verbatim}
See the respective section with processor-specific hints to check whether a
certain target supports such options.

%%- - - - - - - - - - - - - - - - - - - - - - - - - - - - - - - - - - - - - -

\subsection{SUPMODE, FPU, PMMU, CUSTOM}
\ttindex{SUPMODE}\ttindex{FPU}\ttindex{PMMU}\ttindex{CUSTOM}
\label{SectSUPMODE}

{\em
SUPMODE valid for: 680x0, i960, TLCS-900, SH7000, i960, 29K, XA,\\
PowerPC, M*Core, and TMS9900 \\
FPU valid for: 680x0, NS32xxx, 80x86 \\
PMMU valid for: 680x0, NS32xxx \\
CUSTOM valid for: NS32xxx \\
}

These three switches allow to define which parts of the instruction set
shall be disabled because the necessary preconditions are not valid for
the following piece of code.  The parameter for these instructions may be
either \tty{ON} or \tty{OFF}, the current status can be read out of a
variable which is either TRUE or FALSE.

The commands have the following meanings in detail:
\begin{itemize}
\item{\tty{SUPMODE}: allows or prohibits commands, for whose execution the
      processor has to be within the supervisor mode.  The status
      variable is called \tty{INSUPMODE}.}
\item{\tty{FPU}: allows or prohibits the commands of the numerical
      coprocessors 8087, NS32081/32381 resp. 68881 or 68882.  The status
      variable is called \tty{FPUAVAIL}.  For NS32xxx as target, specifying
      the explicit FPU type (\tty{NS32081}, \tty{NS32181}, \tty{NS32381},
      or \tty{NS32580}) is also possible, to enable or disable the additional
      registers and instructions.}
\item{\tty{PMMU}: allows or prohibits the commands of the memory
      management unit 68851 resp. of  the built-in MMU of the 68030.
      \bb{CAUTION!} The 68030-MMU supports only a relatively small subset
      of the 68851 instructions.  This is controlled via the \tty{FULLPMMU}
      statement. The status variable is called \tty{PMMUAVAIL}. For NS32xxx
      as target, specifying the explicit MMU type as target (\tty{NS32082},
      \tty{NS32381}, or \tty{NS32352}) is also possible, to enable access to
      the MMU-type-specific register set.}
\item{\tty{CUSTOM}: allows or prohibits the commands reserved for custom
      slave processors.}
\end{itemize}
The usage of of instructions prohibited in this manner will generate a
warning at \tty{SUPMODE}, at \tty{PMMU} and \tty{FPU} a real error
message.

%%- - - - - - - - - - - - - - - - - - - - - - - - - - - - - - - - - - - - - -

\subsection{FULLPMMU}
\ttindex{FULLPMMU}

{\em valid for: 680x0}

Motorola integrated the MMU into the processor starting with the 68030, but
the built-in FPU is equipped only with a relatively small subset of the
68851 instruction set.  AS will therefore disable all extended MMU
instructions when the target processor is 68030 or higher.  It is however
possible that the internal MMU has been disabled in a 68030-based system
and the processor operates with an external 68851.  One can the use a
\tty{FULLPMMU ON} to tell AS that the complete MMU instruction set is
allowed.  Vice versa, one may use a \tty{FULLPMMU OFF} to disable all
additional instruction in spite of a 68020 target platform to assure that
portable code is written.  The switch between full and reduced instruction
set may be done as often as needed, and the current setting may be read
from a symbol with the same name.  \bb{CAUTION!} The \tty{CPU} instruction
implicitly sets or resets this switch when its argument is a 68xxx
processor!  \tty{FULLPMMU} therefore has to be written after the \tty{CPU}
instruction!

%%- - - - - - - - - - - - - - - - - - - - - - - - - - - - - - - - - - - - - -

\subsection{PADDING}
\ttindex{PADDING}

{\em valid for: 680x0, 68xx, M*Core, XA, H8, SH7000, MSP430(X), TMS9900,\\
      ST7/STM8, AVR (only if code segment granularity is 8 bits)}

Various processor families have a requirement that objects of more than
one byte length must be located on a n even address.  Aside from data
objects, this may also include instruction words.  For instance, word
accesses to an odd address result in an exception on a 68000, while other
processors like the H8 force the lowest address bit to zero.

The \tty{PADDING} instruction allows to activate a mechanism that tries to
avoid such misalignments.  If the situation arises that an instruction
word, or a data object of 16 bits or more (created e.g. via \tty{DC}) would
be stored on an odd address, a padding byte is automatically inserted before.
Such a padding byte is displayed in the listing in a separate line that
contains the remark
\begin{verbatim}
<padding>
\end{verbatim}
If the source line also contained a label, the label still points to the
address of the code or data object, i.e. right behind the pad byte.  The same
is true for a label in a source line immediately before, as long as this
line {\em} only holds the label and no other instruction.  So, in the
follwing example:
\begin{verbatim}
       padding  on
       org      $1000

       dc.b     1
adr1:  nop

       dc.b     1
adr2:
       nop

       dc.b     1
adr3:  equ      *
       nop
\end{verbatim}
the labels \tty{adr1} and \tty{adr2} hold the addresses of the respective
\tty{NOP} instructions, which were made even by inserting a pad byte.
\tty{adr3} in contrast holds the address of the pad byte preceding the
third \tty{NOP}.

Similar to the previous instructions, the argument to \tty{PADDING} may be
either \tty{ON} or \tty{OFF}, and the current setting may be read from a
symbol with the same name.  \tty{PADDING} is by default only enabled for
the 680x0 family, it has to be turned on explicitly for all other families.

%%- - - - - - - - - - - - - - - - - - - - - - - - - - - - - - - - - - - - - -

\subsection{PACKING}
\ttindex{PACKING}

{\em valid for: AVR}

In some way, {\tt PACKING} is similar to {\tt PADDING}, it just has a
somewhat opposite effect: While {\tt PADDING} extends the disposed data to
get full words and keep a possible alignment, {\tt PACKING} squeezes
several values into a single word.  This makes sense for the AVR's code
segment since the CPU has a special instruction ({\tt LPM}) to access
single bytes within a 16-bit word.  In case this option is turned on
(argument {\tt ON}), two byte values are packed into a single word by {\tt
DATA}, similar to the single characters of string arguments.  The value
range of course reduces to -128...+255.  If this option is turned off
(argument {\tt OFF}), each integer argument obtains its own word and may
take values from -32768...+65535.

This distinctin is only made for integer arguments of {\tt DATA}, strings
will always be packed..  Keep further in mind that packing of values only
works within the arguments of a {\tt DATA} statement; if one has
subsequent {\tt DATA} statements, there will still be half-filled words
when the argument count is odd!

%%- - - - - - - - - - - - - - - - - - - - - - - - - - - - - - - - - - - - - -

\subsection{MAXMODE}
\ttindex{MAXMODE}

{\em valid for: TLCS-900, H8}

The processors of the TLCS-900-family are able to work in 2 modes, the
minimum and maximum mode.  Depending on the actual mode, the execution
environment and the assembler are a little bit different. Along with this
instruction and the parameter \tty{ON} or \tty{OFF}, AS is informed that the
following code will run in maximum resp. minimum mode.  The actual setting
can be read from the variable \tty{INMAXMODE}.  Presetting is \tty{OFF},
i.e. minimum mode.

Similarly, one uses this instruction to tell AS in H8 mode whether the
address space is 64K or 16 Mbytes.  This setting is always \tty{OFF} for
the 'small' 300 version and cannot be changed.

%%- - - - - - - - - - - - - - - - - - - - - - - - - - - - - - - - - - - - - -

\subsection{EXTMODE and LWORDMODE}
\ttindex{EXTMODE}\ttindex{LWORDMODE}

{\em valid for: Z380}

The Z380 may operate in altogether 4 modes, which are the result of
setting two flags: The XM flag rules whether the processor shall operate
wit an address space of 64 Kbytes or 4 Gbytes and it may only be set to 1
(after a reset, it is set to 0 for compatibility with the Z80).  The LW
flag in turn rules whether word operations shall work with a word size of
16 or 32 bits.  The setting of these two flags influences range checks of
constants and addresses, which is why one has to tell AS the setting of
these two flags via these instructions.  The default assumption is that
both flags are 0, the current setting (\tty{ON} or \tty{OFF}) may be read
from the predefined symbols \tty{INEXTMODE} resp. \tty{INLWORDMODE.}

%%- - - - - - - - - - - - - - - - - - - - - - - - - - - - - - - - - - - - - -

\subsection{SRCMODE}
\ttindex{SRCMDE}

{\em valid for: MCS-251}

Intel substantially extended the 8051 instruction set with the 80C251, but
unfortunately there was only a single free opcode for all these new
instructions.  To avoid a processor that will be eternally crippled by a
prefix, Intel provided two operating modes: the binary and the source
mode.  The new processor is fully binary compatible to the 8051 in binary
mode, all new instructions require the free opcode as prefix.  In source
mode, the new instructions exchange their places in the code tables with
the corresponding 8051 instructions, which in turn then need a prefix.
One has to inform AS whether the processor operates in source mode
(\tty{ON}) or binary mode (\tty{OFF}) to enable AS to add prefixes when
required.  The current setting may be read from the variable
\tty{INSRCMODE}.  The default is \tty{OFF}.

%%- - - - - - - - - - - - - - - - - - - - - - - - - - - - - - - - - - - - - -

\subsection{BIGENDIAN}
\ttindex{BIGENDIAN}

{\em valid for: MCS-51/251, PowerPC, SC/MP}

Intel broke with its own principles when the 8051 series was designed: in
contrast to all traditions, the processor uses big-endian ordering for all
multi-byte values!  While this was not a big deal for MCS-51 processors
(the processor could access memory only in 8-bit portions, so everyone was
free to use whichever endianess one wanted), it may be a problem for the
251 as it can fetch whole (long-)words from memory and expects the MSB to
be first.  As this is not the way of constant disposal earlier versions of
AS used, one can use this instruction to toggle between big and
little endian mode for the instructions \tty{DB, DW, DD, DQ,} and
\tty{DT}.  \tty{BIGENDIAN OFF} (the default) puts the LSB first into
memory as it used to be on earlier versions of AS, \tty{BIGENDIAN ON}
engages the big-endian mode compatible to the MCS-251.  One may of course
change this setting as often as one wants; the current setting can be read
from the symbol with the same name.

%%- - - - - - - - - - - - - - - - - - - - - - - - - - - - - - - - - - - - - -

\subsection{WRAPMODE}
\ttindex{WRAPMODE}

{\em valid for: Atmel AVR}

After this switch has been set to {\tt ON}, AS will assume that the
processor's program counter does not have the full length of 16 bits given
by the architecture, but instead a length that is exactly sufficient to
address the internal ROM.  For example, in case of the AT90S8515, this
means 12 bits, corresponding to 4 Kwords or 8 Kbytes.  This assumption
allows relative branches from the ROM's beginning to the end and vice
versa which would result in an out-of-branch error when using strict
arithmetics.  Here, they work because the carry bits resulting from the
target address computation are discarded.  Assure that the target
processor you are using works in the outlined way before you enable this
option!  In case of the abovementioned AT90S8515, this option is even
necessary because it is the only way to perform a direct jump through
the complete address space...

This switch is set to {\tt OFF} by default, and its current setting may be
read from a symbol with same name.

%%- - - - - - - - - - - - - - - - - - - - - - - - - - - - - - - - - - - - - -

\subsection{SEGMENT}
\ttindex{SEGMENT}
\label{SEGMENT}

{\em valid for: all processors}

Some microcontrollers and signal processors know various address ranges,
which do not overlap with each other and require also different
instructions and addressing modes for access.  To manage these ones also,
the assembler provides various program counters, you can switch among
them to and from by the use of the \tty{SEGMENT} instruction.  For subroutines
included with \tty{INCLUDE}, this e.g. allows to define data used by the
main program or subroutines near to the place they are used.  In detail,
the following segments with the following names are supported:
\begin{itemize}
\item{\tty{CODE}: program code;}
\item{\tty{DATA}: directly addressable data (including SFRs);}
\item{\tty{XDATA}: data in externally connected RAM or
      X-addressing space of the DSP56xxx or ROM data for the $\mu$PD772x;}
\item{\tty{YDATA}: Y-addressing space of the DSP56xxx;}
\item{\tty{IDATA}: indirectly addressable (internal) data; }
\item{\tty{BITDATA}: the part of the 8051-internal RAM that is bitwise
      addressable;}
\item{\tty{IO}: I/O-address range;}
\item{\tty{REG}: register bank of the ST9;}
\item{\tty{ROMDATA}: constant ROM of the NEC signal processors;}
\item{\tty{EEDATA}: built-in EEPROM.}
\end{itemize}
See also section \ref{SectORG} (\tty{ORG}) for detailed information about
address ranges and initial values of the segments. Depending on the
processor family, not all segment types will be permitted.

The bit segment is managed as if it would be a byte segment, i.e. the
addresses will be incremented by 1 per bit.

Labels get the same type as attribute as the segment that was active
when the label was defined.  So the assembler has a limited ability
to check whether you access symbols of a certain segment with wrong
instructions.  In such cases the assembler issues a warning.

Example:
\begin{verbatim}
        CPU     8051    ; MCS-51-code

        segment code    ; test code

        setb    flag    ; no warning
        setb    var     ; warning : wrong segment

        segment data

var     db      ?

        segment bitdata

flag    db      ?
\end{verbatim}

%%- - - - - - - - - - - - - - - - - - - - - - - - - - - - - - - - - - - - - -

\subsection{PHASE and DEPHASE}
\ttindex{PHASE}\ttindex{DEPHASE}

{\em valid for: all processors}

For some applications (especially on Z80 systems), the code must be moved
to another address range before execution.  If the assembler didn't know
about this, it would align all labels to the load address (not the start
address).  The programmer is then forced to write jumps within this area
either independent of location or has to add the offset at each symbol
manually.  The first one is not possible for some processors, the last one
is extremely error-prone.  With the commands \tty{PHASE} and
\tty{DEPHASE}, it is possible to inform the assembler at which address the
code will really be executed on the target system:
\begin{verbatim}
        phase   <address>
\end{verbatim}
informs the assembler that the following code shall be executed at the
specified address.  The assembler calculates thereupon the difference to
the real program counter and adds this difference for the following
operations:
\begin{itemize}
\item{address values in the listing}
\item{filing of label values}
\item{program counter references in relative jumps and address expressions}
\item{readout of the program counter via the symbols * or \$}
\end{itemize}
By using the instruction
\begin{verbatim}
        DEPHASE
\end{verbatim}
, this ''shifting'' is reverted to the value previous to the most
recent \tty{PHASE} instruction.  \tty{PHASE} und \tty{DEPHASE} may be used
in a nested manner.
\par
The assembler keeps phase values for all defined segments, although
this instruction pair only makes real sense in the code segment.

%%- - - - - - - - - - - - - - - - - - - - - - - - - - - - - - - - - - - - - -

\subsection{SAVE and RESTORE}
\ttindex{SAVE}\ttindex{RESTORE}

{\em valid for: all processors}

The command \tty{SAVE} forces the assembler to push the contents of
following variables onto an internal stack:
\begin{itemize}
\item{currently selected processor type (set by \tty{CPU});}
\item{currently active memory area (set by \tty{SEGMENT});}
\item{the flag whether listing is switched on or off (set by \tty{LISTING});}
\item{the flags that define which part of expanded macros shall be
      printed in the assembly listing (set by
      \tty{/MACEXP\_DFT/MACEXP\_OVR}).}
\item{currently active character translation table (set by
      \tty{CODEPAGE}).}
\end{itemize}
The counterpart \tty{RESTORE} pops the values saved last from this stack.
These two commands were primarily designed for include files, to change
the above mentioned variables in any way inside of these files, without
loosing their original content.  This may be helpful e.g. in include files
with own, fully debugged subroutines, to switch the listing generation
off:
\begin{verbatim}
        SAVE            ; save old status

        LISTING OFF     ; save paper

        .               ; the actual code
        .

        RESTORE         ; restore
\end{verbatim}
In opposite to a simple \tty{LISTING OFF .. ON}-pair, the correct status
will be restored, in case the listing generation was switched off already
before.

The assembler checks if the number of \tty{SAVE}-and
\tty{RESTORE}-commands corresponds and issues error messages in the
following cases:
\begin{itemize}
\item{\tty{RESTORE}, but the internal stack is empty;}
\item{the stack not empty at the end of a pass.}
\end{itemize}

%%- - - - - - - - - - - - - - - - - - - - - - - - - - - - - - - - - - - - - -

\subsection{ASSUME}
\ttindex{ASSUME}

{\em valid for: various}

This instruction allows to tell AS the current setting of certain
registers whose contents cannot be described with a simple \tty{ON} or
\tty{OFF}.  These are typically registers that influence addressing modes
and whose contents are important to know for AS in order to generate
correct addressing.  It is important to note that \tty{ASSUME} only
informs AS about these, \bb{no} machine code is generated that actually
loads these values into the appropriate registers!

A value defined with \tty{ASSUME} can be queried or integrated
into expressions via the built-in function \tty{ASSUMEDVAL}.
This is the case for all architectures listed in the following
sub-sections except for the 8086.

%%. . . . . . . . . . . . . . . . . . . . . . . . . . . . . . . . . . . . . .

\subsubsection{65CE02}

The 65CE02 features a a register named 'B' that is used to set the 'base page'.
In comparison to the original 6502, this allows the programmer to place the 
memory page addressable with short (8 bit) addresses anywhere in the 64K address
space.  This register is set to zero after a reset, so the 65CE02 behaves like
its predecessor.  A base page at zero is also the default assumption of the
assembler.  It may be informed about its actual contents via a \tty{ASSUME B:xx}
statement.  Addresses located in this page will then automatically be addressed
via short addressing modes.

%%. . . . . . . . . . . . . . . . . . . . . . . . . . . . . . . . . . . . . .

\subsubsection{6809}

In contrast to its 'predecessors' like the 6800 and 6502, the position of
the direct page, i.e. the page of memory that can be reached with
single-byte addresses, can be set freely.  This is done via the 'direct
page register' that sets the page number.  One has to assign a
corresponding value to this register via \tty{ASSUME} is the contents are
different from the default of 0, otherwise wrong addresses will be
generated!

%%. . . . . . . . . . . . . . . . . . . . . . . . . . . . . . . . . . . . . .

\subsubsection{68HC11K4}

Also for the HC11, the designers finally weren't able to avoid the major
sin: using a banking scheme to address more than 64 Kbytes with only 16
address lines.  The registers {\tt MMSIZ}, {\tt MMWBR}, {\tt MM1CR}, and
{\tt MM2CR} control whether and how the additional 512K address ranges are
mapped into the physical address space.  AS intially assumes the reset
state of these registers, i.e. all are set to \$00 and windowing is
disabled.

%%. . . . . . . . . . . . . . . . . . . . . . . . . . . . . . . . . . . . . .

\subsubsection{68HC12X}

Similar to its cousin without the appended 'X', the HC12X supports a short
direct addressing mode.  In this case however, it can be used to address
more than just the first 256 bytes of the address space.  The {\tt DIRECT}
register specifices which 256 byte page of the address space is addressed
by this addressing mode.  {\tt ASSUME} is used to tell AS the current
value of this register, so it is able to automatically select the most
efficient address ing mode when absolute addresses are used.  The default
is 0, which corresponds to the reset state.

%%. . . . . . . . . . . . . . . . . . . . . . . . . . . . . . . . . . . . . .

\subsubsection{68HC16}

The 68HC16 employs a set of bank registers to address a space of 1
Mbyte with its registers that are only 16 bits wide.  These registers
supply the upper 4 bits.  Of these, the EK register is responsible
for absolute data accesses (not jumps!).  AS checks for each absolute
address whether the upper 4 bits of the address are equal to the
value of EK specified via \tty{ASSUME}.  AS issues a warning if they
differ.  The default for EK is 0.

%%. . . . . . . . . . . . . . . . . . . . . . . . . . . . . . . . . . . . . .

\subsubsection{H8/500}

In maximum mode, the extended address space of these processors is
addressed via a couple of bank registers.  They carry the names DP
(registers from 0..3, absolute addresses), EP (register 4 and 5), and TP
(stack).  AS needs the current value of DP to check if absolute addresses
are within the currently addressable bank; the other two registers are
only used for indirect addressing and can therefore not be monitored; it
is a question of personal taste whether one specifies their values or not.
The BR register is in contrast important because it rules which 256-byte
page may be accessed with short addresses.  It is common for all registers
that AS does not assume \bb{any} default value for them as they are
undefined after a CPU reset.  Everyone who wants to use absolute addresses
must therefore assign values to at least DR and DP!

%%. . . . . . . . . . . . . . . . . . . . . . . . . . . . . . . . . . . . . .

\subsubsection{MELPS740}

Microcontrollers of this series know a ''special page'' addressing mode
for the \tty{JSR} instruction that allows a shorter coding for jumps into
the last page of on-chip ROM.  The size of this ROM depends of course
on the exact processor type, and there are more derivatives than it
would be meaningful to offer via the CPU instruction...we therefore
have to rely on \tty{ASSUME} to define the address of this page, e.g.
\begin{verbatim}
        ASSUME  SP:$1f
\end{verbatim}
in case the internal ROM is 8K.

%%. . . . . . . . . . . . . . . . . . . . . . . . . . . . . . . . . . . . . .

\subsubsection{MELPS7700/65816}

These processors contain a lot of registers whose contents AS has to know
in order to generate correct machine code.  These are the registers
in question:
\begin{center}\begin{tabular}{|l|l|l|l|}
\hline
name   & function             & value range   & default \\
\hline
\hline
DT/DBR & data bank            & 0-\$ff        &  0 \\
PG/PBR & code Bank            & 0-\$ff        &  0 \\
DPR    & directly addr. page  & 0-\$ffff      &  0 \\
X      & index register width & 0 or 1        &  0 \\
M      & accumulator width    & 0 or 1        &  0 \\
\hline
\end{tabular}\end{center}
\par
To avoid endless repetitions, see section \ref{MELPS7700Spec} for
instructions how to use these registers.  The handling is otherwise
similar to the 8086, i.e. multiple values may be set with one instruction
and no code is generated that actually loads the registers with the given
values.  This is again up to the programmer!

%%. . . . . . . . . . . . . . . . . . . . . . . . . . . . . . . . . . . . . .

\subsubsection{MCS-196/296}

Starting with the 80196, all processors of the MCS-96 family have a
register 'WSR' that allows to map memory areas from the extended
internal RAM or the SFR range into areas of the register file which
may then be accessed with short addresses.  If one informs AS about
the value of the WSR register, it can automatically find out whether
an absolute address can be addressed with a single-byte address via
windowing; consequently, long addresses will be automatically generated
for registers covered by windowing.  The 80296 contains an additional
register WSR1 to allow simultaneous mapping of two memory areas into
the register file.  In case it is possible to address a memory cell
via both areas, AS will always choose the way via WSR!

For indirect addressing, displacements may be either short (8 bits,
-128 to +127) or long (16 bits).  The assembler will automatically
use the shortest possible encoding for a given displacement.  It
is however possible to enforce a 16-bit coding by prefixing the
displacement argument with a bigger sign ((\verb!>!).  Similarly,
absolute addresses in the area from 0ff80h to 0ffffh may be reached
via a short offset relative to the "null register".

%%. . . . . . . . . . . . . . . . . . . . . . . . . . . . . . . . . . . . . .

\subsubsection{8086}

The 8086 is able to address data from all segments in all
instructions, but it however needs so-called ''segment prefixes'' if
another segment register than DS shall be used.  In addition it is
possible that the DS register is adjusted to another segment, e.g. to
address data in the code segment for longer parts of the program.  As
AS cannot analyze the code's meaning, it has to informed via this
instruction to what segments the segment registers point at the
moment, e.g.:
\begin{verbatim}
        ASSUME  CS:CODE, DS:DATA    .
\end{verbatim}
It is possible to assign assumptions to all four segment registers in
this way.  This instruction produces \bb{no} code, so the program itself
has to do the actual load of the registers with the values.

The usage of this instruction has on the one hand the result that AS is
able to automatically put ahead prefixes at sporadic accesses into the
code segment, or on the other hand, one can inform AS that the DS-register
was modified and you can save explicit \tty{CS:}-instructions.

Valid arguments behind the colon are \tty{CODE}, \tty{DATA} and
\tty{NOTHING}.  The latter value informs AS that a segment register
contains no usable value (for AS).  The following values are
preinitialized:
\begin{verbatim}
  CS:CODE, DS:DATA, ES:NOTHING, SS:NOTHING
\end{verbatim}

%%. . . . . . . . . . . . . . . . . . . . . . . . . . . . . . . . . . . . . .

\subsubsection{XA}

The XA family has a data address space of 16 Mbytes, a process however
can always address within a 64K segment only that is given by the DS
register.  One has to inform AS about the current value of this
register in order to enable it to check accesses to absolute
addresses.

%%. . . . . . . . . . . . . . . . . . . . . . . . . . . . . . . . . . . . . .

\subsubsection{29K}

The processors of the 29K family feature a register RBP that allows
to protect banks of 16 registers against access from user mode.  The
corresponding bit has to be set to achieve the protection.  \tty{ASSUME}
allows to tell AS which value RBP currently contains.  AS can warn
this way in case a try to access protected registers from user mode
is made.

%%. . . . . . . . . . . . . . . . . . . . . . . . . . . . . . . . . . . . . .

\subsubsection{80C166/167}

Though none of the 80C166/167's registers is longer than sixteen bits,
this processor has 18/24 address lines and can therefore address up
to 256Kbytes/16Mbytes.  To resolve this contradiction, it neither
uses the well-known (and ill-famed) Intel method of segmentation nor
does it have inflexible bank registers...no, it uses paging!  To accomplish
this, the logical address space of 64 Kbytes is split into 4 pages of
16 Kbytes, and for each page there is a page register (named
DPP0..DPP3) that rules which of the 16/1024 physical pages shall be
mapped to this logical page.  AS always tries to present the address
space with a size of 256Kbytes/16MBytes in the sight of the
programmer, i.e. the physical page is taken for absolute accesses and
the setting of bits 14/15 of the logical address is deduced.  If no
page register fits, a warning is issued.  AS assumes by default that
the four registers linearly map the first 64 Kbytes of memory, in the
following style:
\begin{verbatim}
        ASSUME  DPP0:0,DPP1:1,DPP2:2,DPP3:3
\end{verbatim}
The 80C167 knows some additional instructions that can override the
page registers' function.  The chapter with processor-specific hints
describes how these instructions influence the address generation.
\par
Some machine instructions have a shortened form that can be used if
the argument is within a certain range:
\begin{itemize}
\item{\verb!MOV Rn,#<0..15>!}
\item{\verb!ADD/ADDC/SUB/SUBC/CMP/XOR/AND/OR Rn, #<0..7>!}
\item{\verb!LOOP Rn,#<0..15>!}
\end{itemize}
The assembler automatically uses to the shorter coding if possible.
If one wants to enforce the longer coding, one may place a 'bigger'
character right before the expression (behind the double cross character!).
Vice versa, a 'smaller' character can be used to assure the shorter
coding is used.  In case the operand does not fulfill the range
restrictions for the shorter coding, an error is generated.  This syntax
may also be used for branches and calls which may either have a short
displacement or a long absolute argument.

%%. . . . . . . . . . . . . . . . . . . . . . . . . . . . . . . . . . . . . .

\subsubsection{TLCS-47}

The direct data address space of these processors (it makes no
difference whether you address directly or via the HL register) has a
size of only 256 nibbles.  Because the ''better'' family members have
up to 1024 nibbles of RAM on chip, Toshiba was forced to introduce a
banking mechanism via the DMB register.  AS manages the data segment
as a continuous addressing space and checks at any direct addressing
if the address is in the currently active bank.  The bank AS
currently expects can be set by means of
\begin{verbatim}
        ASSUME  DMB:<0..3>
\end{verbatim}
The default value is 0.

%%. . . . . . . . . . . . . . . . . . . . . . . . . . . . . . . . . . . . . .

\subsubsection{ST6}
\label{ST6Assume}

The microcontrollers of the ST62 family are able to map a part (64 bytes)
of the code area into the data area, e.g. to load constants from the ROM.
This means also that at one moment only one part of the ROM can be
addressed.  A special register rules which part it is.  AS cannot check
the contents of this register directly, but it can be informed by this
instruction that a new value has been assigned to the register.  AS then
can test and warn if necessary, in case addresses of the code segment are
accessed, which are not located in the ''announced'' window.  If, for
example, the variable \tty{VARI} has the value 456h, so
\begin{verbatim}
        ASSUME  ROMBASE:VARI>>6
\end{verbatim}
sets the AS-internal variable to 11h, and an access to \tty{VARI}
generates an access to address 56h in the data segment.

It is possible to assign a simple \tty{NOTHING} instead of a value, e.g.
if the bank register is used temporarily as a memory cell.  This value is
also the default.

The program counter of these controller only has a width of 12 bits.  This
means that some sort of banking scheme had to be introduced if a device
includes more than 4 KBytes of program memory.  The banking scheme splits
both proram space and program memory in pages of 2 KBytes.  Page one of the
program space always accesses page one of program memory.  The \tty{PRPR}
register present on such devices selects which page of program memory is
accessed via addresses 000h to 7ffh of program space.  As an initial
approcimation, AS regards program space to be linear and of the size of
program memory.  If a jump or call from page one is made to code in one
of the other pages, it checks whether the assumed contents of the \tty{PRPR}
register match the destination address.  If a jump or call is done from
one of the other pages to an address outside of page one, it checks whether
the destination address is within the same page.  {\bf IMPORTANT}: The
program counter itself is only 12 bits wide.  It is therefore not possible
to jump from one page to another one, without an intermediate step of
jumping back to page one.  Changing the \tty{PRPR} register while operating
outside of page one would result in ''pulling out'' the code from under
one's feet.

%%. . . . . . . . . . . . . . . . . . . . . . . . . . . . . . . . . . . . . .

\subsubsection{ST9}

The ST9 family uses exactly the same instructions to address code and
data area.  It depends on the setting of the flag register's DP flag
which address space is referenced.  To enable AS to check if one
works with symbols from the correct address space (this of course
\bb{only} works with absolute accesses!), one has to inform AS whether the
DP flag is currently 0 (code) or 1 (data).  The initial value of this
assumption is 0.

%%. . . . . . . . . . . . . . . . . . . . . . . . . . . . . . . . . . . . . .

\subsubsection{78K2}

78K2 is an 8/16 bit architecture, which has later been extended to a
one-megabyte addres space via banking.  Banking is realized with the
registers PM6 (normal case) resp. P6 (alternate case with \verb!&! as
prefix) that supply the missing upper four address bits.  At least for
absolute addresses, AS can check whether the current, linear 20-bit
address is within the given 64K window.

%%. . . . . . . . . . . . . . . . . . . . . . . . . . . . . . . . . . . . . .

\subsubsection{78K3}

Processors witrh a 78K3 core have register banks that consist of
16 registers.  These registers may be used via their numbers
(\tty{R0} to \tty{R15}) or their symbolic names (\tty{X=R0, A=R1,
C=R2, B=R3, VPL=R8, VPH=R9, UPL=R10, UPH=R11, E=R12,
D=R13, L=R14, H=R15}).  The processor core has a register select
bit (\tty{RSS}) to switch the mapping of A/X and B/C from R0..R3
to R4..R7.  This is mainly important for instructions that
implicitly use one of these registers (i.e. instruction that do
not encode the register number in the machine code).  However, it
is also possible to inform the assembler about the changed
mapping via a

\begin{verbatim}
  assume rss:1
\end{verbatim}

The assmebler will then insert the alternate register numbers
into machine instructions that explicitly encode the register
numbers.  Vice versa, \tty{R5} will be treated like \tty{A} instead
of \tty{R1} in the source code.

%%. . . . . . . . . . . . . . . . . . . . . . . . . . . . . . . . . . . . . .

\subsubsection{78K4}

78K4 was designed as an 'upgrade path' from 78K3, which is why
this processor core contains the same RSS bit to control the
mapping of registers AX and BC (though NEC discourages use of it
in new code).

Aside from many new instructins and addressing modes, the most
significant extension is the larger address space of 16 MBytes,
of which only the first MByte may be used for program code.  The
CPU-internal RAM and all special function registers may be
positioned either at the top of the first MByte or the top of the
first 64 KByte page.  Choice is made via the \tty{LOCATION}
machine instruction that either takes a 0 or 15 as argument. 
Together with remapping RAM and SFRs, the processor also switches
the address ranges that may be reached with short (8 bit)
addresses.  Parallel to using \tty{LOCATION}, one has to inform
the assembler about this setting via a \tty{ASSUME LOCATION:..}
statement.  It will then use short addressing for the proper
ranges.  The assembler will assume a default of 0 for LOCATION.

%%. . . . . . . . . . . . . . . . . . . . . . . . . . . . . . . . . . . . . .

\subsubsection{320C3x/C4x}

As all instruction words of this processor family are only 32 bits
long (of which only 16 bits were reserved for absolute addresses),
the missing upper 8/16 bits have to be added from the DP register.  It
is however still possible to specify a full 24/32-bit address when
addressing, AS will check then whether the upper 8 bits are equal to
the DP register's assumed values.  \tty{ASSUME} is different to the
\tty{LDP} instruction in the sense that one cannot specify an arbitrary
address out of the bank in question, one has to extract the upper bits by
hand:
\begin{verbatim}
        ldp     @addr
        assume  dp:addr>>16
        .
        .
        ldi     @addr,r2
\end{verbatim}

%%. . . . . . . . . . . . . . . . . . . . . . . . . . . . . . . . . . . . . .

\subsubsection{uPD78(C)10}

These processors have a register (V) that allows to move the ''zero
page'', i.e. page of memory that is addressable by just one byte,
freely in the address space, within page limits.  By reasons of
comforts you don't want to work with expressions such as
\begin{verbatim}
        inrw    Lo(counter)
\end{verbatim}
so AS takes over this job, but only under the premise that it is informed
via the \tty{ASSUME}-command about the contents of the V register.  If an
instruction with short addressing is used, it will be checked if the upper
half of the address expression corresponds to the expected content.  A
warning will be issued if both do not match.

%%. . . . . . . . . . . . . . . . . . . . . . . . . . . . . . . . . . . . . .

\subsubsection{75K0}

As the whole address space of 12 bits could not be addressed even by
the help of register pairs (8 bits), NEC had to introduce banking
(like many others too...): the upper 4 address bits are fetched from
the MBS register (which can be assigned values from 0 to 15 by the
\tty{ASSUME} instruction), which however will only be regarded if the MBE
flag has been set to 1.  If it is 0 (default), the lowest and highest
128 nibbles of the address space can be reached without banking.  The
\tty{ASSUME} instruction is undefined for the 75402 as it contains neither
a MBE flag nor an MBS register; the initial values cannot be changed
therefore.

%%. . . . . . . . . . . . . . . . . . . . . . . . . . . . . . . . . . . . . .

\subsubsection{F$^2$MC16L}

Similar to many other families of microcontrollers, this family suffers
somewhat from its designers miserliness: registers of only 16 bits width
are faced with an address space of 24 bits.  Once again, bank registers
had to fill the gap.  In detail, these are PCB for the progam code, DTB
for all data accesses, ADB for indirect accesses via RW2/RW6, and SSB/USB
for the stacks.  They may all take values from 0 to 255 and are by default
assumed to be 0, with the exception of 0ffh for PCB.

Furthermore, a DPR register exists that specifies which memory page within
the 64K bank given by DTB may be reached with 8 bit addresses.  The
default for DPR is 1, resulting in a default page of 0001xxh when one
takes DTB's default into account.

%%. . . . . . . . . . . . . . . . . . . . . . . . . . . . . . . . . . . . . .

\subsubsection{MN1613}

The MN1613 is an extension of an architecture with 16 bit addresses.  The
address extension is done by a set of "segment registers" (CSBR, SSBR, TSR0, and
TSR1), each of which is four bits wide.  The contents of a segment register,
left-shifted by 14 bits, is added to the 16 bit addresses.  This way, a process may
access a memory window of 64 KWords within the address space of 256 KWords. The
assembler uses segment register values reported via \tty{ASSUME} to warn whether
an absolute address is outside the window defined by the used segment register.
If the address is within the window, it will compute the correc t16-bit offset.
Naturally, this cannot be done when indirect addressing is used.

%%- - - - - - - - - - - - - - - - - - - - - - - - - - - - - - - - - - - - - -

\subsection{CKPT}
\ttindex{CKPT}

{\em valid for: TI990/12}

Type 12 instructions require a {\em checkpoint register} for execution.  This
register may either be specified explicitly as fourth argument, or a default for
all following code may be given via this instruction.  If neither a \tty{CKPT}
instruction nor an explicit checkpoint register was used, an error
is reported.  The default of no default register may be restored by using
{\tt NOTHING} as argument to \tty{CKPT}.

%%- - - - - - - - - - - - - - - - - - - - - - - - - - - - - - - - - - - - - -

\subsection{EMULATED}

{\em valid for: 29K}

AMD defined the 29000's series exception handling for undefined
instructions in a way that there is a separate exception vector for
each instruction.  This allows to extend the instruction set of a
smaller member of this family by a software emulation.  To avoid that
AS quarrels about these instructions as being undefined, the
\tty{EMULATED} instruction allows to tell AS that certain instructions are
allowed in this case.  The check if the currently set processors knows the
instruction is then skipped.  For example, if one has written a module
that supports 32-bit IEEE numbers and the processor does not have a FPU,
one writes
\begin{verbatim}
        EMULATED FADD,FSUB,FMUL,FDIV
        EMULATED FEQ,FGE,FGT,SQRT,CLASS
\end{verbatim}

%%- - - - - - - - - - - - - - - - - - - - - - - - - - - - - - - - -

\subsection{BRANCHEXT}
\ttindex{BRANCHEXT}

{\em valid for: XA}

{\tt BRANCHEXT} with either \tty{ON} or \tty{OFF} as argument tells AS
whether short branches that are only available with an 8-bit displacement
shall automatically be 'extended', for example by replacing a single
instruction like
\begin{verbatim}
        bne     target
\end{verbatim}
with a longer sequence of same functionality, in case the branc target is
out of reach for the instruction's displacement.  For example, the
replacement sequence for {\tt bne} would be
\begin{verbatim}
        beq     skip
        jmp     target
skip:
\end{verbatim}
In case there is no fitting 'opposite' for an instruction, the sequence
may become even longer, e.g. for {\tt jbc}:
\begin{verbatim}
        jbc     dobr
        bra     skip
dobr:   jmp     target
skip:
\end{verbatim}
This feature however has the side effect that there is no unambigious
assignment between machine and assembly code any more.  Furthermore,
additional passes may be the result if there are forward branches.  One
should therefore use this feature with caution!

%%- - - - - - - - - - - - - - - - - - - - - - - - - - - - - - - - -

\subsection{Z80SYNTAX}
\ttindex{Z80SYNTAX}

{\em G"ultigkeit: 8008, 8080/8085}

With \tty{ON} as argument, one can optionally write (almost) all
8008/8080 instructions in the form Zilog defined them for the Z80.
For instance, you simply use \tty{LD} with self-explaining
operands instead of \tty{MVI, LXI, MOV, STA, LDA, SHLD, LHLD,
LDAX, STAX} or \tty{SPHL}.

Since some mnemonics have a different meaning in 8008/8080 and Z80
syntax, it is not possible to program in 'Z80 style' all the
time, unless the '8080 syntax' is turned off entirely by using
 \tty{EXCLUSIVE} as argument.  The details of this operation mode
can be looked up in section \ref{8080Spec}.

A built-in symbol of same name allows to query the operation mode.
The mapping is \tty{0=OFF}, \tty{1=ON}, and \tty{2=EXCLUSIVE}.

%%- - - - - - - - - - - - - - - - - - - - - - - - - - - - - - - - -

\subsection{EXPECT and ENDEXPECT}
\ttindex{EXPECT}
\ttindex{ENDEXPECT}

This pair of instructions may be used to frame a piece of code that
is {\em expected} to trigger one or more error or warning messages.
If the errors or warnings (identified by their numbers, see chapter
\ref{ChapErrMess}) do occur, they are suppressed and assembly continues
without any error (naturally, without creating code at the erroneous
places).  However, if warnings or errors that were expected do not
occur, \tty{ENDEXPECT} will emit errors about them.  The main usage
scenario of these instructions are the self tests in the tests/
subdirectory.  For instance, one may check this way if range
checking of operands works as expected:
\begin{verbatim}
       cpu      68000
       expect   1320     ; immediate shift only for 1..8
       lsl.l    #10,d0
       endexpect
\end{verbatim}


%%---------------------------------------------------------------------------

\section{Data Definitions}

The instructions described in this section partially overlap in their
functionality, but each processor family defines other names for the
same function.  To stay compatible with the standard assemblers, this
way of implementation was chosen.

If not explicitly mentioned otherwise, all instructions for data
deposition (not those for reservation of memory!) allow an arbitrary
number of parameters which are being processed from left to right.

%%- - - - - - - - - - - - - - - - - - - - - - - - - - - - - - - - - - - - - -

\subsection{DC[.Size]}
\ttindex{DC}

{\em valid for: 680x0, M*Core, 68xx, H8, SH7x00, DSP56xxx, XA, ST7/STM8, MN161x}

This instruction places one or several constants of the type
specified by the attribute into memory.  The attributes are the same ones as
defined in section \ref{AttrTypes}, and there is additionally the
possibility for byte constants to place string constants in memory, like
\begin{verbatim}
String  dc.B "Hello world!\0"
\end{verbatim}
The parameter count may be between 1 and 20.  A repeat count enclosed
in brackets may additionally be prefixed to each parameter; for
example, one can for example fill the area up to the next page
boundary with zeroes with a statement like
\begin{verbatim}
        dc.b    [(*+255)&$ffffff00-*]0
\end{verbatim}
\bb{CAUTION!}  This function easily allows to reach the limit of 1 Kbyte
of generated code per line!

The assembler can automatically add another byte of data in case the byte sum
should become odd, to keep the word alignment.  This behaviour may be
turned on and off via the \tty{PADDING} instruction.

Decimal floating point numbers stored with this instruction (\tty{DC.P...})
can cover the whole range of extended precision, one however has to
pay attention to the detail that the coprocessors currently available
from Motorola (68881/68882) ignore the thousands digit of the
exponent at the read of such constants!

The default attribute is \tty{W}, that means 16-bit-integer numbers.

For the DSP56xxx, the data type is fixed to integer numbers (an attribute is
therefore neither necessary nor allowed), which may be in the range
of -8M up to 16M-1.  String constants are also allowed, whereby three characters
are packed into each word.

Opposed to the standar Motorola ssembler, it is also valid to reserve
memory space with this statement, by using a question mark as operand.
This is an extension added by some third-party suppliers for 68K
assemblers, similar to what Intel assemblers provide.  However, it should
be clear that usage of this feature may lead to portability problems.
Furthermore, question marks as operands must not be mixed with 'normal'
constants in a single statement.

%%- - - - - - - - - - - - - - - - - - - - - - - - - - - - - - - - - - - - - -

\subsection{DS[.Size]}
\ttindex{DS}

{\em valid for: 680x0, M*Core, 68xx, H8, SH7x00, DSP56xxx, XA, ST7/STM8, MN161x}

On the one hand, this instruction enables to reserve memory space for
the specified count of numbers of the type given by the attribute.
Therefore,
\begin{verbatim}
        DS.B    20
\end{verbatim}
for example reserves 20 bytes of memory, but
\begin{verbatim}
        DS.X    20
\end{verbatim}
reserves 240 bytes!

The other purpose is the alignment of the program counter which is
achieved by a count specification of 0.  In this way, with a
\begin{verbatim}
        DS.W    0  ,
\end{verbatim}
the program counter will be rounded up to the next even address, with
a
\begin{verbatim}
        DS.D 0
\end{verbatim}
in contrast to the next double word boundary.  Memory cells possibly
staying unused thereby are neither zeroed nor filled with NOPs, they
simply stay undefined.

The default for the operand length is - as usual - \tty{W}, i.e. 16 bits.

For the 56xxx, the operand length is fixed to words (of 24 bit),
attributes therefore do not exist just as in the case of \tty{DC}.

%%- - - - - - - - - - - - - - - - - - - - - - - - - - - - - - - - - - - - - -

\subsection{DN,DB,DW,DD,DQ, and DT}
\ttindex{DN}\ttindex{DB}\ttindex{DW}\ttindex{DD}\ttindex{DQ}\ttindex{DT}

{\em\begin{tabbing}
valid for: \= Intel (except for 4004/4040), Zilog, Toshiba,\\
           \> NEC, TMS370, Siemens, AMD, MELPS7700/65816,\\
           \> M16(C), National, ST9, Atmel, TMS70Cxx, TMS1000,\\
           \> Signetics, $\mu$PD77230, Fairchild, Intersil,\\
           \> XS1
\end{tabbing}}

These commands are - one could say - the Intel counterpart to \tty{DS} and
\tty{DC}, and as expected, their logic is a little bit different: First,
the specification of the operand length is moved into the mnemonic:
\begin{itemize}
\item{\tty{DN}: 4-bit integer}
\item{\tty{DB}: byte or ASCII string similar to \tty{DC.B}}
\item{\tty{DW}: 16-bit integer or half precision}
\item{\tty{DD}: 32-bit integer or single precision}
\item{\tty{DQ}: double precision   (64 bits)}
\item{\tty{DT}: extended precision (80 bits)}
\end{itemize}
Second, the distinction between constant definition and memory
reservation is done by the operand.  A reservation of memory is
marked by a \tty{?} :
\begin{verbatim}
        db      ?       ; reserves a byte
        dw      ?,?     ; reserves memory for 2 words (=4 byte)
        dd      -1      ; places the constant -1 (FFFFFFFFH) !
\end{verbatim}
Reserved memory and constant definitions \bb{must not} be mixed within one
instruction:
\begin{verbatim}
        db      "hello",?       ; --> error message
\end{verbatim}
Additionally, the \tty{DUP} Operator permits the repeated placing of
constant sequences or the reservation of whole memory blocks:
\begin{verbatim}
        db      3 dup (1,2)     ; --> 1 2 1 2 1 2
        dw      20 dup (?)      ; reserves 40 bytes of memory
\end{verbatim}
As you can see, the \tty{DUP}-argument must be enclosed in parentheses,
which is also why it may consist of several components, that may
themselves be \tty{DUP}s...the stuff therefore works recursively.

\tty{DUP} is however also a place where one can get in touch with another
limit of the assembler: a maximum of 1024 bytes of code or data may be
generated in one line.  This is not valid for the reservation of memory,
only for the definition of constant arrays!

The \tty{DUP} operator only gets recognized if it is itself not enclosed
in parentheses, and if there is a non-empty argument to its left.  This
way, it is possible to use a symbol of same name as argument.

In order to be compatible to the M80, \tty{DEFB/DEFW} may be used instead of
\tty{DB/DW} in Z80-mode.

Similarly, \tty{BYTE/ADDR} resp. \tty{WORD/ADDRW} in COP4/8 mode are an
alias for \tty{DB} resp. \tty{DW}, with the pairs differing in byte order:
instructions defined by National for address storage use big endian,
\tty{BYTE} resp. \tty{WORD} in contrast use little endian.

If \tty{DB} is used in an address space that is not byte addressable (like
the Atmel AVR's \tty{CODE} segment), bytes are packed in pairs into 16 bit
words, according to the endianess given by the architecture: for little
endian, the LSB is filled first.  If the total number of bytes is odd, one
half of the last word remains unused, just like the argument list had been
padded.  It will also not be used if another \tty{DB} immediately follows
in source code.  The analogous is true for \tty{DN}, just with the difference
that two or four nibbles are packet into a byte or 16 bit word.

The NEC 77230 is special with its \tty{DW} instruction: It more works like
the \tty{DATA} statement of its smaller brothers, but apart from string
and integer arguments, it also accepts floating point values (and stores
them in the processor's proprietary 32-bit format). There is {\em no}
\tty{DUP} operator!

%%- - - - - - - - - - - - - - - - - - - - - - - - - - - - - - - - - - - - - -

\subsection{DS, DS8}
\ttindex{DS}
\ttindex{DS8}

{\em\begin{tabbing}
valid for: \= Intel, Zilog, Toshiba, NEC, TMS370, Siemens, AMD, \\
           \> M16(C), National, ST9, TMS7000, TMS1000, Intersil
\end{tabbing}}

With this instruction, you can reserve a memory area:
\begin{verbatim}
        DS      <count>
\end{verbatim}
It is an abbreviation of
\begin{verbatim}
        DB      <count> DUP (?)
\end{verbatim}
Although this could easily be made by a macro, some people grown up
with Motorola CPUs (Hi Michael!) suggest \tty{DS} to be a built-in
instruction...I hope they are satisfied now \tty{;-)}

{\tt DS8} is defined as an alias for {\tt DS} on the National SC14xxx.
Beware that the code memory of these processors is organized in words of
16 bits, it is therefore impossible to reserve individual bytes.  In case
the argument of {\tt DS} is odd, it will be rounded up to the next even
number.

%%- - - - - - - - - - - - - - - - - - - - - - - - - - - - - - - - - - - - - -

\subsection{BYT or FCB}
\ttindex{BYT}\ttindex{FCB}

{\em valid for: 6502, 68xx}

By this instruction, byte constants or ASCII strings are placed in
65xx/68xx-mode, it therefore corresponds to \tty{DC.B} on the 68000 or
\tty{DB} on Intel.  Similarly to \tty{DC}, a repetition factor enclosed
in brackets ([..]) may be prepended to every single parameter.

%%- - - - - - - - - - - - - - - - - - - - - - - - - - - - - - - - - - - - - -

\subsection{BYTE}
\ttindex{BYTE}

{\em valid for: ST6, 320C2(0)x, 320C5x, MSP, TMS9900, CP-1600}

Ditto.  Note that when in 320C2(0)x/5x mode, the assembler assumes that
a label on the left side of this instruction has no type, i.e. it
belongs to no address space. This behaviour is explained in the
processor-specific hints.

The \tty{PADDING} instruction allows to set whether odd counts of bytes
shall be padded with a zero byte in MSP/TMS9900 mode.

The operation of {\tt BYTE} on CP-1600 is somewhat different: the 16-bit
integer arguments are stored byte-wise in two consecutive words of memory
(LSB first).  If individual 8-bit values shall be stored in memor (optionally
packed), use the {\tt TEXT} instruction!

%%- - - - - - - - - - - - - - - - - - - - - - - - - - - - - - - - - - - - - -

\subsection{DC8}
\ttindex{DC8}

{\em valid for: SC144xx}

This statement is an alias for {\tt DB}, i.e. it may be used to dump byte
constants or strings to memory.

%%- - - - - - - - - - - - - - - - - - - - - - - - - - - - - - - - - - - - - -

\subsection{ADR or FDB}
\ttindex{ADR}\ttindex{FDB}

{\em valid for: 6502, 68xx}

\tty{ADR} resp. \tty{FDB} stores word constants when in 65xx/68xx mode.
It is therefore the equivalent to \tty{DC.W} on the 68000 or \tty{DW} on
Intel platforms.  Similarly to \tty{DC}, a repetition factor enclosed
in brackets ([..]) may be prepended to every single parameter.

%%- - - - - - - - - - - - - - - - - - - - - - - - - - - - - - - - - - - - - -

\subsection{WORD}
\ttindex{WORD}

{\em valid for: ST6, i960, 320C2(0)x, 320C3x/C4x/C5x, MSP, CP-1600}

If assembling for the 320C3x/C4x or i960, this command stores 32-bit words,
16-bit words for the other families.  Note that when in 320C2(0)x/5x mode,
the assembler assumes that a label on the left side of this instruction
has no type, i.e. it belongs to no address space.  This behaviour is
explained at the discussion on processor-specific hints.

%%- - - - - - - - - - - - - - - - - - - - - - - - - - - - - - - - - - - - - -

\subsection{DW16}
\ttindex{DW16}

{\em valid for: SC144xx}

This instruction is for SC144xx targets a way to dump word (16 bit)
constants to memory. {\tt CAUTION!!}  It is therefore an alias for {\tt
DW}.

%%- - - - - - - - - - - - - - - - - - - - - - - - - - - - - - - - - - - - - -

\subsection{LONG}
\ttindex{LONG}

{\em valid for: 320C2(0)x, 320C5x}

LONG stores a 32-bit integer to memory with the order LoWord-HiWord.
Note that when in 320C2(0)x/5x mode, the assembler assumes that a label
on the left side of this instruction has no type, i.e. it belongs to
no address space. This behaviour is explained in the
processor-specific hints.

%%- - - - - - - - - - - - - - - - - - - - - - - - - - - - - - - - - - - - - -

\subsection{SINGLE, DOUBLE, and EXTENDED}
\ttindex{SINGLE}\ttindex{DOUBLE}\ttindex{EXTENDED}

{\em valid for: 320C3x/C4x (not {\tt DOUBLE}), 320C6x (not {\tt EXTENDED})}

Both commands store floating-point constants to memory.  In case of the
320C3x/C4x, they are \bb{not} stored in IEEE-format.  Instead the
processor-specific formats with 32 and 40 bit are used.  In case of
\tty{EXTENDED} the resulting constant occupies two memory words.  The most
significant 8 bits (the exponent) are written to the first word while the
other ones (the mantissa) are copied into the second word.

%%- - - - - - - - - - - - - - - - - - - - - - - - - - - - - - - - - - - - - -

\subsection{FLOAT and DOUBLE}
\ttindex{FLOAT}\ttindex{DOUBLE}

{\em valid for: 320C2(0)x, 320C5x}

These two commands store floating-point constants in memory using the
standard IEEE 32-bit and 64-bit IEEE formats.  The least significant
byte is copied to the first allocated memory location.  Note that
when in 320C2(0)x/5x mode the assembler assumes that all labels on the
left side of an instruction have no type, i.e. they belong to no
address space.  This behaviour is explained in the processor-specific
hints.

%%- - - - - - - - - - - - - - - - - - - - - - - - - - - - - - - - - - - - - -

\subsection{SINGLE and DOUBLE}
\ttindex{SINGLE}\ttindex{DOUBLE}

{\em valid for: TMS99xxx}

These two commands store floating-point constants in memory using the
processor's floating point format, which is equal to the IBM/360
floating point format.

%%- - - - - - - - - - - - - - - - - - - - - - - - - - - - - - - - - - - - - -

\subsection{EFLOAT, BFLOAT, and TFLOAT}
\ttindex{EFLOAT}\ttindex{BFLOAT}\ttindex{TFLOAT}

{\em valid for: 320C2(0)x, 320C5x}

Another three floating point commands.  All of them support non-IEEE
formats, which should be easily applicable on signal processors:
\begin{itemize}
\item{\tty{EFLOAT}: mantissa with 16 bits, exponent with 16 bits}
\item{\tty{BFLOAT}: mantissa with 32 bits, exponent with 16 bits}
\item{\tty{DFLOAT}: mantissa with 64 bits, exponent with 32 bits}
\end{itemize}
The three commands share a common storage strategy.  In all cases the
mantissa precedes the exponent in memory, both are stored as 2's
complement with the least significant byte first.  Note that when in
320C2(0)x/5x mode the assembler assumes that all labels on the left side
of an instruction have no type, i.e.  they belong to no address
space. This behaviour is explained in the processor-specific hints.

%%- - - - - - - - - - - - - - - - - - - - - - - - - - - - - - - - - - - - - -

\subsection{Qxx and LQxx}
\ttindex{Qxx}\ttindex{LQxx}

{\em valid for: 320C2(0)x, 320C5x}

\tty{Qxx} and \tty{LQxx} can be used to generate constants in a fixed
point format. \tty{xx} denotes a 2-digit number.  The operand is first
multiplied by $2^{xx}$ before converting it to binary notation.  Thus
\tty{xx} can be viewed as the number of bits which should be reserved for
the fractional part of the constant in fixed point format.  \tty{Qxx}
stores only one word (16 bit) while \tty{LQxx} stores two words (low word
first):
\begin{verbatim}
        q05     2.5     ; --> 0050h
        lq20    ConstPI ; --> 43F7h 0032h
\end{verbatim}
Please do not flame me in case I calculated something wrong on my
HP28...

%%- - - - - - - - - - - - - - - - - - - - - - - - - - - - - - - - - - - - - -

\subsection{DATA}
\ttindex{DATA}

{\em valid for: PIC, 320xx, AVR, MELPS-4500, H8/500,
     HMCS400, 4004/4040, $\mu$PD772x, OLMS-40/50, Padauk}

This command stores data in the current segment.  Both integer
values as well as character strings are supported.  On
16C5x/16C8x, 17C4x in data segment, and on the 4500, 4004, and
HMCS400 in code segment, characters occupy one word.  On AVR,
17C4x in code segment, $\mu$PD772x in the data segments, and on
3201x/3202x, in general two characters fit into one word (LSB
first).  The $\mu$PD77C25 can hold three bytees per word in the
code segment.  When in 320C3x/C4x, mode the assembler puts four
characters into one word (MSB first).  In contrast to this
characters occupy two memory locations in the data segment of the
4500, similar in the 4004 and HMCS400.  The range of integer
values corresponds to the word width of each processor in a
specific segment.  This means that \tty{DATA} has the same result
than \tty{WORD} on a 320C3x/C4x (and that of \tty{SINGLE} if AS
recognizes the operand as a floating-point constant).

%%- - - - - - - - - - - - - - - - - - - - - - - - - - - - - - - - - - - - - -

\subsection{ZERO, CP-1600}
\ttindex{ZERO}

{\em valid for: PIC}

Generates a continuous string of zero words in memory (which equals a NOP
on PIC).

%%- - - - - - - - - - - - - - - - - - - - - - - - - - - - - - - - - - - - - -

\subsection{FB and FW}
\ttindex{FB}\ttindex{FW}

{\em valid for: COP4/8}

These instruction allow to fill memory blocks with a byte or word
constant. The first operand specifies the size of the memory block
while the second one sets the filling constant itself.

%%- - - - - - - - - - - - - - - - - - - - - - - - - - - - - - - - - - - - - -

\subsection{ASCII and ASCIZ}
\ttindex{ASCII}\ttindex{ASCIZ}

{\em valid for: ST6}

Both commands store string constants to memory.  While \tty{ASCII} writes
the character information only, \tty{ASCIZ} additionally appends a zero to
the end of the string.

%%- - - - - - - - - - - - - - - - - - - - - - - - - - - - - - - - - - - - - -

\subsection{STRING and RSTRING}
\ttindex{STRING}\ttindex{RSTRING}

{\em valid for: 320C2(0)x, 320C5x}

These commands are functionally equivalent to \tty{DATA}, but integer
values are limited to the range of byte values. This enables two
characters or numbers to be packed together into one word. Both commands
only differ in the order they use to write bytes: \tty{STRING} stores the
upper one first then the lower one, \tty{RSTRING} does this vice versa.
Note that when in 320C2(0)x/5x mode the assembler assumes that a label on the
left side of this instruction has no type, i.e. it belongs to no address
space.  This behaviour is explained in the processor-specific hints.

%%- - - - - - - - - - - - - - - - - - - - - - - - - - - - - - - - - - - - - -

\subsection{FCC}
\ttindex{FCC}

{\em valid for: 6502, 68xx}

When in 65xx/68xx mode, string constants are generated using this
instruction. In contrast to the original assembler AS11 from Motorola
(this is the main reason why AS understands this command, the
functionality is contained within the \tty{BYT} instruction) you must
enclose the string argument by double quotation marks instead of single
quotation marks or slashes.  Similarly to \tty{DC}, a repetition factor
enclosed in brackets ([..]) may be prepended to every single parameter.

%%- - - - - - - - - - - - - - - - - - - - - - - - - - - - - - - - - - - - - -

\subsection{TEXT}

In CP-1600 mode, This instruction is used to store string constants in
packed format, i.e. two characters per word.

%%- - - - - - - - - - - - - - - - - - - - - - - - - - - - - - - - - - - - - -

\subsection{DFS or RMB}
\ttindex{DFS}\ttindex{RMB}

{\em valid for: 6502, 68xx}

Reserves a memory block when in 6502/68xx mode.  It is therefore the
equivalent to \tty{DS.B} on the 68000 or \tty{DB ?} on Intel platforms.

%%- - - - - - - - - - - - - - - - - - - - - - - - - - - - - - - - - - - - - -

\subsection{BLOCK}
\ttindex{BLOCK}

{\em valid for: ST6}

Ditto.

%%- - - - - - - - - - - - - - - - - - - - - - - - - - - - - - - - - - - - - -

\subsection{SPACE}
\ttindex{SPACE}

{\em valid for: i960}

Ditto.

%%- - - - - - - - - - - - - - - - - - - - - - - - - - - - - - - - - - - - - -

\subsection{RES}
\ttindex{RES}

{\em valid for: PIC, MELPS-4500, HMCS400, 3201x, 320C2(0)x, 320C5x, AVR,
 $\mu$PD772x, OLMS-40/50, Padauk, CP-1600}

This command allocates memory.  When used in code segments the
argument counts words (10/12/14/16 bit).  In data segments it counts
bytes for PICs, nibbles for 4500 and OLMS-40/50 and words for the TI devices.

%%- - - - - - - - - - - - - - - - - - - - - - - - - - - - - - - - - - - - - -

\subsection{BSS}
\ttindex{BSS}

{\em valid for: 320C2(0)x, 320C3x/C4x/C5x/C6x, MSP}

\tty{BSS} works like \tty{RES}, but when in 320C2(0)x/5x mode, the assembler
assumes that a label on the left side of this instruction has no type, i.e
it belongs to no address space. This behaviour is explained in the
processor-specific hints.

%%- - - - - - - - - - - - - - - - - - - - - - - - - - - - - - - - - - - - - -

\subsection{DSB and DSW}
\ttindex{DSB}\ttindex{DSW}

{\em valid for: COP4/8}

Both instructions allocate memory and ensure compatibility to ASMCOP from
National.  While \tty{DSB} takes the argument as byte count, \tty{DSW}
uses it as word count (thus it allocates twice as much memory than
\tty{DSB}).

%%- - - - - - - - - - - - - - - - - - - - - - - - - - - - - - - - - - - - - -

\subsection{DS16}
\ttindex{DS16}

{\em valid for: SC144xx}

This instruction reserves memory in steps of full words, i.e. 16 bits.  It
is an alias for {\tt DW}.

%%- - - - - - - - - - - - - - - - - - - - - - - - - - - - - - - - - - - - - -

\subsection{ALIGN}
\ttindex{ALIGN}

{\em valid for: all processors}

Takes the argument to align the program counter to a certain address
boundary.  AS increments the program counter to the next multiple of the
argument.  So, \tty{ALIGN} corresponds to \tty{DS.x} on 68000, but is much
more flexible at the same time.

Example:
\begin{verbatim}
        align     2
\end{verbatim}
aligns to an even address (PC mod 2 = 0).  If Align is used in this form with
only one argument, the contents of the skipped memory space is not defined. 
An optinal second argument may be used to define the (byte) value
used to fill the area.

%%- - - - - - - - - - - - - - - - - - - - - - - - - - - - - - - - - - - - - -

\subsection{LTORG}
\ttindex{LTORG}

{\em valid for: SH7x00}

Although the SH7000 processor can do an immediate register load with
8 bit only, AS shows up with no such restriction.  This behaviour is
instead simulated through constants in memory.  Storing them in
the code segment (not far away from the register load instruction)
would require an additional jump.  AS Therefore gathers the constants
an stores them at an address specified by \tty{LTORG}.  Details are
explained in the processor-specific section somewhat later.

%%---------------------------------------------------------------------------

\section{Macro Instructions}

{\em valid for: all processors}

Now we finally reach the things that make a macro assembler different
from an ordinary assembler: the ability to define macros (guessed
it !?).

When speaking about 'macros', I generally mean a sequence of (machine
or pseudo) instructions which are united to a block by special
statements and can then be treated in certain ways.  The assembler
knows the following statements to work with such blocks:

%%- - - - - - - - - - - - - - - - - - - - - - - - - - - - - - - - - - - - - -

\subsection{MACRO}
\ttindex{MACRO}\ttindex{ENDM}
\label{SectMacros}

is probably the most important instruction for macro programming.
The instruction sequence
\begin{verbatim}
<name>  MACRO   [parameter list]
        <instructions>
        ENDM
\end{verbatim}
defines the macro \tty{$<$name$>$} to be the enclosed instruction sequence.
This definition by itself does not generate any code!  In turn, from
now on the instruction sequence can simply be called by the name, the
whole construct therefore shortens and simplifies programs.  A
parameter list may be added to the macro definition to make things
even more useful.  The parameters' names have to be separated by
commas (as usual) and have to conform to the conventions for symbol
names (see section \ref{SectSymConv}) - like the macro name itself.

A switch to case-sensitive mode influences both macro names and
parameters.

Similar to symbols, macros are local, i.e. they are only known in a
section and its subsections when the definition is done from within
a section.  This behaviour however can be controlled in wide limits
via the options \tty{PUBLIC} and \tty{GLOBAL} described below.

A default value may be provided for each macro parameter
(appended via an equal sign).  This value is used if there is no
argument for this parameter at macro call or if the positional
argument (see below) for this parameter is empty.

Apart from the macro parameters themselves, the parameter list may
contain control parameters which influence the processing of the
macro.  These parameters are distinguished from normal parameters by
being enclosed in curly braces.  The following control parameters are
defined:
\begin{itemize}
\item{\tty{EXPAND/NOEXPAND}: rule whether the enclosed code shall
      be written to the listing when the macro is expanded.  The
      default is the value set by the pseudo instruction
      \tty{MACEXP\_DFT}.}
\item{\tty{EXPIF/NOEXPIF}: rule whether instructions for
      conditional assembly and code excluded by it shall
      be written to the listing when the macro is expanded.  The
      default is the value set by the pseudo instruction
      \tty{MACEXP\_DFT}.}
\item{\tty{EXPMACRO/NOEXPMACRO}: rule whether macros defined in
      the macro's body shall be written to the listing when the macro
      is expanded.  The default is the value set by the pseudo instruction
      \tty{MACEXP\_DFT}.}
\item{\tty{EXPREST/NOEXPREST} : rule whether a macro body's lines
      not fitting into the first two categories shall be written to the
      listing when the macro is expanded. The default is the value set by
      the pseudo instruction \tty{MACEXP\_DFT}.}
\item{\tty{PUBLIC[:section name]}: assigns the macro to a parent section
      instead of the current section.  A section can make macros
      accessible for the outer code this way.  If the section
      specification is missing, the macro becomes completely global, i.e.
      it may be referenced from everywhere.}
\item{\tty{GLOBAL[:section name]}: rules that in addition to the macro
      itself, another macro shall be generated that has the same contents
      but is assigned to the specified section.  Its name is constructed by
      concatenating the current section's name to the macro name.  The
      section specified must be a parent section of the current section;
      if the specification is missing, the additional macro becomes
      globally visible.  For example, if a macro \tty{A} is defined in a
      section \tty{B} that is a child section of section \tty{C}, an additional
      global macro named \tty{C\_B\_A} would be generated.  In contrast, if
      \tty{C} had been specified as target section, the macro would be named \tty{B\_A}
      and be assigned to section \tty{C}.  This option is turned off by default
      and it only has an effect when it is used from within a section.
      The macro defined locally is not influenced by this option.}
\item{\tty{EXPORT/NOEXPORT}: rules whether the definition of this macro
      shall be written to a separate file in case the \tty{-M} command line
      option was given.  This way, definitions of 'private' macros may
      be mapped out selectively.  The default is FALSE, i.e. the
      definition will not be written to the file.  The macro will be
      written with the concatenated name if the \tty{GLOBAL} option was
      additionally present.}
\item{\tty{INTLABEL/NOINTLABEL} : rules whether a label defined in a line
      that calls this macro may be used as an additional parameter inside
      the label or not, instead of simply 'labeling' the line.}
\item{\tty{GLOBALSYMBOLS/NOGLOBALSYMBOLS} : rules whether labels
      defined in the macro's body shall be local to this macro or
      also be available outside the macro.  The default is to
      keep them local, since using a macro multiple time would be
      difficult otherwise.}
\end{itemize}
The control parameters described above are removed from the parameter
list by AS, i.e. they do not have a further influence on processing
and usage.

When a macro is called, the parameters given for the call are
textually inserted into the instruction block and the resulting
assembler code is assembled as usual.  Zero length parameters are
inserted in case too few parameters are specified.  It is important
to note that string constants are not protected from macro
expansions.  The old IBM rule:
\begin{quote}{\it
 It's not a bug, it's a feature!
}\end{quote}
applies for this detail.  The gap was left to allow checking of
parameters via string comparisons.  For example, one can analyze a
macro parameter in the following way:
\begin{verbatim}
mul     MACRO   para,parb
        IF      UpString("PARA")<>"A"
         MOV    a,para
        ENDIF
        IF      UpString("PARB")<>"B"
         MOV    b,parb
        ENDIF
        !mul     ab
        ENDM
\end{verbatim}
It is important for the example above that the assembler converts all
parameter names to upper case when operating in case-insensitive
mode, but this conversion never takes place inside of string constants.
Macro parameter names therefore have to be written in upper case when
they appear in string constants.

Macro arguments may be given in either of two forms: {\em
positional} or {\em keyword} arguments.

For positional arguments, the assignment of arguments to macro
parameters simply results from the position of arguments, i.e.
the first argument is assigned to the first parameter, the second
argument to the second parameter and so on.  If the number of
arguments is smaller than the number of parameters, eventually
defined default values or simply an empty string are inserted.
The same is valid for empty arguments in the argument list.

Keyword arguments on the other hand explicitly define which
parameter they relate to, by being prefixed with the parameter's
name:
\begin{verbatim}
       mul  para=r0,parb=r1
\end{verbatim}
Again, non-assigned parameters will use an eventually defined
default or an empty string.

As a difference to positional arguments, keyword arguments allow
to assign an empty string to a parameter with a non-empty
default.

Mixing of positional and keyword arguments in one macro call is
possible, however it is not allowed to use positional arguments
after the first keyword argument.

The same naming rules as for usual symbols also apply for macro
parameters, with the exception that only letters and numbers are
allowed, i.e. dots and underscores are forbidden.  This constraint
has its reason in a hidden feature: the underscore allows to
concatenate macro parameter names to a symbol, like in the following
example:
\begin{verbatim}
concat  macro   part1,part2
        call    part1_part2
        endm
\end{verbatim}
The call
\begin{verbatim}
        concat  module,function
\end{verbatim}
will therefore result in
\begin{verbatim}
        call    module_function
\end{verbatim}
Apart from the parameters explicitly declared for a macro, four more
'implicitly' declared parameters exist.   Since they are always present,
they cannot not be redeclared as explicit parameters:
\begin{itemize}
\item{{\tt ATTRIBUTE} refers to the attribute appended to the macro call,
      in case the currently active architecture supports attributes for
      machine instructions.  See below for an example!}
\item{{\tt ALLARGS} refers to a comma-separated list of all arguments
      passed to a macro, usable e.g. to pass them on to a IRP statement.}
\item{{\tt ARGCOUNT} refers to the actual count of parameters passed to
      a macro.  Note however that this number is never lower than the
      formal parameter count of the macro, since AS will fill up missing
      arguments with empty strings!}
\item{{\tt \_\_LABEL\_\_} refers to a label present in a line that calls the
      macro. This replacement only takes place if the {\tt INTLABEL}
      option was set for this macro!}
\end{itemize}
{\bf IMPORTANT:} the names of these implicit parameters are also
case-insensitive if AS was told to operate case-sensitive!

The purpose of being able to 'internally' use a label in a macro is surely
not immediately obvious.  There might be cases where moving the macro's
entry point into its body may be useful.  The most important application
however are TI signal processors that use a double pipe symbol in the
label's column to mark parallelism, like this:
\begin{verbatim}
    instr1
||  instr2
\end{verbatim}
(since both instructions merge into a single word of machine code, you
cannot branch to the second instruction - so occupying the label's
position doesn't hurt).  The problem is however that some 'convenience
instructions' are realized as macros.  A prallelization symbol written in
front of a macro call normally would be assigned to the macro itself, {\it
not to the macro body's first instruction}.  However, things work with
this trick:
\begin{verbatim}
myinstr    macro {INTLABEL}
__LABEL__  instr2
           endm

           instr1
||         myinstr
\end{verbatim}
The result after expanding {\tt myinstr} is identical to the previous
example without macro.

Recursion of macros, i.e. the repeated call of a macro from within its own
body is completely legal.  However, like for any other sort of recursion,
one has to assure that there is an end at someplace.  For cases where one
forgot this, AS keeps an internal counter for every macro that is
incremented when an expansion of this macro is begun and decremented again
when the expansion is completed.  In case of recursive calls, this counter
reaches higher and higher values, and at a limit settable via {\tt
NESTMAX}, AS will refuse to expand. Be careful when you turn off this
emergency brake: the memory consumption on the heap may go beyond all
limits and even shut down a Unix system...

A small example to remove all clarities ;-)

A programmer braindamaged by years of programming Intel processors
wants to have the instructions \tty{PUSH/POP} also for the 68000.  He
solves the 'problem' in the following way:
\begin{verbatim}
push    macro   op
        move.ATTRIBUTE op,-(sp)
        endm

pop     macro   op
        move.ATTRIBUTE (sp)+,op
        endm
\end{verbatim}
If one writes
\begin{verbatim}
        push    d0
        pop.l   a2    ,
\end{verbatim}
this results in
\begin{verbatim}
        move.   d0,-(sp)
        move.l  (sp)+,a2
\end{verbatim}
A macro definition must not cross include file boundaries.

Labels defined in macros always are regarded as being local,
unless the \tty{GLOBALSYMBOLS} was used in the macro's
definition.  If a single label shall be made public in a macro
that uses local labels otherwise, it may be defined with a
\tty{LABEL} statement which always creates global symbols
(similar to \tty{BIT, SFR...}):
\begin{verbatim}
<Name>  label   $
\end{verbatim}
When parsing a line, the assembler first checks the macro list
afterwards looks for processor instructions, which is why macros
allow to redefine processor instructions.  However, the definition
should appear previously to the first invocation of the instruction
to avoid phase errors like in the following example:
\begin{verbatim}
        bsr     target

bsr     macro   targ
        jsr     targ
        endm

        bsr     target
\end{verbatim}
In the first pass, the macro is not known when the first \tty{BSR}
instruction is assembled; an instruction with 4 bytes of length is
generated.  In the second pass however, the macro definition is
immediately available (from the first pass), a \tty{JSR} of 6 bytes length
is therefore generated.  As a result, all labels following are too low
by 2 and phase errors occur for them.  An additional pass is
necessary to resolve this.

Because a machine or pseudo instruction becomes hidden when a macro
of same name is defined, there is a backdoor to reach the original
meaning: the search for macros is suppressed if the name is prefixed
with an exclamation mark (!).  This may come in handy if one wants to
extend existing instructions in their functionality, e.g. the
TLCS-90's shift instructions:
\begin{verbatim}
srl     macro   op,n            ; shift by n places
        rept    n               ; n simple instructions
         !srl   op
        endm
        endm
\end{verbatim}
From now on, the \tty{SRL} instruction has an additional parameter...

\subsubsection{Macro Expansion in the Listing}

If a macro is being called, the macro's body is included in the assembly
listing, after arguemnts have been expanded.  This can significantly increase
the listing's size and make it hard to read.  It is therefore possible to
suppress this expansion totally or in parts.  Fundamentally, AS divides
the source lines contained in a macro's body into three classes:
\begin{itemize}
\item{Macro definitions, i.e. the macro is used to define another macro,
      or it contains \tty{REPT/IRP/IRPC/WHILE} blocks.}
\item{Instructions for conditional assembly plus any source lines that have
      {\it not} been assembled due to conditional assembly.  Since conditional
      assembly may depend on macro arguments, this subset may also vary.}
\item{All remaining sourc elines that do not fall under the first two
      categories.}
\end{itemize}
Which parts occur in the listing may be defined individually for every macro.
When defining a macro, the default is the set defined by the most recent
\tty{MACEXP\_DFT} instruction (\ref{MACEXPDFT}).  If one of the \tty{EXPAND/NOEXPAND},
\tty{EXPIF/NOEXPIF} \tty{EXPMACRO/NOEXPMACRO}, or \tty{EXPREST/NOEXPREST}
directives is used in the macro's definition, they act {\em additionally},
but with higher preference.  For instance, if expansion had been disabled
completely (\tty{MACEXP\_DFT OFF}), adding the directive \tty{EXPREST} has the
effect that when using this macro, only lines are written to the listing
that remain after conditional assembly and are no macro definitions themselves.

In consequence, changing the set via \tty{MACEXP\_DFT} has no effect on macros
that have been {\it defined before} this statement.  The listing's section shows
for defined macros the effective set of expansion directives.  The list given
in curly braces is shorted so that it only conatins the last (and therefore
valid) directive for a certain class of source lines.  A \tty{NOIF} given
via \tty{MACEXP\_DFT} will therefore not show up if the directive \tty{EXPIF}
had been given specifically for this macro.

There might be cases where it is useful to override the expansion rules for
a certain macro, regardless whether they were given by \tty{MACEXP\_DFT} or
individual directives.  The statement \tty{MACEXP\_OVR} (\ref{MACEXPOVR})
exists for such cases.  It only has an effects on macros subsequently being
{\it expanded}.  Once again, directives given by this instruction are regarded
in addition to a macro's rules, and they do with higher priority.  A \tty{MACEXP\_OVR}
without any arguments disables such an override.

%%- - - - - - - - - - - - - - - - - - - - - - - - - - - - - - - - - - - - - -

\subsection{IRP}
\ttindex{IRP}

is a simplified macro definition for the case that an instruction sequence
shall be applied to a couple of operands and the the code is not needed
any more afterwards.  \tty{IRP} needs a symbol for the operand as its
first parameter, and an (almost) arbitrary number of parameters that are
sequentially inserted into the block of code.  For example, one can write
\begin{verbatim}
        irp     op, acc,b,dpl,dph
        push    op
        endm
\end{verbatim}
to push a couple of registers to the stack, what results in
\begin{verbatim}
        push    acc
        push    b
        push    dpl
        push    dph
\end{verbatim}
Analog to a macro definition, the argument list may contain the
control parameters \tty{GLOBALSYMBOLS} resp.
\tty{NOGLOBALSYMBOLS} (marked as such by being enclosed in curly
braces).  This allows to control whether used labels are local for
every pass or not.

%%- - - - - - - - - - - - - - - - - - - - - - - - - - - - - - - - - - - - - -

\subsection{IRPC}
\ttindex{IRPC}

\tty{IRPC} is a variant of \tty{IRP} where the first argument's occurences
in the lines up to \tty{ENDM} are successively replaced by the characters
of a string instead of further parameters.  For example, an especially
complicated way of placing a string into memory would be:
\begin{verbatim}
        irpc    char,"Hello World"
        db      'CHAR'
        endm
\end{verbatim}
\bb{CAUTION!} As the example already shows, \tty{IRPC} only inserts the
pure character; it is the programmer's task to assure that valid code
results (in this example by inserting quotes, including the detail that no
automatic conversion to uppercase characters is done).

%%- - - - - - - - - - - - - - - - - - - - - - - - - - - - - - - - - - - - - -

\subsection{REPT}
\ttindex{REPT}

is the simplest way to employ macro constructs.  The code between
\tty{REPT} and \tty{ENDM} is assembled as often as the integer argument of
\tty{REPT} specifies.  This statement is commonly used in small loops to
replace a programmed loop to save the loop overhead.

An example for the sake of completeness:
\begin{verbatim}
        rept    3
        rr      a
        endm
\end{verbatim}
rotates the accumulator to the right by three digits.

The optional control parameters \tty{GLOBALSYMBOLS}
resp. \tty{NOGLOBALSYMBOLS} (marked as such by being enclosed in curly
braces) may also be used here to decide whether labels are local to the
individual repetitions.

In case \tty{REPT}'s argument is equal to or smaller than 0, no expansion
at all is done.  This is different to older versions of AS which used to
be a bit 'sloppy' in this respect and always made a single expansion.

%%- - - - - - - - - - - - - - - - - - - - - - - - - - - - - - - - - - - - - -

\subsection{WHILE}
\ttindex{WHILE}

\tty{WHILE} operates similarly to \tty{REPT}, but the fixed number of
repetitions given as an argument is replaced by a boolean expression.  The
code framed by \tty{WHILE} and \tty{ENDM} is assembled until the
expression becomes logically false.  This may mean in the extreme case
that the enclosed code is not assembled at all in case the expression was
already false when the construct was found.  On the other hand, it may
happen that the expression stays true forever and AS will run
infinitely...one should apply therefore a bit of accuracy when one uses
this construct, i.e. the code must contain a statement that influences the
condition, e.g. like this:
\begin{verbatim}
cnt     set     1
sq      set     cnt*cnt
        while   sq<=1000
         dc.l    sq
cnt      set     cnt+1
sq       set     cnt*cnt
        endm
\end{verbatim}
This example stores all square numbers up to 1000 to memory.

Currently there exists a little ugly detail for \tty{WHILE}: an additional
empty line that was not present in the code itself is added after the last
expansion.  This is a 'side effect' based on a weakness of the macro
processor and it is unfortunately not that easy to fix.  I hope noone
minds...

%%- - - - - - - - - - - - - - - - - - - - - - - - - - - - - - - - - - - - - -

\subsection{EXITM}
\ttindex{EXITM}

\tty{EXITM} offers a way to terminate a macro expansion or one of the
instructions \tty{REPT, IRP,} or \tty{WHILE} prematurely.  Such an option
helps for example to replace encapsulations with \tty{IF-ENDIF}-ladders in
macros by something more readable.  Of course, an \tty{EXITM} itself
always has to be conditional, what leads us to an important detail: When
an \tty{EXITM} is executed, the stack of open \tty{IF} and
\tty{SWITCH} constructs is reset to the state it had just before the macro
expansion started.  This is imperative for conditional \tty{EXITM}'s as
the \tty{ENDIF} resp. \tty{ENDCASE} that frames the \tty{EXITM} statement
will not be reached any more; AS would print an error message without this
trick.  Please keep also in mind that \tty{EXITM} always only terminates
the innermost construct if macro constructs are nested!  If one want to
completely break out of a nested construct, one has to use additional
\tty{EXITM}'s on the higher levels!

%%- - - - - - - - - - - - - - - - - - - - - - - - - - - - - - - - - - - - - -

\subsection{SHIFT}
\ttindex{SHIFT}\ttindex{SHFT}

{\tt SHIFT} is a tool to construct macros with variable argument lists: it
discards the first parameter, with the result that the second parameter
takes its place and so on.  This way one could process a variable argument
list...if you do it the right way.  For example, the following does not
work...
\begin{verbatim}
pushlist  macro reg
          rept  ARGCOUNT
          push  reg
          shift
          endm
          endm
\end{verbatim}
...because the macro gets expanded {\tt once}, its output is captured by
{\tt REPT} and then executed n times.  Therefore, the first argument is
saved n times...the following approach works better:
\begin{verbatim}
pushlist  macro reg
          if      "REG"<>""
           push    reg
           shift
           pushlist ALLARGS
          endif
          endm
\end{verbatim}
Effectively, this is a recursion that shortens the argument list once per
step.  The important trick is that a new macro expansion is started in
each step...

In case {\tt SHIFT} ist already a machine instruction for a certain target,
one has to use {\tt SHFT} instead.

%%- - - - - - - - - - - - - - - - - - - - - - - - - - - - - - - - - - - - - -

\subsection{MAXNEST}
\ttindex{MAXNEST}

{\tt MAXNEST} allows to adjust how often a mcro may be called recursively
before AS terminates with an error message.  The argument may be an
arbitrary positive integer value, with the special value 0 turning the
this security brake completely off (be careful with that...).  The default
value for the maximum nesting level is 256; its current value may be read
from the integer symbol of same name.

%%- - - - - - - - - - - - - - - - - - - - - - - - - - - - - - - - - - - - - -

\subsection{FUNCTION}
\label{SectFUNCTION}
\ttindex{FUNCTION}

Though \tty{FUNCTION} is not a macro statement in the inner sense, I will
describe this instruction at this place because it uses similar principles
like macro replacements.

This instruction is used to define new functions that may then be
used in formula expressions like predefined functions.  The
definition must have the following form:
\begin{verbatim}
<name>  FUNCTION <arg>,..,<arg>,<expression>
\end{verbatim}
The arguments are the values that are 'fed into' the function.  The
definition uses symbolic names for the arguments.  The assembler
knows by this that where to insert the actual values when the
function is called.  This can be seen from the following example:
\begin{verbatim}
isdigit FUNCTION ch,(ch>='0')&&(ch<='9')
\end{verbatim}
This function checks whether the argument (interpreted as a character) is
a number in the currently valid character set (the character set can be
modified via \tty{CHARSET}, therefore the careful wording).

The arguments' names (\tty{CH} in this case) must conform to the stricter
rules for macro parameter names, i.e. the special characters . and \_
are not allowed.

User-defined functions can be used in the same way as builtin
functions, i.e. with a list of parameters, separated by commas,
enclosed in parentheses:
\begin{verbatim}
        IF isdigit(char)
         message "\{char} is a number"
        ELSEIF
         message "\{char} is not a number"
        ENDIF
\end{verbatim}
When the function is called, all parameters are calculated once and
are then inserted into the function's formula.  This is done to
reduce calculation overhead and to avoid side effects.  The
individual arguments have to be separated by commas when a function
has more than one parameter.

\bb{CAUTION!}  Similar to macros, one can use user-defined functions to
override builtin functions.  This is a possible source for phase
errors.  Such definitions therefore should be done before the first
call!

The result's type may depend on the type of the input arguments as
the arguments are textually inserted into the function's formula.
For example, the function
\begin{verbatim}
double  function x,x+x
\end{verbatim}
may have an integer, a float, or even a string as result, depending
on the argument's type!

When AS operates in case-sensitive mode, the case matters when
defining or referencing user-defined functions, in contrast to
builtin functions!

%%---------------------------------------------------------------------------

\section{Structures}
\ttindex{STRUCT}\ttindex{ENDSTRUCT}\ttindex{UNION}\ttindex{ENDUNION}
\ttindex{STRUC}\ttindex{ENDSTRUC}\ttindex{ENDS}
\ttindex{DOTTEDSTRUCTS}

{\em valid for: all processors}

Even in assembly language programs, there is sometimes the necessity to
define composed data structures, similar to high-level languages.  AS
supports both the definition and usage of structures with a couple of
statements.  These statements shall be explained in the following section.

\subsection{Definition}

The definiton of a structure is begun with the statement
\tty{STRUCT} and ends with \tty{ENDSTRUCT} (lazy people may also
write {\tt STRUC} resp.  {\tt ENDSTRUC} or {\tt ENDS} instead).
A optional label preceding these instructions is taken as the
name of the structure to be defined; it is optional at the end of
the definition and may be used to redefine the length symbol's
name (see below).  The remaining procedure is simple: Together
with \tty{STRUCT}, the cuurent program counter is saved and reset
to zero.  All labels defined between \tty{STRUCT} and
\tty{ENDSTRUCT} therefore are the offsets of the structure's data
fields.  Reserving space is done via the same instructions that
are also otherwise used for reserving space, like e.g.
\tty{DS.x} for Motorola CPUs or \tty{DB} \& co.  for Intel-style
processors.  The rules for rounding up lengths to assure certain
alignments also apply here - if one wants to define 'packed'
structures, a preceding {\tt PADDING OFF} may be necessary.  Vice
versa, alignments may be forced with {\tt ALIGN} or similar
instructions.

Since such a definition only represents a sort of 'prototype', only
instructions that reserve memory may be used, no instructions that dispose
constants or generate code.

Labels defined inside structures (i.e. the elements' names) are not
stored as-is.  Instead, the structure's name is prepended to them,
separated with a special character.  By default, this is the underbar
(\_).  This behaviour however may be modified with two arguments passed
to the
\tty{STRUCT} statement:
\begin{itemize}
\item{\tty{NOEXTNAMES} suppressed the prepending of the structure's name.
      In this case, it is the programmer's responsibility to assure that
      field names are not used more than once.}
\item{\tty{DOTS} instructs AS to use the dot as connecting character
      instead of the underbar.  It should however be pointed out that
      on certain target architectures, the dot has a special meaning
      for bit addressing, which may lead to problems!}
\end{itemize}
It is futhermore possible to turn the usage of a dot on resp. off for all
following structures:
\begin{verbatim}
        dottedstructs <on|off>
\end{verbatim}

Aside from the element names, AS also defines a further symbol with the
structure's overall length when the definition has been finished.  This
symbol has the name {\tt LEN}, which is being extended with the
structure's name via the same rules - or alternitavely with the label name
given with the \tty{ENDSTRUCT} statement.

In practice, this may things may look like in this example:
\begin{verbatim}
Rec     STRUCT
Ident   db      ?
Pad     db      ?
Pointer dd      ?
Rec     ENDSTRUCT
\end{verbatim}
In this example, the symbol {\tt REC\_LEN} would be assigned the value 6.

\subsection{Usage}

Once a structure has been assigned, usage is as simple as possible and
similar to a macro: a simple
\begin{verbatim}
thisrec Rec
\end{verbatim}
reserves as much memory as needed to hold an instance of the structure,
and additionally defines a symbol for every element of the structure with
its address, in this case {\tt THISREC\_IDENT, THISREC\_PAD}, and {\tt
THISREC\_POINTER}.  A label naturally must not be omitted when calling a
structure; if it is missing, an error will be emitted.

Additional arguments allow to reserve memory for a whole array of structures.
The dimensions (up to three) are defined via arguments in square brackets:
\begin{verbatim}
thisarray Rec [10],[2]
\end{verbatim}
In this example, space for $2*10=20$ structures is reserved. For each individual
structure in the array, proper symbols are generated that have the array
indices in their name.

\subsection{Nested Structures}

Is is perfectly valid to call an already defined structure within the
definition of another structure.  The procedure that is taking place then
is a combination of the definition and calling described in the previous
two sections: elements of the substructure are being defined, the name of
the instance is being prepended, and the name of the super-structure is
once again geing prepended to this concatenated name.  This may look like
the following:
\begin{verbatim}
TreeRec struct
left    dd         ?
right   dd         ?
data    Rec
TreeRec endstruct
\end{verbatim}

It is also allowed to define one structure inside of another
structure:
\begin{verbatim}
TreeRec struct
left    dd         ?
right   dd         ?
TreeData struct
name      db         32 dup(?)
id        dw         ?
TreeData endstruct
TreeRec endstruct
\end{verbatim}

\subsection{Unions}

A union is a special form of a structure whose elements are not laid out
sequentially in memory.  Instead all elements occupy the {\em same}
memory and are located at offset 0 in the structure.  Naturally, suich a
defnition basically does nothing more than to assign the value of zero to
a couple of symbols.  It may however be useful to clarify the overlap in a
program and therefore to make it more 'readable'.  The size of a union is
the maximum of all elements' lengths.

\subsection{Nameless Structures}

The name of a structure or union is optional if it is part of
another (named) structure or union.  Elements of this structure
will then become part of of the 'next higher' named structure.
For example,
\begin{verbatim}
TreeRec struct
left    dd         ?
right   dd         ?
        struct
name      db         32 dup(?)
id        dw         ?
        endstruct
TreeRec endstruct
\end{verbatim}
generates the symbols {\tt TREEREC\_NAME} and {\em TREEREC\_ID}.

Futhermore, no symbol holding its length is generated for an
unnamed structure or union.

\subsection{Structures and Sections}

Symbols that are created in the course of defining or usage of structures
are treated just like normal symbols, i.e. when used within a section,
these symbols are local to the section.  The same is however also true for
the structures themselves, i.e. a structure defined within a section
cannot be used outside of the section.

\subsection{Structures and Macros}

If one wants to instantiate structures via macros, one has to use
the \tty{GLOBALSYMBOLS} options when defining the macro to make
the defined symbols visible outside the macro.  For instance, a
list of structures can be defined in the following way:

\begin{verbatim}
        irp     name,{GLOBALSYMBOLS},rec1,rec2,rec3
name    Rec
        endm
\end{verbatim}

%%---------------------------------------------------------------------------

\section{Conditional Assembly}

{\em valid for: all processors}

The assembler supports conditional assembly with the help of statements
like \tty{IF...} resp. \tty{SWITCH...} .  These statements work at
assembly time allowing or disallowing the assembly of program parts based
on conditions.  They are therefore not to be compared with IF statements
of high-level languages (though it would be tempting to extend assembly
language with structurization statements of higher level languages...).

The following constructs may be nested arbitrarily (until a memory
overflow occurs).

%%- - - - - - - - - - - - - - - - - - - - - - - - - - - - - - - - - - - - - -

\subsection{IF / ELSEIF / ENDIF}
\ttindex{IF}
\ttindex{ENDIF}
\ttindex{ELSEIF}\ttindex{ELSE}

\tty{IF} is the most common and most versatile construct.  The general
style of an \tty{IF} statement is as follows:
\begin{verbatim}
        IF      <expression 1>
        .
        .
        <block 1>
        .
        .
        ELSEIF  <expression 2>
        .
        .
        <block 2>
        .
        .
        (possibly more ELSEIFs)

        .
        .
        ELSEIF
        .
        .
        <block n>
        .
        .
        ENDIF
\end{verbatim}
\tty{IF} serves as an entry, evaluates the first expression, and assembles
block 1 if the expression is true (i.e. not 0).  All further
\tty{ELSEIF}-blocks will then be skipped.  However, if the expression is
false, block 1 will be skipped and expression 2 is evaluated.  If this
expression turns out to be true, block 2 is assembled.  The number of
\tty{ELSEIF} parts is variable and results in an \tty{IF-THEN-ELSE} ladder
of an arbitrary length.  The block assigned to the last \tty{ELSEIF}
(without argument) only gets assembled if all previous expressions
evaluated to false; it therefore forms a 'default' branch.  It is
important to note that only \bb{one} of the blocks will be assembled: the
first one whose \tty{IF/ELSEIF} had a true expression as argument.

The \tty{ELSEIF} parts are optional, i.e. \tty{IF} may directly be
followed by an \tty{ENDIF}.  An \tty{ELSEIF} without parameters must be
the last branch.

\tty{ELSEIF} always refers to the innermost, unfinished \tty{IF} construct
in case \tty{IF}'s are nested.

\ttindex{IFDEF}\ttindex{IFNDEF}\ttindex{IFUSED}\ttindex{IFNUSED}
\ttindex{IFEXIST}\ttindex{IFNEXIST}\ttindex{IFB}\ttindex{IFNB}
In addition to \tty{IF}, the following further conditional statements are
defined:
\begin{itemize}
\item{\tty{IFDEF $<$symbol$>$}: true if the given symbol has been defined.
      The definition has to appear before \tty{IFDEF}.}
\item{\tty{IFNDEF $<$symbol$>$}: counterpart to \tty{IFDEF}.}
\item{\tty{IFUSED $<$symbol$>$}: true if if the given symbol has been
      referenced at least once up to now.}
\item{\tty{IFNUSED $<$symbol$>$}: counterpart to \tty{IFUSED}.}
\item{\tty{IFEXIST $<$name$>$}: true if the given file exists.  The same
           rules for search paths and syntax apply as for the
           \tty{INCLUDE} instruction (see section \ref{SectInclude}).}
\item{\tty{IFNEXIST $<$name$>$}: counterpart to \tty{IFEXIST}.}
\item{\tty{IFB $<$arg-list$>$}: true if all arguments of the parameter
           list are empty strings.}
\item{\tty{IFNB $<$arg-list$>$}: counterpart to \tty{IFB}.}
\end{itemize}

It is valid to write {\tt ELSE} instead of {\tt ELSEIF} since everybody
seems to be used to it...

For every {IF...} statement, there has to be a corresponding {\tt ENDIF}.
'Open' constructs will lead to an error message at the end of an assembly
path.  The way AS has 'paired' {\tt ENDIF} statements with {\tt IF}s may
be deduced from the assembly listing: for {\tt ENDIF}, the line number of
the corresponding {\tt IF...} will be shown.

%%- - - - - - - - - - - - - - - - - - - - - - - - - - - - - - - - - - - - - -

\subsection{SWITCH / CASE / ELSECASE / ENDCASE}
\ttindex{SWITCH}\ttindex{SELECT}\ttindex{CASE}\ttindex{ELSECASE}\ttindex{ENDCASE}

\tty{CASE} is a special case of \tty{IF} and is designed for situations
when an expression has to be compared with a couple of values.  This could
of course also be done with a series of \tty{ELSEIF}s, but the following
form
\begin{verbatim}
        SWITCH  <expression>
        .
        .
        CASE    <value 1>
        .
        <block 1>
        .
        CASE    <value 2>
        .
        <block 2>
        .
        (further CASE blocks)
        .
        CASE    <value n-1>
        .
        <block n-1>
        .
        ELSECASE
        .
        <block n>
        .
        ENDCASE
\end{verbatim}
has the advantage that the expression is only written once and also only
gets evaluated once.  It is therefore less error-prone and slightly faster
than an \tty{IF} chain, but obviously not as flexible.

It is possible to specify multiple values separated by commas to a
\tty{CASE} statement in order to assemble the following block in multiple
cases.  The \tty{ELSECASE} branch again serves as a 'trap' for the case
that none of the \tty{CASE} conditions was met.  AS will issue a warning
in case it is missing and all comparisons fail.

Even when value lists of \tty{CASE} branches overlap, only \bb{one} branch
is executed, which is the first one in case of ambiguities.

\tty{SWITCH} only serves to open the whole construct; an arbitrary number
of statements may be between \tty{SWITCH} and the first \tty{CASE} (but
don't leave other \tty{IF}s open!), for the sake of better readability
this should however not be done.

In case that \tty{SWITCH} is already a machine instruction on the
selected processor target, the construct is started instead with
\tty{SELECT}.

Similarly to {\tt IF} constructs, there must be exactly one {\tt ENDCASE}
for every {\tt SWITCH}.  Analogous to {\tt ENDIF}, for {\tt ENDCASE} the
line number of the corresponding {\tt SWITCH} is shown in the listing.

%%---------------------------------------------------------------------------

\section{Listing Control}

{\em valid for: all processors}

%%- - - - - - - - - - - - - - - - - - - - - - - - - - - - - - - - - - - - - -

\subsection{PAGE}
\ttindex{PAGE}

\tty{PAGE} is used to tell AS the dimensions of the paper that is used to
print the assembly listing.  The first parameter is thereby the
number of lines after which AS shall automatically output a form
feed.  One should however take into account that this value does \bb{not}
include heading lines including an eventual line specified with
\tty{TITLE}.  The minimum number of lines is 5, and the maximum value is
255.  A specification of 0 has the result that AS will not do any form
feeds except those triggered by a \tty{NEWPAGE} instruction or those
implicitly engaged at the end of the assembly listing (e.g. prior to the
symbol table).

The specification of the listing's length in characters is an
optional second parameter and serves two purposes: on the one hand,
the internal line counter of AS will continue to run correctly when a
source line has to be split into several listing lines, and on
the other hand there are printers (like some laser printers) that do
not automatically wrap into a new line at line end but instead simply
discard the rest.  For this reason, AS does line breaks by itself,
i.e. lines that are too long are split into chunks whose lengths are
equal to or smaller than the specified width.  This may lead to
double line feeds on printers that can do line wraps on their own if
one specifies the exact line width as listing width.  The solution
for such a case is to reduce the assembly listing's width by 1.  The
specified line width may lie between 5 and 255 characters; a line
width of 0 means similarly to the page length that AS shall not do
any splitting of listing lines; lines that are too long of course
cannot be taken into account of the form feed then any more.

The default setting for the page length is 60 lines, the default for the
line width is 0; the latter value is also assumed when \tty{PAGE} is
called with only one parameter.

In case \tty{PAGE} is already a machine instruction on the
selected processor target, use instead \tty{PAGESIZE} to define
the paper size.

\bb{CAUTION!}  There is no way for AS to check whether the specified
listing length and width correspond to the reality!

%%- - - - - - - - - - - - - - - - - - - - - - - - - - - - - - - - - - - - - -

\subsection{NEWPAGE}
\ttindex{NEWPAGE}

\tty{NEWPAGE} can be used to force a line feed though the current line is
not full up to now.  This might be useful to separate program parts
in the listing that are logically different.  The internal line
counter is reset and the page counter is incremented by one.  The
optional parameter is in conjunction with a hierarchical page
numbering AS supports up to a chapter depth of 4.  0 always refers to
the lowest depth, and the maximum value may vary during the assembly
run.  This may look a bit puzzling, as the following example shows:
\begin{quote}\begin{tabbing}
\hspace{2.5cm} \= \hspace{4.5cm} \= \kill
page 1,   \> instruction \tty{NEWPAGE 0} \>  $\rightarrow$ page 2 \\
page 2,   \> instruction \tty{NEWPAGE 1} \>  $\rightarrow$ page 2.1 \\
page 2.1, \> instruction \tty{NEWPAGE 1} \>  $\rightarrow$ page 3.1 \\
page 3.1, \> instruction \tty{NEWPAGE 0} \>  $\rightarrow$ page 3.2 \\
page 3.2, \> instruction \tty{NEWPAGE 2} \>  $\rightarrow$ page 4.1.1 \\
\end{tabbing}\end{quote}
\tty{NEWPAGE $<$number$>$} may therefore result in
changes in different digits, depending on the current chapter depth.  An
automatic form feed due to a line counter overflow or a \tty{NEWPAGE}
without parameter is equal to \tty{NEWPAGE 0}.  Previous to the output of
the symbol table, an implicit \tty{NEWPAGE $<$maximum up to now$>$} is
done to start a new 'main chapter'.

%%- - - - - - - - - - - - - - - - - - - - - - - - - - - - - - - - - - - - - -

\subsection{MACEXP\_DFT and MACEXP\_OVR}
\ttindex{MACEXP}
\ttindex{MACEXP\_DFT}
\ttindex{MACEXP\_OVR}
\label{MACEXPDFT}
\label{MACEXPOVR}

Once a macro is tested and 'done', one might not want to see it
in the listing when it is used.  As described in the section about
defining and using macros (\ref{SectMacros}), additional arguments to
the \tty{MACRO} statement allow to control whether a macro's body is
expanded upon its usage and if yes, which parts of it.  In case that
several macros are defined in a row, it is not necessary to give
these directives for every single macro.  The pseudo instruction
\tty{MACEXP\_DFT} defines for all following macros which parts shall
be expanded upon invocation of the macro:
\begin{itemize}
\item{\tty{ON} resp. \tty{OFF} enable or disable expansion
      completely.}
\item{The arguments \tty{IF} resp. \tty{NOIF} enable or disable
      expansion of instructions for conditional assembly, plus
      the expansion of code parts the were excluded because of
      conditional assembly.}
\item{Macro definitions (which includes \tty{REPT}, \tty{WHILE}
      and \tty{IRP(C)}) may be excluded from or included in the
      expanded parts via the arguments \tty{MACRO} resp.
      \tty{NOMACRO}.}
\item{All other lines not fitting into the first two categories
      may be excluded from or included in the expanded parts via
      the arguments \tty{REST} resp. \tty{NOREST}.}
\end{itemize}
The default is \tty{ON}, i.e. defined macros will be expanded
completely, of course unless specific expansion arguments were given
to individual macros.  Furthermore, arguments given to
\tty{MACEXP\_DFT} work relative to the current setting: for instance,
if expansion is turned on completely initially, the statement
\begin{verbatim}
	MACEXP_DFT  noif,nomacro
\end{verbatim}
has the result that for macros defined in succession, only code parts
that are no macro definition and that are not excluded via
conditional assembly will be listed.
\par
This instruction plus the per-macro directives provide fine-grained
per-macro over the parts being expanded.  However, there may be cases
in practice where one wants to see the expanded code of a macro at
one place and not at the other.  This is possible by using the
statement \tty{MACEXP\_OVR}: it accepts the same arguemnts like
\tty{MACEXP\_DFT}, these however act as overrides for all macros being
{\em expanded} in the following code.  This is in contrast to
\tty{MACEXP\_DFT} which influences macros being {\em defined} in the
following code.  For instance, if one defined for a macro that
neither macro definitions nor conditional assembly shall be expanded
in the listing, a
\begin{verbatim}
	MACEXP_OVR  MACRO
\end{verbatim}
re-enables expansion of macro definitions for its following usages,
while a
\begin{verbatim}
	MACEXP_OVR  ON
\end{verbatim}
forces expansion of the complete macro body in the listing.
\tty{MACEXP\_OVR} without arguments again disables all overrides,
macros will again behave as individually specified upon definition.
\par
Both statements also have an effect on other macro-like constructs
(\tty{REPT, IRP, IRPC WHILE}).  However, since these are expanded
only one and ,,in-place'', the functional difference of these two
statements becomes minimal.  In case of differences, the override set
via \tty{MACEXP\_OVR} has a higher priority.

The Setting currently made via \tty{MACEXP\_DFT} may be read from the
predefined symbol \tty{MACEXP}.  For backward compatibility reasons,
it is possible to use the statement \tty{MACEXP} instead of
\tty{MACEXP\_DFT}.  However, one should not make use of this in new
programs.

%%- - - - - - - - - - - - - - - - - - - - - - - - - - - - - - - - - - - - - -

\subsection{LISTING}
\ttindex{LISTING}

works like \tty{MACEXP} and accepts the same parameters, but is much more
radical: After a
\begin{verbatim}
        listing off   ,
\end{verbatim}
nothing at all will be written to the listing.  This directive makes sense
for tested code parts or include files to avoid a paper consumption going
beyond all bounds.  \bb{CAUTION!} If one forgets to issue the counterpart
somewhere later, even the symbol table will not be written any more!  In
addition to \tty{ON} and \tty{OFF}, \tty{LISTING} also accepts
\tty{NOSKIPPED} and \tty{PURECODE} as arguments.  Program parts that were
not assembled due to conditional assembly will not be written to the
listing when \tty{NOSKIPPED} is set, while \tty{PURECODE} - as the name
indicates - even suppresses the \tty{IF} directives themselves in the
listing.  These options are useful if one uses macros that act differently
depending on parameters and one only wants to see the used parts in the
listing.

The current setting may be read from the symbol \tty{LISTING} (0=\tty{OFF},
1=\tty{ON}, 2=\tty{NOSKIPPED}, 3=\tty{PURECODE}).

%%- - - - - - - - - - - - - - - - - - - - - - - - - - - - - - - - - - - - - -

\subsection{PRTINIT and PRTEXIT}
\ttindex{PRTINIT}\ttindex{PRTEXIT}

Quite often it makes sense to switch to another printing mode (like
compressed printing) when the listing is sent to a printer and to
deactivate this mode again at the end of the listing.  The output of
the needed control sequences can be automated with these instructions
if one specifies the sequence that shall be sent to the output device
prior to the listing with \tty{PRTINIT $<$string$>$} and similarly the
deinitialization string with \tty{PRTEXIT $<$string$>$}.
\tty{$<$string$>$} has to be a string expression in both cases.  The syntax
rules for string constants allow to insert control characters into the
string without too much tweaking.

When writing the listing, the assembler does \bb{not} differentiate where
the listing actually goes, i.e. printer control characters are sent to the
screen without mercy!

Example:

For Epson printers, it makes sense to switch them to compressed
printing because listings are so wide.  The lines
\begin{verbatim}
        prtinit "\15"
        prtexit "\18"
\end{verbatim}
assure that the compressed mode is turned on at the beginning of the
listing and turned off afterwards.

%%- - - - - - - - - - - - - - - - - - - - - - - - - - - - - - - - - - - - - -

\subsection{TITLE}
\ttindex{TITLE}

The assembler normally adds a header line to each page of the listing
that contains the source file's name, date, and time.  This
statement allows to extend the page header by an arbitrary additional
line.  The string that has to be specified is an arbitrary string
expression.

Example:

For the Epson printer already mentioned above, a title line shall be
written in wide mode, which makes it necessary to turn off the
compressed mode before:
\begin{verbatim}
        title   "\18\14Wide Title\15"
\end{verbatim}
(Epson printers automatically turn off the wide mode at the end of a
line.)

%%- - - - - - - - - - - - - - - - - - - - - - - - - - - - - - - - - - - - - -

\subsection{RADIX}
\ttindex{RADIX}

\tty{RADIX} with a numerical argument between 2 and 36 sets the default
numbering system for integer constants, i.e. the numbering system used if
nothing else has been stated explicitly.  The default is 10, and there are
some possible pitfalls to keep in mind which are described in section
\ref{SectIntConsts}.

Independent of the current setting, the argument of {\tt RADIX} is {\em
always decimal}; furthermore, no symbolic or formula expressions may be
used as argument. Only use simple constant numbers!

%%- - - - - - - - - - - - - - - - - - - - - - - - - - - - - - - - - - - - - -

\subsection{OUTRADIX}
\ttindex{OUTRADIX}

\tty{OUTRADIX} can in a certain way be regarded as the opposite to
\tty{RADIX}: This statement allows to configure which numbering system to
use for integer results when \verb!\{...}! constructs are used in string
constants (see section \ref{SectStringConsts}).  Valid arguments range
again from 2 to 36, while the default is 16.

%%---------------------------------------------------------------------------

\section{Local Symbols}
\label{ChapLocSyms}

{\em valid for: all processors}

local symbols and the section concept introduced with them are a
completely new function that was introduced with version 1.39.  One
could say that this part is version ''1.0'' and therefore probably not
the optimum.  Ideas and (constructive) criticism are therefore
especially wanted.  I admittedly described the usage of sections how
I imagined it.  It is therefore possible that the reality is not
entirely equal to the model in my head.  I promise that in case of
discrepancies, changes will occur that the reality gets adapted to
the documentation and not vice versa (I was told that the latter
sometimes takes place in larger companies...).

AS does not generate linkable code (and this will probably not change
in the near future \tty{:-(}).  This fact forces one to always assemble a
program in a whole.  In contrast to this technique, a separation into
linkable modules would have several advantages:
\begin{itemize}
\item{shorter assembly times as only the modified modules have to be
      reassembled;}
\item{the option to set up defined interfaces among modules by definition
      of private and public symbols;}
\item{the smaller length of the individual modules reduces the number of
      symbols per module and therefore allows to use shorter symbol names
      that are still unique.}
\end{itemize}
Especially the last item was something that always nagged me: once
there was a label's name defined at the beginning of a 2000-lines
program, there was no way to reuse it somehow - even not at the
file's other end where routines with a completely different context
were placed.  I was forced to use concatenated names in the style of
\begin{verbatim}
   <subprogram name>_<symbol name>
\end{verbatim}
that had lengths ranging from 15 to 25 characters and made the
program difficult to overlook.  The concept of section described in
detail in the following text was designed to cure at least the second
and third item of the list above.  It is completely optional: if you
do not want to use sections, simply forget them and continue to work
like you did with previous versions of AS.

%%- - - - - - - - - - - - - - - - - - - - - - - - - - - - - - - - - - - - - -

\subsection{Basic Definition (SECTION/ENDSECTION)}
\ttindex{SECTION}\ttindex{ENDSECTION}

A section represents a part of the assembler program enclosed by
special statements and has a unique name chosen by the programmer:
\begin{verbatim}
        .
        .
        <other code>
        .
        .
        SECTION <section's name>
        .
        .
        <code inside of the section>
        .
        .
        ENDSECTION [section's name]
        .
        .
        <other code>
        .
        .
\end{verbatim}
The name of a section must conform to the conventions for s symbol
name; AS stores section and symbol names in separate tables which is
the reason why a name may be used for a symbol and a section at the
same time.  Section names must be unique in a sense that there must
not be more than one section on the same level with the same name (I
will explain in the next part what ''levels'' mean).  The argument of
\tty{ENDSECTION} is optional, it may also be omitted; if it is omitted, AS
will show the section's name that has been closed with this
\tty{ENDSECTION}.  Code inside a section will be processed by AS exactly
as if it were outside, except for three decisive differences:
\begin{itemize}
\item{Symbols defined within a section additionally get an internally
      generated number that corresponds to the section.  These symbols
      are not accessible by code outside the section (this can be
      changed by pseudo instructions, later more about this).}
\item{The additional attribute allows to define symbols of the same
      name inside and outside the section; the attribute makes it
      possible to use a symbol name multiple times without getting error
      messages from AS.}
\item{If a symbol of a certain name has been defined inside and outside
      of a section, the ''local'' one will be preferred inside the
      section, i.e. AS first searches the symbol table for a symbol of
      the referenced name that also was assigned to the section.  A
      search for a global symbol of this name only takes place if the
      first search fails.}
\end{itemize}
This mechanism e.g. allows to split the code into modules as one
might have done it with linkable code.  A more fine-grained approach
would be to pack every routine into a separate section.  Depending on
the individual routines' lengths, the symbols for internal use may
obtain very short names.

AS will by default not differentiate between upper and lower case in
section names; if one however switches to case-sensitive mode, the
case will be regarded just like for symbols.

The organization described up to now roughly corresponds to what is
possible in the C language that places all functions on the same
level.  However, as my ''high-level'' ideal was Pascal and not C, I
went one step further:

%%- - - - - - - - - - - - - - - - - - - - - - - - - - - - - - - - - - - - - -

\subsection{Nesting and Scope Rules}

It is valid to define further sections within a section.  This is
analog to the option given in Pascal to define procedures inside a
procedure or function.  The following example shows this:
\begin{verbatim}
sym     EQU        0

        SECTION    ModuleA

         SECTION    ProcA1

sym       EQU        5

         ENDSECTION ProcA1

         SECTION    ProcA2

sym       EQU        10

         ENDSECTION ProcA2

        ENDSECTION ModuleA


        SECTION    ModuleB

sym      EQU        15

         SECTION    ProcB

         ENDSECTION ProcB

        ENDSECTION ModuleB
\end{verbatim}
When looking up a symbol, AS first searches for a symbol assigned to
the current section, and afterwards traverses the list of parent
sections until the global symbols are reached.  In our example, the
individual sections see the values given in table \ref{TabSymErg} for
the symbol \tty{sym}:
\begin{table*}[htb]
\begin{center}\begin{tabular}{|l|l|l|}
\hline
section        &   value &   from section... \\
\hline
\hline
Global         &     0   &   Global \\
\hline
\tty{ModuleA}  &     0   &   Global \\
\hline
\tty{ProcA1}   &     5   &   \tty{ProcA1} \\
\hline
\tty{ProcA2}   &    10   &   \tty{ProcA2} \\
\hline
\tty{ModuleB}  &    15   &   \tty{ModuleB} \\
\hline
\tty{ProcB}    &    15   &   \tty{ModuleB} \\
\hline
\end{tabular}\end{center}
\caption{Valid values for the Individual Sections\label{TabSymErg}}
\end{table*}
This rule can be overridden by explicitly appending a section's name
to the symbol's name.  The section's name has to be enclosed in
brackets:
\begin{verbatim}
        move.l  #sym[ModulB],d0
\end{verbatim}
Only sections that are in the parent section path of the current
section may be used.  The special values \tty{PARENT0..PARENT9} are allowed
to reference the n-th ''parent'' of the current section; \tty{PARENT0} is
therefore equivalent to the current section itself, \tty{PARENT1} the
direct parent and so on.  \tty{PARENT1} may be abbreviated as \tty{PARENT}.  If
no name is given between the brackets, like in this example:
\begin{verbatim}
        move.l  #sym[],d0 ,
\end{verbatim}
one reaches the global symbol.  \bb{CAUTION!}  If one explicitly
references a symbol from a certain section, AS will only seek for
symbols from this section, i.e. the traversal of the parent sections
path is omitted!

Similar to Pascal, it is allowed that different sections have
subsections of the same name; the principle of locality avoids
irritations.  One should IMHO still use this feature as seldom as
possible: Symbols listed in the symbol resp. cross reference list are
only marked with the section they are assigned to, not with the
''section hierarchy'' lying above them (this really would have busted
the available space); a differentiation is made very difficult this
way.

As a \tty{SECTION} instruction does not define a label by itself, the
section concept has an important difference to Pascal's concept of
nested procedures: a pascal procedure can automatically ''see'' its
subprocedures(functions), AS requires an explicit definition of an
entry point.  This can be done e.g. with the following macro pair:
\begin{verbatim}
proc    MACRO   name
        SECTION name
name    LABEL   $
        ENDM

endp    MACRO   name
        ENDSECTION name
        ENDM
\end{verbatim}
This example also shows that the locality of labels inside macros
is not influenced by sections.  It makes the trick with the \tty{LABEL}
instruction necessary.

This does of course not solve the problem completely.  The label is
still local and not referencable from the outside.  Those who think
that it would suffice to place the label in front of the \tty{SECTION}
statement should be quiet because they would spoil the bridge to the
next theme:

%%- - - - - - - - - - - - - - - - - - - - - - - - - - - - - - - - - - - - - -

\subsection{PUBLIC and GLOBAL}
\ttindex{PUBLIC}\ttindex{GLOBAL}

The \tty{PUBLIC} statement allows to change the assignment of a symbol to
a certain section.  It is possible to treat multiple symbols with one
statement, but I will use an example with only one symbol in the following
(not hurting the generality of this discussion).  In the simplest case,
one declares a symbol to be global, i.e. it can be referenced from
anywhere in the program:
\begin{verbatim}
        PUBLIC  <name>
\end{verbatim}
As a symbol cannot be moved in the symbol table once it has been sorted
in, this statement has to appear \bb{before} the symbol itself is
defined.  AS stores all \tty{PUBLICs} in a list and removes an entry from
this list when the corresponding symbol is defined.  AS prints errors at
the end of a section in case that not all \tty{PUBLICs} have been
resolved.

Regarding the hierarchical section concept, the method of defining a
symbol as purely global looks extremely brute.  There is fortunately
a way to do this in a bit more differentiated way: by appending a
section name:
\begin{verbatim}
        PUBLIC  <name>:<section>
\end{verbatim}
The symbol will be assigned to the referenced section and therefore also
becomes accessible for all its subsections (except they define a symbol of
the same name that hides the ''more global'' symbol).  AS will naturally
protest if several subsections try to export a symbol of same name to the
same level.  The special \tty{PARENTn} values mentioned in the previous
section are also valid for \tty{$<$section$>$} to export a symbol exactly
\tty{n} levels up in the section hierarchy.  Otherwise only sections that
are parent sections of the current section are valid for
\tty{$<$section$>$}.  Sections that are in another part of the section
tree are not allowed.  If several sections in the parent section path
should have the same name (this is possible), the lowest level will be
taken.

This tool lets the abovementioned macro become useful:
\begin{verbatim}
proc    MACRO   name
        SECTION name
        PUBLIC  name:PARENT
name    LABEL   $
        ENDM
\end{verbatim}
This setting is equal to the Pascal model that also only allows the
''father'' to see its children, but not the ''grandpa''.

AS will quarrel about double-defined symbols if more than one section
attempts to export a symbol of a certain name to the same upper section.
This is by itself a correct reaction, and one needs to ''qualify'' symbols
somehow to make them distinguishable if these exports were deliberate.  A
\tty{GLOBAL} statement does just this.  The syntax of \tty{GLOBAL} is
identical to \tty{PUBLIC}, but the symbol stays local instead of being
assigned to a higher section.  Instead, an additional symbol of the same
value but with the subsection's name appended to the symbol's name is
created, and only this symbol is made public according to the section
specification.  If for example two sections \tty{A} and \tty{B} both
define a symbol named \tty{SYM} and export it with a \tty{GLOBAL}
statement to their parent section, the symbols are sorted in under the
names \tty{A\_SYM} resp. \tty{B\_SYM} .

In case that source and target section are separated by more than one
level, the complete name path is prepended to the symbol name.

%%- - - - - - - - - - - - - - - - - - - - - - - - - - - - - - - - - - - - - -

\subsection{FORWARD}
\ttindex{FORWARD}

The model described so far may look beautiful, but there is an
additional detail not present in Pascal that may spoil the happiness:
Assembler allows forward references.  Forward references may lead to
situations where AS accesses a symbol from a higher section in the
first pass.  This is not a disaster by itself as long as the correct
symbol is used in the second pass, but accidents of the following
type may happen:
\begin{verbatim}
loop:   .
        <code>
        .
        .
        SECTION sub
        .               ; ***
        .
        bra.s   loop
        .
        .
loop:   .
        .
        ENDSECTION
        .
        .
        jmp     loop    ; main loop
\end{verbatim}
AS will take the global label \tty{loop} in the first pass and will
quarrel about an out-of-branch situation if the program part at
\tty{$<$code$>$} is long enough.  The second pass will not be
started at all.  One way to avoid the ambiguity would be to
explicitly specify the symbol's section:
\begin{verbatim}
        bra.s   loop[sub]
\end{verbatim}
If a local symbol is referenced several times, the brackets can be saved
by using a \tty{FORWARD} statement.  The symbol is thereby explicitly
announced to be local, and AS will only look in the local symbol table
part when this symbol is referenced.  For our example, the statement
\begin{verbatim}
        FORWARD loop
\end{verbatim}
should be placed at the position marked with \tty{***}.

\tty{FORWARD} must not only be stated prior to a symbol's definition, but
also prior to its first usage in a section to make sense.  It does not
make sense to define a symbol private and public; this will be regarded as
an error by AS.

%%- - - - - - - - - - - - - - - - - - - - - - - - - - - - - - - - - - - - - -

\subsection{Performance Aspects}

The multi-stage lookup in the symbol table and the decision to which
section a symbol shall be assigned of course cost a bit of time to
compute.  An 8086 program of 1800 lines length for example took 34.5
instead of 33 seconds after a modification to use sections (80386 SX,
16MHz, 3 passes).  The overhead is therefore limited.  As it has
already been stated at the beginning, is is up to the programmer if
(s)he wants to accept it.  One can still use AS without sections.

%%---------------------------------------------------------------------------

\section{Miscellaneous}

%%- - - - - - - - - - - - - - - - - - - - - - - - - - - - - - - - - - - - - -

\subsection{SHARED}
\label{ChapShareOrder}
\ttindex{SHARED}

{\em valid for: all processors}

This statement instructs AS to write the symbols given in the
parameter list (regardless if they are integer, float or string
symbols) together with their values into the share file.  It depends
upon the command line parameters described in section
\ref{SectCallConvention} whether such a file is generated at all and in
which format it is written.  If AS detects this instruction and no share
file is generated, a warning is the result.

\bb{CAUTION!}  A comment possibly appended to the statement itself will be
copied to the first line outputted to the share file (if \tty{SHARED}'s
argument list is empty, only the comment will be written).  In case a
share file is written in C or Pascal format, one has to assure that
the comment itself does not contain character sequences that close
the comment (''*/'' resp. ''*)'').  AS does not check for this!

%%- - - - - - - - - - - - - - - - - - - - - - - - - - - - - - - - - - - - - -

\subsection{INCLUDE}
\label{SectInclude}
\ttindex{INCLUDE}

{\em valid for: all processors}

This instruction inserts the file given as a parameter into the just as
if it would have been inserted with an editor (the file name may
optionally be enclosed with '' characters).  This instruction is
useful to split source files that would otherwise not fit into the
editor or to create ''tool boxes''.

In case that the file name does not have an extension, it will
automatically be extended with \tty{INC}.

The assmebler primarily tries to open the file in the directory
containing the source file with the \tty{INCLUDE} statenemt.  This
means that a path contained in the file specification is relative
to this file's directory, not to the directory the assembler was
called from.  Via the \tty{-i $<$path list$>$} option, one can
specify a list of directories that will automatically be searched
for the file.  If the file is not found, a \bb{fatal} error occurs,
i.e. assembly terminates immediately.

For compatibility reasons, it is valid to enclose the file name in ''
characters, i.e.
\begin{verbatim}
        include stddef51
\end{verbatim}
and
\begin{verbatim}
        include "stddef51.inc"
\end{verbatim}
are equivalent.  \bb{CAUTION!} This freedom of choice is the reason why
only a string constant but no string expression is allowed!

The search list is ignored if the file name itself contains a path
specification.

%%- - - - - - - - - - - - - - - - - - - - - - - - - - - - - - - - - - - - - -

\subsection{BINCLUDE}
\ttindex{BINCLUDE}

{\em valid for: all processors}

\tty{BINCLUDE} can be used to embed binary data generated by other programs
into the code generated by AS (this might theoretically even be code
created by AS itself...).  \tty{BINCLUDE} has three forms:
\begin{verbatim}
        BINCLUDE <file>
\end{verbatim}
This way, the file is completely included.
\begin{verbatim}
        BINCLUDE <file>,<offset>
\end{verbatim}
This way, the file's contents are included starting at \tty{<offset>} up to
the file's end.
\begin{verbatim}
        BINCLUDE <file>,<offset>,<length>
\end{verbatim}
This way, \tty{$<$length$>$} bytes are included starting at
\tty{$<$offset$>$}.

The same rules regarding search paths apply as for \tty{INCLUDE}.

%%- - - - - - - - - - - - - - - - - - - - - - - - - - - - - - - - - - - - - -

\subsection{MESSAGE, WARNING, ERROR, and FATAL}
\ttindex{MESSAGE}\ttindex{WARNING}\ttindex{ERROR}\ttindex{FATAL}
{\em valid for: all processors}

Though the assembler checks source files as strict as possible and
delivers differentiated error messages, it might be necessary from
time to time to issue additional error messages that allow an
automatic check for logical error.  The assembler distinguishes
among three different types of error messages that are accessible to
the programmer via the following three instructions:
\begin{itemize}
\item{\tty{WARNING}: Errors that hint at possibly wrong or inefficient
      code.  Assembly continues and a code file is generated.}
\item{\tty{ERROR}: True errors in a program.  Assembly continues to
      allow detection of possible further errors in the same pass.
      A code file is not generated.}
\item{\tty{FATAL}: Serious errors that force an immediate termination
      of assembly.  A code file may be generated but will be incomplete.}
\end{itemize}
All three instructions have the same format for the message that shall
be issued: an arbitrary (possibly computed?!) string expression which
may therefore be either a constant or variable.

These instructions generally only make sense in conjunction wit
conditional assembly.  For example, if there is only a limited
address space for a program, one can test for overflow in the
following way:
\begin{verbatim}
ROMSize equ     8000h   ; 27256 EPROM

ProgStart:
        .
        .
        <the program itself>
        .
        .
ProgEnd:

        if      ProgEnd-ProgStart>ROMSize
         error  "\athe program is too long!"
        endif
\end{verbatim}
Apart from the instructions generating errors, there is also an
instruction \tty{MESSAGE} that simply prints a message to the assembly
listing and th ecosole (the latter only if the quiet mode is not used).
Its usage is equal to the other three instructions.

%%- - - - - - - - - - - - - - - - - - - - - - - - - - - - - - - - - - - - - -

\subsection{READ}
\ttindex{READ}

{\em valid for: all processors}

One could say that \tty{READ} is the counterpart to the previous
instruction group: it allows to read values from the keyboard during
assembly.  You might ask what this is good for.  I will break with
the previous principles and put an example before the exact
description to outline the usefulness of this instruction:

A program needs for data transfers a buffer of a size that should be
set at assembly time.  One could store this size in a symbol defined
with \tty{EQU}, but it can also be done interactively with \tty{READ}:
\begin{verbatim}
        IF      MomPass=1
         READ    "buffer size",BufferSize
        ENDIF
\end{verbatim}
Programs can this way configure themselves dynamically during assembly
and one could hand over the source to someone who can assemble it
without having to dive into the source code.  The \tty{IF} conditional
shown in the example should always be used to avoid bothering the
user multiple times with questions.

\tty{READ} is quite similar to \tty{SET} with the difference that the
value is read from the keyboard instead of the instruction's arguments.
This for example also implies that AS will automatically set the symbol's
type (integer, float or string) or that it is valid to enter formula
expressions instead of a simple constant.

\tty{READ} may either have one or two parameters because the prompting
message is optional.  AS will print a message constructed from the
symbol's name if it is omitted.

%%- - - - - - - - - - - - - - - - - - - - - - - - - - - - - - - - - - - - - -

\subsection{INTSYNTAX}
\label{SectINTSYNTAX}
\ttindex{INTSYNTAX}

{\em valid for: all processors}

This instruction allows to modify the set of notations for integer constants
in various number systems. - After selection of a CPU target, a certain default
set is instelled (see section \ref{SectPseudoInst}).  This set may be augmented
with other notations, or notations may be removeved from it.  \tty{INTSYNTAX}
takes an arbitrary list of arguments which either begin with aplus or minus
character, followed by the notation's identifier.  For instance, the following
statement
\begin{verbatim}
       intsyntax    -0oct,+0hex
\end{verbatim}
has the result that a leading zero marks a hexadecimal instead of an octal
constant, a common usage on some assemblers for the SC/MP.  The notations'
identifiers can be found in table \ref{TabSystems}.  There is no limit on
combining notations, except when they contradict each other.  For instance, it
would not be allowed to enable \tty{0oct} and \tty{0hex} at the same time.

%%- - - - - - - - - - - - - - - - - - - - - - - - - - - - - - - - - - - - - -

\subsection{RELAXED}
\label{SectRELAXED}
\ttindex{RELAXED}

{\em valid for: all processors}

By default, AS assigns a distinct syntax for integer constants to a
processor family (which is in general equal to the manufacturer's
specifications, as long as the syntax is not too bizarre...).
Everyone however has his own preferences for another syntax and may
well live with the fact that his programs cannot be translated any
more with the standard assembler.  If one places the instruction
\begin{verbatim}
        RELAXED ON
\end{verbatim}
right at the program's beginning, one may furtherly use any syntax
for integer constants, even mixed in a program.  AS tries to guess
automatically for every expression the syntax that was used.  This
automatism does not always deliver the result one might have in mind,
and this is also the reason why this option has to be enable
explicitly: if there are no prefixes or postfixes that unambiguously
identify either Intel or Motorola syntax, the C mode will be used.
Leading zeroes that are superfluous in other modes have a meaning in
this mode:
\begin{verbatim}
        move.b  #08,d0
\end{verbatim}
This constant will be understood as an octal constant and will result
in an error message as octal numbers may only contain digits from 0
to 7.  One might call this a lucky case; a number like 077 would
result in trouble without getting a message about this.  Without the
relaxed mode, both expressions unambiguously would have been
identified as decimal constants.

The current setting may be read from a symbol with the same name.

%%- - - - - - - - - - - - - - - - - - - - - - - - - - - - - - - - - - - - - -

\subsection{COMPMODE}
\label{SectCompMode}
\ttindex{COMPMODE}

{\em valid for: various processors}

Though the assember strives to behave like the correspondig "original
assemblers", there are cases when emulating the original assembler's
behaviour would forbid code optimizations which are valid and useful in
my opinion.  Use the statement
\begin{verbatim}
        compmode on
\end{verbatim}
to switch to a 'compatibility mode' which prioritizes 'original behaviour'
to most efficient code.  See the respective section with processor-specific
hints whether there are any situations for the specific target.
\par
Compatibility mode is disabled by default, unless it was activated by the
command line switch of same name.  The current setting may be read from a
symbol with the same name.

%%- - - - - - - - - - - - - - - - - - - - - - - - - - - - - - - - - - - - - -

\subsection{END}
\ttindex{END}

{\em valid for: all processors}

\tty{END} marks the end of an assembler program.  Lines that eventually
follow in the source file will be ignored.  \bb{IMPORTANT:} \tty{END} may
be called from within a macro, but the \tty{IF}-stack for conditional
assembly is not cleared automatically.  The following construct therefore
results in an error:
\begin{verbatim}
        IF      DontWantAnymore
         END
        ELSEIF
\end{verbatim}
\tty{END} may optionally have an integer expression as argument that marks
the program's entry point.  AS stores this in the code file with a special
record and it may be post-processed e.g. with P2HEX.

\tty{END} has always been a valid instruction for AS, but the only reason
for this in earlier releases of AS was compatibility; \tty{END} had no
effect.

%%===========================================================================

\cleardoublepage
\chapter{Processor-specific Hints}

When writing the individual code generators, I strived for a maximum
amount of compatibility to the original assemblers.  However, I only did this
as long as it did not mean an unacceptable additional amount of work.
I listed important differences, details and pitfalls in the following
chapter.

%%---------------------------------------------------------------------------

\section{6811}

''Where can I buy such a beast, a HC11 in NMOS?'', some of you might
ask.  Well, of course it does not exist, but an H cannot be
represented in a hexadecimal number (older versions of AS would not
have accepted such a name because of this), and so I decided to omit
all the letters...
\par
\begin{quote}{\it
''Someone stating that something is impossible should be at least as
 cooperative as not to hinder the one who currently does it.''
}\end{quote}
From time to time, one is forced to revise one's opinions.  Some versions
earlier, I stated at his place that I couldn't use AS's parser in a way
that it is also possible to to separate the arguments of \tty{BSET/BCLR}
resp. \tty{BRSET/BRCLR} with spaces.  However, it seems that it can do
more than I wanted to believe...after the n+1th request, I sat down once
again to work on it and things seem to work now.  You may use either
spaces or commas, but not in all variants, to avoid ambiguities: for
every variant of an instruction, it is possible to use only commas or a
mixture of spaces and commas as Motorola seems to have defined it (their
data books do not always have the quality of the corresponding
hardware...):
\begin{verbatim}
 Bxxx  abs8 #mask         is equal to Bxxx  abs8,#mask
 Bxxx  disp8,X #mask      is equal to Bxxx  disp8,X,#mask
 BRxxx abs8 #mask addr    is equal to BRxxx abs8,#mask,addr
 BRxxx disp8,X #mask addr is equal to BRxxx disp8,X,#mask,addr
\end{verbatim}
In this list, \tty{xxx} is a synonym either for \tty{SET} or \tty{CLR};
\tty{\#mask} is the bit mask to be applied (the \# sign is optional).  Of
course, the same statements are also valid for Y-indexed expression (not
listed here).

With the K4 version  of the HC11, Motorola has introduced a banking
scheme, which one one hand easily allows to once again extend an
architecture that has become 'too small', but on the other hand not really
makes programmers' and tool developers' lifes simpler...how does one
sensibly map something like this on a model for a programmer?

The K4 architecture {\em extends} the HC11 address space by 2x512 Kbytes,
which means that we now have a total address space of 64+1024=1088 Kbytes.
AS acts like this were one large unified addres space, with the following
layout:
\begin{itemize}
\item{\$000000...\$00ffff: the old HC11 address space}
\item{\$010000...\$08ffff: Window 1}
\item{\$090000...\$10ffff: Window 2}
\end{itemize}
Via the {\tt ASSUME} statement, one tells AS how the banking registers are
set up, which in turn describes which extended areas are mapped to which
physical addresses.  For absolute addresses modes with addresses beyond
\$10000, AS automatically computes the address within the first 64K that
is to be used.  Of course this only works for direct addressing modes, it
is the programmer's responsibility to keep the overview for indirect or
indexed addressing modes!

In case one is not really sure if the current mapping is really the
desired one, the pseudo instruction {\tt PRWINS} may be used, which prints
the assumes MMxxx register contents plus the current mapping(s), like
this:
\begin{verbatim}
MMSIZ $e1 MMWBR $84 MM1CR $00 MM2CR $80
Window 1: 10000...12000 --> 4000...6000
Window 1: 90000...94000 --> 8000...c000
\end{verbatim}
An instruction
\begin{verbatim}
        jmp     *+3
\end{verbatim}
located at \$10000 would effectively result in a jump to address \$4003.

%%---------------------------------------------------------------------------

\section{PowerPC}

Of course, it is a bit crazy idea to add support in AS for a
processor that was mostly designed for usage in work stations.
Remember that AS mainly is targeted at programmers of single board
computers.  But things that today represent the absolute high end in
computing will be average tomorrow and maybe obsolete the next day,
and in the meantime, the Z80 as the 8088 have been retired as CPUs
for personal computers and been moved to the embedded market;
modified versions are marketed as microcontrollers.  With the
appearance of the MPC505 and PPC403, my suspicion has proven to be
true that IBM and Motorola try to promote this architecture in as
many fields as possible.

However, the current support is a bit incomplete: Temporarily, the
Intel-style mnemonics are used to allow storage of data and the more
uncommon RS/6000 machine instructions mentioned in \cite{Mot601} are
missing (hopefully noone misses them!).  I will finish this as soon
as information about them is available!

%%---------------------------------------------------------------------------

\section{DSP56xxx}

Motorola, which devil rode you!  Which person in your company had the
''brilliant'' idea to separate the parallel data transfers with spaces!
In result, everyone who wants to make his code a bit more readable,
e.g. like this:
\begin{verbatim}
        move    x:var9 ,r0
        move    y:var10,r3   ,
\end{verbatim}
is p****ed because the space gets recognized as a separator for
parallel data transfers!

Well...Motorola defined it that way, and I cannot change it.  Using
tabs instead of spaces to separate the parallel operations is also
allowed, and the individual operations' parts are again separated
with commas, as one would expect it.

\cite{Mot56} states that instead of using \tty{MOVEC, MOVEM, ANDI} or
\tty{ORI}, it is also valid to use the more general Mnemonics \tty{MODE,
AND} or \tty{OR}.
AS (currently) does not support this.

%%---------------------------------------------------------------------------

\section{H8/300}

Regarding the assembler syntax of these processors, Hitachi generously
copied from Motorola (that wasn't by far the worst choice...),
unfortunately the company wanted to introduce its own format for
hexadecimal numbers.  To make it even worse, it is a format that uses
unbalanced single quotes, just like Microchip does:
\begin{verbatim}
   mov.w #h'ff,r0
\end{verbatim}
This format is not supported by default.  Instead, one has to write
hexadecimal numbers in the well-known Motorola syntax: with a leading
dollar sign.  If you really need the 'Hitachi Syntax', e.g. to assemble
existing code, enable the RELAXED mode.  Bear in mind that this syntax has
received few testing so far.  I can therefore not guarantee that it will
work in all cases!

%%---------------------------------------------------------------------------

\section{H8/500}

The H8/500's {\tt MOV} instruction features an interesting and uncommon
optimization: If the target operand has a size of 16 bits, it is still possible
to use an 8-bit (immediate) source operand.  For example, for an instruction
like this:
\begin{verbatim}
   mov.w #$ffff,@$1234
\end{verbatim}
it is possible to encode the source just as a single \$ff and to save one
byte in code size.  The processor automatically performs a sign extension,
which turns \$ff into the desired value \$ffff.  AS is awarte of this
optimization and will use it, unless it was explicitly forbidden via a \tty{:16}
suffix at the immediate operand.

Unfortunately, the original Hitachi assembler seems to implement this optimization
in another way: it assumes a zero instead of a sign extension.  This means that
values from 0 to 255 (\$0000 to \$00ff) and not from -128 to +127 (\$ff80 to
\$007f) are encoded as one byte.  I do not know whether the Hitachi assembler
or the H8/500 Programmers Manual is right, and I have no way to verify this.
Anyway, if one wants the ''Hitachi Assembler Behaviour'' for existing source
code, one may enable the compatibility mode, either by the statement
\begin{verbatim}
  compmode on
\end{verbatim}
or by the respective command line switch.

Aside from this, the same remarks regarding hexadecimal number syntax apply
as for H8/500.

%%---------------------------------------------------------------------------

\section{SH7000/7600/7700}

Unfortunately, Hitachi once again used their own format for
hexadecimal numbers, and once again I was not able to reproduce this
with AS...please use Motorola syntax!

When using literals and the \tty{LTORG} instruction, a few things have to
be kept in mind if you do not want to suddenly get confronted with strange
error messages:

Literals exist due to the fact that the processor is unable to load
constants out of a range of -128 to 127 with immediate addressing.
AS (and the Hitachi assembler) hide this inability by the automatic
placement of constants in memory which are then referenced via
PC-relative addressing.  The question that now arises is where to
locate these constants in memory.  AS does not automatically place a
constant in memory when it is needed; instead, they are collected
until an LTORG instruction occurs.  The collected constants are then
dumped en bloc, and their addresses are stored in ordinary labels
which are also visible in the symbol table.  Such a label's name is
of the form
\begin{verbatim}
    LITERAL_s_xxxx_n  .
\end{verbatim}
In this name, \tty{s} represents the literal's type.  Possible values are
\tty{W} for 16-bit constants, \tty{L} for 32-bit constants and \tty{F} for
forward references where AS cannot decide in anticipation which size is
needed.  In case of \tty{s=W} or \tty{L}, \tty{xxxx} denotes the
constant's value in a hexadecimal notation, whereas \tty{xxxx} is a simple
running number for forward references (in a forward reference, one does
not know the value of a constant when it is referenced, so one obviously
cannot incorporate its value into the name).  \tty{n} is a counter that
signifies how often a literal of this value previously occurred in the
current section.  Literals follow the standard rules for localization by
sections.  It is therefore absolutely necessary to place literals that
were generated in a certain section before the section is terminated!

The numbering with \tty{n} is necessary because a literal may occur
multiple times in a section.  One reason for this situation is that
PC-relative addressing only allows positive offsets; Literals that
have once been placed with an \tty{LTORG} can therefore not be referenced
in the code that follows.  The other reason is that the displacement
is generally limited in length (512 resp. 1024 bytes).

An automatic \tty{LTORG} at the end of a program or previously to
switching to a different target CPU does not occur; if AS detects unplaced
literals in such a situation, an error message is printed.

As the PC-relative addressing mode uses the address of the current
instruction plus 4, it is not possible to access a literal that is
stored directly after the instruction, like in the following example:
\begin{verbatim}
        mov     #$1234,r6
        ltorg
\end{verbatim}
This is a minor item since the CPU anyway would try to execute the
following data as code.  Such a situation should not occur in a real
program...another pitfall is far more real: if PC-relative addressing
occurs just behind a delayed branch, the program counter is already
set to the destination address, and the displacement is computed
relative to the branch target plus 2.  Following is an example where
this detail leads to a literal that cannot be addressed:
\begin{verbatim}
        bra     Target
        mov     #$12345678,r4        ; is executed
        .
        .
        ltorg                        ; here is the literal
        .
        .
Target: mov     r4,r7                ; execution continues here
\end{verbatim}
As \tty{Target}+2 is on an address behind the literal, a negative
displacement would result.  Things become especially hairy when one
of the branch instructions \tty{JMP, JSR, BRAF, or BSRF} is used: as AS
cannot calculate the target address (it is generated at runtime from
a register's contents), a PC value is assumed that should never fit,
effectively disabling any PC-relative addressing at this point.

It is not possible to deduce the memory usage from the count and size
of literals.  AS might need to insert a padding word to align a long
word to an address that is evenly divisible by 4; on the other hand,
AS might reuse parts of a 32-bit literal for other 16-bit literals.
Of course multiple use of a literal with a certain value will create
only one entry.  However, such optimizations are completely
suppressed for forward references as AS does not know anything about
their value.

As literals use the PC-relative addressing which is only allowed for
the \tty{MOV} instruction, the usage of literals is also limited to
\tty{MOV} instructions.  The way AS uses the operand size is a bit tricky:
A specification of a byte or word move means to generate the shortest
possible instruction that results in the desired value placed in the
register's lowest 8 resp. 16 bits.  The upper 24 resp. 16 bits are treated
as ''don't care''.  However, if one specifies a longword move or omits the
size specification completely, this means that the complete 32-bit
register should contain the desired value.  For example, in the following
sequence
\begin{verbatim}
        mov.b   #$c0,r0
        mov.w   #$c0,r0
        mov.l   #$c0,r0   ,
\end{verbatim}
the first instruction will result in true immediate addressing, the
second and third instruction will use a word literal:  As bit 7 in
the number is set, the byte instruction will effectively create the
value \$FFFFFFC0 in the register.  According to the convention, this
wouldn't be the desired value in the second and third example.
However, a word literal is also sufficient for the third case because
the processor will copy a cleared bit 15 of the operand to bits
16..31.

As one can see, the whole literal stuff is rather complex; I'm sorry but
there was no chance of making things simpler.  It is unfortunately a
part of its nature that one sometimes gets error messages about
literals that were not found, which logically should not occur because
AS does the literal processing completely on his own.  However, if
other errors occur in the second pass, all following labels will move
because AS does not generate any code any more for statements that
have been identified as erroneous.  As literal names are partially built
from other symbols' values, other errors might follow because literal
names searched in the second pass differ from the names stored in the
first pass and AS quarrels about undefined symbols...if such errors
should occur, please correct all other errors first before you start
cursing on me and literals...

People who come out of the Motorola scene and want to use PC-relative
addressing explicitly (e.g. to address variables in a position-independent
way) should know that if this addressing mode is written like in the
programmer's manual:
\begin{verbatim}
        mov.l   @(Var,PC),r8
\end{verbatim}
\bb{no} implicit conversion of the address to a displacement will occur,
i.e. the operand is inserted as-is into the machine code (this will
probably generate a value range error...).  If you want to use
PC-relative addressing on the SH7x00, simply use ''absolute''
addressing (which does not exist on machine level):
\begin{verbatim}
        mov.l   Var,r8
\end{verbatim}
In this example, the displacement will be calculated correctly (of
course, the same limitations apply for the displacement as it was the
case for literals).

%%---------------------------------------------------------------------------

\section{HMCS400}

The instruction set of these 4 bit processors spontaneously reminded
me of the 8080/8085 - many mnemonics, the addressing mode (e.g.
direct or indirect) is coded into the instruction, and the
instructions are sometimes hard to memorize.  AS or course
supports this syntax as Hitachi defined it.  I however
implemented another variant for most instructions that is - in my
opinion - more beautiful and better to read.  The approach is
similar to what Zilog did back then for the Z80.  For instance,
all machine instructions that transfer data in some form, may the
operands be constants, registers, or memory cells, may be used
via the \tty{LD} instruction.  Similar 'meta instructions' exist
for arithmetic and logical instructions.  A complete list of all
meta instructions and their operands can be found in the tables
\ref{TabHMCS400Meta} and \ref{TabHMCS400MetaOps}, their practical
use can be seen in the file \tty{t\_hmcs4x.asm}.

\begin{table*}
\begin{center}\begin{tabular}{|l|l|}
\hline
Meta Instruction          & Replaces \\
\hline
\tty{LD} {\em src, dest}        & \tty{LAI, LBI, LMID, LMIIY,} \\
                                & \tty{LAB, LBA, LAY, LASPX, LASPY, LAMR,} \\
                                & \tty{LWI, LXI, LYI, LXA, LYA, LAM, LAMD} \\
                                & \tty{LBM, LMA, LMAD, LMAIY, LMADY} \\
\tty{XCH} {\em src, dest}       & \tty{XMRA, XSPX, XSPY, XMA, XMAD, XMB} \\
\tty{ADD} {\em src, dest}       & \tty{AYY, AI, AM, AMD} \\
\tty{ADC} {\em src, dest}       & \tty{AMC, AMCD} \\
\tty{SUB} {\em src, dest}       & \tty{SYY} \\
\tty{SBC} {\em src, dest}       & \tty{SMC, SMCD} \\
\tty{OR}  {\em src, dest}       & \tty{OR, ORM, ORMD} \\
\tty{AND} {\em src, dest}       & \tty{ANM, ANMD} \\
\tty{EOR} {\em src, dest}       & \tty{EORM, EORMD} \\
\tty{CP}  {\em cond, src, dest} & \tty{INEM, INEMD, ANEM, ANEMD, BNEM,} \\
                                & \tty{YNEI, ILEM, ILEMD, ALEM, ALEMD,} \\
                                & \tty{BLEM, ALEI} \\
\tty{BSET} {\em bit}            & \tty{SEC, SEM, SEMD} \\
\tty{BCLR} {\em bit}            & \tty{REC, REM, REMD} \\
\tty{BTST} {\em bit}            & \tty{TC, TM, TMD} \\
\hline
\end{tabular}\end{center}
\caption{Meta Instructions HMCS400}
\label{TabHMCS400Meta}
\end{table*}

\begin{table*}
\begin{center}\begin{tabular}{|l|l|}
\hline
Operand                 & Types \\
\hline
{\em src, dest}         & \tty{A, B, X, Y, W, SPX, SPY} (register) \\
                        & \tty{M} (memory addressed by X/Y/W) \\
                        & \tty{M+} (ditto, with auto increment) \\
                        & \tty{M-} (ditto, with auto decrement) \\
                        & \tty{\#val} (2/4 bits immediate) \\
                        & \tty{addr10} (memory direct) \\
                        & \tty{MRn} (memory register 0..15) \\
{\em cond}              & \tty{NE} (unequal) \\
                        & \tty{LE} (less or equal) \\
{\em bit}               & \tty{CA} (carry) \\
                        & {\em bitpos},\tty{M} \\
                        & {\em bitpos},\tty{addr10} \\
{\em bitpos}            & \tty{0..3} \\
\hline
\end{tabular}\end{center}
\caption{Operand Types for HMCS400 Meta Instructions}
\label{TabHMCS400MetaOps}
\end{table*}

%%---------------------------------------------------------------------------

\section{H16}

The instruction set of the H16's core well deserves the label ''CISC'': complex
addressing modes, instructions of extremely variable length, and
there are many shortforms for instructions with common operands.  For
instance, several instructions know different ''formats'', depending
on the type of source and destination operand.  The general rule is
that AS will always use the shortest possible format, unless it was
specified explicitly:
angegeben:
\begin{verbatim}
       mov.l     r4,r7     ; uses R format
       mov.l     #4,r7     ; uses RQ format
       mov.l     #4,@r7    ; uses Q format
       mov.l     @r4,@r7   ; uses G format
       mov:q.l   #4,r7     ; forces Q instead of RQ format
       mov:g.l   #4,r7     ; forces G instead of RQ format
\end{verbatim}
For immediate arguments, the ''natural' argument length is used, e.g.
2 bytes for 16 bits.  Shorter or longer arguments may be forced by an
appended operand size (.b, .w, .l or :8, :16, :32).  However, the
rule for displacements and absolute addresses is that the shortest
form will be used if no explicit size is given.  This includes
exploiting that the processor does not output the uppermost eight
bits of an address.  Therefore, an absolute address of \$ffff80 can
be coded as a single byte (\$80).

Furthermore, AS knows the ''accumulator bit'', i.e. the second
operand of a two-operand instruction my be left away if the
destination is register zero.  There is currently no override this
behaviour.

Additionally, the following optimizations are performed:
\begin{itemize}
\item{\tty{MOV R0,<ea>} gets optimized to \tty{MOVF <ea>}, unless
      \tty{<ea>} is a PC-relative expression and the size of the
      displacement would change.  This optimization may be disabled
      by specifying an explicit format.}
\item{\tty{SUB} does not support the Q format, however it may be
      replaced by \tty{ADD:Q} with a negated immediate argument,
      given the argument is in the range -127...+128.  This
      optimization may as well be disabled by specifying an explicit
      format.}
\end{itemize}

%%---------------------------------------------------------------------------

\section{OLMS-40}

Similar to the HMCS400, addressing modes are largely encoded (or
rather encrypted..) into into the mnemonics, and also here I
decided to provide an alternate notation that is more modern and
better to read.  A complete list of all meta instructions and
their operands can be found in the tables \ref{TabOLMS40Meta} and
\ref{TabOLMS40MetaOps}, their practical use can be seen in the
file \tty{t\_olms4.asm}.

\begin{table*}
\begin{center}\begin{tabular}{|l|l|}
\hline
Meta Instruction          & Replaces \\
\hline
\tty{LD} {\em dest, src}        & \tty{LAI, LLI, LHI, L,} \\
                                & \tty{LAL, LLA, LAW, LAX, LAY, LAZ,} \\
                                & \tty{LWA, LXA, LYA, LPA, LTI, RTH, RTL} \\
\tty{DEC} {\em dest}            & \tty{DCA, DCL, DCM, DCW, DCX, DCY, DCZ, DCH} \\
\tty{INC} {\em dest}            & \tty{INA, INL, INM, INW, INX, INY, INZ} \\
\tty{BSET} {\em bit}            & \tty{SPB, SMB, SC} \\
\tty{BCLR} {\em bit}            & \tty{RPB, RMB, RC} \\
\tty{BTST} {\em bit}            & \tty{TAB, TMB, Tc} \\
\hline
\end{tabular}\end{center}
\caption{Meta Instructions OLMS-40}
\label{TabOLMS40Meta}
\end{table*}

\begin{table*}
\begin{center}\begin{tabular}{|l|l|}
\hline
Operand                 & Types \\
\hline
{\em src, dest}         & \tty{A, W, X, Y, Z, DPL, DPH} (Register) \\
                        & \tty{T, TL, TH} (Timer, obere/untere H"alfte) \\
                        & \tty{(DP), M} (Speicher adressiert durch DPH/DPL) \\
                        & \tty{\#val} (4/8 bit immediate) \\
                        & \tty{PP} (Port-Pointer) \\
{\em bit}               & \tty{C} (Carry) \\
                        & \tty{(PP)},{\em bitpos} \\
                        & \tty{(DP)},{\em bitpos} \\
                        & \tty{(A)},{\em bitpos} \\
{\em bitpos}            & \tty{0..3} \\
\hline
\end{tabular}\end{center}
\caption{Operand Types for OLMS-40 Meta Instructions}
\label{TabOLMS40MetaOps}
\end{table*}

%%---------------------------------------------------------------------------

\section{OLMS-50}

The data memory of these 4 bit controllers consists of up to 128
nibbles.  However, only a very small subset of the machine
instructions have enough space to accomodate seven address bits,
which menas that - once again - banking must help out.  The
majority of instructions that address memory only contain the
lower four bits of the RAM address, and unless the lowest 16
nibbles of the memory shall be addressed, the P register delivers
the necessary upper address bits. The assembler is told about its
current value via an
\begin{verbatim}
   assume  p:<value>
\end{verbatim}
statement, e.g. directly after a \tty{PAGE} instruction.

Speaking of \tty{PAGE}: both \tty{PAGE} and \tty{SWITCH} are
machine instructions on these controllers, i.e. the do not have
the function known from other targets.  The pseudo instruction to
start a \tty{SWITCH/CASE} construct is \tty{SELECT} in OLMS-50
mode, and the listing's page size is set via \tty{PAGESIZE}.

%%---------------------------------------------------------------------------

\section{MELPS-4500}

The program memory of these microcontrollers is organized in pages of
128 words.  Honestly said, this organization only exists because there
are on the one hand branch instructions with a target that must lie
within the same page, and on the other hand ''long'' branches that can
reach the whole address space.  The standard syntax defined by
Mitsubishi demands that page number and offset have to be written as
two distinct arguments for the latter instructions.  As this is
quite inconvenient (except for indirect jumps, a programmer has no
other reason to deal with pages), AS also allows to write the target
address in a ''linear'' style, for example
\begin{verbatim}
        bl      $1234
\end{verbatim}
instead of
\begin{verbatim}
        bl      $24,$34 .
\end{verbatim}

%%---------------------------------------------------------------------------

\section{6502UNDOC}

Since the 6502's undocumented instructions naturally aren't listed in
any data book, they shall be listed shortly at this place.  Of
course, you are using them on your own risk.  There is no guarantee
that all mask revisions will support all variants!  They anyhow do
not work for the CMOS successors of the 6502, since they allocated
the corresponding bit combinations with "official" instructions...

The following symbols are used:

\begin{tabbing}
\hspace{2cm} \= \kill
\&           \> binary AND \\
|            \> binary OR \\
\verb!^!     \> binary XOR \\
$<<$         \> logical shift left \\
$>>$         \> logical shift right \\
$<<<$        \> rotate left \\
$>>>$        \> rotate right \\
$\leftarrow$ \> assignment \\
 (..)        \> contents of .. \\
 {..}        \> bits .. \\
 A           \> accumulator \\
 X,Y         \> index registers X,Y \\
 S           \> stack pointer \\
 An          \> accumulator bit n \\
 M           \> operand \\
 C           \> carry \\
 PCH         \> upper half of program counter \\
\end{tabbing}

\begin{tabbing}
Addressing Modes \= : \= \kill
Instruction      \> : \> \tty{JAM} or \tty{KIL} or \tty{CRS} \\
Function         \> : \> none, prozessor is halted \\
Addressing Modes \> : \> implicit \\
\end{tabbing}

\begin{tabbing}
Addressing Modes \= : \= \kill
Instruction      \> : \> \tty{SLO} \\
Function         \> : \> $M\leftarrow((M)<<1)|(A)$ \\
Addressing Modes \> : \> absolute long/short, X-indexed long/short, \\
                 \>   \> Y-indexed long, X/Y-indirect \\
\end{tabbing}

\begin{tabbing}
Addressing Modes \= : \= \kill
Instruction      \> : \> \tty{ANC} \\
Function         \> : \> $A\leftarrow(A)\&(M), C\leftarrow A7$ \\
Addressing Modes \> : \> immediate \\
\end{tabbing}

\begin{tabbing}
Addressing Modes \= : \= \kill
Instruction      \> : \> \tty{RLA} \\
Function         \> : \> $M\leftarrow((M)<<1)\&(A)$ \\
Addressing Modes \> : \> absolute long/short, X-indexed long/short, \\
                 \>   \> Y-indexed long, X/Y-indirect \\
\end{tabbing}

\begin{tabbing}
Addressing Modes \= : \= \kill
Instruction      \> : \> \tty{SRE} \\
Function         \> : \> $M\leftarrow((M)>>1)$\verb!^!$(A)$ \\
Addressing Modes \> : \> absolute long/short, X-indexed long/short, \\
                 \>   \> Y-indexed long, X/Y-indirect \\
\end{tabbing}

\begin{tabbing}
Addressing Modes \= : \= \kill
Instruction      \> : \> \tty{ASR} \\
Function         \> : \> $A\leftarrow((A)\&(M))>>1$ \\
Addressing Modes \> : \> immediate \\
\end{tabbing}

\begin{tabbing}
Addressing Modes \= : \= \kill
Instruction      \> : \> \tty{RRA} \\
Function         \> : \> $M\leftarrow((M)>>>1)+(A)+(C)$ \\
Addressing Modes \> : \> absolute long/short, X-indexed long/short, \\
                 \>   \> Y-indexed long, X/Y-indirect \\
\end{tabbing}

\begin{tabbing}
Addressing Modes \= : \= \kill
Instruction      \> : \> \tty{ARR} \\
Function         \> : \> $A\leftarrow((A)\&(M))>>>1$ \\
Addressing Modes \> : \> immediate \\
\end{tabbing}

\begin{tabbing}
Addressing Modes \= : \= \kill
Instruction      \> : \> \tty{SAX} \\
Function         \> : \> $M\leftarrow(A)\&(X)$ \\
Addressing Modes \> : \> absolute long/short, Y-indexed short, \\
                 \>   \> Y-indirect \\
\end{tabbing}

\begin{tabbing}
Addressing Modes \= : \= \kill
Instruction      \> : \> \tty{ANE} \\
Function         \> : \> $M\leftarrow((A)\&\$ee)|((X)\&(M))$ \\
Addressing Modes \> : \> immediate \\
\end{tabbing}

\begin{tabbing}
Addressing Modes \= : \= \kill
Instruction      \> : \> \tty{SHA} \\
Function         \> : \> $M\leftarrow(A)\&(X)\&(PCH+1)$ \\
Addressing Modes \> : \> X/Y-indexed long \\
\end{tabbing}

\begin{tabbing}
Addressing Modes \= : \= \kill
Instruction      \> : \> \tty{SHS} \\
Function         \> : \> $X\leftarrow(A)\&(X), S\leftarrow(X), M\leftarrow(X)\&(PCH+1)$ \\
Addressing Modes \> : \> Y-indexed long \\
\end{tabbing}

\begin{tabbing}
Addressing Modes \= : \= \kill
Instruction      \> : \> \tty{SHY} \\
Function         \> : \> $M\leftarrow(Y)\&(PCH+1)$ \\
Addressing Modes \> : \> Y-indexed long \\
\end{tabbing}

\begin{tabbing}
Addressing Modes \= : \= \kill
Instruction      \> : \> \tty{SHX} \\
Function         \> : \> $M\leftarrow(X)\&(PCH+1)$ \\
Addressing Modes \> : \> X-indexed long \\
\end{tabbing}

\begin{tabbing}
Addressing Modes \= : \= \kill
Instruction      \> : \> \tty{LAX} \\
Function         \> : \> $A,X\leftarrow(M)$ \\
Addressing Modes \> : \> absolute long/short, Y-indexed long/short, \\
                 \>   \> X/Y-indirect \\
\end{tabbing}

\begin{tabbing}
Addressing Modes \= : \= \kill
Instruction      \> : \> \tty{LXA} \\
Function         \> : \> $X{04}\leftarrow(X){04}\&(M){04}$, \\
                 \>   \> $A{04}\leftarrow(A){04}\&(M){04}$ \\
Addressing Modes \> : \> immediate \\
\end{tabbing}

\begin{tabbing}
Addressing Modes \= : \= \kill
Instruction      \> : \> \tty{LAE} \\
Function         \> : \> $X,S,A\leftarrow((S)\&(M))$ \\
Addressing Modes \> : \> Y-indexed long \\
\end{tabbing}

\begin{tabbing}
Addressing Modes \= : \= \kill
Instruction      \> : \> \tty{DCP} \\
Function         \> : \> $M\leftarrow(M)-1, Flags\leftarrow((A)-(M))$ \\
Addressing Modes \> : \> absolute long/short, X-indexed long/short, \\
                 \>   \> Y-indexed long, X/Y-indirect \\
\end{tabbing}

\begin{tabbing}
Addressing Modes \= : \= \kill
Instruction      \> : \> \tty{SBX} \\
Function         \> : \> $X\leftarrow((X)\&(A))-(M)$ \\
Addressing Modes \> : \> immediate \\
\end{tabbing}

\begin{tabbing}
Addressing Modes \= : \= \kill
Instruction      \> : \> \tty{ISB} \\
Function         \> : \> $M\leftarrow(M)+1, A\leftarrow(A)-(M)-(C)$ \\
Addressing Modes \> : \> absolute long/short, X-indexed long/short, \\
                 \>   \> Y-indexed long, X/Y-indirect \\
\end{tabbing}

%%---------------------------------------------------------------------------

\section{MELPS-740}

Microcontrollers of this family have a quite nice, however well-hidden
feature: If one sets bit 5 of the status register with the \tty{SET}
instruction, the accumulator will be replaced with the memory cell
addressed by the X register for all load/store and arithmetic
instructions.  An attempt to integrate this feature cleanly into the
assembly syntax has not been made so far, so the only way to use it
is currently the ''hard'' way (\tty{SET}...instructions with accumulator
addressing...\tty{CLT}).

Not all MELPS-740 processors implement all instructions.  This is a
place where the programmer has to watch out for himself that no
instructions are used that are unavailable for the targeted
processor; AS does not differentiate among the individual processors
of this family.  For a description of the details regarding special
page addressing, see the discussion of the \tty{ASSUME} instruction.

%%---------------------------------------------------------------------------

\section{MELPS-7700/65816}
\label{MELPS7700Spec}

As it seems, these two processor families took disjunct development
paths, starting from the 6502 via their 8 bit predecessors.  Shortly
listed, the following differences are present:
\begin{itemize}
\item{The 65816 does not have a B accumulator.}
\item{The 65816 does not have instructions to multiply or divide.}
\item{The 65816 misses the instructions \tty{SEB, CLB, BBC, BBS, CLM, SEM,
      PSH, PUL} and \tty{LDM}.  Instead, the instructions \tty{TSB, TRB, BIT, CLD,
      SED, XBA, XCE} and \tty{STZ} take their places in the opcode table.}
\end{itemize}
The following instructions have identical function, yet different
names:
\par
\begin{center}\begin{tabular}{|c|c||c|c|}
\hline
   65816  &  MELPS-7700 & 65816 &  MELPS-7700 \\
\hline
\hline
    \tty{REP}  &  \tty{CLP}  &  \tty{PHK}  &  \tty{PHG} \\
    \tty{TCS}  &  \tty{TAS}  &  \tty{TSC}  &  \tty{TSA} \\
    \tty{TCD}  &  \tty{TAD}  &  \tty{TDC}  &  \tty{TDA} \\
    \tty{PHB}  &  \tty{PHT}  &  \tty{PLB}  &  \tty{PLT} \\
    \tty{WAI}  &  \tty{WIT}  &             & \\
\hline
\end{tabular}\end{center}
\par
Especially tricky are the instructions \tty{PHB, PLB} and \tty{TSB}: these
instructions have a totally different encoding and meaning on both
processors!

Unfortunately, these processors address their memory in a way that is
IMHO even one level higher on the open-ended chart of perversity than
the Intel-like segmentation: They do banking!  Well, this seems to
be the price for the 6502 upward-compatibility; before one can use AS
to write code for these processors, one has to inform AS about the
contents of several registers (using the \tty{ASSUME} instruction):

The M flag rules whether the accumulators A and B should be used with
8 bits (1) or 16 bits (0) width.  Analogously, the X flag decides the
width of the X and Y index registers.  AS needs this information for
the decision about the argument's width when immediate addressing
(\verb!#<constant>!) occurs.

The memory is organized in 256 banks of 64 KBytes.  As all registers
in the CPU core have a maximum width of 16 bits, the upper 8 bits
have to be fetched from 2 special bank registers: DT delivers the
upper 8 bits for data accesses, and PG extends the 16-bit program
counter to 24 bits.  A 16 bits wide register DPR allows to move the
zero page known from the 6502 to an arbitrary location in the first
bank.  If AS encounters an address (it is irrelevant if this address
is part of an absolute, indexed, or indirect expression), the
following addressing modes will be tested:
\begin{enumerate}
\item{Is the address in the range of DPR..DPR+\$ff?  If yes, use direct
      addressing with an 8-bit address.}
\item{Is the address contained in the page addressable via DT (resp.
      PG for branch instructions)? If yes, use absolute addressing
      with a 16-bit address.}
\item{If nothing else helps, use long addressing with a 24-bit
      address.}
\end{enumerate}
As one can see from this enumeration, the knowledge about the current
values of DT, PG and DPR is essential for a correct operation of AS;
if the specifications are incorrect, the program will probably do
wrong addressing at runtime.  This enumeration also implied that all
three address lengths are available; if this is not the case, the
decision chain will become shorter.

The automatic determination of the address length described above may
be overridden by the usage of prefixes.  If one prefixes the address
by a $<$, $>$, or $>>$ without a separating space, an address with 1, 2, or
3 bytes of length will be used, regardless if this is the optimal
length.  If one uses an address length that is either not allowed for
the current instruction or too short for the address, an error
message is the result.

To simplify porting of 6502 programs, AS uses the Motorola syntax for
hexadecimal constants instead of the Intel/IEEE syntax that is
the format preferred by Mitsubishi for their 740xxx series.  I still
think that this is the better format, and it looks as if the
designers of the 65816 were of the same opinion (as the \tty{RELAXED}
instruction allows the alternative use of Intel notation, this
decision should not hurt anything).  Another important detail for the
porting of programs is that it is valid to omit the accumulator A as
target for operations.  For example, it is possible to simply write
\verb!LDA #0! instead of \verb!LDA A,#0!.

A real goodie in the instruction set are the instructions \tty{MVN} resp.
\tty{MVP} to do block transfers.  However, their address specification
rules are a bit strange: bits 0--15 are stored in index registers,
bits 16--23 are part of the instruction.  When one uses AS, one
simply specifies the full destination and source addresses.  AS will
then automatically grab the correct bits.  This is a fine yet
important difference  Mitsubishi's assembler where you have to
extract the upper 8 bits on your own.  Things become really
convenient when a macro like the following is used:
\begin{verbatim}
mvpos   macro   src,dest,len
        if      MomCPU=$7700
         lda    #len
        elseif
         lda    #(len-1)
        endif
        ldx     #(src&$ffff)
        ldy     #(dest&$ffff)
        mvp     dest,src
        endm
\end{verbatim}
Caution, possible pitfall: if the accumulator contains the value n,
the Mitsubishi chip will transfer n bytes, but the 65816 will
transfer n+1 bytes!

The \tty{PSH} and \tty{PUL} instructions are also very handy because they
allow to save a user-defined set to be saved to the stack resp. to be
restored from the stack.  According to the Mitsubishi data book
\cite{Mit16}, the bit mask has to be specified as an immediate operand, so
the programmer either has to keep all bit$\leftrightarrow$register
assignments in mind or he has to define some appropriate symbols.  To make
things simpler, I decided to extend the syntax at this point: It is valid
to use a list as argument which may contain an arbitrary sequence of
register names or immediate expressions.  Therefore, the following
instructions
\begin{verbatim}
        psh     #$0f
        psh     a,b,#$0c
        psh     a,b,x,y

\end{verbatim}
are equivalent.  As immediate expressions are still valid, AS stays
upward compatible to the Mitsubishi assemblers.

One thing I did not fully understand while studying the Mitsubishi
assembler is the treatment of the \tty{PER} instruction: this instruction
allows to push a 16-bit variable onto the stack whose address is
specified relative to the program counter.  Therefore, it is an
absolute addressing mode from the programmer's point of view.
Nevertheless, the Mitsubishi assembler requests immediate addressing,
and the instructions argument is placed into the code just as-is.
One has to calculate the address in his own, which is something
symbolic assemblers were designed for to avoid...as I wanted to stay
compatible, AS contains a compromise:  If one chooses immediate
addressing (with a leading \# sign), AS will behave like the original
from Mitsubishi.  But if the \# sign is omitted, as will calculate the
difference between the argument's value and the current program
counter and insert this difference instead.

A similar situation exists for the \tty{PEI} instruction that pushes the
contents of a 16-bit variable located in the zero page: Though the operand
represents an address, once again immediate addressing is required.  In
this case, AS will simply allow both variants (i.e. with or without a \#
sign).

%%---------------------------------------------------------------------------

\section{M16}

The M16 family is a family of highly complex CISC processors with an
equally complicated instruction set.  One of the instruction set's
properties is the detail that in an instruction with two operands,
both operands may be of different sizes.  The method of appending the
operand size as an attribute of the instruction (known from Motorola
and adopted from Mitsubishi) therefore had to be extended: it is
valid to append attributes to the operands themselves.  For example,
the following instruction
\begin{verbatim}
        mov     r0.b,r6.w
\end{verbatim}
reads the lowest 8 bits of register 0, sign-extends them to 32 bits
and stores the result into register 6.  However, as one does not need
this feature in 9 out of 10 cases, it is still valid to append the
operand size to the instruction itself, e.g.
\begin{verbatim}
        mov.w   r0,r6
\end{verbatim}
Both variants may be mixed; in such a case, an operand size appended
to an operand overrules the ''default''.  An exception are instructions
with two operands.  For these instructions, the default for the
source operand is the destination operand's size.  For example, in
the following example
\begin{verbatim}
        mov.h   r0,r6.w
\end{verbatim}
register 0 is accessed with 32 bits, the size specification appended
to the instruction is not used at all.  If an instruction does not
contain any size specifications, word size (\tty{w}) will be used.
Remember: in contrast to the 68000 family, this means 32 bits instead
of 16 bits!

The chained addressing modes are also rather complex; the ability of
AS to automatically assign address components to parts of the chain
keeps things at least halfway manageable.  The only way of influencing
AS allows (the original assembler from Mitsubishi/Green Hills allows
a bit more in this respect) is the explicit setting of displacement
lengths by appending \tty{:4, :16} and \tty{:32}.

%%---------------------------------------------------------------------------

\section{4004/4040}

Thanks to John Weinrich, I now have the official Intel data sheets
describing these 'grandfathers' of all microprocessors, and the questions
about the syntax of register pairs (for 8-bit operations) have been weeded
out for the moment: It is \tty{RnRm} with \tty{n} resp. \tty{m} being even
integers in the range from 0 to E resp. 1 to F.  The equation {\tt m = n +
1} must be fulfilled.

%%---------------------------------------------------------------------------

\section{MCS-48}

The maximum address space of these processors is 4 Kbytes, resp.  up to
8 Kbytes on some Philips varaints.  This address space is not organized
in a linear way (how could this be on an Intel CPU...).  Instead, it is
split into 2 banks of 2 Kbytes.  The only way to change the program
counter from one bank to the other are the instructions \tty{CALL} and
\tty{JMP}, by setting the most significant bit of the address with the
instructions \tty{SEL MB0} to \tty{SEL MB3}.

The assembler may be informed about the bank currently being selected for
jumps and calls, via an {\tt ASSUME} statement:
\begin{verbatim}
         ASSUME MB:<0..3>
\end{verbatim}
If one tries to jump to an address in a different bank, a warnig is
issued.

If the special value {\tt NOTHING} is used (this is by the way the
default), an automatism uilt into \tty{JMP} and \tty{CALL} is activated.
It will insert a {\tt SEL MBx} instruction if the current program counter
and the target address are located in different banks.  Explicit usage of
\tty{SEL MBx} instructions is no longer necessary (though it remains possible),
and it might interfere with this mechanism, like in the following example:
\begin{verbatim}
 000:  SEL      MB1
       JMP      200h
\end{verbatim}
AS assumes that the MB flag is 0 and therefore does not insert a \tty{SEL
MB0} instruction, with the result that the CPU jumps to address
A00h.

Furthermore, one should keep in mind that a jump instruction might
become longer (3 instead of 2 bytes).

%%---------------------------------------------------------------------------

\section{MCS-51}

The assembler is accompanied by the files \tty{STDDEF51.INC} resp.
\tty{80C50X.INC} that define all bits and SFRs of the processors 8051,
8052, and 80515 resp. 80C501, 502, and 504.  Depending on the target
processor setting (made with the \tty{CPU} statement), the correct subset
will be included.  Therefore, the correct order for the instructions
at the beginning of a program is
\begin{verbatim}
        CPU     <processor type>
        INCLUDE stddef51.inc   .
\end{verbatim}
Otherwise, the MCS-51 pseudo instructions will lead to error
messages.

As the 8051 does not have instructions to to push the registers 0..7
onto the stack, one has to work with absolute addresses.  However,
these addresses depend on which register bank is currently active.
To make this situation a little bit better, the include files define
the macro \tty{USING} that accepts the symbols \tty{Bank0...Bank3} as arguments.
In response, the macro will assign the registers' correct absolute
addresses to the symbols \tty{AR0..AR7}.  This macro should be used after
every change of the register banks.  The macro itself does \bb{not}
generate any code to switch to the bank!

The macro also makes bookkeeping about which banks have been used.
The result is stored in the integer variable \tty{RegUsage}: bit 0
corresponds to bank 0, bit 1 corresponds to bank 1. and so on.  To
output its contents after the source has been assembled, use
something like the following piece of code:
\begin{verbatim}
        irp       BANK,Bank0,Bank1,Bank2,Bank3
         if        (RegUsage&(2^BANK))<>0
          message   "bank \{BANK} has been used"
         endif
        endm
\end{verbatim}
The multipass feature introduced with version 1.38 allowed to introduce
the additional instructions \tty{JMP} and \tty{CALL}.  If branches are
coded using these instructions, AS will automatically use the variant that
is optimal for the given target address.  The options are \tty{SJMP,
AJMP}, or \tty{LJMP} for \tty{JMP} resp. \tty{ACALL} or \tty{LCALL} for
\tty{CALL}.  Of course it is still possible to use these variants
directly, in case one wants to force a certain coding.

%%---------------------------------------------------------------------------

\section{MCS-251}

When designing the 80C251, Intel really tried to make the move to
the new family as smooth as possible for programmers.  This
culminated in the fact that old applications can run on the new
processor without having to recompile them.  However, as soon as one
wants to use the new features, some details have to be regarded which
may turn into hidden pitfalls.

The most important thing is the absence of a distinct address space
for bits on the 80C251.  All SFRs can now be addressed bitwise,
regardless of their address.  Furthermore, the first 128 bytes of the
internal RAM are also bit addressable.  This has become possible
because bits are not any more handled by a separate address space
that overlaps other address spaces.  Instead, similar to other
processors, bits are addressed with a two-dimensional address that
consists of the memory location containing the bit and the bit's
location in the byte.  One result is that in an expression like
\tty{PSW.7}, AS will do the separation of address and bit position itself.
Unlike to the 8051, it is not any more necessary to explicitly
generate 8 bit symbols.  This has the other result that the \tty{SFRB}
instruction does not exist any more.  If it is used in a program that
shall be ported, it may be replaced with a simple \tty{SFR} instruction.

Furthermore, Intel cleaned up the cornucopia of different address
spaces on the 8051: the internal RAM (\tty{DATA} resp. \tty{IDATA}), the
\tty{XDATA} space and the former \tty{CODE} space were unified to a single
\tty{CODE} space that is now 16 Mbytes large.  The internal RAM starts at
address 0, the internal ROM starts at address ff0000h, which is the
address code has to be relocated to.  In contrast, the SFRs were moved to
a separate address space (which AS refers to as the \tty{IO} segment).
However, they have the same addresses in this new address space as they
used to have on the 8051.  The \tty{SFR} instructions knows of this
difference and automatically assigns symbols to either the \tty{DATA} or
\tty{IO} segment, depending on the target processor.  As there is no
\tty{BIT} segment any more, the \tty{BIT} instruction operates completely
different: Instead of a linear address ranging from 0..255, a bit symbol
now contains the byte's address in bit 0..7, and the bit position in bits
24..26.  Unfortunately, creating arrays of flags with a symbolic address
is not that simple any more: On an 8051, one simply wrote:
\begin{verbatim}
        segment bitdata

bit1    db      ?
bit2    db      ?

or

defbit  macro   name
name    bit     cnt
cnt     set     cnt+1
        endm
\end{verbatim}
On a 251, only the second way still works, like this:
  \begin{verbatim}
adr     set     20h     ; start address of flags
bpos    set     0       ; in the internal RAM

defbit  macro   name
name    bit     adr.bpos
bpos    set     bpos+1
        if      bpos=8
bpos     set     0
adr      set     adr+1
        endif
        endm
\end{verbatim}
Another small detail: Intel now prefers \tty{CY} instead of \tty{C} as a
symbolic name for the carry, so you might have to rename an already
existing variable of the same name in your program.  However, AS will
continue to understand also the old variant when using the instructions
\tty{CLR, CPL, SETB, MOV, ANL,} or \tty{ORL}.  The same is conceptually
true for the additional registers \tty{R8..R15, WR0..WR30, DR0..DR28, DR56,
DR60, DPX,} and \tty{SPX}.

Intel would like everyone to write absolute addresses in a syntax of
\tty{XX:YYYY}, where \tty{XX} is a 64K bank in the address space resp.
signifies addresses in the I/O space with an \tty{S}.  As one might guess,
I am not amused about this, which is why it is legal to alternitavely use
linear addresses in all places.  Only the \tty{S} for I/O addresses is
incircumventable, like in this case:
\begin{verbatim}
Carry   bit     s:0d0h.7
\end{verbatim}
Without the prefix, AS would assume an address in the \tty{CODE} segment,
and only the first 128 bits in this space are bit-addressable...

Like for the 8051, the generic branch instructions \tty{CALL} and
\tty{JMP} exist that automatically choose the shortest machine code
depending on the address layout.  However, while \tty{JMP} also may use
the variant with a 24-bit address, \tty{CALL} will not do this for a good
reason: In contrast to \tty{ACALL} and \tty{LCALL}, \tty{ECALL} places an
additional byte onto the stack.  A \tty{CALL} instruction would result where
you would not know what it will do.  This problem does not exist for the
\tty{JMP} instructions.

There is one thing I did not understand: The 80251 is also able to
push immediate operands onto the stack, and it may push either single
bytes or complete words.  However, the same mnemonic (\tty{PUSH}) is
assigned to both variants - how on earth should an assembler know if
an instruction like
\begin{verbatim}
        push    #10
\end{verbatim}
shall push a byte or a word containing the value 10?  So the current
rule is that \tty{PUSH} always pushes a byte; if one wants to push a word,
simply use \tty{PUSHW} instead of \tty{PUSH}.

Another well-meant advise: If you use the extended instruction set,
be sure to operate the processor in source mode; otherwise, all
instructions will become one byte longer!  The old 8051 instructions
that will in turn become one byte longer are not a big matter:  AS
will either replace them automatically with new, more general
instructions or they deal with obsolete addressing modes (indirect
addressing via 8 bit registers).

%%---------------------------------------------------------------------------

\section{8080/8085}
\label{8080Spec}

As mentioned before, the statement
\begin{verbatim}
       Z80SYNTAX <ON|OFF|EXCLUSIVE>
\end{verbatim}
makes it possible to write the vast majority of 8080/8085
instructions in 'Z80 s		tyle', i.e. with less mnemonics but with
operands that are easier to understand.  In non-exclusive mode,
the Z80 syntax is not allowed for the following instructions,
because they conflict with existing 8080 mnemonics:
\begin{itemize}
\item{\tty{CP} in 'Intel syntax' means 'Call on Positive', in
      Zilog syntax however it means 'Compare'.  If you use
      \tty{CP} with a numeric value, it is not possible for the
      assembler to recognize whether a jump to an absolute
      address or a compare with an immediate value is meant.
      The assembler will generate a jump in this case, since the
      Intel syntax has precedence in case of ambiguities. If
      one wants the comparison, one may explicitly write down the
      accumulator as destination operand, e.g. \tty{CP A,12h}
      instead of \tty{CP 12h}.}
\item{\tty{JP} in Intel syntax means 'Jump on Positive', in Zilog
      syntax however, this is the jump instruction in general.
      Conditional jumps in Zilog syntax (\tty{JP cond,addr}) are
      unambigious because of the two arguments.  With only one
      argument, the assembler will however always generate the
      conditional jump.  If you want an unconditional jump to
      an absolute address, you still have to use the Intel syntax
      ((\tty{JMP addr}).}
\end{itemize}
The 8085 supports the instructions \tty{RIM} and \tty{SIM} that are
not part of the Z80 instruction set.  They may be written in 'Z80 style'
as \tty{LD A,IM} resp. \tty{LD IM,A}.

%%---------------------------------------------------------------------------

\section{8085UNDOC}
\label{8085Spec}

Similarly to the Z80 or 6502, Intel did not further specify the
undocumented 8085 instructions.  This however means that other assemblers
might use different mnemonics for the same function.  Therefore, I will
list the nistructions in the following.  Once again, usage of these
instructions is at one's own risk - even the Z80 which is principally
upward compatible to the 8085 uses the opcodes for entirely different
functions...

\begin{tabbing}
Arguments         \= : \= \kill \\
Instruction       \> : \> \tty{DSUB [reg]} \\
Z80 Syntax        \> : \> \tty{SUB HL,reg} \\
Function          \> : \> HL $\leftarrow$ HL - reg \\
Flags             \> : \> CY, S, X5, AC, Z, V, P \\
Arguments         \> : \> \tty{reg} = B for BC (optional for non-Z80 syntax) \\
\end{tabbing}

\begin{tabbing}
Arguments         \= : \= \kill \\
Instruction       \> : \> \tty{ARHL} \\
Z80 Syntax        \> : \> \tty{SRA HL} \\
Function          \> : \> HL,CY $\leftarrow$ HL $>>$ 1 (arithmetisch) \\
Flags             \> : \> CY \\
Arguments         \> : \> none resp. fixed for Z80 syntax \\
\end{tabbing}

\begin{tabbing}
Arguments         \= : \= \kill \\
Instruction       \> : \> \tty{RDEL} \\
Z80 Syntax        \> : \> \tty{RLC DE} \\
Function          \> : \> CY,DE $\leftarrow$ DE $<<$ 1 \\
Flags             \> : \> CY, V \\
Arguments         \> : \> none resp. fixed for Z80 syntax \\
\end{tabbing}

\begin{tabbing}
Arguments         \= : \= \kill \\
Instruction       \> : \> \tty{LDHI d8} \\
Z80 Syntax        \> : \> \tty{ADD DE,HL,d8} \\
Function          \> : \> DE $\leftarrow$ HL + {\tt d8} \\
Flags             \> : \> none \\
Arguments         \> : \> {\tt d8} = 8-bit constant, registers fixed for Z80 syntax \\
\end{tabbing}

\begin{tabbing}
Arguments         \= : \= \kill \\
Instruction       \> : \> \tty{LDSI d8} \\
Z80 Syntax        \> : \> \tty{ADD DE,SP,d8} \\
Function          \> : \> DE $\leftarrow$ SP + {\tt d8} \\
Flags             \> : \> none \\
Arguments         \> : \> {\tt d8} = 8-bit constant, registers fixed for Z80 syntax \\
\end{tabbing}

\begin{tabbing}
Arguments         \= : \= \kill \\
Instruction       \> : \> \tty{RSTflag} \\
Z80 Syntax        \> : \> \tty{RST flag} \\
Function          \> : \> restart to 40h if {\tt flag}=1 \\
Flags             \> : \> none \\
Arguments         \> : \> {\tt flag} = V for overflow bit \\
\end{tabbing}

\begin{tabbing}
Arguments         \= : \= \kill \\
Instruction       \> : \> \tty{SHLX [reg]} \\
Z80 Syntax        \> : \> \tty{LD (reg),HL} \\
Function          \> : \> [reg] $\leftarrow$ HL \\
Flags             \> : \> none \\
Arguments         \> : \> \tty{reg} = D/DE for DE (optional for non-Z80 syntax) \\
\end{tabbing}

\begin{tabbing}
Arguments         \= : \= \kill \\
Instruction       \> : \> \tty{LHLX [reg]} \\
Z80 Syntax        \> : \> \tty{LD HL,(reg)} \\
Function          \> : \> HL $\leftarrow$ [reg] \\
Flags             \> : \> none \\
Arguments         \> : \> \tty{reg} = D/DE for DE (optional for non-Z80 syntax) \\
\end{tabbing}

\begin{tabbing}
Arguments         \= : \= \kill \\
Instruction       \> : \> \tty{JNX5 addr} \\
Z80 Syntax        \> : \> \tty{JP NX5, addr} \\
Function          \> : \> jump to {\tt addr} if X5=0 \\
Flags             \> : \> none \\
Arguments         \> : \> {\tt addr} = absolute 16-bit address \\
\end{tabbing}

\begin{tabbing}
Arguments         \= : \= \kill \\
Instruction       \> : \> \tty{JX5 addr} \\
Z80 Syntax        \> : \> \tty{JP X5,addr} \\
Function          \> : \> jump to {\tt addr} if X5=1 \\
Flags             \> : \> none \\
Arguments         \> : \> {\tt addr} = absolute 16-bit address \\
\end{tabbing}

X5 refers to the otherwise unused bit 5 in the processor status word (PSW).

%%---------------------------------------------------------------------------

\section{8086..V35}

Actually, I had sworn myself to keep the segment disease of Intel's
8086 out of the assembler.  However, as there was a request and as
students are more flexible than the developers of this processor
obviously were, there is now a rudimentary support of these
processors in AS.  When saying, 'rudimentary', it does not mean that
the instruction set is not fully covered.  It means that the whole
pseudo instruction stuff that is available when using MASM, TASM, or
something equivalent does not exist.  To put it in clear words, AS
was not primarily designed to write assembler programs for PC's
(heaven forbid, this really would have meant reinventing the wheel!);
instead, the development of programs for single-board computers was
the main goal (which may also be equipped with an 8086 CPU).

For die-hards who still want to write DOS programs with AS, here is a
small list of things to keep in mind:
\begin{itemize}
\item{Only \tty{COM} files may be created.}
\item{Only use the \tty{CODE} segment, and place also all variables in
      this segment.}
\item{DOS initializes all segment registers to the code segment.
      An \tty{ASSUME DS:DATA, SS:DATA} right at the program's beginning
      is therefore necessary.}
\item{DOS loads the code to a start address of 100h.  An \tty{ORG} to this
      address is absolutely necessary.}
\item{The conversion to a binary file is done with P2BIN (see later in
      this document), with an address filter of \tty{\$-\$}.}
\end{itemize}
For these processors, AS only supports a small programming model, i.e.
there is \bb{one} code segment with a maximum of 64 Kbytes and a data
segment of equal size for data (which cannot be set to initial values for
\tty{COM} files).  The \tty{SEGMENT} instruction allows to switch between
these two segments.  From this facts results that branches are always
intrasegment branches if they refer to targets in this single code
segment.  In case that far jumps should be necessary, they are possible
via \tty{CALLF} or \tty{JMPF} with a memory address or a
\tty{Segment:Offset} value as argument.

Another big problem of these processors is their assembler syntax,
which is sometimes ambiguous and whose exact meaning can then only be
deduced by looking at the current context.  In the following example,
either absolute or immediate addressing may be meant, depending on
the symbol's type:
\begin{verbatim}
        mov     ax,value
\end{verbatim}
When using AS, an expression without brackets always is interpreted
as immediate addressing.  For example, when either a variable's
address or its contents shall be loaded, the differences listed in table
\ref{TabMASM} are present between MASM and AS:
\begin{table*}
\begin{center}\begin{tabular}{|l|l|l|}
\hline
assembler  & address             & contents \\
\hline
\hline
MASM       &  \tty{mov ax,offset vari} &  \tty{mov ax,vari} \\
           &  \tty{lea ax,vari}        &  \tty{mov ax,[vari]} \\
           &  \tty{lea ax,[vari]}      & \\
           &                           & \\
AS         &  \tty{mov ax,vari}        &  \tty{mov ax,[vari]} \\
           &  \tty{lea ax,[vari]}      & \\
\hline
\end{tabular}\end{center}
\caption{Differences AS$\leftrightarrow$MASM Concerning Addressing
         Syntax\label{TabMASM}}
\end{table*}
\par
When addressing via a symbol, the assembler checks whether they are
assigned to the data segment and tries to automatically insert an
appropriate segment prefix.  This happens for example when symbols
from the code segment are accessed without specifying a \tty{CS} segment
prefix.  However, this mechanism can only work if the \tty{ASSUME}
instruction (see there) has previously been applied correctly.

The Intel syntax also requires to store whether bytes or words were
stored at a symbol's address.  AS will do this only when the \tty{DB} resp.
\tty{DW} instruction is in the same source line as the label.  For any
other case, the operand size has to be specified explicitly with the
\tty{BYTE PTR, WORD PTR,...} operators.  As long as a register is the other
operator, this may be omitted, as the operand size is then clearly
given by the register's name.

In an 8086-based system, the coprocessor is usually synchronized via
via the processor's TEST input line which is connected to toe
coprocessor's BUSY output line.  AS supports this type of handshaking
by automatically inserting a \tty{WAIT} instruction prior to every 8087
instruction.  If this is undesired for any reason, an \tty{N} has to be
inserted after the \tty{F} in the mnemonic; for example,
\begin{verbatim}
        FINIT
        FSTSW   [vari]
\end{verbatim}
becomes
\begin{verbatim}
        FNINIT
        FNSTSW  [vari]
\end{verbatim}
This variant is valid for \bb{all} coprocessor instructions.

%%---------------------------------------------------------------------------

\section{8X30x}
\label{8X30xSpec}

The processors of this family have been optimized for an easy manipulation
of bit groups at peripheral addresses.  The instructions \tty{LIV} and
\tty{RIV} were introduced to deal with such objects in a symbolic fashion.
They work similar to \tty{EQU}, however they need three parameters:
\begin{enumerate}
\item{the address of the peripheral memory cell that contains the bit
     group (0..255);}
\item{the number of the group's first bit (0..7);}
\item{the length of the group, expressed in bits (1..8).}
\end{enumerate}
\bb{CAUTION!} The 8X30x does not support bit groups that span over more
than one memory address.  Therefore, the valid value range for the
length can be stricter limited, depending on the start position.  AS
does \bb{not} perform any checks at this point, you simply get strange
results at runtime!

Regarding the machine code, length and position are expressed vis a 3
bit field in the instruction word and a proper register number (\tty{LIVx}
resp. \tty{RIVx}).  If one uses a symbolic object, AS will automatically
assign correct values to this field, but it is also allowed to
specify the length explicitly as a third operand if one does not work
with symbolic objects.  If AS finds such a length specification in
spite of a symbolic operand, it will compare both lengths and issue
an error if they do not match (the same will happen for the MOVE
instruction if two symbolic operands with different lengths are used
- the instruction simply only has a single length field...).

Apart from the real machine instructions, AS defines similarly to its
''idol'' MCCAP some pseudo instructions that are implemented as builtin
macros:
\begin{itemize}
\item{\tty{NOP} is a shortform for \tty{MOVE AUX,AUX}}
\item{\tty{HALT} is a shortform for {\tt JMP \verb!*!}}
\item{\tty{XML ii} is a shortform for \tty{XMIT ii,R12} (only 8X305)}
\item{\tty{XMR ii} is a shortform for \tty{XMIT ii,R13} (only 8X305)}
\item{\tty{SEL $<$busobj$>$} is a shortform for \tty{XMIT $<$adr$>$,IVL/IVR},
   i.e. it performs the necessary preselection to access $<$busobj$>$.}
\end{itemize}
The \tty{CALL} and \tty{RTN} instructions MCCAP also implements are
currently missing due to sufficient documentation.  The same is true for a
set of pseudo instructions to store constants to memory.  Time may change
this...

%%---------------------------------------------------------------------------

\section{XA}

Similar to its predecessor MCS/51, but in contrast to its
'competitor' MCS/251, the Philips XA has a separate address space for
bits, i.e. all bits that are accessible via bit instructions have a
certain, one-dimensional address which is stored as-is in the machine
code.  However, I could not take the obvious opportunity to offer
this third address space (code and data are the other two) as a
separate segment.  The reason is that - in contrast to the MCS/51 -
some bit addresses are ambiguous: bits with an address from 256 to 511
refer to the bits of memory cells 20h..3fh in the current data
segment.  This means that these addresses may correspond to different
physical bits, depending on the current state.  Defining bits with
the help of \tty{DC} instructions - something that would be possible with a
separate segment - would not make too much sense.  However, the \tty{BIT}
instruction still exists to define individual bits (regardless if
they are located in a register, the RAM or SFR space) that can then
be referenced symbolically.  If the bit is located in RAM, the
address of the 64K-bank is also stored.  This way, AS can check
whether the DS register has previously be assigned a correct value
with an \tty{ASSUME} instruction.

In contrast, nothing can stop AS's efforts to align potential branch
targets to even addresses.  Like other XA assemblers, AS does this by
inserting \tty{NOP}s right before the instruction in question.

%%---------------------------------------------------------------------------

\section{AVR}

In contrast to the AVR assembler, AS by default uses the Intel format
to write hexadecimal contants instead of the C syntax.  All right, I
did not look into the (free) AVR assembler before, but when I started
with the AVR part, there was hardly mor einformation about the AVR
than a preliminary manual describing processor types that were never
sold...this problem can be solved with a simple RELAXED ON.

Optionally, AS can generate so-called "object files" for the AVRs (it
also works for other CPUs, but it does not make any sense for them...).
These are files containing code and source line info what e.g. allows
a step-by-step execution on source level with the WAVRSIM simulator
delivered by Atmel.  Unfortunately, the simulator seems to have
trouble with source file names longer than approx. 20 characters:
Names are truncated and/or extended by strange special characters
when the maximum length is exceeded.  AS therefore stores file name
specifications in object files without a path specification.
Therefore, problems may arise when files like includes are not in the
current directory.

A small specialty are machine instructions that have already been defined
by Atmel as part of the architecture, but up to now haven't been
implemented in any of the family's members.  The instructions in question
are {\tt MUL, JMP,} and {\tt CALL}.  Considering the latter ones, one may
ask himself how to reach the 4 Kwords large address space of the AT90S8515
when the 'next best' instructions {\tt RJMP} and {\tt RCALL} can only
branch up to 2 Kwords forward or backward.  The trick is named 'discarding
the upper address bits' and described in detail with the {\tt WRAPMODE}
statement.

All AVR targets support the optional CPU argument {\tt CODESEGSIZE}.
Like in this example,
\begin{verbatim}
   cpu atmega8:codesegsize=0
\end{verbatim}
it may be used to instruct the assembler to treat the code segment (i.e.
the internal flash ROM) as being organized in bytes instead of 16 bit words.
This is the view when the {\tt LPM} instruction is used, and which some other
(non Atmel) assemblers use in general.  It has the advantage that addresses 
in the {\tt CODE} segment need not be multiplied by two if used for data
accesses.  On the other hand, care has to be taken that instructions do
not start on an odd address - this would be the equivalent of an instruction
occupying fractions of flash words.  The {\tt PADDING} option is therefore
enabled by default, while it remains possible to define arrays of bytes via
multiple uses of {\tt DB} or {\tt DATA} without the risk of padding bytes
inserted in between.  Target addresses for relative and absolute branches 
automatically get divided by two in this ''byte mode''.  The default is the
organizazion in 16 bit word as used by the original Atmel assembler.  This
may explicitly be selected by using the argument \verb!codesegsize=1!.

%%---------------------------------------------------------------------------

\section{Z80UNDOC}

As one might guess, Zilog did not make any syntax definitions for the
undocumented instructions; furthermore, not everyone might know the
full set.  It might therefore make sense to list all instructions at
this place:

Similar to a Z380, it is possible to access the byte halves of IX and
IY separately.  In detail, these are the instructions that allow
this:
\begin{verbatim}
 INC Rx              LD R,Rx             LD  Rx,n
 DEC Rx              LD Rx,R             LD  Rx,Ry
 ADD/ADC/SUB/SBC/AND/XOR/OR/CP A,Rx
\end{verbatim}
\tty{Rx} and \tty{Ry} are synonyms for \tty{IXL, IXU, IYL} or \tty{IYU}.
Keep however in mind that in the case of \tty{LD  Rx,Ry}, both registers
must be part of the same index register.

The coding of shift instructions leaves an undefined bit combination which
is now accessible as the tty{SL1}, \tty{SLI}, \tty{SLIA}, or \tty{SLS}
instruction.  It works like \tty{SLA} with the difference of entering a 1
into bit position 0.  \bb{CAUTION!} Some sources also name this operation
\tty{SLL}.  It decided to not offer this, since it is misleading: \tty{SLL}
translates into "shift left logically", and the operation performed by this
instruction is no logical left shift.  If one should define \tty{SLL} at
all, then as an alias for \tty{SLA}.  If you have existing code that uses
\tty{SLL} in the meaning of \tty{SL1/SLI}, define it via a macro.

Like all other (docummented) shift instructions, this also works in
another undocumented variant:
\begin{verbatim}
        SLIA    R,(XY+d)
        SLIA    (XY+d),R
\end{verbatim}
In this case, \tty{R} is an arbitrary 8-bit register (excluding index
register halves...), and \tty{(XY+d)} is a normal indexed address.  This
operation has the additional effect of copying the result into the
register.  This also works for the \tty{RES} and \tty{SET} instructions:
\begin{verbatim}
        SET/RES R,n,(XY+d)
        SET/RES n,(XY+d),R
\end{verbatim}
Furthermore, two hidden I/O instructions exist:
\begin{verbatim}
        IN      (C) resp. TSTI
        OUT     (C),0
\end{verbatim}
Their operation should be clear.  \bb{CAUTION!}  Noone can
guarantee that all mask revisions of the Z80 execute these
instructions, and the Z80's successors will react with traps if they
find one of these instructions.  Use them on your own risk...

%%---------------------------------------------------------------------------

\section{GB\_Z80 resp. LR35902}

The LR35902 SoC used in the original Gameboy  was developed by Sharp,
and the CPU core is (probably) the same as in the SM83
microcontrollers.  Regarding its instruction set, it is somewhere
''half way'' between 8080 and Z80, however with its own omissions and
extensions.  Sharp of course defined an assembler syntax for the new
instructions.  However, variations have established itself in the
''Gameboy scene''.  I tried to regard those as well (as far as I am
aware of them):

\begin{center}\begin{tabular}{|l|l|l|}
\hline
Sharp & Alternate & Function \\
\hline
\hline
LD A,(HLD)    & LD A,(HL-)  & A $\longleftarrow$ (HL), \\
              & LDD A,(HL)  & HL $\longleftarrow$ HL-1 \\
\hline
LD A,(HLI)    & LD A,(HL+)  & A $\longleftarrow$ (HL), \\
              & LDI A,(HL)  & HL $\longleftarrow$ HL+1 \\
\hline
LD (HLD),A    & LD (HL-),A  & (HL) $\longleftarrow$ A, \\
              & LDD (HL),A  & HL $\longleftarrow$ HL-1 \\
\hline
LD (HLI),A    & LD (HL+),A  & (HL) $\longleftarrow$ A, \\
              & LDI (HL),A  & HL $\longleftarrow$ HL+1 \\
\hline
LD A,(C)      & LD A,(FF00+C) & A $\longleftarrow$ (0ff00h+C) \\
              & LDH A,(C)     & \\
\hline
LD (C),A      & LD (FF00+C),A & (0ff00h+C) $\longleftarrow$ A \\
              & LDH (C),A     & \\
\hline
LD (FF00+n),A & LDH (n),A     & (0ff00h+n) $\longleftarrow$ A \\
\hline
LD A,(FF00+n) & LDH A,(n)     & A $\longleftarrow$ (0ff00h+n) \\
\hline
LDHL SP,d     & LD HL,SP+d    & HL $\longleftarrow$ SP + d \\
\hline
LDX A,(nn)    & LD A,(nn)     & A $\longleftarrow$ (nn) $^{1}$ \\
\hline
LDX (nn),A    & LD (nn),A     & (nn) $\longleftarrow$ A $^{1}$ \\
\hline
\multicolumn{3}{|l|}{$^{1}$ enforces 16 bit addressing } \\
\hline
\end{tabular}\end{center}
 
%%---------------------------------------------------------------------------

\section{Z380}

As this processor was designed as a grandchild of the still most popular
8-bit microprocessor, it was a sine-qua-non design target to execute
existing Z80 programs without modification (of course, they execute a bit
faster, roughly by a factor of 10...).  Therefore, all extended features
can be enabled after a reset by setting two bits which are named XM
(eXtended Mode, i.e. a 32-bit instead of a 16-bit address space)
respectively LW (long word mode, i.e. 32-bit instead of 16-bit operands).
One has to inform AS about their current setting with the instructions
\tty{EXTMODE} resp. \tty{LWORDMODE}, to enable AS to check addresses and
constants against the correct upper limits.  The toggle between 32- and
16-bit instruction of course only influences instructions that are
available in a 32-bit variant.  Unfortunately, the Z380 currently offers
such variants only for load and store instructions; arithmetic can only be
done in 16 bits.  Zilog really should do something about this, otherwise
the most positive description for the Z380 would be ''16-bit processor
with 32-bit extensions''...

The whole thing becomes complicated by the ability to override the operand
size set by LW with the instruction prefixes \tty{DDIR W} resp.
\tty{DDIR LW}.  AS will note the occurrence of such instructions and will
toggle setting for the instruction following directly.  By the way, one
should never explicitly use other \tty{DDIR} variants than \tty{W} resp.
\tty{LW}, as AS will introduce them automatically when an operand is
discovered that is too long.  Explicit usage might puzzle AS.  The
automatism is so powerful that in a case like this:
\begin{verbatim}
        DDIR    LW
        LD      BC,12345678h   ,
\end{verbatim}
the necessary \tty{IW} prefix will automatically be merged into the previous
instruction, resulting in
\begin{verbatim}
        DDIR    LW,IW
        LD      BC,12345668h   .
\end{verbatim}
The machine code that was first created for \tty{DDIR LW} is retracted and
replaced, which is signified with an \tty{R} in the listing.

%%---------------------------------------------------------------------------

\section{Z8, Super8, and eZ8}
\label{Z8Spec}

The CPU core contained in the Z8 microcontrollers does not
contain any specific registers.  Instead, a block of 16
consecutive cells of the internal address space (contains RAM and
I/O registers) may be used as 'work registers' and be addressed
with 4-bit addresses.  The RP registers define which memory block
is used as work registers: on a classic Z8, bits 4 to 7 of RP
define the 'offset' that is added to a 4-bit work register
address to get a complete 8-bit address.  The Super8 core
features two register pointers (RP0 and RP1), which allow mapping
the lower and upper half of work registers to separate places.

Usually, one refers to work registers as R0..R15 in assembly
statements.  It is however also posssible to regard work registers as
an efficient way to address a block of memory addresses in internal
RAM.

The \tty{ASSUME} statement is used to inform AS about the current
value of RP. AS is then capable to automatically decide whether an
address in internal RAM may be reached with a 4-bit or 8-bit address.
This may be used to assign symbolic names to work registers:
\begin{verbatim}
op1     equ     040h
op2     equ     041h

        srp     #040h
        assume  rp:040h

        ld      op1,op2         ; equal to ld r0,r1
\end{verbatim}
Note that though the Super8 does not have an RP register (only
RP0 and RP1), RP as argument to \tty{ASSUME} is still allowed -
it will set the assumed values of RP0 and RP1 to $value$ resp.
$value+8$, as the \tty{SRP} machine instruction does on the Super
8 core.

Opposed to the original Zilog assembler, it is not necessary to
explicitly specify 'work register addressing' with a prefixed
exclamation mark.  AS however also understands this syntax - a
prefixed exclamation mark enforces 4-bit addressing, even when the
address does not lie within the 16-address block defined by RP (AS will
issue a warning in that case).  Vice versa, a prefixed $>$
character enforces 8-bit addressing even when the address is within
the current 16-address block.

The eZ8 takes this 'game' to the next level: the internal address
space now has 12 instead of 8 bits.  To assure compatibility with the
old Z8 core, Zilog placed the additional 4 bits in the {\em lower}
four bits of RP.  For instance, an RP value of 12h defines an address
window from 210h to 21fh.

At the same time, the lower four bits of RP define a window of 256
addresses that can be addressed with 8-bit addresses.  The mechanism
to automatically select between 8- and 12-bit addresses is analogous.
'Long' 12-bit addresses may be enforced by prefixing two $>$
characters.

%%---------------------------------------------------------------------------

\section{Z8000}
\label{Z8000Spec}

A Z8001/8003 may be operated in one of two modes:

\begin{itemize}
\item{{\em Non-Segmented}: The memory address space is limited to 64 KBytes,
      and all addresses are 'simple' linear 16 bit addresses.  Address
      registers are single 16 bit registers (Rn), and absolute addresses
      within instructions are one byte long.}
\item{{\em Segmented}: Memory is structured into up to 128 segments of up
      to 64 KBytes size.  Addresses consist of a 7 bit segment number and a
      16 bit offset. Address registers are register pairs (RRn).  Absolute
      addresses in instructions occupy two 16 bit words, unless the offset
      is smaller than 256.}
\end{itemize}

The operation mode (segmented or non-segmented) therefore has an influence
on the generated code and is selected implicitly via the selected processor
type.  For instance, if the target is a Z8001 in non-segmented mode, use
Z8002 as target.

However, similar to the 8086, there is no 'real' support for a segmented
memory model in AS. In segmented mode, the segment number is simply interpreted
as the upper seven bits of a virtually linear address space.  Though this is
not what Zilog intended, it is the way the segment number was used on the
Z8001 if the system had no MMU.

AS in general implements the Z8000 machine instruction syntax as it is specified
by Zilog in its manuals.  However, there are assemblers that support extensions
or variations of the syntax.  AS implements a few of them as well:

\subsection{Conditions}

In addition to the conditions defined by Zilog, the following alternative names
are defined:

\begin{center}\begin{tabular}{|l|l|l|}
\hline
Alternate & Zilog & Meaning \\
\hline
\hline
ZR         & Z     & Z = 1 \\
CY         & C     & C = 1 \\
LLE        & ULE   & (C OR Z) = 1 \\
LGE        & UGE   & C = 0 \\
LGT        & UGT   & ((C = 0) AND (Z = 0)) = 1 \\
LLT        & ULT   & C = 1 \\
\hline
\end{tabular}\end{center}

\subsection{Conditions}

\tty{SETFLG}, \tty{COMFLG} und \tty{RESFLG} accept the following alternate names
as arguments:

\begin{center}\begin{tabular}{|l|l|l|}
\hline
Alternate & Zilog & Meaning \\
\hline
\hline
ZR         & Z     & Zero Flag \\
CY         & C     & Carry Flag \\
\hline
\end{tabular}\end{center}

\subsection{Indirect Addressing}

It is valid to write \verb!Rn^! instead of \verb!@Rn!, if the option
\tty{AMDSyntax=1} was given to the \tty{CPU} statement.  If an I/O address
is addressed indirectly, this option even allows to write just \verb!Rn!.

\subsection{Direct versus Immediate Addressing}

The Zilog syntax mandates that immediate addressing has to be done by prefixing
the argument with a hash character.  However, if the \tty{AMDSyntax=1} option
was given to the \tty{CPU} statement, the type of argument (label or constant)
decides whether immediate or direct addressing is to be used.  Immediate addressing
may be forced by prefixing the argument with a circumflex, i.e. to load the address
of a label into a register.

%%---------------------------------------------------------------------------

\section{TLCS-900(L)}
\label{TLCS900Spec}

These processors may run in two operating modes: on the one hand, in
minimum mode, which offers almost complete source code compatibility
to the Z80 and TLCS-90, and on the other hand in maximum mode, which
is necessary to make full use of the processor's capabilities.  The
main differences between these two modes are:
\begin{itemize}
\item{width of the registers WA, BC, DE, and HL: 16 or 32 bits;}
\item{number of register banks: 8 or 4;}
\item{code address space: 64 Kbytes or 16 Mbytes;}
\item{length of return addresses: 16 or 32 bits.}
\end{itemize}
To allow AS to check against the correct limits, one has to inform him
about the current execution mode via the \tty{MAXMODE} instruction (see
there).  The default is the minimum mode.

From this follows that, depending on the operating mode, the 16-bit
resp. 32-bit versions of the bank registers have to be used for
addressing, i.e. WA, BC, DE and HL for the minimum mode resp. XWA,
XBC, XDE and XHL for the maximum mode.  The registers XIX..XIZ and
XSP are \bb{always} 32 bits wide and therefore always have to to be used
in this form for addressing; in this detail, existing Z80 code
definitely has to be adapted (not including that there is no I/O
space and all I/O registers are memory-mapped...).

Absolute addresses and displacements may be coded in different
lengths.  Without an explicit specification, AS will always use
the shortest possible coding.  This includes eliminating a zero
displacement, i.e. \verb!(XIX+0)! becomes \verb!(XIX)!.  If a certain
length is needed, it may be forced by appending a suffix (:8, :16,
:24) to the displacmenet resp. the address.

The syntax chosen by Toshiba is a bit unfortunate in the respect of
choosing an single quote (') to reference the previous register bank.  The
processor independent parts of AS already use this character to mark
character constants.  In an instruction like
\begin{verbatim}
        ld      wa',wa   ,
\end{verbatim}
AS will not recognize the comma for parameter separation.  This
problem can be circumvented by usage of an inverse single quote (`), for
example
\begin{verbatim}
        ld      wa`,wa
\end{verbatim}
Toshiba delivers an own assembler for the TLCS-900 series (TAS900),
which is different from AS in the following points:

\subsubsection{Symbol Conventions}

\begin{itemize}
\item{TAS900 differentiates symbol names only on the first 32
      characters.  In contrast, AS always stores symbol names with the
      full length (up to 255 characters) and uses them all for
      differentiation.}
\item{TAS900 allows to write integer constants either in Intel or C
      notation (with a 0 prefix for octal or a 0x prefix for hexadecimal
      constants).  By default, AS only supports the Intel notation.
      With the help of the \tty{RELAXED} instruction, one also gets the C
      notation (among other).}
\item{AS does not distinguish between upper and lower case.  In
      contrast, TAS900 differentiates between upper- and lowercase
      letters in symbol names.  One needs to engage the \tty{-u} command
      line option to force AS to do this.}
\end{itemize}

\subsubsection{Syntax}

For many instructions, the syntax checking of AS is less strict than
the checking of TAS900.  In some (rare) cases, the syntax is slightly
different.  These extensions and changes are on the one hand for the
sake of a better portability of existing Z80 codes, on the other hand
they provide a simplification and better orthogonality of the
assembly syntax:
\begin{itemize}
\item{In the case of \tty{LDA, JP}, and \tty{CALL}, TAS requires that address
      expressions like \tty{XIX+5} must not be placed in parentheses, as it
      is usually the case.  For the sake of better orthogonality, AS
      requires parentheses for \tty{LDA}.  They are optional if \tty{JP} resp.
      \tty{CALL} are used with a simple, absolute address.}
\item{In the case of \tty{JP, CALL, JR}, and \tty{SCC}, AS leaves the choice to the
      programmer whether to explicitly write out the default condition
      \tty{T} (= true) as first parameter or not.  TAS900 in contrast only
      allows to use the default condition implicitly (e.g. \tty{jp (xix+5)}
      instead of \tty{jp t,(xix+5))}.}
\item{For the \tty{EX} instruction, AS allows operand combinations which are
      not listed in \cite{Tosh900} but can be reduced to a standard
      combination by swapping the operands.  Combinations like \tty{EX f`,f}
      or \tty{EX wa,(xhl)} become possible.  In contrast, TAS900 limits to
      the 'pure' combinations.}
\item{AS allows to omit an increment resp. decrement of 1 when using the
      instructions \tty{INC} and \tty{DEC}.  TAS900 instead forces the programmer to
      explicit usage of '1'.}
\item{The similar is true for the shift instructions: If the operand is
      a register, TAS900 requires that even a shift count of 1 has to
      be written explicitly; however, when the operand is in memory,
      the hardware limits the shift count to 1 which must not be written
      in this case.  With AS, a shift count of 1 is always optional and
      valid for all types of operands.}
\end{itemize}

\subsubsection{Macro Processor}

The macro processor of TAS900 is an external program that operates
like a preprocessor.  It consists of two components: The first one is
a C-like preprocessor, and the second one is a special macro language
(MPL) that reminds of high level languages.  The macro processor of
AS instead is oriented towards ''classic'' macro assemblers like MASM
or M80 (both programs from Microsoft).  It is a fixed component of
AS.

\subsubsection{Output Format}

TAS900 generates relocatable code that allows to link separately
compiled programs to a single application.  AS instead generates
absolute machine code that is not linkable.  There are currently no
plans to extend AS in this respect.

\subsubsection{Pseudo Instructions}

Due to the missing linker, AS lacks a couple of pseudo instructions
needed for relocatable code TAS900 implements.  The following
instructions are available with equal meaning:
\begin{quote}\tt
   EQU, DB, DW, ORG, ALIGN, END, TITLE, SAVE, RESTORE
\rm\end{quote}
The latter two have an extended functionality for AS.  Some TAS900
pseudo instructions can be replaced with equivalent AS instructions (see
table \ref{TabTAS900}).
\par
\begin{table*}[htbp]
\begin{center}\begin{tabular}{|l|l|l|}
\hline
TAS900           & AS                  &     meaning/function \\
\hline
\hline
\tty{DL} $<$Data$>$    & \tty{DD} $<$Data$>$           & define longword constants \\
\hline
\tty{DSB} $<$number$>$ & \tty{DB} $<$number$>$ \tty{DUP} (?) & reserve bytes of memory \\
\hline
\tty{DSW} $<$number$>$ & \tty{DW} $<$number$>$ \tty{DUP} (?) & reserve words of memory \\
\hline
\tty{DSD} $<$number$>$ & \tty{DD} $<$number$>$ \tty{DUP} (?) & reserve longwords of memory \\
\hline
\tty{\$MIN[IMUM]}      & \tty{MAXMODE OFF}             & following code runs \\
                       &                               & in minimum mode \\
\hline
\tty{\$MAX[IMUM]}      & \tty{MAXMODE ON}              & following code runs \\
                       &                               & in maximum mode \\
\hline
\tty{\$SYS[TEM]}       & \tty{SUPMODE ON}              & following code runs \\
                       &                               & in system mode \\
\hline
\tty{\$NOR[MAL]}       & \tty{SUPMODE OFF}             & following code runs \\
                       &                               & in user mode \\
\hline
\tty{\$NOLIST}         & \tty{LISTING OFF}             & turn off assembly listing \\
\hline
\tty{\$LIST}           & \tty{LISTING ON}              & turn on assembly listing \\
\hline
\tty{\$EJECT}          & \tty{NEWPAGE}                 & start new page in listing \\
\hline
\end{tabular}\end{center}
\caption{equivalent instructions TAS900$\leftrightarrow$AS\label{TabTAS900}}
\end{table*}
Toshiba manufactures two versions of the processor core, with the L
version being an ''economy version''.  AS will make the following
differences between TLCS-900 and TLCS-900L:
\begin{itemize}
\item{The instructions \tty{MAX} and \tty{NORMAL} are not allowed for the L version;
      the \tty{MIN} instruction is disabled for the full version.}
\item{The L version does not know the normal stack pointer XNSP/NSP, but
      instead has the interrupt nesting register INTNEST.}
\end{itemize}
The instructions \tty{SUPMODE} and \tty{MAXMODE} are not influenced, just as
their initial setting \tty{OFF}.  The programmer has to take care of the
fact that the L version starts in maximum mode and does not have a
normal mode.  However, AS shows a bit of mercy against the L variant
by suppressing warnings for privileged instructions.

%%---------------------------------------------------------------------------

\section{TLCS-90}

Maybe some people might ask themselves if I mixed up the order a
little bit, as Toshiba first released the TLCS-90 as an extended Z80
and afterwards the 16-bit version TLCS-900.  Well, I discovered the
'90 via the '900 (thank you Oliver!).  The two families are quite
similar, not only regarding their syntax but also in their
architecture.  The hints for the '90 are therefore a subset of of the
chapter for the '900: As the '90 only allows shifts, increments, and
decrements by one, the count need not and must not be written as the
first argument.  Once again, Toshiba wants to omit parentheses for
memory operands of \tty{LDA, JP, and CALL}, and once again AS requires them
for the sake of orthogonality (the exact reason is of course that
this way, I saved an extra in the address parser, but one does not
say such a thing aloud).

Principally, the TLCS-90 series already has an address space of 1
Mbyte which is however only accessible as data space via the index
registers.  AS therefore does not regard the bank registers and
limits the address space to 64 Kbytes.  This should not limit too
much as this area above is anyway only reachable via indirect
addressing.

%%---------------------------------------------------------------------------

\section{TLCS-870}

Once again Toshiba...a company quite productive at the moment!
Especially this branch of the family (all Toshiba microcontrollers
are quite similar in their binary coding and programming model) seems
to be targeted towards the 8051 market: the method of separating the
bit position from the address expression with a dot had its root in
the 8051.  However, it creates now exactly the sort of problems I
anticipated when working on the 8051 part: On the one hand, the dot
is a legal part of symbol names, but on the other hand, it is part of
the address syntax.  This means that AS has to separate address and
bit position and must process them independently.  Currently, I
solved this conflict by seeking the dot starting at the \bb{end} of the
expression.  This way, the last dot is regarded as the separator, and
further dots stay parts of the address.   I continue to urge everyone
to omit dots in symbol names, they will lead to ambiguities:
\begin{verbatim}
        LD      CF,A.7  ; accumulator bit 7 to carry
        LD      C,A.7   ; constant 'A.7' to accumulator
\end{verbatim}

%%---------------------------------------------------------------------------

\section{TLCS-47}

This family of 4-bit microcontrollers should mark the low end of what
is supportable by AS.  Apart from the \tty{ASSUME} instruction for the data
bank register (see there), there is only one thing that is worth
mentioning: In the data and I/O segment, nibbles are reserved instead
of byte (it's a 4-bitter...).  The situation is similar to the bit
data segment of the 8051, where a \tty{DB} reserves a single bit, with the
difference that we are dealing with nibbles.

Toshiba defined an ''extended instruction set'' for this processor
family to facilitate the work with their limited instruction set.  In
the case of AS, it is defined in the include file \tty{STDDEF47.INC}.
However, some instructions that could not be realized as macros are
''builtins'' and are therefore also available without the include file:
\begin{itemize}
\item{the \tty{B} instruction that automatically chooses the optimal version
      of the jump instruction (\tty{BSS; BS}, or \tty{BSL});}
\item{\tty{LD} in the variant of \tty{HL} with an immediate operand;}
\item{\tty{ROLC} and \tty{RORC} with a shift amplitude higher than one.}
\end{itemize}

%%---------------------------------------------------------------------------

\section{TLCS-9000}

This was the first time that I implemented a processor for AS which
was not yet available at that point of time.  And unfortunately,
I received back then information that Toshiba had decided no to
maket this processor at all.  This of course had the result that
the TLCS-9000 part of the assembler
\begin{enumerate}
\item{was a ''paper design'', i.e. there was so far no chance to test
      it on real hardware and}
\item{the documentation for the '9000 I could get hold of \cite{Tosh9000}
      was preliminary and was unclear in a couple of detail
      issues.}
\end{enumerate}
So i effect, this target went into 'dormant mode'...

...cut, 20 years have passed: all of a sudden, people are
contacting me and tell me that Toshiba actually did sell
TLCS-9000 chips to customers, and they ask for documentation to
do reverse engineering.  Maybe this will shed some light on the
remaining unclarities.  Nevertheless, errors in this code generator
are quite possible (and will of course be fixed!).  At least the
few examples listed in \cite{Tosh9000} are assembled correctly.

Displacements included in machine instructions may only have a
certain maximum length (e.g. 9 or 13 bits).  In case the
displacement is longer, a prefix containing the 'upper bits' must
be prepended to the instruction.  AS will automatically insert
such prefixes when necessary, however it is also possible to
force usage of a prefix by adding a leading \verb!'>'!.  An
example for this:

\begin{verbatim}
  ld:g.b  (0h),0       ; no prefix
  ld:g.b  (400000h),0  ; prefix added automatically
  ld:g.b  (>0h),0      ; forced prefix
\end{verbatim}

%%---------------------------------------------------------------------------

\section{TC9331}

Toshiba supplied a (DOS-based) assembler for this processor which
was named ASM31T.  This assembler supports a number of syntax
elements which could not be mapped on the capabilities of AS
without risking incompatibilities for existing source files for
other targets.  The following issues might require changes on
programs written for ASM31T:

\begin{itemize}
\item{ASM31T supports C-like comments (\verb!/* ... */!) which
      may also span multiple lines.  Such comments are not
      supported by AS and have to be replaced by comments
      beginning with a semicolon.}
\item{Similar to ASM31T, AS supports comments with round parentheses
      (\verb!( ... )!), however only within a single command
      argument.  Should such a comment contain a comma, this
      comma will be treated like an argument separator and the
      comment will not be skipped when parsing the arguments.}
\item{ASM31T allows symbol and label names containing a dash.
      AS does not allow this, because the dash is regarded to be
      the subtraction operator.  It would be unclear whether an
      expression like \verb!end-start! represents a single symbol
      or the difference of two symbols.}
\item{ASM31T requires an \tty{END} statement as the last
      statement of the program; this is optional for AS.}
\end{itemize}

Furthermore, AS currently lacks the capabilities to detect
conflicting uses of functional units in a machine instructions.
Toshiba's documentation is a bit difficult to understand in this
respect...

%%---------------------------------------------------------------------------

\section{29xxx}

As it was already described in the discussion of the \tty{ASSUME}
instruction, AS can use the information about the current setting of
the RBP register to detect accesses to privileged registers in user
mode.  This ability is of course limited to direct accesses (i.e.
without using the registers IPA...IPC), and there is one more
pitfall: as local registers (registers with a number $>$127) are
addressed relative to the stack pointer, but the bits in RBP always
refer to absolute numbers, the check is NOT done for local registers.
An extension would require AS to know always the absolute value of
SP, which would at least fail for recursive subroutines...

%%---------------------------------------------------------------------------

\section{80C16x}

As it was already explained in the discussion of the \tty{ASSUME}
instruction, AS tries to hide the fact that the processor has more
physical than logical RAM as far as possible.  Please keep in mind
that the DPP registers are valid only for data accesses and only
have an influence on absolute addressing, neither on indirect nor on indexed
addresses.  AS cannot know which value the computed address may take
at runtime...
The paging unit unfortunately does not operate for code accesses so
one has to work with explicit long or short \tty{CALL}s, \tty{JMP}s, or
\tty{RET}s.  At least for the ''universal'' instructions \tty{CALL} and
\tty{JMP}, AS will automatically use the shortest variant, but at least for the RET one
should know where the call came from.  \tty{JMPS} and \tty{CALLS} principally
require to write segment and address separately, but AS is written in
a way that it can split an address on its own, e.g. one can write
\begin{verbatim}
        jmps    12345h
\end{verbatim}
instead of
\begin{verbatim}
        jmps    1,2345h
\end{verbatim}
Unfortunately, not all details of the chip's internal instruction
pipeline are hidden: if CP (register bank address), SP (stack), or
one of the paging registers are modified, their value is not
available for the instruction immediately following.  AS tries to
detect such situations and will issue a warning in such cases.  Once
again, this mechanism only works for direct accesses.

Bits defined with the \tty{BIT} instruction are internally stored as a
12-bit word, containing the address in bits 4..11 and the bit
position in the four LSBs.  This order allows to refer the next resp.
previous bit by incrementing or decrementing the address.  This will
however not work for explicit bit specifications when a word boundary
is crossed.  For example, the following expression will result in a
range check error:
\begin{verbatim}
        bclr    r5.15+1
\end{verbatim}
We need a \tty{BIT} in this situation:
\begin{verbatim}
msb     bit     r5.15
        .
        .
        bclr    msb+1
\end{verbatim}
The SFR area was doubled for the 80C167/165/163: bit 12 flags that a bit
lies in the second part.  Siemens unfortunately did not foresee that
256 SFRs (128 of them bit addressable) would not suffice for
successors of the 80C166.  As a result, it would be impossible to
reach the second SFR area from F000H..F1DFH with short addresses or
bit instructions if the developers had not included a toggle
instruction:
\begin{verbatim}
        EXTR    #n
\end{verbatim}
This instruction has the effect that for the next \tty{n} instructions
($0<n<5$), it is possible to address the alternate SFR space instead of
the normal one.  AS does not only generate the appropriate machine
code when it encounters this instruction.  It also sets an internal
flag that will only allow accesses to the alternate SFR space for
the next \tty{n} instructions.  Of course, they may not contain jumps...
Of course, it is always possible to define bits from either area at
any place, and it is always possible to reach all registers with
absolute addresses.  In contrast, short and bit addressing only works
for one area at a time, attempts contradicting to this will result in
an error message.

The situation is similar for prefix instructions and absolute resp.
indirect addressing: as the prefix argument and the address
expression cannot always be evaluated at assembly time, chances for
checking are limited and AS will limit itself to warnings...in
detail, the situation is as follows:
\begin{itemize}
\item{fixed specification of a 64K bank with \tty{EXTS} or \tty{EXTSR}: the address
      expression directly contains the lower 16 bits of the target
      address.  If the prefix and the following instruction have a
      constant operand, AS will check if the the prefix argument and bits
      16..23 of the target address are equal.}
\item{fixed specification of a 16K page with \tty{EXTP} or \tty{EXTPR}: the address
      expression directly contains the lower 14 bits of the target
      address.  Bits 14 and 15 are fixed to 0, as the processor ignores
      them in this mode.  If the prefix and the following instruction
      have a constant operand, AS will check if the the prefix argument
      and bits 14..23 of the target address are equal.}
\end{itemize}
An example to clarify things a bit (the DPP registers have their
reset values):
\begin{verbatim}
        extp    #7,#1      ; range from 112K..128K
        mov     r0,1cdefh  ; results in address 0defh in code
        mov     r0,1cdefh  ; -->warning
        exts    #1,#1      ; range from 64K..128K
        mov     r0,1cdefh  ; results in address 0cdefh in code
        mov     r0,1cdefh  ; -->warning
\end{verbatim}

%%---------------------------------------------------------------------------

\section{PIC16C5x/16C8x}

Similar to the MCS-48 family, the PICs split their program memory
into several banks because the opcode does not offer enough space for
a complete address.  AS uses the same automatism for the instructions
\tty{CALL} and \tty{GOTO}, i.e. the PA bits in the status word are set according
to the start and target address.  However, this procedure is far more
problematic compared to the 48's:
\begin{enumerate}
\item{The instructions are not any more one word long (up to three
      words).  Therefore, it is not guaranteed that they can be
      skipped with a conditional branch.}
\item{It is possible that the program counter crosses a page boundary
      while the program sequence is executed.  The setting of PA bits
      AS assumes may be different from reality.}
\end{enumerate}
The instructions that operate on register W and another register
normally require a second parameter that specifies whether the result
shall be stored in W or the register.  Under AS, it is valid to omit
the second parameter.  The assumed target then depends upon the
operation's type: For unary operations, the result is by default
stored back into the register.  These instructions are:
\begin{quote}{\tt
    COMF, DECF, DECFSZ, INCF, INCFSZ, RLF, RRF, and SWAPF
}\end{quote}
The other operations by default regard W as an accumulator:
\begin{quote}{\tt
    ADDWF, ANDWF, IORWF, MOVF, SUBWF, and XORWF
}\end{quote}
The syntax defined by Microchip to write literals is quite obscure
and reminds of the syntax used on IBM 360/370 systems (greetings from
the stone-age...).  To avoid introducing another branch into the
parser, with AS one has to write constants in the Motorola syntax
(optionally Intel or C in \tty{RELAXED} mode).

%%---------------------------------------------------------------------------

\section{PIC 17C4x}

With two exceptions, the same hints are valid as for its two smaller
brothers: the corresponding include file only contains register
definitions, and the problems concerning jump instructions are much
smaller.  The only exception is the \tty{LCALL} instruction, which allows a
jump with a 16-bit address.  It is translated with the following
''macro'':
\begin{verbatim}
        MOVLW   <addr15..8>
        MOWF    3
        LCALL   <addr0..7>
\end{verbatim}

%%---------------------------------------------------------------------------

\section{SX20/28}

The limited length of the instruction word does not permit specifying
a complete program memory address (11 bits) or data memory address (8
bits).  The CPU core augments the truncated address from the
instruction word with the PA bits from the STATUS registers,
respectively with the upper bits of the FSR register.  It is possible
to inform the assembler via \tty{ASSUME} instructions about the
contents of these two registers.  In case that addresses are used
that are inaccessible with th current values, a warning is issued.

%%---------------------------------------------------------------------------

\section{ST6}

These processors have the ability to map their code ROM pagewise into the
data area.  I am not keen on repeating the whole discussion of the
\tty{ASSUME} instruction at this place, so I refer to the corresponding
section (\ref{ST6Assume}) for an explanation how to read constants out of
the code ROM without too much headache.

Some builtin ''macros'' show up when one analyzes the instruction set a
bit more in detail.  The instructions I found are listed in table
\ref{TabHid62} (there are probably even more...):
\par
\begin{table*}[htbp]
\begin{center}\begin{tabular}{|l|l|}
\hline
instruction & in reality \\
\hline
\hline
\tty{CLR A}      & \tty{SUB A,A} \\
\tty{SLA A}      & \tty{ADD A,A} \\
\tty{CLR addr}   & \tty{LDI addr,0} \\
\tty{NOP}        & \tty{JRZ PC+1} \\
\hline
\end{tabular}\end{center}
\caption{Hidden Macros in the ST62's Instruction Set\label{TabHid62}}
\end{table*}
Especially the last case is a bit astonishing...unfortunately, some
instructions are really missing.  For example, there is an \tty{AND}
instruction but no \tty{OR}...not to speak of an \tty{XOR}.  For this reason, the
include file \tty{STDDEF62.INC} contains also some helping macros
(additionally to register definitions).

The original assembler AST6 delivered by SGS-Thomson partially uses
different pseudo instructions than AS.  Apart from the fact that AS
does not mark pseudo instructions with a leading dot, the following
instructions are identical:
\begin{verbatim}
  ASCII, ASCIZ, BLOCK, BYTE, END, ENDM, EQU, ERROR, MACRO,
  ORG, TITLE, WARNING
\end{verbatim}
Table \ref{TabAST6} shows the instructions which have AS counterparts
with similar function.
\par
\begin{table*}[htbp]
\begin{center}\begin{tabular}{|l|l|l|}
\hline
AST6            & AS                     & meaning/function \\
\hline
\hline
\tty{.DISPLAY}  & \tty{MESSAGE}          & output message \\
\hline
\tty{.EJECT}    & \tty{NEWPAGE}          & new page in assembly listing \\
\hline
\tty{.ELSE}     & \tty{ELSEIF}           & conditional assembly \\
\hline
\tty{.ENDC}     & \tty{ENDIF}            & conditional assembly \\
\hline
\tty{.IFC}      & \tty{IF...}            & conditional assembly \\
\hline
\tty{.INPUT}    & \tty{INCLUDE}          & insert include file \\
\hline
\tty{.LIST}     & \tty{LISTING, MACEXP\_DFT}  & settings for listing \\
\hline
\tty{.PL}       & \tty{PAGE}             & page length of listing \\
\hline
\tty{.ROMSIZE}  & \tty{CPU}              & set target processor \\
\hline
\tty{.VERS}     & \tty{VERSION} (symbol) & query version \\
\hline
\tty{.SET}      & \tty{EVAL}             & redefine variables \\
\hline
\end{tabular}\end{center}
\caption{Equivalent Instructions AST6$\leftrightarrow$AS\label{TabAST6}}
\end{table*}

%%---------------------------------------------------------------------------

\section{ST7}

In \cite{ST7Man}, the \tty{.w} postfix to signify 16-bit addresses is only
defined for memory indirect operands.  It is used to mark that a
16-bit address is stored at a zero page address.  AS additionally
allows this postfix for absolute addresses or displacements of
indirect address expressions to force 16-bit displacements in spite
of an 8-bit value (0..255).

%%---------------------------------------------------------------------------

\section{ST9}

The ST9's bit addressing capabilities are quite limited: except for
the \tty{BTSET} instruction, only bits within the current set of working
registers are accessible.  A bit address is therefore of the
following style:
\begin{verbatim}
        rn.[!]b   ,
\end{verbatim}
whereby \tty{!} means an optional complement of a source operand.  If a bit
is defined symbolically, the bit's register number is stored in bits
7..4, the bit's position is stored in bits 3..1 and the optional
complement is kept in bit 0.  AS distinguishes explicit and symbolic
bit addresses by the missing dot.  A bit's symbolic name therefore
must not contain a dot, thought it would be legal in respect to the
general symbol name conventions.  It is also valid to invert a
symbolically referred bit:
\begin{verbatim}
bit2    bit     r5.3
        .
        .
        bld     r0.0,!bit2
\end{verbatim}
This opportunity also allows to undo an inversion that was done at
definition of the symbol.

The include file \tty{REGST9.INC} defines the symbolic names of all on-chip
registers and their associated bits. Keep however in mind that the
bit definitions only work after previously setting the working
register bank to the address of these peripheral registers!

In contrast to the definition file delivered with the AST9 assembler
from SGS-Thomson, the names of peripheral register names are only
defined as general registers (\tty{R...}), not also as working registers
(\tty{r...}).  The reason for this is that AS does not support register
aliases; a tribute to assembly speed.

%%---------------------------------------------------------------------------

\section{6804}

To be honest: I only implemented this processor in AS to quarrel
about SGS-Thomson's peculiar behaviour.  When I first read the 6804's
data book, the ''incomplete'' instruction set and the built-in macros
immediately reminded me of the ST62 series manufactured by the same
company.  A more thorough comparison of the opcodes gave surprising
insights: A 6804 opcode can be generated by taking the equivalent
ST62 opcode and mirroring all the bits!  So Thomson obviously did a
bit of processor core recycling...which would be all right if they
would not try to hide this:  different peripherals, motorola instead
of Zilog-style syntax, and the awful detail of \bb{not} mirroring operand
fields in the opcode (e.g. bit fields containing displacements).  The
last item is also the reason that finally convinced me to support the
6804 in AS.  I personally can only guess which department at Thomson
did the copy...

In contrast to its ST62 counterpart, the include file for the 6804
does not contain instruction macros that help a bit to deal with the
limited machine instruction set.  This is left as an exercise to the
reader!

%%---------------------------------------------------------------------------

\section{TMS3201x}

It seems that every semiconductor's ambition is to invent an own
notation for hexadecimal numbers.  Texas Instrument took an
especially eccentric approach for these processors: a $>$ sign as
prefix!  The support of such a format in AS would have lead to
extreme conflicts with AS's compare and shift operators.  I therefore
decided to use the Intel notation, which is what TI also uses for the
340x0 series and the 3201x's successors...

The instruction word of these processors unfortunately does not have
enough bits to store all 8 bits for direct addressing.  This is why
the data address space is split into two banks of 128 words.  AS
principally regards the data address space as a linear segment of 256
words and automatically clears bit 7 on direct accesses (an exception
is the \tty{SST} instruction that can only write to the upper bank).  The
programmer has to take care that the bank flag always has the correct
value!

Another hint that is well hidden in the data book: The \tty{SUBC}
instruction internally needs more than one clock for completion, but
the control unit already continues to execute the next instruction.
An instruction following \tty{SUBC} therefore may not access the
accumulator.  AS does not check for such conditions!

%%---------------------------------------------------------------------------

\section{TMS320C2x}

As I did not write this code generator myself (that does not lower
its quality by any standard), I can only roughly line out why there
are some instructions that force a prefixed label to be untyped, i.e.
not assigned to any specific address space: The 2x series of TMS
signal processors has a code and a data segment which are both 64
Kbytes large.  Depending on external circuitry, code and data space may
overlap, e.g. to allow storage of constants in the code area and
access them as data.  Data storage in the code segment may be
necessary because older versions of AS assume that the data segment
only consists of RAM that cannot have a defined power-on state in a
single board system.  They therefore reject storage of contents in
other segments than \tty{CODE}.  Without the feature of making symbols
untyped, AS would punish every access to a constant in code space
with a warning (''symbol out of wrong segment'').  To say it in detail,
the following instructions make labels untyped:
\begin{quote}\tt
  BSS, STRING, RSTRING, BYTE, WORD , LONG\\
  FLOAT, DOUBLE, EFLOAT, BFLOAT and TFLOAT
\rm\end{quote}
If one needs a typed label in front of one of these instructions, one
can work around this by placing the label in a separate line just
before the pseudo instruction itself.  On the other hand, it is
possible to place an untyped label in front of another pseudo
instruction by defining the label with \tty{EQU}, e.g.
\begin{verbatim}
<name>  EQU     $        .
\end{verbatim}

%%---------------------------------------------------------------------------

\section{TMS320C3x/C4x}

The syntax detail that created the biggest amount of headache for me
while implementing this processor family is the splitting of parallel
instructions into two separate source code lines.  Fortunately, both
instructions of such a construct are also valid single instructions.
AS therefore first generates the code for the first instruction and
replaces it by the parallel machine code when a parallel construct is
encountered in the second line.  This operation can be noticed in the
assembly listing by the machine code address that does not advance
and the double dot replaced with a \tty{R}.

Compared to the TI assembler, AS is not as flexible regarding the
position of the double lines that signify a parallel operation
(\tty{||}): One either has to place them like a label (starting in the
first column) or to prepend them to the second mnemonic.  The line
parser of AS will run into trouble if you do something else...

%%---------------------------------------------------------------------------

\section{TMS9900}

Similar to most older TI microprocessor families, TI used an own
format for hexadecimal and binary constants.  AS instead favours the
Intel syntax which is also common for newer processor designs from
TI.

The TI syntax for registers allows to use a simple integer number
between 0 and 15 instead of a real name (\tty{Rx} or \tty{WRx}).
This has two consequences:
\begin{itemize}
\item{\tty{R0...R15} resp. \tty{WR0..WR15} are simple predefined integer
      symbols with values from 0 to 15, and the definition of register
      aliases is a simple matter of \tty{EQU}.}
\item{In contrast to several other processors, I cannot offer the
      additional AS feature that allows to omit the character sigifying
      absolute addressing (a \@ sign in this case).  As a missing
      character would mean register numbers (from 0 to 15) in this case,
      it was not possible to offer the optional omission.}
\end{itemize}
Furthermore, TI sometimes uses \tty{Rx} to name registers and \tty{WRx}
at other places...currently both variants are recognized by AS.

%%---------------------------------------------------------------------------

\section{TMS70Cxx}

This processor family belongs to the older families developed by TI
and therefore TI's assemblers use their proprietary syntax for
hexadecimal resp. binary constants (a prefixed $<$ resp. \tty{?} character).
As this format could not be realized for AS, the Intel syntax is used
by default.  This is the format TI to which also switched over when
introducing the successors, of this family, the 370 series of
microcontrollers.  Upon a closer inspection of both's machine
instruction set, one discovers that about 80\% of all instruction are
binary upward compatible, and that also the assembly syntax is almost
identical - but unfortunately only almost.  TI also took the chance to
make the syntax more orthogonal and simple.  I tried to introduce
the majority of these changes also into the 7000's instruction set:
\begin{itemize}
\item{It is valid to use the more common \tty{\#} sign for immediate addressing
      instead of the percent sign.}
\item{If a port address (\tty{P...}) is used as source or destination in a
      \tty{AND, BTJO, BTJZ, MOV, OR}, or \tty{XOR} instruction, it is not necessary
      to use the mnemonic variant with an appended \tty{P} - the general
      form is sufficient.}
\item{The prefixed \tty{@} sign for absolute or B-relative addressing may be
      omitted.}
\item{Instead of \tty{CMPA, CMP} with \tty{A} as target may be written.}
\item{Instead of \tty{LDA} resp. \tty{STA}, one can simply use the
      \tty{MOV} instruction with \tty{A} as source resp. destination.}
\item{One can write \tty{MOVW} instead of \tty{MOVD}.}
\item{It is valid to abbreviate \tty{RETS} resp. \tty{RETI} as \tty{RTS}
      resp. \tty{RTI}.}
\item{\tty{TSTA} resp. \tty{TSTB} may be written as \tty{TST A} resp.
      \tty{TST B}.}
\item{\tty{XCHB B} is an alias for \tty{TSTB}.}
\end{itemize}
An important note: these variants are only allowed for the TMS70Cxx -
the corresponding 7000 variants are not allowed for the 370 series!

%%---------------------------------------------------------------------------

\section{TMS370xxx}

Though these processors do not have specialized instructions for bit
manipulation, the assembler creates (with the help of the \tty{DBIT}
instruction - see there) the illusion as if single bits were
addressable.  To achieve this, the \tty{DBIT} instructions stores an
address along with a bit position into an integer symbol which may
then be used as an argument to the pseudo instructions \tty{SBIT0, SBIT1,
CMPBIT, JBIT0}, and \tty{JBIT1}.  These are translated into the instructions
\tty{OR, AND, XOR, BTJZ}, and \tty{BTJO} with an appropriate bit mask.

There is nothing magic about these bit symbols, they are simple
integer values that contain the address in their lower and the bit
position in their upper half.  One could construct bit symbols
without the \tty{DBIT} instruction, like this:
\begin{verbatim}
defbit  macro   name,bit,addr
name    equ     addr+(bit<<16)
        endm
\end{verbatim}
but this technique would not lead to the \tty{EQU}-style syntax defined by
TI (the symbol to be defined replaces the label field in a line).
\bb{CAUTION!} Though \tty{DBIT} allows an arbitrary address, the pseudo
instructions can only operate with addresses either in the range from
0..255 or 1000h..10ffh.  The processor does not have an absolute
addressing mode for other memory ranges...

%%---------------------------------------------------------------------------

\section{MSP430(X)}
\label{MSPSpec}

The MSP was designed to be a RISC processor with a minimal power
consumption.  The set of machine instructions was therefore reduced
to the absolute minimum (RISC processors do not have a microcode ROM
so every additional instruction has to be implemented with additional
silicon that increases power consumption).  A number of instructions
that are hardwired for other processors are therefore emulated with
other instructions.  Older versions of AS implemented these
instructions via macros in the file \tty{REGMSP.INC}.  If one did
not include this file, you got error messages for more than
half of the instructions defined by TI.  This has been changed in
recent versions: as part of adding the 430X instruction set,
implementation of these instructions was moved into the assmebler's
core.  \tty{REGMSP.INC} now only contains addresses of I/O
registers.  If you need the old macros for some reason, they have
been moved to the file \tty{EMULMSP.INC}.

Instruction emulation also covers some special cases not handled
by the original TI assembler.  For instance,
\begin{verbatim}
    rlc  @r6+
\end{verbatim}
is automatically assembled as
\begin{verbatim}
    addc @r6+,-2(r6)
\end{verbatim}

%%---------------------------------------------------------------------------

\section{TMS1000}

At last, world's first microcontroller finally also supported in
AS - it took long to fill this gap, but now it is done.  This
target has some pitfalls that will be discussed shortly in this
section.

First, the instruction set of these controllers is partially
defined via the ROM mask, i.e. the function of some opcodes may
be freely defined to some degree.  AS only knows the instructions
and codings that are described as default codings in
\cite{TMS1000PGMRef}.  If you have a special application with an
instruction set deviating from this, you may define and modify
instructions via macros and the \tty{DB} instruction.

Furthermore, keep in mind that branches and subroutine calls only
contain the lower 6 bits of the target address.  The upper 4
resp. 5 bits are fetched from page and chapter registers tha
thave to be set beforehand.  AS cannot check whether these
registers have been set correctly by the programmer! At least for
the cas of staying in the same chapter, there are the assmebler
pseudo instructions \tty{CALLL} resp. \tty{BL} that combine an
\tty{LDP} and \tty{CALL/BR} instruction.  Regarding the limited
amount of program memory, this is a convenient yet inefficient
variant.

%%---------------------------------------------------------------------------

\section{COP8 \& SC/MP}
\label{COP8Spec}

National unfortunately also decided to use the syntax well known from
IBM mainframes (and much hated by me..) to write non-decimal integer
constants.  Just like with other processors, this does not work with
AS's parser.  ASMCOP however fortunately also seems to allow the C
syntax, which is why this became the default for the COP series and
the SC/MP...

%%---------------------------------------------------------------------------

\section{SC144xxx}
\label{SC144xxspec}

Originally, National offered a relatively simple assembler for this series
of DECT controllers.  An much more powerful assembler has been announced
by IAR, but it is not available up to now.  However, since the development
tools made by IAR are as much target-independent as possible, one can
roughly estimate the pseudo instructions it will support by looking at
other available target platforms.  With this in mind, the (few)
SC144xx-specific instructions {\tt DC, DC8, DW16, DS, DS8, DS16, DW} were
designed.  Of course, I didn't want to reinvent the wheel for pseudo
instructions whose functionality is already part of the AS core.
Therefore, here is a little table with equivalences.  The statements
\tty{ALIGN, END, ENDM, EXITM, MACRO, ORG, RADIX, SET,} and \tty{REPT} both
exist for the IAR assembler and AS and have same functionality.  Changes
are needed for the following instructions:

\begin{table*}[htb]
\begin{center}\begin{tabular}{|l|l|l|}
\hline
IAR & AS & Funktion\\
\hline
\hline
\tty{\#include} & \tty{include} & include file \\
\tty{\#define} & \tty{SET, EQU} & define symbol \\
\tty{\#elif, ELIF, ELSEIF} & \tty{ELSEIF} & start another \\
                           &              & IF branch \\
\tty{\#else, ELSE} & \tty{ELSE} & last branch of an IF \\
                   &            & construct \\
\tty{\#endif, ENDIF} & \tty{ENDIF} & ends an IF construct \\
\tty{\#error} & \tty{ERROR, FATAL} & create error message \\
\tty{\#if, IF} & \tty{IF} & start an IF construct \\
\tty{\#ifdef} & \tty{IFDEF} & symbol defined ? \\
\tty{\#ifndef} & \tty{IFNDEF} & symbol not defined ? \\
\tty{\#message} & \tty{MESSAGE} & output message \\
\tty{=, DEFINE, EQU} & \tty{=, EQU} & fixed value assignment \\
\tty{EVEN} & \tty{ALIGN 2} & force PC to be equal \\
\tty{COL, PAGSIZ} & \tty{PAGE} & set page size for listing \\
\tty{ENDR} & \tty{ENDM} & end REPT construct \\
\tty{LSTCND, LSTOUT} & \tty{LISTING} & control amount of listing \\
\tty{LSTEXP, LSTREP} & \tty{MACEXP} & list expanded macros? \\
\tty{LSTXRF} & \verb!<command line>! & generate cross reference \\
\tty{PAGE} & \tty{NEWPAGE} & new page in listing \\
\tty{REPTC} & \tty{IRPC} & repetition with character \\
            &            & replacement \\
\hline
\end{tabular}\end{center}
\end{table*}

There is no direct equivalent for {\tt CASEON}, {\tt CASEOFF,}
\tty{LOCAL}, \tty{LSTPAG}, \tty{\#undef,} and {\tt REPTI}.

A 100\% equivalent is of course impossible as long as there is no C-like
preprocessor in AS.  C-like comments unfortunately are also impossible
at the moment.  Caution: When modifying IAR codes for AS, do not forget to
move converted preprocessor statements out of column 1 as AS reserves this
column exclusively for labels!

%%---------------------------------------------------------------------------

\section{NS32xxx}

As one might expect from a CISC processor, the NS32xxx series provides
sophisticated and complex addressing modes.  National defied the assembly syntax
for each of them in its manuals, and this is also the syntax AS implements.
However, as for every architecture that was supported by third-party tools,
there are deviations and extensions, and I added a few of them to AS:

The syntax to use PC-relative addressing, as defined by National, is:
\begin{verbatim}
 movb r0,*+disp
\end{verbatim}
This of course quite clearly expresses what is happening at runtime, one however
has to compute the distance himself if a certain memory location is to be
addressed:
\begin{verbatim}
 movb r0,*+(addr-*)
\end{verbatim}
The first simplification is that under certain conditions, it is sufficient to
just write:
\begin{verbatim}
 movb r0,addr
\end{verbatim}
since absolute addressierung is marked by a \@ prefix.  This is allowed under
the following conditions:
\begin{itemize}
\item{Immediate addressierung is not allowed, e.g. because the operand is
      the destination and there is no risk os ambiguities.}
\item{An index extension is used (appended in square brackets), which must not
      be combined with immediate addressing.}
\end{itemize}
As an alterntative, AS also supports the following way to use PC-relative addressing:
\begin{verbatim}
 movb r0,addr(pc)
\end{verbatim}
Analog to the 68000, the distance is computed automatically.

The external mode, whis written this way in National syntax:
\begin{verbatim}
 movb r0,ext(disp1)+disp2
\end{verbatim}
there is another supported syntax variant:
\begin{verbatim}
 movb r0,disp2(disp1(ext))
\end{verbatim}
which used to be common in UNIX environments.

%%---------------------------------------------------------------------------

\section{uPD78(C)1x}
\label{78C1xSpec}

For relative, unconditional instructions, there is the \tty{JR} instruction
branch distance -32...+31, one byte), and the \tty{JRE} instruction (branch
distance -256...+255, two bytes).  AS furthermore knows the \tty{J} pseudo
instruction, which automatically selects the shortest possible variant.

Architecture and instructon set of these processors are coarsely
related to the Intel 8080/8085 - thi is also true for the
mnemonics.  The adressing mode (direct, indirect, immediate) is
packed into the mnemonic, and 16 bit registers (BC, DE, HL) are
written with just one letter.  However, since NEC itself also
uses at some places written-out register names and parentheses to
signify indirect addressing, I decided to support some
alternative notations next to the 'official' ones.   Some non-NEC
tools like disassemblers seem to use these notations either:

\begin{itemize}
\item{It is allowed to use \tty{BC}, \tty{(B)}, or \tty{(BC)}
      instead of \tty{B}.}
\item{It is allowed to use \tty{DE}, \tty{(D)}, or \tty{(DE)}
      instead of \tty{D}.}
\item{It is allowed to use \tty{HL}, \tty{(H)}, or \tty{(HL)}
      instead of \tty{H}.}
\item{It is allowed to use \tty{DE+}, \tty{(D+)}, \tty{(DE+)},
      or \tty{(DE)+} instead of \tty{D+}.}
\item{It is allowed to use \tty{HL+}, \tty{(H+)}, \tty{(HL+)},
      or \tty{(HL)+} instead of \tty{H+}.}
\item{It is allowed to use \tty{DE-}, \tty{(D-)}, \tty{(DE-)},
      or \tty{(DE)-} instead of \tty{D-}.}
\item{It is allowed to use \tty{HL-}, \tty{(H-)}, \tty{(HL-)},
      or \tty{(HL)-} instead of \tty{H-}.}
\item{It is allowed to use \tty{DE++}, \tty{(D++)}, \tty{(DE++)},
      or \tty{(DE)++} instead of \tty{D++}.}
\item{It is allowed to use \tty{HL++}, \tty{(H++)}, \tty{(HL++)},
      or \tty{(HL)++} instead of \tty{H++}.}
\item{It is allowed to use \tty{DE--}, \tty{(D--)}, \tty{(DE--)},
      or \tty{(DE)--} instead of \tty{D--}.}
\item{It is allowed to use \tty{HL--}, \tty{(H--)}, \tty{(HL--)},
      or \tty{(HL)--} instead of \tty{H--}.}
\item{It is allowed to use \tty{HL+A}, \tty{A+H}, \tty{A+HL},
      \tty{(H+A)}, \tty{(HL+A)}, \tty{(A+H)}, or \tty{(A+HL)}
      instead of \tty{H+A}.}
\item{It is allowed to use \tty{HL+B}, \tty{B+H}, \tty{B+HL},
      \tty{(H+B)}, \tty{(HL+B)}, \tty{(B+H)}, or \tty{(B+HL)}
      instead of \tty{H+B}.}
\item{It is allowed to use \tty{HL+EA}, \tty{EA+H}, \tty{EA+HL},
      \tty{(H+EA)}, \tty{(HL+EA)}, \tty{(EA+H)}, or \tty{(EA+HL)}
      instead of \tty{H+EA}.}
\end{itemize}

%%---------------------------------------------------------------------------

\section{75K0}
\label{75K0Spec}

Similar to other processors, the assembly language of the 75 series
also knows pseudo bit operands, i.e. it is possible to assign a
combination of address and bit number to a symbol that can then be
used as an argument for bit oriented instructions just like explicit
expressions.  The following three instructions for example generate
the same code:
\begin{verbatim}
ADM     sfr     0fd8h
SOC     bit     ADM.3

        skt     0fd8h.3
        skt     ADM.3
        skt     SOC
\end{verbatim}
AS distinguishes direct and symbolic bit accesses by the missing dot
in symbolic names; it is therefore forbidden to use dots in symbol
names to avoid misunderstandings in the parser.

The storage format of bit symbols mostly accepts the binary coding in
the machine instructions themselves:  16 bits are used, and there is
a ''long'' and a ''short'' format.  The short format can store the
following variants:
\begin{itemize}
\item{direct accesses to the address range from 0FBxH to 0FFxH}
\item{indirect accesses in the style of \tty{Addr.@L} (0FC0H $\leq$ \tty{Addr} $\leq$0FFFH)}
\item{indirect accesses in the style of \tty{@H+d4.bit}}
\end{itemize}
The upper byte is set to 0, the lower byte contains the bit
expression coded according to \cite{NEC75}.  The long format in contrast
only knows direct addressing, but it can cover the whole address space
(given a correct setting of MBS and MBE).  A long expression stores
bits 0..7 of the address in the lower byte, the bit position in bits
8 and 9, and a constant value of 01 in bits 10 and 11.  The highest
bits allow to distinguish easily between long and short addresses via
a check if the upper byte is 0.  Bits 12..15 contain bits 8..11 of
the address; they are not needed to generate the code, but they have
to be stored somewhere as the check for correct banking can only
take place when the symbol is actually used.

%%---------------------------------------------------------------------------

\section{78K0}
\label{78K0Spec}

NEC uses different ways to mark absolute addressing in its data
books:
\begin{itemize}
\item{absolute short: no prefix}
\item{absolute long: prefix of \tty{!}}
\item{PC relative: prefix of \tty{\$}}
\end{itemize}
Under AS, these prefixes are only necessary if one wants to force a
certain addressing mode and the instruction allows different
variants.  Without a prefix, AS will automatically select the shortest
variant.  It should therefore rarely be necessary to use a prefix in
practice.

%%---------------------------------------------------------------------------

\section{78K2/78K3/78K4}
\label{78K234Spec}

Analogous to the 78K0, NEC here also uses dollar signs and exclamation
marks to specify different lengths of address expressions.  The selection
between long and short addresses is done automatically (both in RAM and
SFR areas), only relative addressing has to be selected explicitly, if an
instruction supports both variants (like {\tt BR}).

An additional remark (which is also true for the 78K0): Those who want to
use Motorola syntax via {\tt RELAXED}, might have to put hexadecimal
constants in parentheses, since the leading dollar sign might be
misunderstood as relative addressing...

%%---------------------------------------------------------------------------

\section{uPD772x}

Both the 7720 and 7725 are provided by the same code generator and are
extremely similar in their instruction set.  One should however not
beleive that they are binary compatible: To get space for the longer
address fields and additional instructions, the bit positions of some
fields in the instruction word have changed, and the instruction length
has changed from 23 to 24 bits.  The code format therefore uses different
header ids for both CPUs.

They both have in common that in addition to the code and data segment,
there is also a ROM for storage of constants.  In the case of AS, it is
mapped onto the \tty{ROMDATA} segment!

%%---------------------------------------------------------------------------

\section{F2MC16L}

Along with the discussion of the {\tt ASSUME} statement, it has already
been mentioned that it is important to inform AS about the correct current
values of all bank registers - if your program uses more than 64K RAM or
64K ROM.  With these assumptions in mind, AS checks every direct memory
access for attempts to access a memory location that is currently not in
reach.  Of course, standard situations only require knowledge of DTB and
DPR for this purpose, since ADB resp. SSB/USB are only used for indirect
accesses via RW2/RW6 resp. RW3/RW7 and this mechanism anyway doesn't work
for indirect accesses.  However, similar to the 8086, it is possible to
place a prefix in front of an instruction to replace DTB by a different
register.  AS therefore keeps track of used segment prefixes and
toggles appropriately for the next {\em machine instruction}.  A pseudo
instruction placed between the prefix and the machine instruction does
{\em not} reset the toggle.  This is also true for pseudo instructions
that store data or modify the program counter.  Which doesn't make much
sense anyway...

%%---------------------------------------------------------------------------

\section{MN161x}

This target is special because there are two different code generators one may
choose from.  The first one was kindly provided by Haruo Asano and that may be
reached via the CPU names \tty{MN1610} resp.\tty{MN1613}.  The other one was
written by me and is activated via the CPU names \tty{MN1610ALT} resp.
\tty{MN1613ALT}.  If you want to use the MN1613's extended address space of
256 KWords, or if you want to experiment with the MN1613's floating point
formant, you have to use the \tty{ALT} target.

%%---------------------------------------------------------------------------

\section{CDP180x}

This family of processors supports both long and short branches: a short
branch is only possible within the same 256 byte memory page, and a long branch
is possible to any target in the 64K address space.  The assembly syntax provides
different mnemonics for both variants (the long variant with a leading 'L'), but
there is no variant that would let the assembler decide itself between long
or short.  AS supports such 'pseudo instructions' as an extension:
\begin{itemize}
\item{\tty{JMP} becomes \tty{BR} oder \tty{LBR}.}
\item{\tty{JZ} becomes \tty{BZ} oder \tty{LBZ}.}
\item{\tty{JNZ} becomes \tty{BNZ} oder \tty{LBNZ}.}
\item{\tty{JDF} becomes \tty{BDF} oder \tty{LBDF}.}
\item{\tty{JPZ} becomes \tty{BPZ} oder \tty{LBPZ}.}
\item{\tty{JGE} becomes \tty{BGE} oder \tty{LBGE}.}
\item{\tty{JNF} becomes \tty{BNF} oder \tty{LBNF}.}
\item{\tty{JM} becomes \tty{BM} oder \tty{LBM}.}
\item{\tty{JL} becomes \tty{BL} oder \tty{LBL}.}
\item{\tty{JQ} becomes \tty{BQ} oder \tty{LBQ}.}
\item{\tty{JNQ} becomes \tty{BNQ} oder \tty{LBNQ}.}
\end{itemize}


%%---------------------------------------------------------------------------

\section{KENBAK}

The KENBAK-1 was developed in 1970, at a time when the first microprocessor
was still three years away.  One may assume that for the few hobbyists that
could afford the kit back then, this was their first and only computer.  As
a consequence, they had nothing they could run an assembler on, the KENBAK-1 
itself with its 256 bytes of memory was way too small for such a task.  The
preferred method was to use pre-printed tables, which had fields to fill in
instructions and machine codes.  Once this ''programming job'' was done, one
would enter the machine code manually via the computer's switch row.

The effect of this is that though the KENBAK's assembly language is described
in the manual, there is no real formal definition of it.  When Grant Stockly
released new KENBAK kits a few years ago, he did a first implementation of the
KENBAK on my assembler.  Unfortunately, this never went upstream.  I tried
to take up his ideas in my implementation, but on the other hand I also tried to
offer a syntax that should be familiar to programmers of 6502, Z80 or similar
processors.  The following table lists the syntax differences:

\hfuzz=60pt
\begin{center}\begin{longtable}{|l|l|l|}
\hline
Stockly & Alternativ & Bemerkung \\
\hline
\hline
\endhead
\multicolumn{3}{|l|}{\bf Arithmetic/Logic (ADD/SUB/LOAD/STORE/AND/OR/LNEG)} \\
\hline
{\it instr} {\tt Constant}, {\it Reg}, {\it Wert}, & {\it instr} {\it Reg}, {\it \#Wert} & immediate \\
{\it instr} {\tt Memory}, {\it Reg}, {\it Addr}, & {\it instr} {\it Reg}, {\it Addr} & direct \\
{\it instr} {\tt Indirect}, {\it Reg}, {\it Addr}, & {\it instr} {\it Reg}, {\it (Addr)} & direct \\
{\it instr} {\tt Indexed}, {\it Reg}, {\it Addr}, & {\it instr} {\it Reg}, {\it Addr},X & indexed \\
{\it instr} {\tt Indirect-Indexed}, {\it Reg}, {\it Addr}, & {\it instr} {\it Reg}, {\it (Addr)},X & indirect-indexed \\
\hline
\multicolumn{3}{|l|}{\bf Jumps} \\
\hline
{\tt JPD} {\it Reg}, {\it Cond}, {\it Addr} & {\tt JP} {\it Reg}, {\it Cond}, {\it Addr} & conditional-direct \\
{\tt JPI} {\it Reg}, {\it Cond}, {\it Addr} & {\tt JP} {\it Reg}, {\it Cond}, {\it (Addr)} & conditional-indirect \\
{\tt JMD} {\it Reg}, {\it Cond}, {\it Addr} & {\tt JM} {\it Reg}, {\it Cond}, {\it Addr} & conditional-direct \\
{\tt JMI} {\it Reg}, {\it Cond}, {\it Addr} & {\tt JM} {\it Reg}, {\it Cond}, {\it (Addr)} & conditional-indirect \\
{\tt JPD} {\tt Unconditional}, {\it Cond}, {\it Addr} & {\tt JP} {\it Addr} & unconditional-direct \\
{\tt JPI} {\tt Unconditional}, {\it Cond}, {\it Addr} & {\tt JP} {\it (Addr)} & unconditional-indirect \\
{\tt JMD} {\tt Unconditional}, {\it Cond}, {\it Addr} & {\tt JM} {\it Addr} & unconditional-direct \\
{\tt JMI} {\tt Unconditional}, {\it Cond}, {\it Addr} & {\tt JM} {\it (Addr)} & unconditional-indirect \\
\hline
\multicolumn{3}{|l|}{\bf Jump Conditions} \\
\hline
{\tt Non-zero} & {\tt NZ} & $\neq 0$ \\
{\tt Zero} & {\tt Z} & $= 0$ \\
{\tt Negative} & {\tt N} & $< 0$ \\
{\tt Positive} & {\tt P} & $\geq 0$ \\
{\tt Positve-Non-zero} & {\tt PNZ} & $ > 0$ \\
\hline
\multicolumn{3}{|l|}{\bf Skips} \\
\hline
{\tt SKP 0}, {\it bit}, {\it Addr} & {\tt SKP0} {\it bit}, {\it Addr} {\it [,Dest]} & \\
{\tt SKP 1}, {\it bit}, {\it Addr} & {\tt SKP1} {\it bit}, {\it Addr} {\it [,Dest]} & \\
\hline
\multicolumn{3}{|l|}{\bf Bit Manipulation} \\
\hline
{\tt SET 0}, {\it bit}, {\it Addr} & {\tt SET0} {\it bit}, {\it Addr} & \\
{\tt SET 1}, {\it bit}, {\it Addr} & {\tt SET1} {\it bit}, {\it Addr} & \\
\hline
\multicolumn{3}{|l|}{\bf Shifts/Rotates} \\
\hline
{\tt SHIFT LEFT}, {\it cnt}, {\it Reg} & {\tt SFTL} {\it [cnt,]} {\it Reg} & \\
{\tt SHIFT RIGHT}, {\it cnt}, {\it Reg} & {\tt SFTR} {\it [cnt,]} {\it Reg} & arithm. Shift \\
{\tt ROTATE LEFT}, {\it cnt}, {\it Reg} & {\tt ROTL} {\it [cnt,]} {\it Reg} & \\
{\tt ROTATE RIGHT}, {\it cnt}, {\it Reg} & {\tt ROTR} {\it [cnt,]} {\it Reg} & \\
\hline
\caption{KENBAK-Befehlssyntax \label{TabKENBAKSyntax}}
\end{longtable}\end{center}
\hfuzz=0pt

There is no pseudo instruction to switch between these syntax variants.  They may
both be used anytime and in an arbitrary mix.

The target address {\it [Dest]} that may optionally be added to skip instructions
will not become part of the machine code.  The assembler only checks whether the
processor wil actually skip to the given address.  This allows for instance to check
whether one actually tries to skip a one-byte instruction.  If the shift count
argument {\it [cnt]} is omitted, a one-bit shift/rotate is coded.

%%===========================================================================

\cleardoublepage
\chapter{File Formats}

In this chapter, the formats of files AS generates shall be explained
whose formats are not self-explanatory.

%%---------------------------------------------------------------------------

\section{Code Files}
\label{SectCodeFormat}

The format for code files generated by the assembler must be able to
separate code parts that were generated for different target
processors; therefore, it is a bit different from most other formats.
Though the assembler package contains tools to deal with code files,
I think is a question of good style to describe the format in short:

If a code file contains multibyte values, they are stored in little
endian order.  This rule is already valid for the 16-bit magic word
\$1489, i.e. every code file starts with the byte sequence \$89/\$14.

This magic word is followed by an arbitrary number of ''records''.  A
record may either contain a continuous piece of the code or certain
additional information.  Even without switching to different
processor types, a file may contain several code-containing records,
in case that code or constant data areas are interrupted by reserved
memory areas that should not be initialized.  This way, the assembler
tries to keep the file as short as possible.

Common to all records is a header byte which defines the record's type
and its contents.  Written in a PASCALish way, the record structure
can be described in the following way:
\begin{verbatim}
FileRecord = RECORD CASE Header:Byte OF
              $00:(Creator:ARRAY[] OF Char);
              $01..
              $7f:(StartAdr : LongInt;
                   Length   : Word;
                   Data     : ARRAY[0..Length-1] OF Byte);
              $80:(EntryPoint:LongInt);
              $81:(Header   : Byte;
                   Segment  : Byte;
                   Gran     : Byte;
                   StartAdr : LongInt;
                   Length   : Word;
                   Data     : ARRAY[0..Length-1] OF Byte);
             END
\end{verbatim}
This description does not express fully that the length of data
fields is variable and depends on the value of the \tty{Length} entries.

A record with a header byte of \$81 is a record that may contain code
or data from arbitrary segments.  The first byte (\tty{Header}) describes
the processor family the following code resp. data was generated for (see
table \ref{TabHeader}).

\begin{center}\begin{longtable}{|c|l||c|l|}
\hline
Header  & Family & Header  & Family \\
\hline
\hline
\endhead
\$01 &    680x0, 6833x         & \$02 &    ATARI\_VECTOR \\
\$03 &    M*Core               & \$04 &    XGATE \\
\$05 &    PowerPC              & \$06 &    XCore \\
\$07 &    TMS1000              & \$08 &    NS32xxx \\
\$09 &    DSP56xxx             & \$0a &    CP-1600 \\
\$0b &    HP Nano Processor    & \$0c &    IM6100/6120 \\
\$0d &    NEC V60              & \$0e &    IBM PALM \\
\$0f &    CP-3F                & \$11 &    65xx/MELPS-740 \\
\$12 &    MELPS-4500           & \$13 &    M16 \\
\$14 &    M16C                 & \$15 &    F$^{2}$MC8L \\
\$16 &    F$^{2}$MC16L         & \$19 &    65816/MELPS-7700 \\
\$1a &    PDK13                & \$1b &    PDK14 \\
\$1c &    PDK15                & \$1d &    PDK16 \\
\$21 &    MCS-48               & \$25 &    SYM53C8xx \\
\$27 &    KENBAK               & \$29 &    29xxx \\
\$2a &    i960                 & \$31 &    MCS-51 \\
\$32 &    ST9                  & \$33 &    ST7 \\
\$35 &    Z8000                & \$35 &    Super8 \\
\$36 &    MN161x               & \$37 &    2650 \\
\$38 &    1802/1805            & \$39 &    MCS-96/196/296 \\
\$3a &    8X30x                & \$3b &    AVR \\
\$3c &    XA                   & \$3d &    AVR (8-Bit Code-Segment) \\
\$3e &    8008                 & \$3f &    4004/4040 \\
\$40 &    H16                  & \$41 &    8080/8085 \\
\$42 &    8086..V35            & \$43 &    SX20 \\
\$44 &    F8                   & \$45 &    S12Z \\
\$46 &    78K4                 & \$47 &    TMS320C6x \\
\$48 &    TMS9900              & \$49 &    TMS370xxx \\
\$4a &    MSP430               & \$4b &    TMS320C54x \\
\$4c &    80C166/167           & \$4d &    OLMS-50 \\
\$4e &    OLMS-40              & \$4f &    MIL STD 1750 \\
\$50 &    HMCS-400             & \$51 &    Z80/180/380 \\
\$52 &    TLCS-900             & \$53 &    TLCS-90 \\
\$54 &    TLCS-870             & \$55 &    TLCS-47 \\
\$56 &    TLCS-9000            & \$57 &    TLCS-870/C \\
\$58 &    NEC 78K3             & \$59 &    eZ8 \\
\$5a &    TC9331               & \$5b &    KCPSM3 \\
\$5c &    LatticeMico8         & \$5d &    NEC 75xx \\
\$5e &    68RS08               & \$5f &    COP4 \\
\$60 &    78K2                 & \$61 &    6800, 6301, 6811 \\
\$62 &    6805/HC08            & \$63 &    6809 \\
\$64 &    6804                 & \$65 &    68HC16 \\
\$66 &    68HC12               & \$67 &    ACE \\
\$68 &    H8/300(H)            & \$69 &    H8/500 \\
\$6a &    807x                 & \$6b &    KCPSM \\
\$6c &    SH7000               & \$6d &    SC14xxx \\
\$6e &    SC/MP                & \$6f &    COP8 \\
\$70 &    PIC16C8x             & \$71 &    PIC16C5x \\
\$72 &    PIC17C4x             & \$73 &    TMS-7000 \\
\$74 &    TMS3201x             & \$75 &    TMS320C2x \\
\$76 &    TMS320C3x/C4x        & \$77 &    TMS320C20x/C5x \\
\$78 &    ST6                  & \$79 &    Z8 \\
\$7a &    $\mu$PD78(C)10       & \$7b &    75K0 \\
\$7c &    78K0                 & \$7d &    $\mu$PD7720 \\
\$7e &    $\mu$PD7725          & \$7f &    $\mu$PD77230 \\

\\ \hline
\caption{Header Bytes for the Different Processor Families}
\label{TabHeader}
\end{longtable}\end{center}

The \tty{Segment} field signifies the address space the following code
belongs to.  The assignment defined in table \ref{TabSegments} applies.
\begin{table*}[htbp]
\begin{center}\begin{tabular}{|c|l||c|l|}
\hline
number  & segment               & number  & segment \\
\hline
\hline
\$00    & $<$undefined$>$       & \$01    & \tty{CODE} \\
\$02    & \tty{DATA}            & \$03    & \tty{IDATA} \\
\$04    & \tty{XDATA}           & \$05    & \tty{YDATA} \\
\$06    & \tty{BDATA}           & \$07    & \tty{IO} \\
\$08    & \tty{REG}             & \$09    & \tty{ROMDATA} \\
\hline
\end{tabular}\end{center}
\caption{Codings of the {\tt Segment} Field\label{TabSegments}
         \label{TabSegmentNums}}
\end{table*}
The \tty{Gran} field describes the code's ''granularity'', i.e. the size of
the smallest addressable unit in the following set of data.  This
value is a function of processor type and segment and is an important
parameter for the interpretation of the following two fields that
describe the block's start address and its length: While the start
address refers to the granularity, the \tty{Length} value is always
expressed in bytes!  For example, if the start address is \$300 and
the length is 12, the resulting end address would be \$30b for a
granularity of 1, however \$303 for a granularity of 4!  Granularities
that differ from 1 are rare and mostly appear in DSP CPU's that are
not designed for byte processing.  For example, a DSP56K's address
space is organized in 64 Kwords of 16 bits.  The resulting storage
capacity is 128 Kbytes, however it is organized as $2^{16}$ words that
are addressed with addresses 0,1,2,...65535!

The start address is always 32 bits in size, independent of the
processor family.  In contrast, the length specification has only 16
bits, i.e. a record may have a maximum length of 4+4+2+(64K-1) =
65545 bytes.

Data records with a Header ranging from \$01 to \$7f present a shortcut
and preserve backward compatibility to earlier definitions of the
file format: in their case, the Header directly defines the processor
type, the target segment is fixed to \tty{CODE} and the granularity is
implicitly given by the processor type, rounded up to the next power
of two.  AS prefers to use these records whenever data or code should
go into the \tty{CODE} segment.

A record with a Header of \$80 defines an entry point, i.e. the
address where execution of the program should start.  Such a record
is the result of an \tty{END} statement with a corresponding address as
argument.

The last record in a file bears the Header \$00 and has only a string
as data field.  This string does not have an explicit length
specification; its end is equal to the file's end.  The string
contains only the name of the program that created the file and has
no further meaning.

%%---------------------------------------------------------------------------

\section{Debug Files}
\label{SectDebugFormat}

Debug files may optionally be generated by AS.  They deliver important
information for tools used after assembly, like disassemblers or
debuggers.  AS can generate debug files in one of three formats: On the
one hand, the object format used by the AVR tools from Atmel respectively
a NoICE-compatible command file, and on the other hand an own format.  The
first two are described in detail in \cite{AVRObj} resp. the NoICE
documentations, which is why the following description limits itself to
the AS-specific MAP format:

The information in a MAP file is split into three groups:
\begin{itemize}
\item{symbol table}
\item{memory usage per section}
\item{machine addresses of source lines}
\end{itemize}
The second item is listed first in the file.  A single entry in this
list consists of two numbers that are separated by a \tty{:} character:
\begin{verbatim}
 <line number>:<address>
\end{verbatim}
Such an entry states that the machine code generated for the source
statement in a certain line is stored at the mentioned address
(written in hexadecimal notation).  With such an information, a
debugger can display the corresponding source lines while stepping
through a program.  As a program may consist of several include
files, and due to the fact that a lot of processors have more than
one address space (though admittedly only one of them is used to
store executable code), the entries described above have to be
sorted.  AS does this sorting in two levels: The primary sorting
criteria is the target segment, and the entries in one of these
sections are sorted according to files.  The sections resp.
subsections are separated by special lines in the style of
\begin{verbatim}
Segment <segment name>
\end{verbatim}
resp.
\begin{verbatim}
File <file name>   .
\end{verbatim}
The source line info is followed by the symbol table.  Similar to the
source line info, the symbol table is primarily sorted by the
segments individual symbols are assigned to.  In contrast to the
source line info, an additional section \tty{NOTHING} exists which contains
the symbols that are not assigned to any specific segment (e.g.
symbols that have been defined with a simple \tty{EQU} statement).  A
section in the symbol table is started with a line of the following
type:
\begin{verbatim}
Symbols in Segment <segment name>
\end{verbatim}
The symbols in a section are sorted according to the alphabetical
order of their names, and one symbol entry consists of exactly one
line.  Such a line consists of six fields witch are separated by at
least a single space:

The first field is the symbol's name, possibly extended by a section
number enclosed in brackets.  Such a section number limits the
range of validity for a symbol.  The second field designates the
symbol's type: \tty{Int} stands for integer values, \tty{Float} for floating
point numbers, and \tty{String} for character arrays.  The third field
finally contains the symbol's value.  If the symbol contains a
string, it is necessary to use a special encoding for control
characters and spaces.  Without such a coding, spaces in a string
could be misinterpreted as delimiters to the next field.  AS uses the
same syntax that is also valid for assembly source files: Instead of
the character, its ASCII value with a leading backslash (\verb!\!) is
inserted.  For example, the string
\begin{verbatim}
 This is a test
\end{verbatim}
becomes
\begin{verbatim}
 This\032is\032\a\032test   .
\end{verbatim}
The numerical value always has three digits and has to be interpreted
as a decimal value.  Naturally, the backslash itself also has to be
coded this way.

The fourth field specifies - if available - the size of the data
structure placed at the address given by the symbol.   A debugger may
use this information to automatically display variables in their
correct length when they are referred symbolically.  In case AS does
not have any information about the symbol size, this field simply
contains the value -1.

The fifth field states via the values 0 or 1 if the symbol has been
used during assembly.  A program that reads the symbol table can use
this field to skip unused symbols as they are probably unused during
the following debugging/disassembly session.

Finally, the sixth field states via the values 0 or 1 if the symbol
is a constant (0) or variable(1).  Constant symbols are set once, e.g.
via the \tty{EQU} statement or a label, while variables are allowed
to change their value during the course of assembly.  The MAP file
lists the final value.

The third section in a debug file describes the program's sections in
detail.  The need for such a detailed description arises from the
sections' ability to limit the validity range of symbols.  A symbolic
debugger for example cannot use certain symbols for a reverse
translation, depending on the current PC value.  It may also have to
regard priorities for symbol usage when a value is represented by
more than one symbol.  The definition of a section starts with a line
of the following form:
\begin{verbatim}
Info for Section nn ssss pp
\end{verbatim}
\tty{nn} specifies the section's number (the number that is also used in
the symbol table as a postfix for symbol names), \tty{ssss} gives its name
and \tty{pp} the number of its parent section.  The last information is
needed by a retranslator to step upward through a tree of sections
until a fitting symbol is found.  This first line is followed by a
number of further lines that describe the code areas used by this
section.  Every single entry (exactly one entry per line) either
describes a single address or an address range given by a lower and
an upper bound (separation of lower and upper bound by a minus sign).
These bounds are ''inclusive'', i.e. the bounds themselves also belong
to the area.  Is is important to note that an area belonging to a
section is not additionally listed for the section's parent sections
(an exception is of course a deliberate multiple allocation of address
areas, but you would not do this, would you?).  On the one hand, this
allows an optimized storage of memory areas during assembly. On the
other hand, this should not be an obstacle for symbol backtranslation
as the single entry already gives an unambiguous entry point for the
symbol search path.  The description of a section is ended by an
empty line or the end of the debug file.

Program parts that lie out of any section are not listed separately.
This implicit ''root section'' carries the number -1 and is also used
as parent section for sections that do not have a real parent
section.

It is possible that the file contains empty lines or comments (semi
colon at line start).  A program reading the file has to ignore such
lines.

%%===========================================================================

\cleardoublepage
\chapter{Utility Programs}
\label{ChapTools}

To simplify the work with the assembler's code format a bit, I added
some tools to aid processing of code files.  These programs are
released under the same license terms as stated in section
\ref{SectLicense}!

Common to all programs are the possible return codes they may deliver
upon completion (see table \ref{TabToolReturns}).
\par
\begin{table*}[h]
\begin{center}\begin{tabular}{|c|l|}
\hline
return code   & error condition \\
\hline
\hline
0             & no errors \\
1             & error in command line parameters \\
2             & I/O error \\
3             & file format error \\
\hline
\end{tabular}\end{center}
\caption{Return Codes of the Utility Programs\label{TabToolReturns}}
\end{table*}
Just like AS, all programs take their input from STDIN and write
messages to STDOUT (resp. error messages to STDERR).  Therefore,
input and output redirections should not be a problem.

In case that numeric or address specifications have to be given in
the command line, they may also be written in hexadecimal
notation,  either ba allending a \verb!h! or prepending a dollar
character or a \tty{0x} like in C.
(e.g. \verb!10h!, \verb!$10!, or \verb!0x10! instead of 16).

Unix shells however \marginpar{{\em UNIX}} assign a special meaning to the
dollar sign, which makes it necessary to escape a dollar sign with a
backslash.  The \tty{0x} variant is definitely more comfortable in this case.

Otherwise, calling conventions and variations are equivalent to those
of AS (except for PLIST and AS2MSG); i.e. it is possible to store
frequently used parameters in an environment variable (whose name is
constructed by appending CMD to the program's name, i.e. \tty{BINDCMD} for
BIND), to negate options, and to use all upper- resp. lower-case
writing (for details on this, see section \ref{SectCallConvention}).

Address specifications always relate to the granularity of the
processor currently in question; for example, on a PIC, an address
difference of 1 means a word and not a byte.

%%---------------------------------------------------------------------------

\section{PLIST}

PLIST is the simplest one of the five programs supplied: its purpose
is simply to list all records that are stored in a code file.  As the
program does not do very much, calling is quite simple:
\begin{verbatim}
    PLIST <file name>
\end{verbatim}
The file name will automatically be extended with the extension \tty{P} if
it doesn't already have one.

\bb{CAUTION!} At this place, no wildcards are allowed!  If there is a
necessity to list several files with one command, use the following
''mini batch'':
\begin{verbatim}
    for %n in (*.p) do plist %n
\end{verbatim}
PLIST prints the code file's contents in a table style, whereby
exactly one line will be printed per record.  The individual rows
have the following meanings:
\begin{itemize}
\item{code type: the processor family the code has been generated for.}
\item{start address: absolute memory address that expresses the load
      destination for the code.}
\item{length: length of this code chunk in bytes.}
\item{end address: last address of this code chunk.  This address
      is calculated as start address+length-1.}
\end{itemize}
All outputs are in hexadecimal notation.

Finally, PLIST will print a copyright remark (if there is one in the
file), together with a summaric code length.

Simply said, PLIST is a sort of DIR for code files.  One can use it
to examine a file's contents before one continues to process it.

%%---------------------------------------------------------------------------

\section{BIND}

BIND is a program that allows to concatenate the records of several
code files into a single file.  A filter function is available that
can be used to copy only records of certain types.  Used in this way,
BIND can also be used to split a code file into several files.

The general syntax of BIND is
\begin{verbatim}
   BIND <source file(s)> <target file> [options]
\end{verbatim}
Just like AS, BIND regards all command line arguments that do not
start with a \tty{+, -} or \tty{/} as file specifications, of which the last one
must designate the destination file.  All other file specifications
name sources, which may again contain wildcards.

Currently, BIND defines only one command line option:
\begin{itemize}
\item{\tty{f $<$Header[,Header]$>$}: sets a list of record headers that should
      be copied.  Records with other header IDs will
      not be copied.  Without such an option, all
      records will be copied.  The headers given in
      the list correspond to the \tty{HeaderID} field of the
      record structure described in section \ref{SectCodeFormat}.
      Individual headers in this list are separated
      with commas.}
\end{itemize}
For example, to filter all MCS-51 code out of a code file, use BIND
in the following way:
\begin{verbatim}
   BIND <source name> <target name> -f $31
\end{verbatim}
If a file name misses an extension, the extension \tty{P} will be added
automatically.

%%---------------------------------------------------------------------------

\section{P2HEX}

P2HEX is an extension of BIND.  It has all command line options of BIND and
uses the same conventions for file names.  In contrary to BIND, the
target file is written as a Hex file, i.e. as a sequence of lines
which represent the code as ASCII hex numbers.

P2HEX knows nine different target formats, which can be selected via the
command line parameter \tty{F}:
\begin{itemize}
\item{Motorola S-Records (\tty{-F Moto)})}
\item{MOS Hex \tty{(-F MOS)}}
\item{Intel Hex (Intellec-8, \tty{-F Intel})}
\item{16-Bit Intel Hex (MCS-86, \tty{-F Intel16})}
\item{32-Bit Intel Hex (\tty{-F Intel32)})}
\item{Tektronix Hex (\tty{-F Tek})}
\item{Texas Instruments DSK (\tty{-F DSK})}
\item{Atmel AVR Generic (\tty{-F Atmel}, see \cite{AVRObj})}
\item{Lattice Mico8 prom\_init (\tty{-F Mico8})}
\item{C arrays, for inclusion into C(++) source files (\tty{-F C})}
\end{itemize}
If no target format is explicitly specified, P2HEX will automatically
choose one depending in the processor type:  S-Records for Motorola
CPUs, Hitachi, and TLCS-900, MOS for 65xx/MELPS, DSK for the 16 bit
signal processors from Texas, Atmel Generic for the AVRs, and Intel Hex
for the rest.  Depending on the start addresses width, the S-Record
format will use Records of type 1, 2, or 3, however, records in one
group will always be of the same type.  This automatism can be partially
suppressed via the command line option
\begin{verbatim}
  -M <1|2|3>
\end{verbatim}
A value of 2 resp. 3 assures that that S records with a minimum type of 2
resp. 3 will be used, while a value of 1 corresponds to the full
automatism.

Normally, the AVR format always uses an address length of 3 bytes.  Some
programs however do not like that...which is why there is a switch
\begin{verbatim}
  -avrlen <2|3>
\end{verbatim}
that allows to reduce the address length to two bytes in case of
emergency.

The Mico8 format is different from all the other formats in
having no address fields - it is plain list of all instruction
words in program memory.  When using it, be sure that the used
address range (as displeyed e.g. by PLIS) starts at zero and is
continuous.

The Intel, MOS and Tektronix formats are limited to 16 bit addresses, the
16-bit Intel format reaches 4 bits further.  Addresses that are to long
for a given format will be reported by P2HEX with a warning; afterwards,
they will be truncated (!).

For the PIC microcontrollers, the switch
\begin{verbatim}
-m <0..3>
\end{verbatim}
allows to generate the three different variants of the Intel Hex
format.  Format 0 is INHX8M which contains all bytes in a
Lo-Hi-Order.  Addresses become double as large because the PICs have
a word-oriented address space that increments addresses only by one
per word.  This format is also the default.  With Format 1 (INHX16M),
bytes are stored in their natural order.  This is the format
Microchip uses for its own programming devices.  Format 2 (INHX8L)
resp. 3 (INHX8H) split words into their lower resp. upper bytes.
With these formats, P2HEX has to be called twice to get the complete
information, like in the following example:
\begin{verbatim}
  p2hex test -m 2
  rename test.hex test.obl
  p2hex test -m 3
  rename test.hex test.obh
\end{verbatim}
For the Motorola format, P2HEX additionally uses the S5 record type
mentioned in \cite{CPM68K}.  This record contains the number of data
records (S1/S2/S3) to follow.  As some programs might not know how to
deal with this record, one can suppress it with the option
\begin{verbatim}
 +5  .
\end{verbatim}
The C format is different in the sense that it always has to be
selected explicitly.  The output file is basically a complete piece of
C or C++ code that contains the data as a list of C arrays.  Additionally to
the data itself, a list of descriptors is written that describes the
start, length, and end address of each data block.  The contents of these
descriptors may be configured via the option
\begin{verbatim}
 -cformat <format>
\end{verbatim}
Each letter in \verb!format! defines an element of the descriptor:
\begin{itemize}
\item{A \verb!d! or \verb!D! defines a pointer to the data itself.
      Usage of a lower or upper case letter defines whether lowercase
      or uppercase letters are used for hexadecimal constants.}
\item{An \verb!s! or \verb!S! defines the start address of the data,
      either as {\em unsigned} or {\em unsigned long}.}
\item{An \verb!l! or \verb!L! defines the length of the data,
      either as {\em unsigned} or {\em unsigned long}.}
\item{An \verb!e! or \verb!E! defines the end address of the data,
      specifically the last address used by the data, either as
      {\em unsigned} or {\em unsigned long}.} 
\end{itemize}
\par
In case a source file contains code record for different processors,
the different hex formats will also show up in the target file - it
is therefore strongly advisable to use the filter function.

Apart form this filter function, P2HEX also supports an address
filter, which is useful to split the code into several parts (e.g.
for a set of EPROMs):
\begin{verbatim}
-r <start address>-<end address>
\end{verbatim}
The start address is the first address in the window, and the end
address is the last address in the window, \bb{not} the first address
that is out of the window.  For example, to split an 8051 program
into 4 2764 EPROMs, use the following commands:
\begin{verbatim}
p2hex <source file> eprom1 -f $31 -r $0000-$1fff
p2hex <source file> eprom2 -f $31 -r $2000-$3fff
p2hex <source file> eprom3 -f $31 -r $4000-$5fff
p2hex <source file> eprom4 -f $31 -r $6000-$7fff
\end{verbatim}
It is allowed to specifiy a single dollar character or '0x' as start
or stop address.  This means that the lowest resp. highest address found
in the source file shall be taken as start resp. stop address.  The
default range is '0x-0x', i.e. all data from the source file is
transferred.

\bb{CAUTION!} This type of splitting does not change the absolute
addresses that will be written into the files!  If the addresses in
the individual hex files should rather start at 0, one can force this
with the additional switch
\begin{verbatim}
 -a     .
\end{verbatim}
On the other hand, to move the addresses to a different location, one may
use the switch
\begin{verbatim}
 -R <value> .
\end{verbatim}
The value given is an {\em offset}, i.e. it is added to the addresses
given in the code file.
\par
By using an offset, it is possible to move a file's contents to an
arbitrary position.  This offset is simply appended to a file's name,
surrounded with parentheses.  For example, if the code in a file
starts at address 0 and you want to move it to address 1000 hex in the
hex file, append \tty{(\$1000)} to the file's name (without spaces!).
\par
In case the source file(s) not only contain data for the code segment,
the switch
\begin{verbatim}
 -segment <name>
\end{verbatim}
allows to select the segment data is extracted from and converted to
HEX format.  The segment names are the same as for the \tty{SEGMENT}
pseudo instruction (\ref{SEGMENT}).  The TI DSK is a special case
since it has the ability to distinguish between data and code in one
file.  If TI DSK is the output format, P2HEX will automatically
extract data from both segments if no segment was specified explicitly.
\par
Similar to the \verb!-r! option, the argument
\begin{verbatim}
 -d <start>-<end>
\end{verbatim}
allows to designate the address range that should be written as data
instead of code.
\par
The option
\begin{verbatim}
 -e <address>
\end{verbatim}
is valid for the DSK, Intel, and Motorola formats.  Its purpose is to
set the entry address that will be inserted into the hex file.  If
such a command line parameter is missing, P2HEX will search a
corresponding entry in the code file.  If even this fails, no entry
address will be written to the hex file (DSK/Intel) or the field
reserved for the entry address will be set to 0 (Motorola).

Unfortunately, one finds different statements about the last line of
an Intel-Hex file in literature.  Therefore, P2HEX knows three
different variants that may be selected via the command-line
parameter \tty{i} and an additional number:
\begin{verbatim}
 0  :00000001FF
 1  :00000001
 2  :0000000000
\end{verbatim}
By default, variant 0 is used which seems to be the most common one.

If the target file name does not have an extension, an extension of
\tty{HEX} is supposed.

By default, P2HEX will print a maximum of 16 data bytes per line,
just as most other tools that output Hex files.  If you want to
change this, you may use the switch
\begin{verbatim}
-l <count>   .
\end{verbatim}
The allowed range of values goes from 2 to 254 data bytes; odd values
will implicitly be rounded down to an even count.

In most cases, the temporary code files generated by AS are not of
any further need after P2HEX has been run.  The command line option
\begin{verbatim}
-k
\end{verbatim}
allows to instruct P2HEX to erase them automatically after
conversion.

In contrast to BIND, P2HEX will not produce an empty target file if
only one file name (i.e. the target name) has been given.  Instead,
P2HEX will use the corresponding code file.  Therefore, a minimal
call in the style of
\begin{verbatim}
 P2HEX <name>
\end{verbatim}
is possible, to generate \tty{$<$name$>$.hex} out of \tty{$<$name$>$.p}.

%%---------------------------------------------------------------------------

\section{P2BIN}

P2BIN works similar to P2HEX and offers the same options (except for
the a and i options that do not make sense for binary files),
however, the result is stored as a simple binary file instead of a
hex file.  Such a file is for example suitable for programming an
EPROM.

P2BIN knows three additional options to influence the resulting binary
file:
\begin{itemize}
\item{\tty{l $<$8 bit number$>$}: sets the value that should be used to fill
      unused memory areas.  By default, the value
      \$ff is used.  This value assures that every
      half-way intelligent EPROM burner will skip
      these areas.  This option allows to set different values,
      for example if you want to
      generate an image for the EPROM versions of
      MCS-48 microcontrollers (empty cells of their
      EPROM array contain zeroes, so \$00 would be
      the correct value in this case).}
\item{\tty{s}:  commands the program to calculate a checksum
      of the binary file.  This sum is printed as
      a 32-bit value, and the two's complement of
      the least significant bit will be stored in
      the file's last byte.  This way, the modulus-
      256-sum of the file will become zero.}
\item{\tty{m}:  is designed for the case that a CPU with a
      16- or 32-bit data bus is used and the file
      has to be split for several EPROMs.  The
      argument may have the following values:
      \begin{itemize}
      \item{\tty{ALL}: copy everything}
      \item{\tty{ODD}: copy all bytes with an odd address}
      \item{\tty{EVEN}: copy all bytes with an even address}
      \item{\tty{BYTE0..BYTE3}: copy only bytes with an address of
            4n+0 .. 4n+3}
      \item{\tty{WORD0, WORD1}: copy only the lower resp. upper 16-
            bit word of a 32-bit word}
      \end{itemize}}
\end{itemize}
To avoid confusions: If you use this option, the resulting binary file
will become smaller because only a part of the source will be copied.
Therefore, the resulting file will be smaller by a factor of 2 or 4
compared to \tty{ALL}.  This is just natural...

In case the code file does not contain an entry address, one may set
it via the \tty{-e} command line option just like with P2HEX.  Upon
request, P2BIN prepends the resulting image with this address.  The
command line option
\begin{verbatim}
-S
\end{verbatim}
activates this function.  It expects a numeric specification ranging
from 1 to 4 as parameter which specifies the length of the address
field in bytes.  This number may optionally be prepended wit a \tty{L} or
\tty{B} letter to set the endian order of the address.  For example, the
specification \tty{B4} generates a 4 byte address in big endian order,
while a specification of \tty{L2} or simply \tty{2} creates a 2 byte address
in little endian order.

%%---------------------------------------------------------------------------

\section{AS2MSG}

AS2MSG is not a tool in the real sense, it is a filter that was
designed to simplify the work with the assembler for (fortunate)
users of Borland Pascal 7.0.  The DOS IDEs feature a 'tools' menu
that can be extended with own programs like AS.  The filter allows to
directly display the error messages paired with a line
specification delivered by AS in the editor window.  A new entry has
to be added to the tools menu to achieve this (Options/Tools/New).
Enter the following values:
\begin{verbatim}
 - Title: ~m~acro assembler
 - Program path: AS
 - Command line:
      -E !1 $EDNAME $CAP MSG(AS2MSG) $NOSWAP $SAVE ALL
 - assign a hotkey if wanted (e.g. Shift-F7)
\end{verbatim}
The -E option assures that Turbo Pascal will not become puzzled by
STDIN and STDERR.

I assume that AS and AS2MSG are located in a directory listed in the
\tty{PATH} variable.  After pressing the appropriate hotkey (or selecting
AS from the tools menu), as will be called with the name of the file
loaded in the active editor window as parameter.  The error messages
generated during assembly are redirected to a special window that
allows to browse through the errors.  \tty{Ctrl-Enter} jumps to an
erroneous line.  The window additionally contains the statistics AS
prints at the end of an assembly.  These lines obtain the dummy line
number 1.

\tty{TURBO.EXE} (Real Mode) and \tty{BP.EXE} (Protected Mode) may be used for
this way of working with AS.  I recommend however BP, as this version
does not have to 'swap' half of the DOS memory before before AS is
called.

%%===========================================================================
\appendix

\cleardoublepage
\chapter{Error Messages of AS}
\label{ChapErrMess}

Here is a list of all error messages emitted by AS. Each error message is
described by:
\begin{itemize}
\item{the internal error number (it is displayed only if AS is started with the
      \tty{-n} option)}
\item{the text of the error message}
\item{error type:
      \begin{itemize}
      \item{Warning: informs the user that a possible error was
            found, or that some inefficient binary code
            could be generated. The assembly process is not
            stopped.}
      \item{Error: an error was detected. The assembly process
            continues, but no binary code is emitted.}
      \item{Fatal: unrecoverable error. The assembly process is
            terminated.}
      \end{itemize}}
\item{reason of the error: the situation originating the error.}
\item{argument:  a further explanation of the error message.}
\end{itemize}

\par

\newcommand{\errentry}[5]
           {\item[#1]{#2
                      \begin{description}
                      \item[Type:]{\ \\#3}
                      \item[Reason:]{\ \\#4}
                      \item[Argument:]{\ \\#5}
                      \end{description}}
           }

\begin{description}
\errentry{   5}{useless displacement}
               {warning}
               {680x0, 6809 and COP8 CPUs: an address displacement of 0 was
                given.  An address expression without displacement is
                generated, and a convenient number of NOPs are emitted
                to avoid phasing errors.}
               {none}
\errentry{  10}{short addressing possible}
               {warning}
               {680x0-, 6502 and 68xx CPUs: a given memory location can be
                reached using short addressing. A short addressing
                instruction is emitted, together with the required
                number of NOPs to avoid phasing errors.}
               {none}
\errentry{  20}{short jump possible}
               {warning}
               {680x0- and 8086 CPUs can execute jumps using a short or long
                displacement. If a shorter jump was not explicitly
                requested, in the
                first pass room for the long jump is reserved. Then the code
                for the shorter jump is emitted, and the remaining space is
                filled with NOPs to avoid phasing errors.}
               {none}
\errentry{  30}{no sharefile created, SHARED ignored}
               {warning}
               {A \tty{SHARED} directive was found, but on the command line no
                options were specified, to generate a shared file.}
               {none}
\errentry{  40}{FPU possibly cannot read this value ($>$=1E1000)}
               {warning}
               {The BCD-floating point format used by the 680x0-FPU
                allows such a large exponent, but according to the latest
                databooks, this cannot be fully interpreted. The
                corresponding word is assembled, but the associated
                function is not expected to produce the correct result.}
               {none}
\errentry{  50}{privileged instruction}
               {warning}
               {A Supervisor-mode directive was used, that was not preceded
                by an explicit \tty{SUPMODE ON} directive}
               {none}
\errentry{  60}{distance of 0 not allowed for short jump (NOP created instead)}
               {warning}
               {A short jump with a jump distance equal to 0 is not allowed
                by 680x0 resp. COP8 processors, since the associated code word is
                used to identify long jump instruction. Instead of a
                jump instruction, AS emits a NOP}
               {none}
\errentry{  70}{symbol out of wrong segment}
               {warning}
               {The symbol used as an operand comes from an address space
                that cannot be addressed together with the given instruction}
               {none}
\errentry{  75}{segment not accessible}
               {warning}
               {The symbol used as an operand belongs to an address space
                that cannot be accessed with any of the segment registers of
                the 8086}
               {The name of the inaccessible segment}
\errentry{  80}{change of symbol values forces additional pass}
               {warning}
               {A symbol changed value, with respect to previous pass. This
                warning is emitted only if the \tty{-r} option is used.}
               {name of the symbol that changed value.}
\errentry{  90}{overlapping memory usage}
               {warning}
               {The analysis of the usage list shows that part of the
                program memory was used more than once. The reason can be an
                excessive usage of \tty{ORG} directives.}
               {none}
\errentry{  95}{overlapping register usage}
               {warning}
               {The instruction uses whole registers or parts thereof in
                a non-allowed way.}
               {The offending argument}
\errentry{ 100}{none of the CASE conditions was true}
               {warning}
               {A \tty{SWITCH...CASE} directive without \tty{ELSECASE} clause was
                executed, and none of the \tty{CASE} conditions was found
                to be true.}
               {none}
\errentry{ 110}{page might not be addressable}
               {warning}
               {The symbol used as an operand was not found in the memory
                page defined by an \tty{ASSUME} directive (ST6, 78(C)10).}
               {none}
\errentry{ 120}{register number must be even}
               {warning}
               {The CPU allows to concatenate only register pairs, whose
                start address is even (RR0, RR2, ..., only for Z8).}
               {none}
\errentry{ 130}{obsolete instruction, usage discouraged}
               {warning}
               {The instruction used, although supported, was superseded by
                a new instruction. Future versions of the CPU could no more
                implement the old instruction.}
               {none}
\errentry{ 140}{unpredictable execution of this instruction}
               {warning}
               {The addressing mode used for this instruction is allowed,
                however a register is used in such a way that its contents
                cannot be predicted after the execution of the
                instruction.}
               {none}
\errentry{ 150}{localization operator senseless out of a section}
               {warning}
               {An aheaded \@ must be used, so that it is
                explicitly referred to the local symbols used in the
                section. When the operator is used out of a section, there
                are no local symbols, because this operator is useless in
                this context.}
               {none}
\errentry{ 160}{senseless instruction}
               {warning}
               {The instruction used has no meaning, or it can be
                substituted by an other instruction, shorter and more
                rapidly executed.}
               {none}
\errentry{ 170}{unknown symbol value forces additional pass}
               {warning}
               {AS expects a forward definition of a symbol, i.e. a symbol
                was used before it was defined. A further pass must be
                executed. This warning is emitted only if the \tty{-r} option was
                used.}
               {none}
\errentry{ 180}{address is not properly aligned}
               {warning}
               {An address was used that is not an exact multiple of the
                operand size. Although the CPU databook forbids this, the
                address could be stored in the instruction word, so AS
                simply emits a warning.}
               {none.}
\errentry{ 190}{I/O-address must not be used here}
               {warning}
               {The addressing mode or the address used are correct, but the
                address refers to the peripheral registers, and it
                cannot be used in this circumstance.}
               {none.}
\errentry{ 200}{possible pipelining effects}
               {warning}
               {A register is used in a series of instructions, so that a
                sequence of instructions probably does not generate the
                desired result. This usually happens when a register is
                used before its new content was effectively loaded in it.}
               {the register probably causing the problem.}
\errentry{ 210}{multiple use of address register in one instruction}
               {warning}
               {A register used for the addressing is used once more in the
                same instruction, in a way that results in a modification
                of the register value. The resulting address does not have a
                well defined value.}
               {the register used more than once.}
\errentry{ 220}{memory location is not bit addressable}
               {warning}
               {Via a \tty{SFRB} statement, it was tried to declare a memory cell
                as bit addressable which is not bit addressable due to the
                8051's architectural limits.}
               {none}
\errentry{ 230}{stack is not empty}
               {warning}
               {At the end of a pass, a stack defined by the program is
                not empty.}
               {the name of the stack and its remaining depth}
\errentry{ 240}{NUL character in string, result is undefined}
               {warning}
               {A string constant contains a NUL character. Though this
                works with the Pascal version, it is a problem for the
                C version of AS since C itself terminates strings with
                a NUL character. i.e. the string would have its end for
                C just at this point...}
               {none}
\errentry{ 250}{instruction crosses page boundary}
               {warning}
               {The parts of a machine statement partiallly lie on
                different pages.  As the CPU's instruction counter does
                not get incremented across page boundaries, the processor
                would fetch at runtime the first byte of the old page
                instead of the instruction's following byte; the program
                would execute incorrectly.}
               {none}
\errentry{ 255}{range underflow}
               {warning}
               {A numeric value was below the allowed range.  AS brought
                the value back into the allowed range by truncating upper
                bits, but it is not guaranteed that meaningful and correct
                code is generated by this.}
               {none}
\errentry{ 260}{range overflow}
               {warning}
               {A numeric value was above the allowed range.  AS brought
                the value back into the allowed range by truncating upper
                bits, but it is not guaranteed that meaningful and correct
                code is generated by this.}
               {none}
\errentry{ 270}{negative argument for DUP}
               {warning}
               {The repetition argument of a DUP directive was smaller
                than 0.  Analogous to a count of exactly 0, no data is
                stored.}
               {none}
\errentry{ 280}{single X operand interpreted as indexed and not implicit
                addressing}
               {warning}
               {A single X operand may be interpreted either as register X
                or x-indexed addressing with zero displacement, since
                Motorola does not specify this variant.  AS chooses the
                latter, which may not be the desired one.}
               {none}
\errentry{ 300}{bit number will be truncated}
               {warning}
               {This instruction only operates on byte resp. longword
                operands.  bit numbers beyond 7 resp. 31 will be treated
                modulo-8 resp. modulo-32 by the CPU.}
               {none}
\errentry{ 310}{invalid register pointer value}
               {warning}
               {Valid values for the RP register range from 0x00 to 0x70 resp.
                0xf0, because all other areas are unused on the Z8.}
               {none}
\errentry{ 320}{macro argument redefined}
               {warning}
               {A macro parameter was assigned two or more
                different values.  This may happen by usage of
                keyword arguments.  The last argument is actually
                used.}
               {name of the macro parameter}
\errentry{ 330}{deprecated instruction}
               {warning}
               {This instruction is deprecated and should not be used any
                more in new programs.}
               {the instruction that should be used instead.}
\errentry{ 340}{source operand is longer or same size as destination operand}
               {warning}
               {The source operand's size is larger than the destination operand's
                size, expressed in bits.  Sign or zero extension does not make sense
                with these arguments.  See the CPU's reference manual for its behaviour
                in this situation.}
               {none}
\errentry{ 350}{TRAP number represents valid instruction}
               {warning}
               {A TRAP with this number uses the same machine code as a 
                machine instruction supported by the CPU.}
               {none}
\errentry{ 360}{Padding added}
               {warning}
               {The amount of bytes placed in memory is odd; one half of the last
                16 bit word remains unused.}
               {none}
\errentry{ 370}{register number wraparound}
               {warning}
               {The start register number plus the count of registers results
                in a last register beyond the end of the register bank.}
               {the argument holding the register count}
\errentry{ 380}{using indexed instead of indirect addressing}
               {warning}
               {Indirect addressing is not allowed at this place.
                Instead, indexed addressing with a dummy displacement of
                zero will be used.}
               {the argument holding the indirect addressing expression}
\errentry{1000}{symbol double defined}
               {error}
               {A new value is assigned to a symbol, using a label or a
                \tty{EQU, PORT, SFR, LABEL, SFRB} or \tty{BIT} instruction: however this
                can be done only using \tty{SET/EVAL}.}
               {the name of the offending symbol, and the line number where
                it was defined for the first time, according to the symbol
                table.}
\errentry{1010}{symbol undefined}
               {error}
               {A symbol is still not defined in the symbol table, also
                after a second pass.}
               {the name of the undefined symbol.}
\errentry{1020}{invalid symbol name}
               {error}
               {A symbol does not fulfill the requirements that symbols
                must have to be considered valid by AS. Please pay
                attention that more stringent syntax rules exist for
                macros and function parameters.}
               {the wrong symbol}
\errentry{1090}{invalid format}
               {error}
               {The instruction format used does not exist for this
                instruction.}
               {the known formats for this command}
\errentry{1100}{useless attribute}
               {error}
               {The instruction (processor or pseudo) cannot be used with a
                point-suffixed attribute.}
               {none}
\errentry{1105}{attribute may only be one character long}
               {error}
               {The attribute following a point after an instruction must
                not be longer or shorter than one character.}
               {none}
\errentry{1107}{undefined attribute}
               {error}
               {This instruction uses an invalid attribute.}
               {none}
\errentry{1110}{wrong number of operands}
               {error}
               {The number of arguments issued for the instruction (processor or
                pseudo) does not conform with the accepted number of
                operands.}
               {the expected number of arguments resp. operands}
\errentry{1112}{failed splitting argument into parts}
               {error}
               {For some targets (e.g. DSP56000), the
                comma-separated have to be split into individual
                operands, which failed.}
               {none}
\errentry{1115}{wrong number of operations}
               {error}
               {The number of options given with this command is not
                correct.}
               {none}
\errentry{1120}{addressing mode must be immediate}
               {error}
               {The instruction can be used only with immediate operands
                (preceded by \tty{\#}).}
               {none}
\errentry{1130}{invalid operand size}
               {error}
               {Although the operand is of the right type, it does not have
                the correct length (in bits).}
               {none}
\errentry{1131}{conflicting operand sizes}
               {error}
               {The operands used have different length (in bits)}
               {none}
\errentry{1132}{undefined operand size}
               {error}
               {It is not possible to estimate, from the opcode and from
                the operands, the size of the operand (a trouble with
                8086 assembly). You must define it with a \tty{BYTE or WORD}
                \tty{PTR} prefix.}
               {none}
\errentry{1133}{expected integer or string, but got floating point number}
               {error}
               {A floating point number cannot be used as argument at this place.}
               {the argument in question}
\errentry{1134}{expected integer, but got floating point number}
               {error}
               {A floating point number cannot be used as argument at this place.}
               {the argument in question}
\errentry{1136}{expected floating point number, but got string}
               {error}
               {A string cannot be used as argument at this place.}
               {the argument in question}
\errentry{1137}{operand type mismatch}
               {Error}
               {The two arguments of an operator are not of same
                data type (integer/\-float/\-string).}
               {keines}
\errentry{1138}{expected string, but got integer}
               {error}
               {An integer cannot be used as argument at this place.}
               {the argument in question}
\errentry{1139}{expected string, but got floating point number}
               {error}
               {An floating point number cannot be used as argument at this place.}
               {the argument in question}
\errentry{1140}{too many arguments}
               {error}
               {No more than 20 arguments can be given to any instruction}
               {none}
\errentry{1141}{expected integer, but got string}
               {error}
               {A string cannot be used as argument at this place.}
               {the argument in question}
\errentry{1142}{expected integer or floating point number, but got string}
               {error}
               {A string cannot be used as argument at this place.}
               {the argument in question}
\errentry{1143}{expected string}
               {error}
               {Only a string (enclosed in single quotes) may be used as
                argument at this place.}
               {the argument in question}
\errentry{1144}{expected integer}
               {error}
               {Only an integer number may be used as argument at this place.}
               {the argument in question}
\errentry{1145}{expected integer, floating point number or string but got register}
               {error}
               {A register symbol may not be used as argument at this place.}
               {the argument in question}
\errentry{1146}{expected integer or string}
               {error}
               {A floating point number or register symbol may not be used as argument at this place.}
               {the argument in question}
\errentry{1147}{expected register}
               {error}
               {Only an register may be used as argument at this place.}
               {the argument in question}
\errentry{1148}{register symbol for different target}
               {error}
               {The used register symbol was defined for a target different from
                the current one and is not compatible.}
               {the argument in question}
\errentry{1200}{unknown instruction}
               {error}
               {An instruction was used that is neither an AS instruction, nor a
                known macine instruction for the current processor type.}
               {none}
\errentry{1300}{number of opening/closing brackets does not match}
               {error}
               {The expression parser found an expression enclosed by
                parentheses, where the number of opening and closing
                parentheses does not match.}
               {the wrong expression}
\errentry{1310}{division by 0}
               {error}
               {An expression on the right side of a division or modulus
                operation was found to be equal to 0.}
               {none}
\errentry{1315}{range underflow}
               {error}
               {An integer word underflowed the allowed range.}
               {the value of the word and the allowed minimum (in most
                cases, maybe I will complete this one day...)}
\errentry{1320}{range overflow}
               {error}
               {An integer word overflowed the allowed range.}
               {the value of the world, and the allowed maximum (in most
                cases, maybe I will complete this one day...)}
\errentry{1322}{not a power of two}
               {error}
               {only powers of two (1,2,4,8,...) are allowed at this place.}
               {The value in question}
\errentry{1325}{address is not properly aligned}
               {error}
               {The given address does not correspond with the size needed
                by the data transfer, i.e. it is not an integral multiple of
                the operand size. Not all processor types can use unaligned
                data.}
               {none}
\errentry{1330}{distance too big}
               {error}
               {The displacement used for an address is too large.}
               {none}
\errentry{1340}{short addressing not allowed}
               {error}
               {The address of the operand is outside of the address space
                that can be accessed using short-addressing mode.}
               {none}
\errentry{1350}{addressing mode not allowed here}
               {error}
               {the addressing mode used, although usually possible,
                cannot be used here.}
               {none}
\errentry{1351}{address must be even}
               {error}
               {At this point, only even addresses are allowed, since the
                low order bits are used for other purposes or are reserved.}
               {the argument in question}
\errentry{1352}{address must be aligned}
               {error}
               {At this point, only aligned (i.e. a mulitple of 2,4,8...) addresses
                are allowed, since the low order bits are used for other purposes
                or are reserved.}
               {the argument in question}
\errentry{1355}{addressing mode not allowed in parallel operation}
               {error}
               {The addressing mode(s) used are allowed in sequential,
                but not in parallel instructions}
               {none}
\errentry{1360}{undefined condition}
               {error}
               {The branch condition used for a conditional jump does not
                exist.}
               {none}
\errentry{1365}{incompatible conditions}
               {error}
               {The used combination of conditions is not possible
                in a single instruction.}
               {the condition where the incompatibility was detected.}
\errentry{1366}{unknown flag}
               {error}
               {The given flag does not exist.}
               {the argument using the flag in question}
\errentry{1367}{duplicate flag}
               {error}
               {The given flag has already been used in the list of flags.}
               {the argument duplicating the flag}
\errentry{1368}{unknown interrupt}
               {error}
               {The given interrupt does not exist.}
               {the argument using the interrupt in question}
\errentry{1369}{duplicate interrupt}
               {error}
               {The given interrupt has already been used in the list of interrupt.}
               {the argument duplicating the interrupt}
\errentry{1370}{jump distance too big}
               {error}
               {the jump instruction and destination are too apart to
                execute the jump with a single step}
               {none}
\errentry{1375}{jump distance is odd}
               {error}
               {Since instruction must only be located at even addresses,
                the jump distance between two instructions must always be
                even, and the LSB of the jump distance is used otherwise.
                This issue was not verified here. The reason is usually the
                presence of an odd number of data in bytes or a wrong
                \tty{ORG}.}
               {none}
\errentry{1376}{skip target mismatch}
               {error}
               {The gien branch target is not the address the processor would
                jump to if the skip instruction were executed.}
               {the given (intended) jump target}
\errentry{1380}{invalid argument for shifting}
               {error}
               {only a constant or a data register can be used for defining
                the shift size. (only for 680x0)}
               {none}
\errentry{1390}{operand must be in range 1..8}
               {error}
               {constants for shift size or \tty{ADDQ} argument can be only
                within the 1..8 range (only for 680x0)}
               {none}
\errentry{1400}{shift amplitude too big}
               {error}
               {(no more used)}
               {none}
\errentry{1410}{invalid register list}
               {error}
               {The register list argument of \tty{MOVEM} or \tty{FMOVEM} has a
                wrong format (only for 680x0)}
               {none}
\errentry{1420}{invalid addressing mode for CMP}
               {error}
               {The operand combination used with the \tty{CMP} instruction is
                not allowed (only for 680x0)}
               {none}
\errentry{1430}{invalid CPU type}
               {error}
               {The processor type used as argument for \tty{CPU} command is
                unknown to AS.}
               {the unknown processor type}
\errentry{1431}{invalid FPU type}
               {error}
               {The co-processor type used as argument for \tty{FPU} command is
                unknown to AS.}
               {the unknown co-processor type}
\errentry{1432}{invalid PMMU type}
               {error}
               {The MMU type used as argument for \tty{PMMU} command is
                unknown to AS.}
               {the unknown MMU type}
\errentry{1440}{invalid control register}
               {error}
               {The control register used by a \tty{MOVEC} is not (yet) available
                for the processor defined by the \tty{CPU} command.}
               {none}
\errentry{1445}{invalid register}
               {error}
               {The register used, although valid, cannot be used in this
                context.}
               {none}
\errentry{1446}{register(s) listed more than once}
               {error}
               {A register appears more than once in the list of registers
                to be saved or restored.}
               {none}
\errentry{1447}{register bank mismatch}
               {error}
               {An address expression uses registers from different banks.}
               {the register in question}
\errentry{1448}{undefined register length}
               {error}
               {Registers of different size may be used at this place, and
                 the register length cannot be deduced from the address alone.}
               {the argument in question}
\errentry{1449}{invalid operation on register}
               {error}
               {This operation may not be applied to this register, e.g. because
                the register is read-only or write-only.}
               {the register in question}
\errentry{1450}{RESTORE without SAVE}
               {error}
               {A \tty{RESTORE} command was found, that cannot be coupled with a
                corresponding \tty{SAVE}.}
               {none}
\errentry{1460}{missing RESTORE}
               {error}
               {After the assembling pass, a \tty{SAVE} command was missing.}
               {none.}
\errentry{1465}{unknown macro control instruction}
               {error}
               {A macro option parameter is unknown to AS.}
               {the dubious option.}
\errentry{1470}{missing ENDIF/ENDCASE}
               {error}
               {after the assembling, some of the \tty{IF}- or \tty{CASE}- constructs
                were found without the closing command}
               {none}
\errentry{1480}{invalid IF-structure}
               {error}
               {The command structure in a \tty{IF}- or \tty{SWITCH}- sequence is
                wrong.}
               {none}
\errentry{1483}{section name double defined}
               {error}
               {In this program module a section with the same name still
                exists.}
               {the multiple-defined name}
\errentry{1484}{unknown section}
               {error}
               {In the current scope, there are no sections with this name}
               {the unknown name}
\errentry{1485}{missing ENDSECTION}
               {error}
               {Not all the sections were properly closed.}
               {none}
\errentry{1486}{wrong ENDSECTION}
               {error}
               {The given \tty{ENDSECTION} does not refer to the most
                deeply nested one.}
               {none}
\errentry{1487}{ENDSECTION without SECTION}
               {error}
               {An \tty{ENDSECTION} command was found, but the associated section
                was not defined before.}
               {none}
\errentry{1488}{unresolved forward declaration}
               {error}
               {A symbol declared with a \tty{FORWARD} or \tty{PUBLIC} statement could
                not be resolved.}
               {the name of the unresolved symbol, plus the
                position of the forward declaration in the
                source.}
\errentry{1489}{conflicting FORWARD $<->$ PUBLIC-declaration}
               {error}
               {A symbol was defined both as public and private.}
               {the name of the symbol.}
\errentry{1490}{wrong numbers of function arguments}
               {error}
               {The number of arguments used for referencing a function
                does not match the number of arguments defined in the
                function definition.}
               {none}
\errentry{1495}{unresolved literals (missing LTORG)}
               {error}
               {At the end of the program, or just before switching to
                another processor type, unresolved literals still remain.}
               {none}
\errentry{1500}{instruction not allowed on}
               {error}
               {Although the instruction is correct, it cannot be used with
                the selected member of the CPU family.}
               {The processor variants that would support this
                instruction.}
\errentry{1501}{FPU instructions are not enabled}
               {error}
               {FPU instruction set extensions must be enabled to
                use this instruction.}
               {none}
\errentry{1502}{PMMU instructions are not enabled}
               {error}
               {PMMU instruction set extensions must be enabled
                to use this instruction.}
               {none}
\errentry{1503}{full PMMU instruction set is not enabed}
               {error}
               {This instrction is only contained in the 68851's
                instruction set, not in the reduced instruction
                set of the integrated PMMU.}
               {none}
\errentry{1504}{Z80 syntax was not allowed}
               {error}
               {This instruction is only allowed if Z80 syntax
                for 8080/8085 instructions has been enabled.}
               {none}
\errentry{1505}{addressing mode not allowed on}
               {error}
               {Although the addressing mode used is correct, it cannot be
                used with the selected member of the CPU family.}
               {The processor variants that would support this
                addressing mode.}
\errentry{1506}{not allowed in exclusive Z80 syntax mode}
               {error}
               {This instrction is no longer allowed if exclusive
                Z80 syntax mode for 8080/8085 instructions has been set.}
               {none}
\errentry{1507}{FPU instruction not supported on ...}
               {error}
               {Although this FPU instruction exists, it cannot be used on
                the selected type of FPU.}
               {The instruction in question}
\errentry{1508}{Custom instructions are not enabled}
               {error}
               {Custom instruction set extensions must be enabled
                to use this instruction.}
               {The instruction in question}
\errentry{1510}{invalid bit position}
               {error}
               {Either the number of bits specified is not allowed, or
                the command is not completely specified.}
               {none}
\errentry{1520}{only ON/OFF allowed}
               {error}
               {This pseudo command accepts as argument either \tty{ON} or
                \tty{OFF}}
               {none}
\errentry{1530}{stack is empty or undefined}
               {error}
               {It was tried to access a stack via a \tty{POPV} instruction
                that was either never defined or already emptied.}
               {the name of the stack in question}
\errentry{1540}{not exactly one bit set}
               {error}
               {Not exactly one bit was set in a mask passed to the
                \tty{BITPOS} function.}
               {none}
\errentry{1550}{ENDSTRUCT without STRUCT}
               {error}
               {An \tty{ENDSTRUCT} instruction was found though there is
                currently no structure definition in progress.}
               {none}
\errentry{1551}{open structure definition}
               {error}
               {After end of assembly, not all \tty{STRUCT} instructions
                have been closed with appropriate \tty{ENDSTRUCT}s.}
               {the innermost, unfinished structure definition}
\errentry{1552}{wrong ENDSTRUCT}
               {error}
               {the name parameter of an \tty{ENDSTRUCT} instruction does
                not correspond to the innermost open structure
                definition.}
               {none}
\errentry{1553}{phase definition not allowed in structure definition}
               {error}
               {What should I say about that?  \tty{PHASE} inside a record
                simply does not make sense and only leads to
                confusion...}
               {none}
\errentry{1554}{invalid \tty{STRUCT} directive}
               {error}
               {Only \tty{EXTNAMES}, \tty{NOEXTNAMES}, \tty{DOTS},
                and \tty{NODOTS} are allowed as directives of a
                \tty{STRUCT} statement.}
               {the unknown directive}
\errentry{1555}{structure re-defined}
               {error}
               {A structure of this name has already been defined.}
               {the name of the structure}
\errentry{1556}{unresolvable structure element reference}
               {error}
               {An element in a structure references to another
                element, however this referenced element was not
                defined or itself has an unresolvable reference.}
               {the name of the element itself and the referenced one}
\errentry{1557}{duplicate structure element}
               {error}
               {The structure already contains an element of this name.}
               {name of the element}
\errentry{1560}{instruction is not repeatable}
               {error}
               {This machine instruction cannot be repeated via a {\tt
                RPT} construct.}
               {none}
\errentry{1600}{unexpected end of file}
               {error}
               {It was tried to read past the end of a file with a
                \tty{BINCLUDE} statement.}
               {none}
\errentry{1700}{ROM-offset must be in range 0..63}
               {error}
               {The ROM table of the 680x0 coprocessor has only 64 entries.}
               {none}
\errentry{1710}{invalid function code}
               {error}
               {The only function code arguments allowed are SFC, DFC, a
                data register, or a constant in the interval of 0..15 (only
                for 680x0 MMU).}
               {none}
\errentry{1720}{invalid function code mask}
               {error}
               {Only a number in the interval 0..15 can be used as
                function code mask (only for 680x0 MMU)}
               {none}
\errentry{1730}{invalid MMU register}
               {error}
               {The MMU does not have a register with this name (only for
                680x0 MMU).}
               {none}
\errentry{1740}{level must be in range 0..7}
               {error}
               {The level for \tty{PTESTW} and \tty{PTESTR} must be a constant in the
                range of 0...7 (only for 680x0 MMU).}
               {none}
\errentry{1750}{invalid bit mask}
               {error}
               {The bit mask used for a bit field command has a wrong
                format (only for 680x0).}
               {none}
\errentry{1760}{invalid register pair}
               {error}
               {The register here defined cannot be used in this context,
                or there is a syntactic error (only for 680x0).}
               {none}
\errentry{1800}{open macro definition}
               {error}
               {An incomplete macro definition was found. Probably an
                \tty{ENDM} statement is missing.}
               {none}
\errentry{1801}{IRP without ENDM}
               {error}
               {An incomplete IRP block was found. Probably an
                \tty{ENDM} statement is missing.}
               {none}
\errentry{1802}{IRPC without ENDM}
               {error}
               {An incomplete IRPC block was found. Probably an
                \tty{ENDM} statement is missing.}
               {none}
\errentry{1803}{REPT without ENDM}
               {error}
               {An incomplete REPT block was found. Probably an
                \tty{ENDM} statement is missing.}
               {none}
\errentry{1804}{WHILE without ENDM}
               {error}
               {An incomplete WHILE block was found. Probably an
                \tty{ENDM} statement is missing.}
               {none}
\errentry{1805}{EXITM not called from within macro}
               {error}
               {\tty{EXITM} is designed to terminate a macro expansion.  This
                instruction only makes sense within macros and an attempt
                was made to call it in the absence of macros.}
               {none}
\errentry{1810}{more than 10 macro parameters}
               {error}
               {A macro cannot have more than 10 parameters}
               {none}
\errentry{1811}{keyword argument not defined in macro}
               {error}
               {a keyword argument referred to a parameter the
                called macro does not provide.}
               {used keyword resp. macro parameter}
\errentry{1812}{positional argument no longer allowed after keyword argument}
               {Fehler}
               {position and keyword arguments  may be mixed in
                one macro call, however only keyword arguments
                are allowed after the first keyword argument.}
               {none}
\errentry{1815}{macro double defined}
               {error}
               {A macro was defined more than once in a program section.}
               {the multiply defined macro name.}
\errentry{1820}{expression must be evaluatable in first pass}
               {error}
               {The command used has an influence on the length of the
                emitted code, so that forward references cannot be resolved
                here.}
               {none}
\errentry{1830}{too many nested IFs}
               {error}
               {(no more implemented)}
               {none}
\errentry{1840}{ELSEIF/ENDIF without IF}
               {error}
               {A \tty{ELSEIF}- or \tty{ENDIF}- command was found, that is not preceded
                by an \tty{IF}- command.}
               {none}
\errentry{1850}{nested / recursive macro call}
               {error}
               {(no more implemented)}
               {none}
\errentry{1860}{unknown function}
               {error}
               {The function invoked was not defined before.}
               {The name of the unknown function}
\errentry{1870}{function argument out of definition range}
               {error}
               {The argument does not belong to the allowed argument range
                associated to the referenced function.}
               {none}
\errentry{1880}{floating point overflow}
               {error}
               {Although the argument is within the range allowed to the
                function arguments, the result is not valid}
               {none}
\errentry{1890}{invalid value pair}
               {error}
               {The base-exponent pair used in the expression cannot be
                computed}
               {none}
\errentry{1900}{instruction must not start on this address}
               {error}
               {No jumps can be performed by the selected CPU from this
                address.}
               {none}
\errentry{1905}{invalid jump target}
               {error}
               {No jumps can be performed by the selected CPU to this
                address.}
               {none}
\errentry{1910}{jump target not on same page}
               {error}
               {Jump command and destination must be in the same memory
                page.}
               {none}
\errentry{1911}{jump target not in same section}
               {error}
               {Jump command and destination must be in the same (64K)
                memory section.}
               {none}
\errentry{1920}{code overflow}
               {error}
               {An attempt was made to generate more than 1024 code or
                data bytes in a single memory page.}
               {none}
\errentry{1925}{address overflow}
               {error}
               {The address space for the processor type actually used was
                filled beyond the maximum allowed limit.}
               {none}
\errentry{1930}{constants and placeholders cannot be mixed}
               {error}
               {Instructions that reserve memory, and instructions that define
                constants cannot be mixed in a single pseudo instruction.}
               {none}
\errentry{1940}{code must not be generated in structure definition}
               {error}
               {a \tty{STRUCT} construct is only designed to describe a
                data structure and not to create one; therefore, no
                instructions are allowed that generate code.}
               {none}
\errentry{1950}{parallel construct not possible here}
               {error}
               {Either these instructions cannot be executed in parallel,
                or they are not close enough each other, to do parallel
                execution.}
               {none}
\errentry{1960}{invalid segment}
               {error}
               {The referenced segment cannot be used here.}
               {The name of the segment used.}
\errentry{1961}{unknown segment}
               {error}
               {The segment referenced with a \tty{SEGMENT} command does not
                exist for the CPU used.}
               {The name of the segment used}
\errentry{1962}{unknown segment register}
               {error}
               {The segment referenced here does not exist (8086 only)}
               {none}
\errentry{1970}{invalid string}
               {error}
               {The string has an invalid format.}
               {none}
\errentry{1980}{invalid register name}
               {error}
               {The referenced register does not exist, or it cannot
                be used here.}
               {none}
\errentry{1985}{invalid argument}
               {error}
               {The command used cannot be performed with the \tty{REP}-prefix.}
               {none}
\errentry{1990}{indirect mode not allowed}
               {error}
               {Indirect addressing cannot be used in this way}
               {none}
\errentry{1995}{not allowed in current segment}
               {error}
               {(no more implemented)}
               {none}
\errentry{1996}{not allowed in maximum mode}
               {error}
               {This register can be used only in minimum mode}
               {none}
\errentry{1997}{not allowed in minimum mode}
               {error}
               {This register can be used only in maximum mode}
               {none}
\errentry{2000}{execution packet crosses address boundary}
               {error}
               {An execution packet must not cross a 32-byte address
                boundary}
               {none}
\errentry{2001}{multiple use of same execution unit}
               {error}
               {One of the CPU's execution units was used more than
                once in an execution packet}
               {the name of the execution unit}
\errentry{2002}{multiple long read operations}
               {error}
               {An execution packet contains more than one long read
                operation, which is not allowed}
               {one of the functional units executing a long read}
\errentry{2003}{multiple long write operations}
               {error}
               {An execution packet contains more than one long write
                operation, which is not allowed}
               {one of the functional units executing a long write}
\errentry{2004}{long read with write operation}
               {error}
               {An execution packet contains both a long read and a write
                operation, which is not allowed.}
               {one of the execution units executing the conflicting
                operations}
\errentry{2005}{too many reads of one register}
               {error}
               {The same register was referenced more than four times in
                the same execution packet.}
               {the name of the register referenced too often}
\errentry{2006}{overlapping destinations}
               {error}
               {The same register was written more than one time in the
                same instruction packet, which is not allowed.}
               {the name of the register in question}
\errentry{2008}{too many absolute branches in one execution packet}
               {error}
               {An execution packet contains more than one direct branch,
                which is not allowed.}
               {none}
\errentry{2009}{instruction cannot be executed on this unit}
               {error}
               {This instruction cannot be executed on this functional
                unit.}
               {none}
\errentry{2010}{invalid escape sequence}
               {error}
               {The special character defined using a backslash sequence
                is not defined}
               {none}
\errentry{2020}{invalid combination of prefixes}
               {error}
               {The prefix combination here defined is not allowed, or it
                cannot be translated into binary code}
               {none}
\errentry{2030}{constants cannot be redefined as variables}
               {error}
               {A symbol that has once been declared as constant with
                {\tt EQU} must not be modified afterwards with {\tt SET}.}
               {the name of the symbol in question}
\errentry{2035}{variables cannot be redefined as constants}
               {error}
               {A symbol that has once been declared as variable with
                {\tt SET} must not be redeclared afterwards as constant
                (e.g. with {\tt EQU}.}
               {the name of the symbol in question}
\errentry{2040}{structure name missing}
               {error}
               {A structure's definition lacks the identifier name for the
                new structure}
               {none}
\errentry{2050}{empty argument}
               {error}
               {Empty strings must not be used in the argument list for
                this statement}
               {none}
\errentry{2060}{unimplemented instruction}
               {error}
               {The used machinen instruction is principally known
                to the assembler, however, it is currently not
                implemented, du to lack of documentation from the
                processor manufacturer.}
               {the instruction that was used}
\errentry{2070}{unnamed structure is not part of another structure}
               {error}
               {An unnamed structure or union always must be part
                of another structure or union.}
               {none}
\errentry{2080}{STRUCT ended by ENDUNION}
               {error}
               {ENDUNION may only be used to finalize the definition
                of a union and not of a structure.}
               {name of the structure (if available)}
\errentry{2090}{Memory address mot on active memory page}
               {error}
               {The target address is not within the page
                that is currently addressable via the page
                register.}
               {none}
\errentry{2100}{unknown macro expansion argument}
               {error}
               {An argument to \tty{MACEXP} could not be
                interpreted.}
               {the unknown argument}
\errentry{2105}{too many macro expansion arguments}
               {error}
               {The number macro expansion arguments exceeds the allowed limit.}
               {the argument that busted the limit}
\errentry{2110}{contradicting macro expansion specifications}
               {error}
               {A specification about macro expansion and its
                precise opposite may not be used in the same
                \tty{MACEXP} instruction.}
               {none}
\errentry{2130}{erwarteter Fehler nicht eingetreten}
               {error}
               {An error or warning announced via {\tt EXPECT} did not occur in the
                instruction block terminated via {\tt ENDEXPECT}.}
               {The error that was expected}
\errentry{2140}{nesting of EXPECT/ENDEXPECT not allowed}
               {error}
               {Code blocks framed via {\tt EXPECT/ENDEXPECT} must not contain
                nested {\tt EXPECT/ENDEXPECT} blocks.}
               {none}
\errentry{2150}{missing ENDEXPECT}
               {error}
               {An instruction block opened via {\tt EXPECT} was not closed via
                {\tt ENDEXPECT}.}
               {none}
\errentry{2160}{ENDEXPECT without EXPECT}
               {error}
               {There is no matching previous {\tt EXPECT} to an {\tt ENDEXPECT}.}
               {none}
\errentry{2170}{no default checkpoint register defined}
               {error}
               {No checkpoint register was specified for a type 12 instruction
                and no default checkpoint register had previously been defined
                via the {\tt CKPT} statement.}
               {none}
\errentry{2180}{invalid bit field}
               {error}
               {The bit field is not in the required syntax {\tt (start,count)}.}
               {the argument in question}
\errentry{2190}{argument value missing}
               {error}
               {Arguments must have the form 'variable=value'.}
               {the argument in question}
\errentry{2200}{unknown argument}
               {error}
               {This variable is not supported by the selected target platform.}
               {the argument in question}
\errentry{2210}{index register must be 16 bit}
               {error}
               {Z8000 index registers must have a size of 16 bits (Rn).}
               {the argument in question}
\errentry{2211}{I/O address register must be 16 bit}
               {error}
               {Z8000 registers used to address I/O addresses must have a size of 16 bits (Rn).}
               {the argument in question}
\errentry{2212}{address register in segmented mode must be 32 bit}
               {error}
               {Z8000 registers to address memory in segmented mode must have a size of 32 bits (RRn).}
               {the argument in question}
\errentry{2213}{address register in non-segmented mode must be 16 bit}
               {error}
               {Z8000 registers to address memory in non-segmented mode must have a size of 16 bits (Rn).}
               {the argument in question}
\errentry{2220}{invalid structure argument}
               {error}
               {The argument does not match any pattern of allowed arguments
                when expanding a structure.}
               {the argument in question}
\errentry{2221}{too many array dimensions}
               {error}
               {Arrays of structures are limited to being three-dimensional.}
               {the dimension argument that was 'too much'}
\errentry{2230}{unknown integer notation}
               {error}
               {The given integer notation does not exist, or the leading plus resp.
                minus sign is missing.}
               {the argument in question}
\errentry{2231}{invalid list of integer notations}
               {error}
               {The requested changes to the list of usable integer notations cannot
                be applied, because they would result in a contradiction.  Currently,
                the only such case are 0hex und 0oct which cannot be used at the same
                time.}
               {none}
\errentry{2240}{invalid scale}
               {error}
               {The given argument cannot be used as scaling factor.}
               {the argument in question}
\errentry{2250}{conflicting string options}
               {error}
               {The string option is in contradiction to a previously given option.}
               {the option in question}
\errentry{2251}{unknown string option}
               {error}
               {The string option does not exist.}
               {the option in question}
\errentry{2252}{invalid cache invalidate mode}
               {error}
               {Only data, instruction, or both caches may be invalidated.}
               {the argument in question}
\errentry{2253}{invalid config list}
               {error}
               {The configuration list is either syntactically incorrect or
                contains invalid elements.}
               {The list in question or one of its elements}
\errentry{2254}{conflicting config options}
               {error}
               {The option is in contradiction to a previously given option or repeats a previous one.}
               {the option in question}
\errentry{2255}{unknown config option}
               {error}
               {The option does not exist.}
               {the option in question}
\errentry{10001}{error in opening file}
               {fatal}
               {An error was detected while trying to open a file for input.}
               {description of the I/O error}
\errentry{10002}{error in writing listing}
               {fatal}
               {An error happened while AS was writing the listing file.}
               {description of the I/O error}
\errentry{10003}{file read error}
               {fatal}
               {An error was detected while reading a source file.}
               {description of the I/O error}
\errentry{10004}{file write error}
               {fatal}
               {While AS was writing a code or share file, an error happened.}
               {description of the I/O error}
\errentry{10006}{heap overflow}
               {fatal}
               {The memory available is not enough to store all the data
                needed by AS. Try using the DPMI or OS/2 version of AS.}
               {none}
\errentry{10007}{stack overflow}
               {fatal}
               {The program stack crashed, because too complex formulas, or
                a bad disposition of symbols and/or macros were used. Try
                again, using AS with the option \tty{-A}.}
               {none}
\errentry{10008}{INCLUDE nested too deeply}
               {fatal}
               {The include nesting depth has exceeded the given limit (200
                by default). The limit may be raised via the {\tt -maxinclevel}
                command line argument, a wrong (recursive) inclusion is however
                the more probable cause.}
               {the INCLUDE statement that exceeded the limit}
\end{description}

%%===========================================================================

\cleardoublepage
\chapter{I/O Error Messages}

The following error messages are generated not only by AS, but also by
the auxiliary programs, like PLIST, BIND, P2HEX, and P2BIN. Only the most
probable error messages are here explained. Should you meet an undocumented
error message, then you probably met a program bug! Please inform us
immediately about this!!

\begin{description}
\item[2]{file not found\\
         The file requested does not exist, or it is stored on another
         drive.}
\item[3]{path not found\\
         The path of a file does not exist, or it is on another drive.}
\item[4]{too much open files\\
         There are no more file handles available to DOS. Increase
         their number changing the value associated to \tty{FILES=} in the file
         \tty{CONFIG.SYS}.}
\item[5]{file access not allowed\\
         Either the network access rights do not allow the file access, or
         an attempt was done to rewrite or rename a protected file.}
\item[6]{invalid file handler}
\item[12]{invalid access mode}
\item[15]{invalid drive letter\\
         The required drive does not exist.}
\item[16]{The file cannot be deleted}
\item[17]{RENAME cannot be done on this drive}
\item[100]{Unexpected end of file\\
         A file access tried to go beyond the end of file, although according
         to its structure this should not happen. The file is probably
         corrupted.}
\item[101]{disk full\\
         This is self explaining! Please, clean up !}
\item[102]{ASSIGN failed}
\item[103]{file not open}
\item[104]{file not open for reading}
\item[105]{file not open for writing}
\item[106]{invalid numerical format}
\item[150]{the disk is write-protected\\
         When you don't use a hard disk as work medium storage, you should
         sometimes remove the protecting tab from your diskette!}
\item[151]{unknown device\\
         you tried to access a peripheral unit that is unknown to DOS. This
         should not usually happen, since the name should be automatically
         interpreted as a filename.}
\item[152]{drive not ready\\
         close the disk drive door.}
\item[153]{unknown DOS function}
\item[154]{invalid disk checksum\\
         A bad read error on the disk. Try again; if nothing changes,
         reformat the floppy disk resp. begin to take care of your hard
         disk!}
\item[155]{invalid FCB}
\item[156]{position error\\
         the diskette/hard disk controller has not found a disk track. See
         nr. 154 !}
\item[157]{format unknown\\
         DOS cannot read the diskette format}
\item[158]{sector not found\\
         As nr. 156, but the controller this time could not find a disk
         sector in the track.}
\item[159]{end of paper\\
         You probably redirected the output of AS to a printer. Assembler
         printout can be veeery long...}
\item[160]{device read error\\
         The operating system detected an unclassificable read error}
\item[161]{device write error\\
         The operating system detected an unclassificable write error}
\item[162]{general failure error\\
         The operating system has absolutely no idea of what happened to the
         device.}
\end{description}

%%===========================================================================

\cleardoublepage
\chapter{Frequently Asked Questions}

In this chapter, I tried to collect some questions that arise very often
together with their answers.  Answers to the problems presented in
this chapter might also be found at other places in this manual, but
one maybe does not find them immediately...

\begin{description}
\item[Q:]{I am fed up with DOS.  Are there versions of AS for other
   operating systems ?}
\item[A:]{Apart from the protected mode version that offers more memory when
   working under DOS, ports exist for OS/2 and Unix systems like
   Linux (currently in test phase).  Versions that help operating
   system manufacturers located in Redmont to become even richer are
   currently not planned.  I will gladly make the sources of AS
   available for someone else who wants to become active in this
   direction.  The C variant is probably the best way to start a
   port into this direction.  He should however not expect support
   from me that goes beyond the sources themselves...}
\vspace{0.3cm}
\item[Q:]{Is a support of the XYZ processor planned for AS?}
\item[A:]{New processors are appearing all the time and I am trying to keep
   pace by extending AS.  The stack on my desk labeled ''undone''
   however never goes below the 4 inch watermark... Wishes coming
   from users of course play an important role in the decision which
   candidates will be done first.  The internet and the rising amount
   of documentation published in electronic form make the acquisition
   of data books easier than it used to be, but it always becomes
   difficult when more exotic or older architectures are wanted.  If
   the processor family in question is not in the list of families
   that are planned (see chapter 1), adding a data book to a request
   will have a highly positive influence.  Borrowing books is also
   fine.}
\vspace{0.3cm}
\item[Q:]{Having a free assembler is really fine, but I now also had use for
   a disassembler...and a debugger...a simulator would also really be
   cool!}
\item[A:]{AS is a project I work on in leisure time, the time I have when I
   do not have to care of how to make my living.  AS already takes a
   significant portion of that time, and sometimes I make a time-out
   to use my soldering iron, enjoy a Tangerine Dream CD, watch TV, or
   simply to fulfill some basic human needs... I once started to
   write the concept of a disassembler that was designed to create
   source code that can be assembled and that automatically
   separates code and data areas.  I quickly stopped this project
   again when I realized that the remaining time simply did not
   suffice.  I prefer to work on one good program than to struggle for
   half a dozen of mediocre apps.  Regarded that way, the answer to
   the question is unfortunately ''no''...}
\vspace{0.3cm}
\item[Q:]{The screen output of AS is messed up with strange characters, e.g.
   arrows and brackets.  Why?}
\item[A:]{AS will by default use some ANSI control sequences for screen
   control.  These sequences will appear unfiltered on your screen
   if you did not install an ANSI driver.  Either install an ANSI
   driver or use the DOS command \tty{SET USEANSI=N} to turn the
   sequences off.}
\vspace{0.3cm}
\item[Q:]{AS suddenly terminates with a stack overflow error while
   assembling my program.  Did my program become to large?}
\item[A:]{Yes and No.  Your program's symbol table has grown a bit
   unsymmetrically what lead to high recursion depths while accessing
   the table.  Errors of this type especially happen in the
   16-bit-OS/2 version of AS which has a very limited stack area.
   Restart AS with the \tty{-A} command line switch.  If this does not
   help, too complex formula expression are also a possible cause of
   stack overflows.  In such a case, try to split the formula into
   intermediate steps.}
\vspace{0.3cm}
\item[Q:]{It seems that AS does not assemble my program up to the end.  It
   worked however with an older version of AS (1.39).}
\item[A:]{Newer versions of AS no longer ignore the \tty{END} statement; they
   actually terminate assembly when an \tty{END} is encountered.
   Especially older include files made by some users tended to
   contain an \tty{END} statement at their end.  Simply remove the
   superfluous \tty{END} statements.}
\vspace{0.3cm}
\item[Q:]{I made an assembly listing of my program because I had some more
   complicated assembly errors in my program.  Upon closer
   investigation of the listing, I found that some branches do not
   point to the desired target but instead to themselves!}
\item[A:]{This effect happens in case of forward jumps in the first pass.
   The formula parser does not yet have the target address in its symbol
   table, and as it is a completely independent module, it has to think of
   a value that even does not hurt relative branches with short displacement
   lengths.  This is the current program counter itself...in the
   second pass, the correct values would have appeared, but the second
   pass did not happen due to errors in the first one.  Correct the
   other errors first so that AS gets into the second pass, and the
   listing should look more meaningful again.}
\vspace{0.3cm}
\item[Q:]{Assembly of my program works perfectly, however I get an empty
   file when I try to convert it with P2HEX or P2BIN.}
\item[A:]{You probably did not set the address filter correctly.  By
   default, the filter is disabled, i.e. all data is copied to the
   HEX or binary file.  It is however possible to create an empty file
   if a manually set range does not fit to the addresses used by your
   program.}
\vspace{0.3cm}
\item[Q:]{I cannot enter the dollar character when using P2BIN or P2HEX
   under Unix.  The automatic address range setting does not work, instead
   I get strange error messages.}
\item[A:]{Unix shells use the dollar character for expansion of shell
   variables.  If you want to pass a dollar character to an application,
   prefix it with a backslash (\verb!\!).  In the special case of the
   address range specification for P2HEX and P2BIN, you may also use
   \tty{0x} instead of the dollar character, which removes this prblen
   completely.}
\item[Q:]{I use AS on a Linux system, the loader program for my target
          system however runs on a Windows machine. To simplify things,
          both systems access the same network drive.  Unfortunately, the
          Windows side refuses to read the hex files created by the Linux
          side :-(}
\item[A:]{Windows and Linux systems use slightly different formats for
          text files (hex files are a sort of text files).  Windows
          terminates every line with the characters CR (carriage return)
          and LF (linefeed), however Linux only uses the linefeed.  It
          depends on the Windows program's 'goodwill' whether it will
          accept text files in the Linux format or not.  If not, it is
          possible to transfer the files via FTP in ASCII mode instead
          of a network drive.  Alternatively, the hex files can be
          converted to the Windows format.  For example, the program
          {\em unix2dos} can be used to do this, or a small script under
          Linux:
          \begin{verbatim}
          awk '{print $0"\r"}' test.hex >test_cr.hex
          \end{verbatim}}
\end{description}

%%===========================================================================

\cleardoublepage
\chapter{Pseudo-Instructions and Integer Syntax}
\label{SectPseudoInst}

This appendix is designed as a quick reference to look up all pseudo
instructions provided by AS.  The list is ordered in two parts: The
first part lists the instructions that are always available, and this
list is followed by lists that enumerate the instructions
additionally available for a certain processor family.

\subsubsection{Instructions that are always available}
{\tt\begin{tabbing}
\hspace{3cm}\=\hspace{3cm}\=\hspace{3cm}\=\hspace{3cm}\=\kill
=           \> :=          \> ALIGN       \> BINCLUDE    \> CASE \\
CHARSET     \> CPU         \> DEPHASE     \> DOTTEDSTRUCTS\> ELSE \\
ELSECASE    \> ELSEIF      \> END         \> ENDCASE     \> ENDIF \\
ENDM        \> ENDS        \> ENDSECTION  \> ENDSTRUCT   \> ENUM \\
ENUMCONF    \> ERROR       \> EQU         \> .EQU        \> EVAL \\
EXITM       \> FATAL       \> FORWARD     \> FUNCTION    \> GLOBAL \\
IF          \> IFB         \> IFDEF       \> IFEXIST     \> IFNB \\
IFNDEF      \> IFNEXIST    \> IFNUSED     \> IFUSED      \> INCLUDE \\
INTSYNTAX   \> IRP         \> LABEL       \> LISTING     \> MACEXP \\
MACECP\_DFT \> MACEXP\_OVR \> MACRO       \> MESSAGE     \> NEWPAGE \\
NEXTENUM    \> ORG         \> .PAGE       \> PHASE       \> POPV \\
PUSHV       \> PRTEXIT     \> PRTINIT     \> PUBLIC      \> READ \\
RELAXED     \> REPT        \> .RESTORE    \> RESTOREENV  \> RORG \\
.SAVE       \> SAVEENV     \> SECTION     \> SEGMENT     \> .SET \\
SHARED      \> .SHIFT      \> STRUC       \> STRUCT      \> .SWITCH \\
TITLE       \> UNION       \> WARNING     \> WHILE \\
\end{tabbing}}

Additionally, there are:
\begin{itemize}
\item{\tty{SET} resp. \tty{EVAL}, in case \tty{SET} is already a machine
      instruction.}
\item{\tty{SHIFT} resp. \tty{SHFT}, in case \tty{SHIFT} is already a machine
      instruction.}
\end{itemize}

 \subsubsection{Motorola 680x0/MCF5xxx}

{\em Default Integer Syntax: Motorola}

{\tt\begin{tabbing}
\hspace{3cm}\=\hspace{3cm}\=\hspace{3cm}\=\hspace{3cm}\=\kill
DC[.$<$size$>$]\> DS[.$<$size$>$] \> FULLPMMU    \> FPU         \> PADDING \\
PMMU       \> REG         \> SUPMODE \\
\end{tabbing}}

\subsubsection{Motorola 56xxx}

{\em Default Integer Syntax: Motorola}

{\tt\begin{tabbing}
\hspace{3cm}\=\hspace{3cm}\=\hspace{3cm}\=\hspace{3cm}\=\kill
DC         \> DS          \> PACKING     \> XSFR        \> YSFR \\
\end{tabbing}}

\subsubsection{PowerPC}

{\em Default Integer Syntax: C}

{\tt\begin{tabbing}
\hspace{3cm}\=\hspace{3cm}\=\hspace{3cm}\=\hspace{3cm}\=\kill
BIGENDIAN  \> DB          \> DD          \> DN          \> DQ \\
DS         \> DT          \> DW          \> REG         \> SUPMODE \\
\end{tabbing}}

\subsubsection{IBM PALM}

{\em Default Integer Syntax: IBM}

{\tt\begin{tabbing}
\hspace{3cm}\=\hspace{3cm}\=\hspace{3cm}\=\hspace{3cm}\=\kill
DB         \> DD          \> DN          \> DQ          \> DS \\
DT         \> DW          \> PORT        \> REG \\
\end{tabbing}}

\subsubsection{Motorola M-Core}

{\em Default Integer Syntax: Motorola}

{\tt\begin{tabbing}
\hspace{3cm}\=\hspace{3cm}\=\hspace{3cm}\=\hspace{3cm}\=\kill
DC[.$<$size$>$] \> DS[.$<$size$>$] \> REG \> SUPMODE \\
\end{tabbing}}

\subsubsection{Motorola XGATE}

{\em Default Integer Syntax: Motorola}

{\tt\begin{tabbing}
\hspace{3cm}\=\hspace{3cm}\=\hspace{3cm}\=\hspace{3cm}\=\kill
ADR        \> BYT         \> DC[.$<$size$>$] \> DFS     \> DS[.$<$size$>$] \\
FCB        \> FCC         \> FDB         \> PADDING     \> REG \\
RMB \\
\end{tabbing}}

\subsubsection{Motorola 68xx/Hitachi 63xx}

{\em Default Integer Syntax: Motorola}

{\tt\begin{tabbing}
\hspace{3cm}\=\hspace{3cm}\=\hspace{3cm}\=\hspace{3cm}\=\kill
ADR        \> BYT         \> DB          \> DC[.$<$size$>$] \> DFS     \\
DS[.$<$size$>$] \> DW     \> FCB         \> FCC         \> FDB         \\
PADDING     \> PRWINS(68HC11K4) \> RMB \\
\end{tabbing}}

\subsubsection{Motorola/Freescale 6805/68HC(S)08}

{\em Default Integer Syntax: Motorola}

{\tt\begin{tabbing}
\hspace{3cm}\=\hspace{3cm}\=\hspace{3cm}\=\hspace{3cm}\=\kill
ADR        \> BYT         \> DB          \> DC[.$<$size$>$] \> DFS     \\
DS[.$<$size$>$] \> DW     \> FCB         \> FCC         \> FDB         \\
PADDING     \> RMB \\
\end{tabbing}}

\subsubsection{Motorola 6809/Hitachi 6309}

{\em Default Integer Syntax: Motorola}

{\tt\begin{tabbing}
\hspace{3cm}\=\hspace{3cm}\=\hspace{3cm}\=\hspace{3cm}\=\kill
ADR        \> ASSUME      \> BYT         \> DB          \> DC[.$<$size$>$] \\
DFS        \> DS[.$<$size$>$] \> DW      \> FCB         \> FCC \\
FDB        \> PADDING     \> RMB \\
\end{tabbing}}

\subsubsection{Konami 052001}

{\em Default Integer Syntax: Motorola}

{\tt\begin{tabbing}
\hspace{3cm}\=\hspace{3cm}\=\hspace{3cm}\=\hspace{3cm}\=\kill
ADR        \> BYT         \> DB          \> DC[.$<$size$>$] DFS \\
DS[.$<$size$>$] \> DW      \> FCB         \> FCC \> FDB        \\
PADDING    \> RMB \\
\end{tabbing}}

\subsubsection{Motorola 68HC12}

{\em Default Integer Syntax: Motorola}

{\tt\begin{tabbing}
\hspace{3cm}\=\hspace{3cm}\=\hspace{3cm}\=\hspace{3cm}\=\kill
ADR        \> BYT         \> DB          \> DC[.$<$size$>$] \> DFS     \\
DS[.$<$size$>$] DW        \> FCB         \> FCC         \> FDB \\
PADDING     \> RMB \\
\end{tabbing}}

\subsubsection{NXP S12Z}

{\em Default Integer Syntax: Motorola}

{\tt\begin{tabbing}
\hspace{3cm}\=\hspace{3cm}\=\hspace{3cm}\=\hspace{3cm}\=\kill
ADR        \> BYT         \> DB          \> DC[.$<$size$>$] \> DEFBIT \\
DEFBITFIELD \> DFS        \> DS[.$<$size$>$] \> DW          \> FCB \\
FCC     \> FDB \> PADDING \> RMB \\
\end{tabbing}}

\subsubsection{Motorola 68HC16}

{\em Default Integer Syntax: Motorola}

{\tt\begin{tabbing}
\hspace{3cm}\=\hspace{3cm}\=\hspace{3cm}\=\hspace{3cm}\=\kill
ADR        \> ASSUME      \> BYT         \> DB          \> DC[.$<$size$>$] \\
DFS        \> DS[.$<$size$>$] \> DW      \> FCB         \> FCC \\
FDB         \> PADDING    \> RMB \\
\end{tabbing}}

\subsubsection{Freescale 68RS08}

{\em Default Integer Syntax: Motorola}

{\tt\begin{tabbing}
\hspace{3cm}\=\hspace{3cm}\=\hspace{3cm}\=\hspace{3cm}\=\kill
ADR        \> ASSUME      \> BYT         \> DB          \> DC[.$<$size$>$] \\
DFS        \> DS[.$<$size$>$] \> DW      \> FCB         \> FCC \\
FDB         \> PADDING \\
\end{tabbing}}

\subsubsection{Hitachi H8/300(L/H)}

{\em Default Integer Syntax: Motorola}

{\tt\begin{tabbing}
\hspace{3cm}\=\hspace{3cm}\=\hspace{3cm}\=\hspace{3cm}\=\kill
BIT    \> DC[.$<$size$>$] \> DS[.$<$size$>$] \> MAXMODE     \> PADDING \\
REG    \\
\end{tabbing}}

\subsubsection{Hitachi H8/500}

{\em Default Integer Syntax: Motorola}

{\tt\begin{tabbing}
\hspace{3cm}\=\hspace{3cm}\=\hspace{3cm}\=\hspace{3cm}\=\kill
ASSUME          \> BIT    \> COMPMODE \> DATA   \> DC[.$<$size$>$] \\
DS[.$<$size$>$] \> MAXMODE \> PADDING \> REG \\
\end{tabbing}}

\subsubsection{Hitachi SH7x00}

{\em Default Integer Syntax: Motorola}

{\tt\begin{tabbing}
\hspace{3cm}\=\hspace{3cm}\=\hspace{3cm}\=\hspace{3cm}\=\kill
COMPLITERALS \> DC[.$<$size$>$]  \>  DS[.$<$size$>$]  \>  LTORG    \> PADDING \\
REG          \> SUPMODE \\
\end{tabbing}}

\subsubsection{Hitachi HMCS400}

{\em Default Integer Syntax: Motorola}

{\tt\begin{tabbing}
\hspace{3cm}\=\hspace{3cm}\=\hspace{3cm}\=\hspace{3cm}\=\kill
DATA       \> RES         \> SFR \\
\end{tabbing}}

\subsubsection{Hitachi H16}

{\em Default Integer Syntax: Motorola}

{\tt\begin{tabbing}
\hspace{3cm}\=\hspace{3cm}\=\hspace{3cm}\=\hspace{3cm}\=\kill
BIT        \> DC[.$<$size$>$]    \> DS[.$<$size$>$]   \> REG       \> SUPMODE \\
\end{tabbing}}

\subsubsection{65xx/MELPS-740}

{\em Default Integer Syntax: Motorola}

{\tt\begin{tabbing}
\hspace{3cm}\=\hspace{3cm}\=\hspace{3cm}\=\hspace{3cm}\=\kill
ADR        \> ASSUME      \> BYT         \> DB          \> DCB \\
DDB        \> DFS         \> DS          \> DW          \> FCB \\
FCC        \> FDB         \> RMB \\
\end{tabbing}}

\subsubsection{65816/MELPS-7700}

{\em Default Integer Syntax: Motorola}

{\tt\begin{tabbing}
\hspace{3cm}\=\hspace{3cm}\=\hspace{3cm}\=\hspace{3cm}\=\kill
ADR        \> ASSUME      \> BYT         \> DB          \> DD \\
DDB        \> DN          \> DQ          \> DS          \> DT \\
DW         \> DFS         \> FCB         \> FCC         \> FDB \\
RMB \\
\end{tabbing}}

\subsubsection{Mitsubishi MELPS-4500}

{\em Default Integer Syntax: Motorola}

{\tt\begin{tabbing}
\hspace{3cm}\=\hspace{3cm}\=\hspace{3cm}\=\hspace{3cm}\=\kill
DATA       \> RES         \> SFR \\
\end{tabbing}}

\subsubsection{Rockwell PPS-4}

{\em Default Integer Syntax: Intel}

{\tt\begin{tabbing}
\hspace{3cm}\=\hspace{3cm}\=\hspace{3cm}\=\hspace{3cm}\=\kill
DATA       \> DC          \> DS          \> RES \\
\end{tabbing}}


\subsubsection{Mitsubishi M16}

{\em Default Integer Syntax: Intel}

{\tt\begin{tabbing}
\hspace{3cm}\=\hspace{3cm}\=\hspace{3cm}\=\hspace{3cm}\=\kill
DB         \> DD          \> DN          \> DQ          \> DS \\
DT         \> DW          \> REG \\
\end{tabbing}}

\subsubsection{Mitsubishi M16C}

{\em Default Integer Syntax: Intel}

{\tt\begin{tabbing}
\hspace{3cm}\=\hspace{3cm}\=\hspace{3cm}\=\hspace{3cm}\=\kill
DB         \> DD          \> DN          \> DQ          \> DS \\
DT         \> DW          \> REG \\
\end{tabbing}}

\subsubsection{DEC PDP-11}

{\em Default Integer Syntax: C}

{\tt\begin{tabbing}
\hspace{3cm}\=\hspace{3cm}\=\hspace{3cm}\=\hspace{3cm}\=\kill
BYTE       \> CIS         \> EIS         \> FIS         \> FLT2 \\
FLT4       \> FP11        \> REG         \> SUPMODE     \> WORD \\
\end{tabbing}}

\subsubsection{WD16}

{\em Default Integer Syntax: C}

{\tt\begin{tabbing}
\hspace{3cm}\=\hspace{3cm}\=\hspace{3cm}\=\hspace{3cm}\=\kill
BYTE       \> FLT3        \> REG         \> WORD \\
\end{tabbing}}

\subsubsection{Olympia CP-3F/GI LP8000/SGS M380}

{\em Default Integer Syntax: Intel}

{\tt\begin{tabbing}
\hspace{3cm}\=\hspace{3cm}\=\hspace{3cm}\=\hspace{3cm}\=\kill   
DC         \> DS          \> PORT \\
\end{tabbing}}

\subsubsection{Intel 4004/4040}

{\em Default Integer Syntax: Intel}

{\tt\begin{tabbing}
\hspace{3cm}\=\hspace{3cm}\=\hspace{3cm}\=\hspace{3cm}\=\kill   
DATA       \> DS          \> REG \\
\end{tabbing}}

\subsubsection{Intel 8008}

{\em Default Integer Syntax: Intel}

{\tt\begin{tabbing}
\hspace{3cm}\=\hspace{3cm}\=\hspace{3cm}\=\hspace{3cm}\=\kill    
DB         \> DD          \> DN          \> DQ          \> DS \\
DT         \> DW          \> PORT        \> Z80SYNTAX   \\
\end{tabbing}}

\subsubsection{Intel MCS-48}

{\em Default Integer Syntax: Intel}

{\tt\begin{tabbing}
\hspace{3cm}\=\hspace{3cm}\=\hspace{3cm}\=\hspace{3cm}\=\kill
ASSUME     \> DB         \> DD          \> DN          \> DQ \\
DS         \> DT         \> DW          \> REG \\
\end{tabbing}}

\subsubsection{Intel MCS-(2)51}

{\em Default Integer Syntax: Intel}

{\tt\begin{tabbing}
\hspace{3cm}\=\hspace{3cm}\=\hspace{3cm}\=\hspace{3cm}\=\kill
BIGENDIAN  \> BIT         \> DB          \> DD          \> DN \\
DQ         \> DS          \> DT          \> DW          \> PORT \\
REG        \> SFR         \> SFRB        \> SRCMODE \\
\end{tabbing}}

\subsubsection{Intel MCS-96}

{\em Default Integer Syntax: Intel}

{\tt\begin{tabbing}
\hspace{3cm}\=\hspace{3cm}\=\hspace{3cm}\=\hspace{3cm}\=\kill
ASSUME     \> DB          \> DD          \> DN          \> DQ \\
DS         \> DT          \> DW \\
\end{tabbing}}
 
\subsubsection{Intel 8080/8085}

{\em Default Integer Syntax: Intel}

{\tt\begin{tabbing}
\hspace{3cm}\=\hspace{3cm}\=\hspace{3cm}\=\hspace{3cm}\=\kill
DB         \> DD          \> DN          \> DQ          \> DS \\
DT         \> DW          \> PORT \\
\end{tabbing}}

\subsubsection{Intel 8086/80186/NEC V20...V5x}

{\em Default Integer Syntax: Intel}

{\tt\begin{tabbing}
\hspace{3cm}\=\hspace{3cm}\=\hspace{3cm}\=\hspace{3cm}\=\kill
ASSUME     \> DB          \> DD          \> DN          \> DQ \\
DS         \> DT          \> DW          \> PORT \\
\end{tabbing}}

\subsubsection{Intel i960}

{\em Default Integer Syntax: Intel}

{\tt\begin{tabbing}
\hspace{3cm}\=\hspace{3cm}\=\hspace{3cm}\=\hspace{3cm}\=\kill    
DB         \> DD          \> DN          \> DQ          \> DS \\
DT         \> DW          \> FPU         \> REG         \> SPACE \\
SUPMODE    \> WORD \\
\end{tabbing}}

\subsubsection{Signetics 8X30x}

{\em Default Integer Syntax: Motorola}

{\tt\begin{tabbing}
\hspace{3cm}\=\hspace{3cm}\=\hspace{3cm}\=\hspace{3cm}\=\kill
LIV        \> RIV         \\   
\end{tabbing}}

\subsubsection{Signetics 2650}

{\em Default Integer Syntax: Motorola}

{\tt\begin{tabbing}
\hspace{3cm}\=\hspace{3cm}\=\hspace{3cm}\=\hspace{3cm}\=\kill
ACON       \> BIGENDIAN  \> DB          \> DD          \> DN \\
DQ         \> DS         \> DT          \> DW          \> RES \\
\end{tabbing}}

\subsubsection{Philips XA}

{\em Default Integer Syntax: Intel}

{\tt\begin{tabbing}
\hspace{3cm}\=\hspace{3cm}\=\hspace{3cm}\=\hspace{3cm}\=\kill
ASSUME     \> BIT         \> DB          \> DC[.$<$size$>$] \> DD \\
DN         \> DQ          \> DS[.$<$size$>$] \> DT      \> DW \\
PADDING    \> PORT        \> REG         \> SUPMODE \\
\end{tabbing}}

\subsubsection{Atmel AVR}

{\em Default Integer Syntax: C}

{\tt\begin{tabbing}
\hspace{3cm}\=\hspace{3cm}\=\hspace{3cm}\=\hspace{3cm}\=\kill
BIT        \> DATA       \> DB          \> DD           \> DN \\
DQ         \> DS         \> DT          \> DW           \> PACKING \\
PORT       \> REG        \> RES         \> SFR \\
\end{tabbing}}

\subsubsection{AMD 29K}

{\em Default Integer Syntax: C}

{\tt\begin{tabbing}
\hspace{3cm}\=\hspace{3cm}\=\hspace{3cm}\=\hspace{3cm}\=\kill
ASSUME     \> DB          \> DD          \> DN          \> DQ \\
DS         \> DT          \> DW          \> EMULATED    \> ERG \\
SUPMODE \\
\end{tabbing}}

\subsubsection{Siemens 80C166/167}

{\em Default Integer Syntax: Intel}

{\tt\begin{tabbing}
\hspace{3cm}\=\hspace{3cm}\=\hspace{3cm}\=\hspace{3cm}\=\kill
ASSUME     \> BIT         \> DB          \> DD          \> DN \\
DQ         \> DS          \> DT          \> DW          \> REG \\
\end{tabbing}}

\subsubsection{Zilog Zx80}

{\em Default Integer Syntax: Intel}

{\tt\begin{tabbing}
\hspace{3cm}\=\hspace{3cm}\=\hspace{3cm}\=\hspace{3cm}\=\kill
DB         \> DD          \> DEFB        \> DEFW        \> DN \\
DQ         \> DS          \> DT          \> DW          \> EXTMODE \\
LWORDMODE  \> PRWINS(Z180) \> REG \\
\end{tabbing}}

\subsubsection{Zilog Z8}

{\em Default Integer Syntax: Intel}

{\tt\begin{tabbing}
\hspace{3cm}\=\hspace{3cm}\=\hspace{3cm}\=\hspace{3cm}\=\kill
DB         \> DEFBIT      \> DD          \> DN          \> DQ \\
DS         \> DT          \> DW          \> REG         \> SFR \\
\end{tabbing}}

\subsubsection{Zilog Z8000}

{\em Default Integer Syntax: Intel}

{\tt\begin{tabbing}
\hspace{3cm}\=\hspace{3cm}\=\hspace{3cm}\=\hspace{3cm}\=\kill
DB         \> DD          \> DEFBIT      \> DEFBITB    \> DN \\
DQ         \> DS          \> DT          \> DW         \> PORT \\
REG        \\
\end{tabbing}}

\subsubsection{Xilinx KCPSM}

{\em Default Integer Syntax: Intel}

{\tt\begin{tabbing}
\hspace{3cm}\=\hspace{3cm}\=\hspace{3cm}\=\hspace{3cm}\=\kill
CONSTANT   \> NAMEREG     \> REG \\
\end{tabbing}}

\subsubsection{Xilinx KCPSM3}

{\em Default Integer Syntax: Intel}

{\tt\begin{tabbing}
\hspace{3cm}\=\hspace{3cm}\=\hspace{3cm}\=\hspace{3cm}\=\kill
CONSTANT   \> DB         \> DD          \> DN          \> DQ \\
DS         \> DT         \> DW          \> NAMEREG     \> PORT \\
REG \\
\end{tabbing}}

\subsubsection{LatticeMico8}

{\em Default Integer Syntax: C}

{\tt\begin{tabbing}
\hspace{3cm}\=\hspace{3cm}\=\hspace{3cm}\=\hspace{3cm}\=\kill
DB         \> DD          \> DN         \> DQ          \> DS \\
DT         \> DW          \> PORT       \> REG \\
\end{tabbing}}

\subsubsection{Toshiba TLCS-900}

{\em Default Integer Syntax: Intel}

{\tt\begin{tabbing}
\hspace{3cm}\=\hspace{3cm}\=\hspace{3cm}\=\hspace{3cm}\=\kill
DB         \> DD          \> DN         \> DQ          \> DS \\
DT         \> DW          \> MAXIMUM    \> SUPMODE \\
\end{tabbing}}

\subsubsection{Toshiba TLCS-90}

{\em Default Integer Syntax: Intel}

{\tt\begin{tabbing}
\hspace{3cm}\=\hspace{3cm}\=\hspace{3cm}\=\hspace{3cm}\=\kill
DB         \> DD          \> DN         \> DQ          \> DS \\
DT         \> DW \\
\end{tabbing}}

\subsubsection{Toshiba TLCS-870(/C)}

{\em Default Integer Syntax: Intel}

{\tt\begin{tabbing}
\hspace{3cm}\=\hspace{3cm}\=\hspace{3cm}\=\hspace{3cm}\=\kill
DB         \> DD          \> DN         \> DQ          \> DS \\
DT         \> DW \\
\end{tabbing}}

\subsubsection{Toshiba TLCS-47(0(A))}

{\em Default Integer Syntax: Intel}

{\tt\begin{tabbing}
\hspace{3cm}\=\hspace{3cm}\=\hspace{3cm}\=\hspace{3cm}\=\kill
ASSUME     \> DB          \> DN         \> DD          \> DQ \\
DS         \> DT          \> DW         \> PORT \\
\end{tabbing}}

\subsubsection{Toshiba TLCS-9000}

{\em Default Integer Syntax: Intel}

{\tt\begin{tabbing}
\hspace{3cm}\=\hspace{3cm}\=\hspace{3cm}\=\hspace{3cm}\=\kill
DB         \> DD          \> DN         \> DQ          \> DS \\
DT         \> DW          \> REG \\
\end{tabbing}}

\subsubsection{Toshiba TC9331}

{\em Default Integer Syntax: Intel}

{\tt\begin{tabbing}
\hspace{3cm}\=\hspace{3cm}\=\hspace{3cm}\=\hspace{3cm}\=\kill
\end{tabbing}}

\subsubsection{Microchip PIC16C5x}

{\em Default Integer Syntax: Motorola}

{\tt\begin{tabbing}
\hspace{3cm}\=\hspace{3cm}\=\hspace{3cm}\=\hspace{3cm}\=\kill
DATA       \> RES         \> SFR         \> ZERO \\
\end{tabbing}}

\subsubsection{Microchip PIC16C8x}

{\em Default Integer Syntax: Motorola}

{\tt\begin{tabbing}
\hspace{3cm}\=\hspace{3cm}\=\hspace{3cm}\=\hspace{3cm}\=\kill
DATA       \> RES         \> SFR         \> ZERO \\
\end{tabbing}}

\subsubsection{Microchip PIC17C42}

{\em Default Integer Syntax: Motorola}

{\tt\begin{tabbing}
\hspace{3cm}\=\hspace{3cm}\=\hspace{3cm}\=\hspace{3cm}\=\kill
DATA       \> RES         \> SFR         \> ZERO \\
\end{tabbing}}

\subsubsection{Parallax SX20}

{\em Default Integer Syntax: Motorola}

{\tt\begin{tabbing}
\hspace{3cm}\=\hspace{3cm}\=\hspace{3cm}\=\hspace{3cm}\=\kill
BIT        \> DATA        \> SFR         \> ZERO \\
\end{tabbing}}

\subsubsection{SGS-Thomson ST6}

{\em Default Integer Syntax: Intel}

{\tt\begin{tabbing}
\hspace{3cm}\=\hspace{3cm}\=\hspace{3cm}\=\hspace{3cm}\=\kill
ASCII      \> ASCIZ       \> ASSUME      \> BIT  \>  BYTE \\
BLOCK      \> SFR         \> WORD \\
\end{tabbing}}

\subsubsection{SGS-Thomson ST7/STM8}

{\em Default Integer Syntax: Intel}

{\tt\begin{tabbing}
\hspace{3cm}\=\hspace{3cm}\=\hspace{3cm}\=\hspace{3cm}\=\kill
DC[.$<$size$>$] \> DS[.$<$size$>$]   \> PADDING \\
\end{tabbing}}

\subsubsection{SGS-Thomson ST9}

{\em Default Integer Syntax: Intel}

{\tt\begin{tabbing}
\hspace{3cm}\=\hspace{3cm}\=\hspace{3cm}\=\hspace{3cm}\=\kill
ASSUME     \> BIT         \> DB          \> DD          \> DN \\
DQ         \> DS          \> DT          \> DW          \> REG \\
\end{tabbing}}

\subsubsection{6804}

{\em Default Integer Syntax: Motorola}

{\tt\begin{tabbing}
\hspace{3cm}\=\hspace{3cm}\=\hspace{3cm}\=\hspace{3cm}\=\kill
ADR        \> BYT         \> DB          \> DFS         \> DS \\
DW         \> FCB         \> FCC         \> FDB         \> RMB \\
SFR \\
\end{tabbing}}

\subsubsection{Texas Instruments TMS3201x}

{\em Default Integer Syntax: Intel}

{\tt\begin{tabbing}
\hspace{3cm}\=\hspace{3cm}\=\hspace{3cm}\=\hspace{3cm}\=\kill
DATA       \> PORT        \> RES \\
\end{tabbing}}

\subsubsection{Texas Instruments TMS32C02x}

{\em Default Integer Syntax: Intel}

{\tt\begin{tabbing}
\hspace{3cm}\=\hspace{3cm}\=\hspace{3cm}\=\hspace{3cm}\=\kill
BFLOAT     \> BSS         \> BYTE        \> DATA        \> DOUBLE \\ 
EFLOAT     \> TFLOAT      \> LONG        \> LQxx        \> PORT \\
Qxx        \> RES         \> RSTRING     \> STRING      \> WORD \\
\end{tabbing}}

\subsubsection{Texas Instruments TMS320C3x/C4x}

{\em Default Integer Syntax: Intel}

{\tt\begin{tabbing}
\hspace{3cm}\=\hspace{3cm}\=\hspace{3cm}\=\hspace{3cm}\=\kill
ASSUME     \> BSS         \> DATA        \> EXTENDED    \> PACKING \\
SINGLE     \> WORD \\
\end{tabbing}}

\subsubsection{Texas Instruments TM32C020x/TM32C05x/TM32C054x}

{\em Default Integer Syntax: Intel}

{\tt\begin{tabbing}
\hspace{3cm}\=\hspace{3cm}\=\hspace{3cm}\=\hspace{3cm}\=\kill
BFLOAT     \> BSS         \> BYTE        \> DATA        \> DOUBLE \\
EFLOAT     \> TFLOAT      \> LONG        \> LQxx        \> PORT \\
Qxx        \> RES         \> RSTRING     \> STRING      \> WORD \\
\end{tabbing}}

\subsubsection{Texas Instruments TMS320C6x}

{\em Default Integer Syntax: Intel}

{\tt\begin{tabbing}
\hspace{3cm}\=\hspace{3cm}\=\hspace{3cm}\=\hspace{3cm}\=\kill
BSS         \> DATA        \> DOUBLE     \> PACKING     \> SINGLE \\
WORD \\
\end{tabbing}}

\subsubsection{Texas Instruments TMS99xx}

{\em Default Integer Syntax: Intel}

{\tt\begin{tabbing}
\hspace{3cm}\=\hspace{3cm}\=\hspace{3cm}\=\hspace{3cm}\=\kill
BSS        \> BYTE        \> DOUBLE      \> PADDING     \> SINGLE \\
WORD \\
\end{tabbing}}

\subsubsection{Texas Instruments Instruments TMS1000}

{\em Default Integer Syntax: Intel}

{\tt\begin{tabbing}
\hspace{3cm}\=\hspace{3cm}\=\hspace{3cm}\=\hspace{3cm}\=\kill
DB         \> DD          \> DN          \> DQ          \> DS \\
DT         \> DW \\
\end{tabbing}}

\subsubsection{Texas Instruments TMS70Cxx}

{\em Default Integer Syntax: Intel}

{\tt\begin{tabbing}
\hspace{3cm}\=\hspace{3cm}\=\hspace{3cm}\=\hspace{3cm}\=\kill
DB         \> DD          \> DN          \> DQ          \> DS \\
DT         \> DW \\
\end{tabbing}}

\subsubsection{Texas Instruments TMS370}

{\em Default Integer Syntax: Intel}

{\tt\begin{tabbing}
\hspace{3cm}\=\hspace{3cm}\=\hspace{3cm}\=\hspace{3cm}\=\kill
DB         \> DBIT        \> DN          \> DD          \> DQ \\
DS         \> DT          \> DW \\
\end{tabbing}}

\subsubsection{Texas Instruments MSP430}

{\em Default Integer Syntax: Intel}

{\tt\begin{tabbing}
\hspace{3cm}\=\hspace{3cm}\=\hspace{3cm}\=\hspace{3cm}\=\kill
BSS        \> BYTE        \> PADDING     \> REG         \> WORD \\
\end{tabbing}}

\subsubsection{National IMP-16}

{\em Default Integer Syntax: IBM}

{\tt\begin{tabbing}
\hspace{3cm}\=\hspace{3cm}\=\hspace{3cm}\=\hspace{3cm}\=\kill
ASCII      \> LTORG       \> PORT        \> WORD \\
\end{tabbing}}

\subsubsection{National IPC-16/INS8900}

{\em Default Integer Syntax: IBM}

{\tt\begin{tabbing}
\hspace{3cm}\=\hspace{3cm}\=\hspace{3cm}\=\hspace{3cm}\=\kill
ASCII      \> ASSUME      \> LTORG       \> WORD \\
\end{tabbing}}

\subsubsection{National SC/MP}

{\em Default Integer Syntax: C}

{\tt\begin{tabbing}
\hspace{3cm}\=\hspace{3cm}\=\hspace{3cm}\=\hspace{3cm}\=\kill
BIGENDIAN  \> DB         \> DD          \> DN          \> DQ \\
DS         \> DT         \> DW \\
\end{tabbing}}
	
\subsubsection{National INS807x}

{\em Default Integer Syntax: C}

{\tt\begin{tabbing}
\hspace{3cm}\=\hspace{3cm}\=\hspace{3cm}\=\hspace{3cm}\=\kill
DB         \> DD          \> DN          \> DQ          \> DS \\
DT         \> DW \\
\end{tabbing}}
        
\subsubsection{National COP4}

{\em Default Integer Syntax: C}

{\tt\begin{tabbing}
\hspace{3cm}\=\hspace{3cm}\=\hspace{3cm}\=\hspace{3cm}\=\kill
ADDR       \> ADDRW       \> BYTE        \> DB          \> DD \\
DQ         \> DS          \> DSB         \> DSW         \> DT \\
DW         \> FB          \> FW          \> SFR         \> WORD \\
\end{tabbing}}

\subsubsection{National COP8}

{\em Default Integer Syntax: C}

{\tt\begin{tabbing}
\hspace{3cm}\=\hspace{3cm}\=\hspace{3cm}\=\hspace{3cm}\=\kill
ADDR       \> ADDRW       \> BYTE        \> DB          \> DD \\
DQ         \> DS          \> DSB         \> DSW         \> DT \\
DW         \> FB          \> FW          \> SFR         \> WORD \\
\end{tabbing}}

\subsubsection{National SC14xxx}

{\em Default Integer Syntax: C}

{\tt\begin{tabbing}
\hspace{3cm}\=\hspace{3cm}\=\hspace{3cm}\=\hspace{3cm}\=\kill
DC         \> DC8         \> DS          \> DS8         \> DS16 \\
DW         \> DW16 \\
\end{tabbing}}

\subsubsection{National NS32xxx}
{\tt\begin{tabbing}
\hspace{3cm}\=\hspace{3cm}\=\hspace{3cm}\=\hspace{3cm}\=\kill
BYTE       \> CUSTOM      \> DB         \> DD           \> DOUBLE \\
DQ         \> DS          \> DT         \> DW           \> FLOAT \\
FPU        \> LONG        \> PMMU       \> REG          \> SUPMODE \\
WORD \\
\end{tabbing}}

\subsubsection{Fairchild ACE}

{\em Default Integer Syntax: Intel}

{\tt\begin{tabbing}
\hspace{3cm}\=\hspace{3cm}\=\hspace{3cm}\=\hspace{3cm}\=\kill
DB          \> DD          \> DN         \> DQ          \> DS \\
DT          \> DW \\
\end{tabbing}}

\subsubsection{Fairchild F8}

{\em Default Integer Syntax: Intel}

{\tt\begin{tabbing}
\hspace{3cm}\=\hspace{3cm}\=\hspace{3cm}\=\hspace{3cm}\=\kill
DB          \> DD          \> DN         \> DQ          \> DS \\
DT          \> DW          \> PORT \\
\end{tabbing}}

\subsubsection{NEC $\mu$PD7800...$\mu$PD7806, $\mu$PD78(C)1x}

{\em Default Integer Syntax: Intel}

{\tt\begin{tabbing}
\hspace{3cm}\=\hspace{3cm}\=\hspace{3cm}\=\hspace{3cm}\=\kill
ASSUME     \> DB          \> DD          \> DN          \> DQ \\
DS         \> DT          \> DW \\
\end{tabbing}}

\subsubsection{NEC $\mu$PD7807...$\mu$PD7809}

{\em Default Integer Syntax: Intel}

{\tt\begin{tabbing}
\hspace{3cm}\=\hspace{3cm}\=\hspace{3cm}\=\hspace{3cm}\=\kill
ASSUME     \> DB          \> DEFBIT      \>DD          \> DN \\
DQ         \> DS         \> DT          \> DW \\
\end{tabbing}}

\subsubsection{NEC 75xx}

{\em Default Integer Syntax: Intel}

{\tt\begin{tabbing}
\hspace{3cm}\=\hspace{3cm}\=\hspace{3cm}\=\hspace{3cm}\=\kill
DB         \> DD          \> DN          \> DQ          \>DS \\
DT         \> DW \\
\end{tabbing}}

\subsubsection{NEC 75K0}

{\em Default Integer Syntax: Intel}

{\tt\begin{tabbing}
\hspace{3cm}\=\hspace{3cm}\=\hspace{3cm}\=\hspace{3cm}\=\kill
ASSUME     \> BIT         \> DB          \> DD          \> DN \\
DQ         \> DS          \> DT          \> DW          \> SFR \\
\end{tabbing}}

\subsubsection{NEC 78K0}

{\em Default Integer Syntax: Intel}

{\tt\begin{tabbing}
\hspace{3cm}\=\hspace{3cm}\=\hspace{3cm}\=\hspace{3cm}\=\kill
DB         \> DD          \> DN          \> DQ          \> DS \\
DT         \> DW \\
\end{tabbing}}

\subsubsection{NEC 78K2}

{\em Default Integer Syntax: Intel}

{\tt\begin{tabbing}
\hspace{3cm}\=\hspace{3cm}\=\hspace{3cm}\=\hspace{3cm}\=\kill
BIT        \> DB         \> DD          \> DN          \> DQ \\
DS         \> DT         \> DW \\
\end{tabbing}}

\subsubsection{NEC 78K3}

{\em Default Integer Syntax: Intel}

{\tt\begin{tabbing}
\hspace{3cm}\=\hspace{3cm}\=\hspace{3cm}\=\hspace{3cm}\=\kill
BIT        \> DB         \> DD          \> DN          \> DQ \\
DS         \> DT         \> DW \\
\end{tabbing}}

\subsubsection{NEC 78K4}

{\em Default Integer Syntax: Intel}

{\tt\begin{tabbing}
\hspace{3cm}\=\hspace{3cm}\=\hspace{3cm}\=\hspace{3cm}\=\kill
BIT        \> DB         \> DD          \> DN          \> DQ \\
DS         \> DT         \> DW \\
\end{tabbing}}

\subsubsection{NEC $\mu$PD772x}

{\em Default Integer Syntax: Intel}

{\tt\begin{tabbing}
\hspace{3cm}\=\hspace{3cm}\=\hspace{3cm}\=\hspace{3cm}\=\kill
DATA       \> PACKING     \> RES \\
\end{tabbing}}

\subsubsection{NEC $\mu$PD77230}

{\em Default Integer Syntax: Intel}

{\tt\begin{tabbing} 
\hspace{3cm}\=\hspace{3cm}\=\hspace{3cm}\=\hspace{3cm}\=\kill
DS         \> DW          \> PACKING \\
\end{tabbing}}

\subsubsection{NEC V60}

{\em Default Integer Syntax: Intel}

{\tt\begin{tabbing}
\hspace{3cm}\=\hspace{3cm}\=\hspace{3cm}\=\hspace{3cm}\=\kill
DC[.$<$size$>$]\> DS[.$<$size$>$] \> PADDING \> REG         \> SUPMODE \\
\end{tabbing}}

\subsubsection{Symbios Logic SYM53C8xx}

{\em Default Integer Syntax: C}

{\tt\begin{tabbing} 
\hspace{3cm}\=\hspace{3cm}\=\hspace{3cm}\=\hspace{3cm}\=\kill
DB         \> DD          \> DN          \> DQ          \> DS \\
DT         \> DW \\
\end{tabbing}}

\subsubsection{Fujitsu F$^{2}$MC8L}

{\em Default Integer Syntax: Intel}

{\tt\begin{tabbing} 
\hspace{3cm}\=\hspace{3cm}\=\hspace{3cm}\=\hspace{3cm}\=\kill
DB         \> DD          \> DN          \> DQ          \> DS \\
DT         \> DW \\
\end{tabbing}}

\subsubsection{Fujitsu F$^{2}$MC16L}

{\em Default Integer Syntax: Intel}

{\tt\begin{tabbing} 
\hspace{3cm}\=\hspace{3cm}\=\hspace{3cm}\=\hspace{3cm}\=\kill
DB         \> DD          \> DN          \> DQ          \> DS \\
DT         \> DW \\
\end{tabbing}}

\subsubsection{OKI OLMS-40}

{\em Default Integer Syntax: Intel}

{\tt\begin{tabbing} 
\hspace{3cm}\=\hspace{3cm}\=\hspace{3cm}\=\hspace{3cm}\=\kill
DATA       \> RES         \> SFR \\
\end{tabbing}}

\subsubsection{OKI OLMS-50}

{\em Default Integer Syntax: Intel}

{\tt\begin{tabbing} 
\hspace{3cm}\=\hspace{3cm}\=\hspace{3cm}\=\hspace{3cm}\=\kill
DATA       \> RES         \> SFR \\
\end{tabbing}}

\subsubsection{Panafacom MN161x}

{\em Default Integer Syntax: IBM}

{\tt\begin{tabbing}
\hspace{3cm}\=\hspace{3cm}\=\hspace{3cm}\=\hspace{3cm}\=\kill
DC         \> DS          \> PACKING \\
\end{tabbing}}

\subsubsection{Padauk PMC/PMS/PFSxxx}

{\em Default Integer Syntax: C}

{\tt\begin{tabbing}
\hspace{3cm}\=\hspace{3cm}\=\hspace{3cm}\=\hspace{3cm}\=\kill
BIT        \> DATA       \> RES       \> SFR \\
\end{tabbing}}

\subsubsection{Intersil 180x}

{\em Default Integer Syntax: Intel}

{\tt\begin{tabbing}
\hspace{3cm}\=\hspace{3cm}\=\hspace{3cm}\=\hspace{3cm}\=\kill
DB         \> DD          \> DN          \> DQ          \> DS \\
DT         \> DW \\
\end{tabbing}}


\subsubsection{XMOS XS1}

{\em Default Integer Syntax: Motorola}

{\tt\begin{tabbing}
\hspace{3cm}\=\hspace{3cm}\=\hspace{3cm}\=\hspace{3cm}\=\kill
DB         \> DD          \> DQ          \> DN          \> DS \\
DT         \> DW          \> REG \\
\end{tabbing}}

\subsubsection{ATARI Vector}

{\em Default Integer Syntax: Motorola}

{\tt\begin{tabbing}
\hspace{3cm}\=\hspace{3cm}\=\hspace{3cm}\=\hspace{3cm}\=\kill
\end{tabbing}}

\subsubsection{MIL STD 1750}

{\em Default Integer Syntax: Intel}

{\tt\begin{tabbing}
\hspace{3cm}\=\hspace{3cm}\=\hspace{3cm}\=\hspace{3cm}\=\kill
DATA       \> EXTENDED    \> FLOAT \\
\end{tabbing}}

\subsubsection{KENBAK}

{\em Default Integer Syntax: Intel}

{\tt\begin{tabbing}
\hspace{3cm}\=\hspace{3cm}\=\hspace{3cm}\=\hspace{3cm}\=\kill
BIT        \> DB          \> DD          \> DN          \> DQ \\
DS         \> DT          \> DW          \> REG \\
\end{tabbing}}

\subsubsection{CP-1600}

{\em Default Integer Syntax: IBM (hex), C (oct) }

{\tt\begin{tabbing}
\hspace{3cm}\=\hspace{3cm}\=\hspace{3cm}\=\hspace{3cm}\=\kill
BYTE       \> PACKING     \> RES         \> TEXT        \> WORD \\
ZERO \\
\end{tabbing}}

\subsubsection{HP Nano Processor}

{\em Default Integer Syntax: C }

{\tt\begin{tabbing}
\hspace{3cm}\=\hspace{3cm}\=\hspace{3cm}\=\hspace{3cm}\=\kill
\end{tabbing}}

\subsubsection{IM61x0}

{\em Default Integer Syntax: C}

{\tt\begin{tabbing}
\hspace{3cm}\=\hspace{3cm}\=\hspace{3cm}\=\hspace{3cm}\=\kill
DC         \> DECIMAL     \> DS          \> LTORG       \> OCTAL \\
ZERO \\
\end{tabbing}}

\subsubsection{Renesas RX}

{\em Default Integer Syntax: Intel}

{\tt\begin{tabbing}
\hspace{3cm}\=\hspace{3cm}\=\hspace{3cm}\=\hspace{3cm}\=\kill
BLKB       \> BLKW        \> BLKL        \> BLKD        \> BYTE \\
DOUBLE     \> ENDIAN      \> FLOAT       \> LWORD       \> WORD \\
\end{tabbing}}


%%===========================================================================

\cleardoublepage
\chapter{Predefined Symbols}
\label{AppInternSyms}

\begin{table*}[htb]
\begin{center}\begin{tabular}{|l|l|l|l|}
\hline
name          & data type & definition & meaning \\
\hline
\hline
ARCHITECTURE  & string    & predef.    & target platform AS was \\
              &           &            & compiled for, in the style \\
              &           &            & processor-manufacturer- \\
              &           &            & operating system \\
BIGENDIAN     & boolean   & dyn.(0)    & storage of constants MSB \\
              &           &            & first ? \\
CASESENSITIVE & boolean   & normal     & case sensitivity in symbol \\
              &           &            & names ? \\
CONSTPI       & float     & normal     & constant Pi (3.1415.....) \\
DATE          & string    & predef.    & date of begin of assembly \\
FALSE         & boolean   & predef.    & 0 = logically ''false'' \\
HASFPU        & boolean   & dyn.(0)    & coprocessor instructions \\
              &           &            & enabled ? \\
HASPMMU       & boolean   & dyn.(0)    & MMU instructions enabled ? \\
INEXTMODE     & boolean   & dyn.(0)    & XM flag set for 4 Gbyte \\
              &           &            & address space ? \\
INLWORDMODE   & boolean   & dyn.(0)    & LW flag set for 32 bit \\
              &           &            & instructions ? \\
INMAXMODE     & boolean   & dyn.(0)    & processor in maximum \\
              &           &            & mode ? \\
INSUPMODE     & boolean   & dyn.(0)    & processor in supervisor \\
              &           &            & mode ? \\
\hline
\end{tabular}\end{center}
\caption{Predefined Symbols - Part 1\label{TabInternSyms1}}
\end{table*}

\begin{table*}[htb]
\begin{center}\begin{tabular}{|l|l|l|l|}
\hline
name             &  data type   & definition & meaning \\
\hline\hline
INSRCMODE     & boolean   & dyn.(0)    & processor in source mode ? \\
FULLPMMU      & boolean   & dyn.(0/1)  & full PMMU instruction set \\
              &           &            & allowed ? \\
LISTON        & boolean   & dyn.(1)    & listing enabled ? \\
MACEXP        & boolean   & dyn.(1)    & expansion of macro con- \\
              &           &            & structs in listing enabled ? \\
MOMCPU        & integer   & dyn.       & number of target CPU \\
	      & 	  & (68008)    & currently set \\
MOMCPUNAME    & string    & dyn.       & name of target CPU \\
              &           & (68008)    & currently set \\
MOMFILE       & string    & special    & current source file \\
              &           &            & (including include files) \\
MOMLINE       & integer   & special    & current line number in  \\
              &           &            & source file \\
MOMPASS       & integer   & special    & number of current pass \\
MOMSECTION    & string    & special    & name of current section or \\
	      & 	  &	       & empty string if out of any \\
              &           &            & section \\
MOMSEGMENT    & string    & special    & name of address space \\
              &           &            & currently selected \\
              &           &            & with \tty{SEGMENT} \\
              &           &            & \\
\hline
\end{tabular}\end{center}
\caption{Predefined Symbols - Part 2\label{TabInternSyms2}}
\end{table*}

\begin{table*}[htb]
\begin{center}\begin{tabular}{|l|l|l|l|}
\hline
name             &  data type   & definition & meaning \\
\hline\hline
\tty{NESTMAX} &  Integer  & dyn.(256)  & maximum nesting level \\
              &           &            & of macro expansions \\
\hline
PADDING       & boolean   & dyn.(1)    & pad byte field to even \\
              &           &            & count ? \\
RELAXED       & boolean   & dyn.(0)    & any syntax allowed integer \\
              &           &            & constants ? \\
PC            & integer   & special    & curr. program counter \\
              &           &            & (Thomson) \\
TIME          & string    & predef.    & time of begin of assembly \\
              &           &            & (1. pass) \\
TRUE          & integer   & predef.    & 1 = logically ''true'' \\
VERSION       & integer   & predef.    & version of AS in BCD \\
	      &  	  &	       & coding, e.g. 1331 hex for \\
              &           &            & version 1.33p1 \\
WRAPMODE      &  Integer  & predef.    & shortened program counter \\
              &           &            & assumed? \\
*             & integer   & special    & curr. program counter \\
              &           &            &  (Motorola, Rockwell, Micro- \\
              &           &            & chip, Hitachi) \\
\$            & integer   & special    & curr. program counter (Intel, \\
              &           &            & Zilog, Texas, Toshiba, NEC, \\
              &           &            & Siemens, AMD) \\
\hline
\end{tabular}\end{center}
\caption{Predefined Symbols - Part 3\label{TabInternSyms3}}
\end{table*}

To be exact, boolean symbols are just ordinary integer symbols with the
difference that AS will assign only two different values to them (0 or 1,
corresponding to False or True).  AS does not store special symbols
in the symbol table.  For performance reasons, they are realized with
hardcoded comparisons directly in the parser.  They therefore do not
show up in the assembly listing's symbol table.  Predefined symbols
are only set once at the beginning of a pass.  The values of dynamic
symbols may in contrast change during assembly as they reflect
settings made with related pseudo instructions.  The values added in
parentheses give the value present at the beginning of a pass.

The names given in this table also reflect the valid way to reference
these symbols in case-sensitive mode.

The names listed here should be avoided for own symbols; either one
can define but not access them (special symbols), or one will receive
an error message due to a double-defined symbol.  The ugliest case is
when the redefinition of a symbol made by AS at the beginning of a
pass leads to a phase error and an infinite loop...

%%===========================================================================

\cleardoublepage
\chapter{Shipped Include Files}

The distribution of AS contains a couple of include files.  Apart from
include files that only refer to a specific processor family (and whose
function should be immediately clear to someone who works with this
family), there are a few processor-independent files which include useful
functions.  The functions defined in these files shall be explained
briefly in the following sections:

\section{BITFUNCS.INC}

This file defines a couple of bit-oriented functions that might be
hardwired for other assemblers.  In the case of AS however, thaey are
implemented with the help of user-defined functions:

\begin{itemize}
\item{{\em mask(start,bits)} returns an integer with {\em bits} bits set
      starting at position {\em start};}
\item{{\em invmask(start,bits)} returns one's complement to {\em
      mask()};}
\item{{\em cutout(x,start,bits)} returns {\em bits} bits masked out from
      {\em x} starting at position {\em start} without shifting them to
      position 0;}
\item{{\em hi(x)} returns the second lowest byte (bits 8..15) of {\em
      x};}
\item{{\em lo(x)} returns the lowest byte (bits 8..15) of {\em x};}
\item{{\em hiword(x)} returns the second lowest word (bits 16..31) of
      {\em x};}
\item{{\em loword(x)} returns the lowest word (bits 0..15) of {\em x};}
\item{{\em odd(x)} returns TRUE if {\em x} is odd;}
\item{{\em even(x)} returns TRUE if {\em x} is even;}
\item{{\em getbit(x,n)} extracts bit {\em n} out of {\em x} and returns
      it as 0 or 1;}
\item{{\em shln(x,size,n)} shifts a word {\em x} of length {\em size} to
      the left by {\em n} places;}
\item{{\em shrn(x,size,n)} shifts a word {\em x} of length {\em size} to
      the right by {\em n} places;}
\item{{\em rotln(x,size,n)} rotates the lowest {\em size} bits of an
      integer {\em x} to the left by {\em n} places;}
\item{{\em rotrn(x,size,n)} rotates the lowest {\em size} bits of an
      integer {\em x} to the right by {\em n} places;}
\end{itemize}

\section{CTYPE.INC}

This include file is similar to the C include file {\tt ctype.h} which
offers functions to classify characters.  All functions deliver either
TRUE or FALSE:

\begin{itemize}
\item{{\em isdigit(ch)} becomes TRUE if {\em ch} is a valid decimal
      digit (0..9);}
\item{{\em isxdigit(ch)} becomes TRUE if {\em ch} is a valid hexadecimal
      digit (0..9, A..F, a..f);}
\item{{\em isupper(ch)} becomes TRUE if {\em ch} is an uppercase
      letter, excluding special national characters);}
\item{{\em islower(ch)} becomes TRUE if {\em ch} is a lowercase
      letter, excluding special national characters);}
\item{{\em isalpha(ch)} becomes TRUE if {\em ch} is a letter, excluding
      special national characters);}
\item{{\em isalnum(ch)} becomes TRUE if {\em ch} is either a letter or
      a valid decimal digit;}
\item{{\em isspace(ch)} becomes TRUE if {\em ch} is an 'empty' character
      (space, form feed, line feed, carriage return, tabulator);}
\item{{\em isprint(ch)} becomes TRUE if {\em ch} is a printable character,
      i.e. no control character up to code 31;}
\item{{\em iscntrl(ch)} is the opposite to {\em isprint()};}
\item{{\em isgraph(ch)} becomes TRUE if {\em ch} is a printable and
      visible character;}
\item{{\em ispunct(ch)} becomes TRUE if {\em ch} is a printable special
      character (i.e. neither space nor letter nor number);}
\end{itemize}

%%===========================================================================

\cleardoublepage
\chapter{Acknowledgments}

\begin{quote}\it
''If I have seen farther than other men, \\
it is because I stood on the shoulders of giants.'' \\
\hspace{2cm} --Sir Isaac Newton
\rm\end{quote}
\begin{quote}\it
''If I haven't seen farther than other men, \\
it is because I stood in the footsteps of giants.'' \\
\hspace{2cm} --unknown
\rm\end{quote}
\par
If one decides to rewrite a chapter that has been out of date for two
years, it is almost unavoidable that one forgets to mention some of
the good ghosts who contributed to the success this project had up
to now.  The first ''thank you'' therefore goes to the people whose
names I unwillingly forgot in the following enumeration!

The concept of AS as a universal cross assembler came from Bernhard
(C.) Zschocke who needed a ''student friendly'', i.e. free cross
assembler for his microprocessor course and talked me into extending
an already existing 68000 assembler.  The rest is history...
The microprocessor course held at RWTH Aachen also always provided the
most engaged users (and bug-searchers) of new AS features and
therefore contributed a lot to today's quality of AS.

The internet and FTP have proved to be a big help for spreading AS and
reporting of bugs.  My thanks therefore go to the FTP admins (Bernd
Casimir in Stuttgart, Norbert Breidor in Aachen, and J\"urgen Mei\ss\-burger
in J\"ulich).  Especially the last one personally engaged a lot to
establish a practicable way in J\"ulich.

As we are just talking about the ZAM: Though Wolfgang E. Nagel is not
personally involved into AS, he is at least my boss and always puts
at least four eyes on what I am doing.  Regarding AS, there seems to
be at least one that smiles...

A program like AS cannot be done without appropriate data books and
documentation.  I received information from an enormous amount of
people, ranging from tips up to complete data books.  An enumeration
follows (as stated before, without guarantee for completelessness!):

Ernst Ahlers, Charles Altmann, Marco Awater, Len Bayles,
Andreas Bolsch, Rolf Buchholz, Bernd Casimir, Nils Eilers,
Gunther Ewald, Michael Haardt, Stephan Hruschka, Peter Kliegelh\"ofer,
Ulf Meinke, Udo M\"oller, Matthias Paul, Norbert Rosch, Curt J. Sampson,
Steffen Schmid, Leonhard Schneider, Ernst Schwab, Michael Schwingen,
Oliver Sellke, Christian Stelter, Patrik Str\"omdahl,
Tadashi G. Takaoka, Oliver Thamm, Thorsten Thiele, Leszek Ulman,
Rob Warmelink, Andreas Wassatsch, John Weinrich.

...and an ironic ''thank you'' to Rolf-Dieter-Klein and Tobias Thiel who
demonstrated with their ASM68K how one should \bb{not} do it and thereby
indirectly gave me the impulse to write something better!

I did not entirely write AS on my own. The DOS version of AS contained
the OverXMS routines from Wilbert van Leijen which can move the overlay
modules into the extended memory.  A really nice library, easy to use
without problems!

The TMS320C2x/5x code generators and the file \tty{STDDEF2x.INC} come
from Thomas Sailer, ETH Zurich.  It's surprising, he only needed one
weekend to understand my coding and to implement the new code generator.
Either that was a long nightshift or I am slowly getting old...the same
praise goes to Haruo Asano for providing the MN1610/MN1613 code generator.

%%===========================================================================

\cleardoublepage
\chapter{Changes since Version 1.3}

\begin{itemize}
\item{version 1.31:
      \begin{itemize}
      \item{additional MCS-51 processor type 80515.  The number
            is again only stored by the assembler.  The file
            \tty{STDDEF51.INC} was extended by the necessary SFRs.
            \bb{CAUTION!} Some of the 80515 SFRs have moved to other
            addresses!}
      \item{additional support for the Z80 processor;}
      \item{faster 680x0 code generator.}
      \end{itemize}}
\item{version 1.32:
      \begin{itemize}
      \item{syntax for zero page addresses for the 65xx family
            was changed from \tty{addr.z} to \tty{$<$addr} (similar to 68xx);}
      \item{additional support for the 6800, 6805, 6301, and
            6811 processors;}
      \item{the 8051 part now also understands \tty{DJNZ, PUSH}, and
            \tty{POP} (sorry);}
      \item{the assembly listing now not also list the symbols
            but also the macros that have been defined;}
      \item{additional instructions \tty{IFDEF/IFNDEF} for conditional
            assembly based on the existence of a symbol;}
      \item{additional instructions \tty{PHASE/DEPHASE} to support code
            that shall be moved at runtime to a different address;}
      \item{additional instructions \tty{WARNING, ERROR}, and \tty{FATAL} to print
            user-defined error messages;}
      \item{the file \tty{STDDEF51.INC} additionally contains the macro
            \tty{USING} to simplify working with the MCS-51's register
            banks;}
      \item{command line option \tty{u} to print segment usage;}
      \end{itemize}}
\item{version 1.33:
      \begin{itemize}
      \item{additionally supports the 6809 processor;}
      \item{added string variables;}
      \item{The instructions \tty{TITLE, PRTINIT, PRTEXIT, ERROR},
            \tty{WARNING}, and \tty{FATAL} now expect a string expression.
            Constants therefore now have to be enclosed in
            '' instead of ' characters.  This is also true
            for \tty{DB}, \tty{DC.B}, and \tty{BYT};}
      \item{additional instruction \tty{ALIGN} to align the program
            counter for Intel processors;}
      \item{additional instruction \tty{LISTING} to turn the generation
            of an assembly listing on or off;}
      \item{additional instruction \tty{CHARSET} for user-defined
            character sets.}
      \end{itemize}}
\item{version 1.34:
      \begin{itemize}
      \item{the second pass is now omitted if there were errors
            in the first pass;}
      \item{additional predefined symbol \tty{VERSION} that contains
            the version number of AS;}
      \item{additional instruction \tty{MESSAGE} to generate additional
            messages under program control;}
      \item{formula parser is now accessible via string constants;}
      \item{if an error in a macro occurs, additionally the line
            number in the macro itself is shown;}
      \item{additional function \tty{UPSTRING} to convert a string to
            all upper-case.}
      \end{itemize}}
\item{version 1.35:
      \begin{itemize}
      \item{additional function \tty{TOUPPER} to convert a single
            character to upper case;}
      \item{additional instruction \tty{FUNCTION} for user-defined
            functions;}
      \item{additional command line option \tty{D} to define symbols
            from outside;}
      \item{the environment variable \tty{ASCMD} for commonly used
            command line options was introduced;}
      \item{the program will additionally be checked for double
            usage of memory areas if the u option is enabled;}
      \item{additional command line option \tty{C} to generate a cross
            reference list.}
      \end{itemize}}
\item{version 1.36:
      \begin{itemize}
      \item{additionally supports the PIC16C5x and PIC17C4x
            processor families;}
      \item{the assembly listing additionally shows the nesting
            depth of include files;}
      \item{the cross reference list additionally shows the
            definition point of a symbol;}
      \item{additional command line option \tty{A} to force a more
            compact layout of the symbol table.}
      \end{itemize}}
\item{version 1.37:
      \begin{itemize}
      \item{additionally supports the processors 8086, 80186,
            V30, V35, 8087, and Z180;}
      \item{additional instructions \tty{SAVE} and \tty{RESTORE} for an
            easier switching of some flags;}
      \item{additional operators for logical shifts and bit
            mirroring;}
      \item{command line options may now be negated with a
            plus sign;}
      \item{additional filter AS2MSG for a more comfortable
            work with AS under Turbo-Pascal 7.0;}
      \item{\tty{ELSEIF} now may have an argument for construction
            of \tty{IF\--THEN\--ELSE} ladders;}
      \item{additional \tty{CASE} construct for a more comfortable
            conditional assembly;}
      \item{user-defined functions now may have more than one
            argument;}
      \item{P2HEX can now additionally generate hex files in
            a format suitable for 65xx processors;}
      \item{BIND, P2HEX, and P2BIN now have the same scheme
            for command line processing like AS;}
      \item{additional switch \tty{i} for P2HEX to select one out
            three possibilities for the termination record;}
      \item{additional functions \tty{ABS} and \tty{SGN};}
      \item{additional predefined symbols \tty{MOMFILE} and \tty{MOMLINE};}
      \item{additional option to print extended error messages;}
      \item{additional instruction \tty{IFUSED} and \tty{IFNUSED} to check
            whether a symbol has been used so far;}
      \item{The environment variables \tty{ASCMD, BINDCMD} etc. now
            optionally may contain the name of a file that
            provides more space for options;}
      \item{P2HEX can now generate the hex formats specified
            by Microchip (p4);}
      \item{a page length specification of 0 now allows to
            suppress automatic formfeeds in the assembly listing
            completely (p4);}
      \item{symbols defined in the command line now may be
            assigned an arbitrary value (p5).}
      \end{itemize}}
\item{version 1.38:
      \begin{itemize}
      \item{changed operation to multipass mode.  This enables
            AS to generate optimal code even in case of forward
            references;}
      \item{the 8051 part now also knows the generic \tty{JMP} and
            \tty{CALL} instructions;}
      \item{additionally supports the Toshiba TLCS-900 series
            (p1);}
      \item{additional instruction \tty{ASSUME} to inform the assembler
            about the 8086's segment register contents (p2);}
      \item{additionally supports the ST6 series from
            SGS-Thomson (p2);}
      \item{..and the 3201x signal processors from Texas
            Instruments (p2);}
      \item{additional option \tty{F} for P2HEX to override the
            automatic format selection (p2);}
      \item{P2BIN now can automatically set the start resp.
            stop address of the address window by specifying
            dollar signs (p2);}
      \item{the 8048 code generator now also knows the 8041/42
            instruction extensions (p2);}
      \item{additionally supports the Z8 microcontrollers (p3).}
      \end{itemize}}
\item{version 1.39:
      \begin{itemize}
      \item{additional opportunity to define sections and local
            symbols;}
      \item{additional command line switch \tty{h} to force hexadecimal
            numbers to use lowercase;}
      \item{additional predefined symbol \tty{MOMPASS} to read the
            number of the currently running pass;}
      \item{additional command line switch \tty{t} to disable
            individual parts of the assembly listing;}
      \item{additionally knows the L variant of the TLCS-900
            series and the MELPS-7700 series from Mitsubishi
            (p1);}
      \item{P2HEX now also accepts dollar signs as start resp.
            stop address (p2);}
      \item{additionally supports the TLCS-90 family from
            Toshiba (p2);}
      \item{P2HEX now also can output data in Tektronix and
            16 bit Intel Hex format (p2);}
      \item{P2HEX now prints warnings for address overflows
            (p2);}
      \item{additional include file \tty{STDDEF96.INC} with address
            definitions for the TLCS-900 series (p3);}
      \item{additional instruction \tty{READ} to allow interactive
            input of values during assembly (p3);}
      \item{error messages are written to the STDERR channel
            instead of standard output (p3);}
      \item{the \tty{STOP} instruction missing for the 6811 is now
            available (scusi, p3);}
      \item{additionally supports the $\mu$PD78(C)1x family from
            NEC (p3);}
      \item{additionally supports the PIC16C84 from NEC (p3);}
      \item{additional command line switch \tty{E} to redirect error
            messages to a file (p3);}
      \item{The MELPS-7700's 'idol' 65816 is now also available
            (p4);}
      \item{the ST6 pseudo instruction \tty{ROMWIN} has been removed
            was integrated into the \tty{ASSUME} instruction (p4);}
      \item{additionally supports the 6804 from SGS-Thomson (p4);}
      \item{via the \tty{NOEXPORT} option in a macro definition, it is
            now possible to define individually for every macro
            whether it shall appear in the \tty{MAC} file or not (p4);}
      \item{the meaning of \tty{MACEXP} regarding the expansion of
            macros has changed slightly due to the additional
            \tty{NOEXPAND} option in the macro definition (p4);}
      \item{The additional \tty{GLOBAL} option in the macro definition
            now additionally allows to define macros that are
            uniquely identified by their section name (p4).}
      \end{itemize}}
\item{version 1.40:
      \begin{itemize}
      \item{additionally supports the DSP56000 from Motorola;}
      \item{P2BIN can now also extract the lower resp. upper
            half of a 32-bit word;}
      \item{additionally supports the TLCS-870 and TLCS-47
            families from Toshiba (p1);}
      \item{a prefixed \tty{!} now allows to reach machine instructions
            hidden by a macro (p1);}
      \item{the \tty{GLOBAL} instruction now allows to export symbols
            in a qualified style (p1);}
      \item{the additional \tty{r} command line switch now allows to
            print a list of constructs that forced additional
            passes (p1);}
      \item{it is now possible to omit an argument to the \tty{E}
            command line option; AS will then choose a fitting
            default (p1);}
      \item{the \tty{t} command line option now allows to suppress
            line numbering in the assembly listing (p1);}
      \item{escape sequences may now also be used in ASCII style
            integer constants (p1);}
      \item{the additional pseudo instruction \tty{PADDING} now allows
            to enable or disable the insertion of padding bytes
            in 680x0 mode (p2);}
      \item{\tty{ALIGN} is now a valid instruction for all targets
            (p2);}
      \item{additionally knows the PIC16C64's SFRs (p2);}
      \item{additionally supports the 8096 from Intel (p2);}
      \item{\tty{DC} additionally allows to specify a repetition factor
            (r3);}
      \item{additionally supports the TMS320C2x family from Texas
            Instruments (implementation done by Thomas Sailer, ETH
            Zurich, r3); P2HEX has been extended appropriately;}
      \item{an equation sign may be used instead of \tty{EQU} (r3);}
      \item{additional \tty{ENUM} instruction to define enumerations
            (r3);}
      \item{\tty{END} now has a real effect (r3);}
      \item{additional command line switch \tty{n} to get the internal
            error numbers in addition to the error messages (r3);}
      \item{additionally supports the TLCS-9000 series from
            Toshiba (r4);}
      \item{additionally supports the TMS370xxx series from Texas
            Instruments, including a new \tty{DBIT} pseudo instruction
            (r5);}
      \item{additionally knows the DS80C320's SFR's (r5);}
      \item{the macro processor is now also able to include files
            from within macros.  This required to modify the
            format of error messages slightly.  If you use
            AS2MSG, replace it with the new version! (r5)}
      \item{additionally supports the 80C166 from Siemens (r5);}
      \item{additional \tty{VAL} function to evaluate string
            expressions (r5);}
      \item{it is now possible to construct symbol names with the
            help of string expressions enclosed in braces (r5);}
      \item{additionally knows the 80C167's peculiarities (r6);}
      \item{the MELPS740's special page addressing mode is now
            supported (r6);}
      \item{it is now possible to explicitly reference a symbol
            from a certain section by appending its name enclosed
            in brackets.  The construction with an \tty{@} sign has
            been removed! (r6)}
      \item{additionally supports the MELPS-4500 series from
            Mitsubishi (r7);}
      \item{additionally supports H8/300 and H8/300H series from
            Hitachi (r7);}
      \item{settings made with \tty{LISTING} resp. \tty{MACEXP} may now be
            read back from predefined symbols with the same names
            (r7);}
      \item{additionally supports the TMS320C3x series from Texas
            Instruments (r8);}
      \item{additionally supports the SH7000 from Hitachi (r8);}
      \item{the Z80 part has been extended to also support the
            Z380 (r9);}
      \item{the 68K part has been extended to know the
            differences of the 683xx micro controllers (r9);}
      \item{a label not any more has to be placed in the first
            row if it is marked with a double dot (r9);}
      \item{additionally supports the 75K0 series from NEC (r9);}
      \item{the additional command line option o allows to set
            a user-defined name for the code file (r9);}
      \item{the \verb!~~! operator has been moved to a bit more senseful
            ranking (r9);}
      \item{\tty{ASSUME} now also knows the 6809's DPR register and its
            implications (pardon, r9);}
      \item{the 6809 part now also knows the 6309's secret
            extensions (r9);}
      \item{binary constants now also may be written in a C-like
            notation (r9);}
      \end{itemize}}
\item{version 1.41:
      \begin{itemize}
      \item{the new predefined symbol \tty{MOMSEGMENT} allows to
            inquire the currently active segment;}
      \item{\tty{:=} is now allowed as a short form for \tty{SET/EVAL};}
      \item{the new command line switch \tty{q} allows to force a
            ''silent'' assembly;}
      \item{the key word \tty{PARENT} to reference the parent section
            has been extended by \tty{PARENT0..PARENT9};}
      \item{the PowerPC part has been extended by the
            microcontroller versions MPC505 and PPC403;}
      \item{symbols defined with \tty{SET} or \tty{EQU} may now be assigned
            to a certain segment (r1);}
      \item{the SH7000 part now also knows the SH7600's
            extensions (and should compute correct
            displacements...) (r1);}
      \item{the 65XX part now differentiates between the 65C02
            and 65SC02 (r1);}
      \item{additionally to the symbol \tty{MOMCPU}, there is now also
            a string symbol \tty{MOMCPUNAME} that contains the
            processor's full name (r1);}
      \item{P2HEX now also knows the 32-bit variant of the Intel
            hex format (r1);}
      \item{additionally knows the 87C750's limitations (r2);}
      \item{the internal numbers for fatal errors have been moved
            to the area starting at 10000, making more space for
            normal error messages (r2);}
      \item{unused symbols are now marked with a star in the
            symbol table (r2);}
      \item{additionally supports the 29K family from AMD (r2);}
      \item{additionally supports the M16 family from Mitsubishi
            (r2);}
      \item{additionally supports the H8/500 family from Hitachi
            (r3);}
      \item{the number of data bytes printed per line by P2HEX
            can now be modified (r3);}
      \item{the number of the pass that starts to output warnings
            created by the \tty{r} command line switch is now variable
            (r3);}
      \item{the macro processor now knows a \tty{WHILE} statement that
            allows to repeat a piece of code a variable number of
            times (r3);}
      \item{the \tty{PAGE} instruction now also allows to set the line
            with of the assembly listing (r3);}
      \item{CPU aliases may now be defined to define new pseudo
            processor devices (r3);}
      \item{additionally supports the MCS/251 family from Intel
            (r3);}
      \item{if the cross reference list has been enabled, the
            place of the first definition is given for double
            definitions of symbols (r3);}
      \item{additionally supports the TMS320C5x family from Texas
            Instruments (implementation done by Thomas Sailer,
            ETH Zurich, r3);}
      \item{the OS/2 version should now also correctly work with
            long file names.  If one doesn't check every s**t
            personally... (r3);}
      \item{the new pseudo instruction \tty{BIGENDIAN} now allows to
            select in MCS-51/251 mode whether constants should
            be stored in big endian or little endian format (r3);}
      \item{the 680x0 part now differentiates between the full
            and reduced MMU instruction set; a manual toggle can
            be done via the \tty{FULLPMMU} instruction (r3);}
      \item{the new command line option \tty{I} allows to print a list
            of all include files paired with their nesting level
            (r3);}
      \item{additionally supports the 68HC16 family from Motorola
            (r3);}
      \item{the \tty{END} statement now optionally accepts an argument
            as entry point for the program (r3);}
      \item{P2BIN and P2HEX now allow to move the contents of a
            code file to a different address (r4);}
      \item{comments appended to a \tty{SHARED} instruction are now
            copied to the share file (r4);}
      \item{additionally supports the 68HC12 family from Motorola
            (r4);}
      \item{additionally supports the XA family from Philips
            (r4);}
      \item{additionally supports the 68HC08 family from Motorola
            (r4);}
      \item{additionally supports the AVR family from Atmel (r4);}
      \item{to achieve better compatibility to the AS11 from
            Motorola, the pseudo instructions \tty{FCB, FDB, FCC}, and
            \tty{RMB} were added (r5);}
      \item{additionally supports the M16C from Mitsubishi (r5);}
      \item{additionally supports the COP8 from National
            Semiconductor (r5);}
      \item{additional instructions \tty{IFB} and \tty{IFNB} for conditional
            assembly (r5);}
      \item{the new \tty{EXITM} instruction now allows to terminate a
            macro expansion (r5);}
      \item{additionally supports the MSP430 from Texas
            Instruments (r5);}
      \item{\tty{LISTING} now knows the additional variants
            \tty{NOSKIPPED} and \tty{PURECODE} to remove code that
            was not assembled from the listing (r5);}
      \item{additionally supports the 78K0 family from NEC (r5);}
      \item{\tty{BIGENDIAN} is now also available in PowerPC mode
            (r5);}
      \item{additional \tty{BINCLUDE} instruction to include binary
            files (r5);}
      \item{additional \tty{TOLOWER} and \tty{LOWSTRING} functions to convert
            characters to lower case (r5);}
      \item{it is now possible to store data in other segments
            than \tty{CODE}.  The file format has been extended
            appropriately (r5);}
      \item{the \tty{DS} instruction to reserve memory areas is now
            also available in Intel mode (r5);}
      \item{the \tty{U} command line switch now allows to switch AS
            into a case sensitive mode that differentiates
            between upper and lower case in the names of symbols,
            user-defined functions, macros, macro parameters, and
            sections (r5);}
      \item{\tty{SFRB} now also knows the mapping rules for bit
            addresses in the RAM areas; warnings are generated
            for addresses that are not bit addressable (r5);}
      \item{additional instructions \tty{PUSHV} and \tty{POPV} to save symbol
            values temporarily (r5);}
      \item{additional functions \tty{BITCNT, FIRSTBIT, LASTBIT}, and
            \tty{BITPOS} for bit processing (r5);}
      \item{the 68360 is now also known as a member of the CPU32
            processors (r5);}
      \item{additionally supports the ST9 family from SGS-Thomson
            (r6);}
      \item{additionally supports the SC/MP from National
            Semiconductor (r6);}
      \item{additionally supports the TMS70Cxx family from Texas
            Instruments (r6);}
      \item{additionally supports the TMS9900 family from Texas
            Instruments (r6);}
      \item{additionally knows the 80296's instruction set
            extensions (r6);}
      \item{the supported number of Z8 derivatives has been
            extended (r6);}
      \item{additionally knows the 80C504's mask defects (r6);}
      \item{additional register definition file for Siemens' C50x
            processors (r6);}
      \item{additionally supports the ST7 family from SGS-Thomson
            (r6);}
      \item{the Tntel pseudo instructions for data disposal are
            now also valid for the 65816/MELPS-7700 (r6);}
      \item{for the 65816/MELPS-7700, the address length may now
            be set explicitly via prefixes (r6);}
      \item{additionally supports the 8X30x family from Signetics
            (r6);}
      \item{from now on, \tty{PADDING} is enabled by default only
            for the 680x0 family (r7);}
      \item{the new predefined symbol \tty{ARCHITECTURE} can now be
            used to query the platform AS was compiled for (r7);}
      \item{additional statements \tty{STRUCT} and \tty{ENDSTRUCT}
            to define data structures (r7);}
      \item{hex and object files for the AVR tools may now be generated
            directly (r7);}
      \item{\tty{MOVEC} now also knows the 68040's control registers
            (r7);}
      \item{additional \tty{STRLEN} function to calculate the length
            of a string (r7);}
      \item{additional ability to define register symbols (r7 currently
            only Atmel AVR);}
      \item{additionally knows the 6502's undocumented instructions (r7);}
      \item{P2HEX and P2BIN now optionally can erase the input files
            automatically (r7);}
      \item{P2BIN can additionally prepend the entry address to the
            resulting image (r7);}
      \item{additionally supports the ColdFire family from Motorola as a
            variation of the 680x0 core (r7);}
      \item{\tty{BYT/FCB, ADR/FDB}, and \tty{FCC} now also allow the
            repetition factor known from DC (r7);}
      \item{additionally supports Motorola's M*Core (r7);}
      \item{the SH7000 part now also knows the SH7700's
            extensions (r7);}
      \item{the 680x0 part now also knows the 68040's additional
            instructions (r7);}
      \item{the 56K part now also knows the instruction set extensions
            up to the 56300 (r7).}
      \item{the new \tty{CODEPAGE} statement now allows to keep several
            character sets in parallel (r8);}
      \item{The argument variations for \tty{CHARSET} have been extended
            (r8);}
      \item{New string functions \tty{SUBSTR} and \tty{STRSTR} (r8);}
      \item{additional \tty{IRPC} statement in the macro processor (r8);}
      \item{additional \tty{RADIX} statement to set the default numbering
            system for integer constants (r8);}
      \item{instead of {\tt ELSEIF}, it is now valid to simply write {\tt
            ELSE} (r8);}
      \item{$==$ may be used as equality operator instead of $=$ (r8);}
      \item{\tty{BRANCHEXT} for the Philips XA now allows to automatically
            extend the reach of short branches (r8);}
      \item{debug output is now also possible in NoICE format (r8);}
      \item{additionally supports the i960 family from Intel (r8);}
      \item{additionally supports the $\mu$PD7720/7725 signal processors
            from NEC (r8);}
      \item{additionally supports the $\mu$PD77230 signal processor from
            NEC (r8);}
      \item{additionally supports the SYM53C8xx SCSI processors from
            Symbios Logic (r8);}
      \item{additionally supports the 4004 from Intel (r8);}
      \item{additionally supports the SC14xxx series of National (r8);}
      \item{additionally supports the instruction extensions of the PPC
            403GC (r8);}
      \item{additional command line option {\tt cpu} to set the default
            target processor (r8);}
      \item{key files now also may be referenced from the command line
            (r8);}
      \item{additional command line option {\tt shareout} to set the
            output file for SHARED definitions (r8);}
      \item{new statement {\tt WRAPMODE} to support AVR processors with
            a shortened program counter (r8);}
      \item{additionally supports the C20x instruction subset in the C5x
            part (r8);}
      \item{hexadecimal address specifications for the tools now may also
            be made in C notation (r8);}
      \item{the numbering system for integer results in \verb!\{...}!
            expressions is now configurable via \tty{OUTRADIX} (r8);}
      \item{the register syntax for 4004 register pairs has been corrected
            (r8);}
      \item{additionally supports the F$^{2}$MC8L family from Fujitsu
            (r8);}
      \item{P2HEX now allows to set the minimum address length for S
            record addresses (r8);}
      \item{additionally supports the ACE family from Fairchild (r8);}
      \item{{\tt REG} is now also allowed for PowerPCs (r8);}
      \item{additional switch in P2HEX to relocate all addresses (r8);}
      \item{The switch \tty{x} now additionally allows a second level
            of detailness to print the source line in question (r8).}
      \end{itemize}}
\item{version 1.42:
      \begin{itemize}
      \item{the default integer syntax for Atmel AVR is now the C Syntax;}
      \item{additional command line option {\tt olist} to set the
            list file's name and location;}
      \item{additionally supports the F$^{2}$MC16L family from Fujitsu;}
      \item{additional instruction {\tt PACKING} for the AVR family;}
      \item{additional implicit macro parameters {\tt ALLARGS} and
            {\tt ARGCOUNT};}
      \item{additional instruction {\tt SHIFT} to process variable macro
            argument lists;}
      \item{support for temporary symbols;}
      \item{additional instruction {\tt MAXNEST} to set the maximum
            nesting depth of macro expansions;}
      \item{additional command line argument {\tt noicemask} to control
            the amount of segments listed in a NoICE debug info file;}
      \item{additionally supports the 180x family from Intersil;}
      \item{additionally supports the 68HC11K4 address windowing;}
      \item{P2HEX now allows to vary the address field length of AVR HEX
            files;}
      \item{the new command line option {\tt -gnuerrors} allows to output
            error messages in a GNU C-style format;}
      \item{additionally supports the TMS320C54x family from Texas
            Instruments;}
      \item{new macro option {\tt INTLABEL};}
      \item{added Atmel MegaAVR 8/16 instructions and register
            definitions;}
      \item{{\tt ENDIF/ENDCASE} show the line number of the corresponding
            opening statement in the listing;}
      \item{the 8051 part now also supports the extended address space of
            the Dallas DS80C390;}
      \item{added nameless temporary smbols;}
      \item{additionally supports the undocumented 8085 instructions;}
      \item{improved structure handling;}
      \item{added EXPRTYPE() function;}
      \item{allow line continuation;}
      \item{integrated support for KCPSM/PicoBlaze provided by Andreas
            Wassatsch;}
      \item{additionally supports the 807x family from National
            Semiconductor;}
      \item{additionally supports the Intel 4040;}
      \item{additionally supports the Zilog eZ8;}
      \item{additionally supports the 78K2 family from NEC;}
      \item{additionally supports the KCPSM3 variant from Xilinx;}
      \item{additionally supports the LatticeMico8;}
      \item{additionally supports the 12X instruction extensions and the
            XGATE core of the 68HC12 family;}
      \item{additionally supports the Signetics 2650;}
      \item{additionally supports the COP4 family from National
            Semiconductor;}
      \item{additionally supports the HCS08 extensions by Freesacle;}
      \item{additionally supports the RS08 family by Freescale;}
      \item{additionally supports the Intel 8008;}
      \item{add another optional syntax for integer constants;}
      \item{added function \tty{CHARFROMSTR};}
      \item{additionally allow Q for octal constants in Intel mode;}
      \item{add another variant for temporary symbols;}
      \item{the PowerPC part has been extended by the MPC821
            (contribution by Marcin Cieslak);}
      \item{implicit macro parameters are always case-insensitive;}
      \item{add \tty{REG} statement to MSP430;}
      \item{additionally supports the XMOS XS1;}
      \item{additional parameters \tty{GLOBALSYMBOLS} and
            \tty{NOGLOBALSYMBOLS} to control whether labels in
            macros are local or not;}
      \item{additionally supports the NEC 75xx series;}
      \item{additionally supports the TMS1000 controllers from
            TI;}
      \item{additionally supports the 78K2 family from NEC;}
      \item{all newer changes are only documented in the separate
            changelog file.}
      \end{itemize}}
\end{itemize}

%%===========================================================================

\cleardoublepage
\chapter{Hints for the AS Source Code}
\label{ChapSource}

As I already mentioned in the introduction, I release the source code of
AS on request.  The following shall give a few hints to their usage.

%%---------------------------------------------------------------------------

\section{Language Preliminaries}

In the beginning, AS was a program written in Turbo-Pascal.  This was
roughly at the end of the eighties, and there were a couple of reasons for
this choice: First, I was much more used to it than to any C compiler, and
compared to Turbo Pascal's IDE, all DOS-based C compilers were just
crawling along.  In the beginning of 1997 however, it became clear that
things had changed: One factor was that Borland had decided to let its
confident DOS developers down (once again, explicitly no 'thank you', you
boneheads from Borland!) and replaced version 7.0 of Borland Pascal with
something called 'Delphi', which is probably a wonderful tool to develop
Windows programs which consist of 90\% user interface and accidentaly a
little bit of content, however completely useless for command-line driven
programs like AS.  Furthermore, my focus of operating systems had made a
clear move towards Unix, and I probably could have waited arbitrarily long
for a Borland Pascal for Linux (to all those remarking now that Borland
would be working on something like that: this is {\em Vapourware}, don\'t
believe them anything until you can go into a shop and actually buy it!).
It was therefore clear that C was the way to go.

After this eperience what results the usage of 'island systems' may have,
I put a big emphasize on portability while doing the translation to C;
however, since AS for example deals with binary data in an exactly format
and uses operating systen-specific functions at some places which may need
adaptions when one compliles AS the first time for a new platform.

AS is tailored for a C compiler that conforms to the ANSI C standard; C++
is explicitly not required.  If you are still using a compiler conforming
to the outdated Kernighan\&Ritchie standard, you should consider getting a
newer compiler: The ANSI C standard has been fixed in 1989 and there
should be an ANSI C compiler for every contemporary platform, maybe by
using the old compiler to build GNU-C.  Though there are some switches in
the source code to bring it nearer to K\&R, this is not an officially
supported feature which I only use internally to support a quite antique
Unix.  Everything left to say about K\&R is located in the file {\tt
README.KR}.

The inclusion of some additional features not present in the Pascal
version (e.g. dynamically loadable message files, test suite, automatic
generation of the documentation from {\em one} source format) has made the
source tree substantially more complicated.  I will attempt to unwire
everything step by step:

%%---------------------------------------------------------------------------

\section{Capsuling System dependencies}

As I already mentioned, As has been tailored to provide maximum platform
independence and portability (at least I believe so...).  This means
packing all platform dependencies into as few files as possible.  I will
describe these files now, and this section is the first one because it is
probably one of the most important:

The Build of all components of AS takes place via a central {\tt
Makefile}.  To make it work, it has to be accompanied by a fitting {\tt
Makefile.def} that gives the platform dependent settings like compiler
flags.  The subdirectory {\tt Makefile.def-samples} contains a couple of
includes that work for widespread platforms (but which need not be
optimal...).  In case your platform is not among them, you may take the
file {\tt Makefile.def.tmpl} as a starting point (and send me the
result!).

A further component to capure system dependencies is the file {\tt
sysdefs.h}.  Practically all compilers predefine a couple of preprocessor
symbols that describe the target processor and the used operating system.
For example, on a Sun Sparc under Solaris equipped with the GNU compiler,
the symbols \verb!__sparc! and \verb!__SVR4!.  {\tt sysdefs.h} exploits
these symbols to provide a homogeneous environment for the remaining,
system-independent files.  Especially, this covers integer datatypes of a
specific length, but it may also include the (re)definition of C functions
which are not present or non-standard-like on a specific platform.  It's
best to read this files yourself if you like to know which things may
occur...  Generally, the \verb!#ifdef! statement are ordered in two
levels: First, a specific processor platform is selected, the the
operating systems are sorted out in such a section.

If you port AS to a new platform, you have to find two symbols typical for
this platform and extend {\tt sysdefs.h} accordingly.  Once again, I'm
interested in the result...

%%---------------------------------------------------------------------------

\section{System-Independent Files}

...represent the largest part of all modules.  Describing all functions in
detail is beyond the scope of this description (those who want to know
more probably start studying the sources, my programming style isn't that
horrible either...), which is why I can only give a short list at this
place with all modules their function:

\subsection{Modules Used by AS}

\subsubsection{as.c}

This file is AS's root: it contains the {\em main()} function of AS, the
processing of all command line options, the overall control of all passes
and parts of the macro processor.

\subsubsection{asmallg.c}

This module processes all statements defined for all processor targets,
e.g. \tty{EQU} and \tty{ORG}.  The \tty{CPU} pseudo-op used to switch
among different processor targets is also located here.

\subsubsection{asmcode.c}

This module contains the bookkeping needed for the code output file.  It
exports an interface that allows to open and close a code file and offers
functions to write code to (or take it back from) the file.  An important
job of this module is to buffer the write process, which speeds up
execution by writing the code in larger blocks.

\subsubsection{asmdebug.c}

AS can optionally generate debug information for other tools like
simulators or debuggers, allowing a backward reference to the source code.
They get collected in this module and can be output after assembly in one
of several formats.

\subsubsection{asmdef.c}

This modules only contains declarations of constants used in different
places and global variables.

\subsubsection{asmfnums.c}

AS assigns internally assigns incrementing numbers for each used source
file.  These numbers are used for quick referencing.  Assignment of
numbers and the conversion between names and numbers takes place here.

\subsubsection{asmif.c}

Here ara ll routines located controlling conditional assembly.  The most
important exported variable is a flag called \tty{IfAsm} which controls
whether code generation is currently turned on or off.

\subsubsection{asminclist.c}

This module holds the definition of the list stucture that allows AS to
print the nesting of include files to the assembly list file.

\subsubsection{asmitree.c}

When searching for the mnemonic used in a line of code, a simple linear
comparison with all available machine instructions (as it is still done in
most code generators, for reasons of simplicity and laziness) is not
necessary the most effective method.  This module defines two improved
structures (binary tree and hash table) which provide a more efficient
search and are destined to replace the simple linear search on a
step-by-step basis...priorities as needed...

\subsubsection{asmmac.c}

Routines to store and execute macro constructs are located in this module.
The real macro processor is (as already mentioned) in {\tt as.c}.

\subsubsection{asmpars.c}

Here we really go into the innards: This module stores the symbol tables
(global and local) in two binary trees.  Further more, there is a quite
large procedure {\tt EvalExpression} which analyzes and evaluates a (formula)
expression.  The procedure returns the result (integer, floating point, or
string) in a varaint record.  However, to evaluate expressions during code
generation, one should better use the functions  \tty{EvalIntExpression,
EvalFloatExpression}, and \tty{EvalStringExpression}.  Modifications for
tha esake of adding new target processors are unnecessary in this modules
and should be done with extreme care, since you are touching something
like 'AS's roots'.

\subsubsection{asmsub.c}

This module collects a couple of commonly used subroutines which primarily
deal with error handling and 'advanced' string processing.

\subsubsection{bpemu.c}

As already mentioned at the beginning, AS originally was a program written
in Borland Pascal.  For some intrinsic functions of the compiler, it was
simpler to emulate those than to touch all places in the source code where
they are used.  Well...

\subsubsection{chunks.c}

This module defines a data type to deal with a list of address ranges.
This functionality is needed by AS for allocation lists; furthermore,
P2BIN and P2HEX use such lists to warn about overlaps.

\subsubsection{cmdarg.c}

This module implements the overall mechanism of command line arguments.
It needs a specification of allowed arguments, splits the command line and
triggers the appropriate callbacks.   In detail, the mechanism includes
the following:
\begin{itemize}
\item{Processing of arguments located in an environment variable or
      a corresponding file;}
\item{Return of a set describing which command line arguments have not
      been processed;}
\item{A backdoor for situations when an overlaying IDE converts the passed
      command line completely into upper or lower case.}
\end{itemize}

\subsubsection{codepseudo.c}

You will find at this place pseudo instructions that are used by
a subset of code generators.  On the one hand, this is the Intel group of
\tty{DB..DT}, and on the other hand their counterparts for 8/16 bit CPUs
from Motorola or Rockwell.  Someone who wants to extend AS by a
processor fitting into one of these groups can get the biggest part
of the necessary pseudo instructions with one call to this module.

\subsubsection{codevars.c}

For reasons of memory efficiency, some variables commonly used by diverse
code generators.

\subsubsection{endian.c}

Yet another bit of machine dependence, however one you do not have to
spend attention on: This module automatically checks at startup whether
a host machine is little or big endian.  Furthermore, checks are made if
the type definitions made for integer variables in {\tt sysdefs.h} really
result in the correct lengths.

\subsubsection{headids.c}

At this place, all processor families supported by AS are collected with
their header IDs (see chapter \ref{SectCodeFormat}) and the output format
to be used by default by P2HEX.  The target of this table is to centralize
the addition of a new processor as most as possible, i.e. in contrast to
earlier versions of AS, no further modifications of tool sources are
necessary.

\subsubsection{ioerrs.c}

The conversion from error numbers to clear text messages is located here.
I hope I'll never hit a system that does not define the numbers as macros,
because I would have to rewrite this module completely...

\subsubsection{nlmessages.c}

The C version of AS reads all messages from files at runtime after the
language to be used is clear.  The format of message files is not a simple
one, but instead a special compact and preindexed format that is generated
at runtime by a program called 'rescomp' (we will talk about it later).
This module is the counterpart to rescomp that reads the correct language
part into a character field and offers functions to access the messages.

\subsubsection{nls.c}

This module checks which country-dependent settings (date and time format,
country code) are present at runtime.  Unfortunately, this is a highly
operating system-dependend task, and currently, there are only three
methods defines: The MS-DOS method, the OS/2 method and the typical Unix
method via {\em locale} functions.  For all other systems, there is
unfortunately currently only \verb!NO_NLS! available...

\subsubsection{stdhandl.c}

On the one hand, here is a special open function located knowing the
special strings {\tt !0...!2} as file names and creating duplicates of the
standard file handles {\em stdin, stdout,} and {\em stderr}.  On the other
hand, investiagations are done whether the standard output has been
redirected to a device or a file.  On no-Unix systems, this unfortunately
also incorporates some special operations.

\subsubsection{stringlists.c}

This is just a little 'hack' that defines routines to deal with linear
lists of strings, which are needed e.g. in the macro processor of AS.

\subsubsection{strutil.c}

Some commonly needed string operations have found their home here.

\subsubsection{version.c}

The currently valid version is centrally stored here for AS and all other
tools.

\subsubsection{code????.c}

These modules form the main part of AS: each module contains the code
generator for a specific processor family.

\subsection{Additional Modules for the Tools}

\subsubsection{hex.c}

A small module to convert integer numbers to hexadecimal strings.  It's
not absolutely needed in C any more (except for the conversion of {\em
long long} variables, which unfortunately not all {\tt printf()}'s
support), but it somehow survived the porting from Pascal to C.

\subsubsection{p2bin.c}

The sources of P2BIN.

\subsubsection{p2hex.c}

The sources of P2HEX.

\subsubsection{pbind.c}

The sources of BIND.

\subsubsection{plist.c}

The sources of PLIST.

\subsubsection{toolutils.c}

All subroutines needed by several tools are collected here, e.g. for
reading of code files.

\section{Modules Needed During the Build of AS}

\subsubsection{a2k.c}

This is a minimal filter converting ANSI C source files to
Kernighan-Ritchie style.  To be exact: only function heads are converted,
even this only when they are roughly formatted like my programming style.
Noone should therefore think this were a universal C parser!

\subsubsection{addcr.c}

A small filter needed during installation on DOS- or OS/2-systems.  Since
DOS and OS/2 use a CR/LF for a newline, inc ontrast to the single LF of
Unix systems, all assembly include files provided with AS are sent through
this filter during assembly.

\subsubsection{bincmp.c}

For DOS and OS/2, this module takes the task of the {\em cmp} command,
i.e. the binary comparison of files during the test run.  While this would
principally be possible with the {\em comp} command provided with the OS,
{\em bincmp} does not have any nasty interactive questions (which seem to
be an adventure to get rid of...)

\subsubsection{findhyphen.c}

This is the submodule in {\em tex2doc} providing hyphenation of words.
The algorithm used for this is shamelessly stolen from TeX.

\subsubsection{grhyph.c}

The definition of hyphenation rules for the german language.

\subsubsection{rescomp.c}

This is AS's 'resource compiler', i.e. the tool that converts a readable
file with string resources into a fast, indexed format.

\subsubsection{tex2doc.c}

A tool that converts the LaTeX documentation of AS into an ASCII format.

\subsubsection{tex2html.c}

A tool that converts the LaTeX documentation of AS into an HTML document.

\subsubsection{umlaut.c and unumlaut.c}

These tiny programs convert national special characters between their
coding in ISO8859-1 (all AS files use this format upon delivery) and their
system-specific coding.  Apart from a plain ASCII7 variant, there are
currently the IBM character sets 437 and 850.

\subsubsection{ushyph.c}

The definition of hyphenation rules for the english language.

%%---------------------------------------------------------------------------

\section{Generation of Message Files}

As already mentioned, the C source tree of AS uses a dynamic load
principle for all (error) messages.  In contrast to the Pasacl sources
where all messages were bundled in an include file and compiled into the
programs, this method eliminates the need to provide AS in multiple
language variants; there is only one version which checks for the
langugage to be used upon runtime and loads the corresponding component
from the message files.  Just to remind: Under DOS and OS/2, the {\tt
COUNTRY} setting is queried, while under Unix, the environment variables
{\tt LC\_MESSAGES, LC\_ALL,} and {\tt LANG} are checked.

\subsection{Format of the Source Files}

A source file for the message compiler {\em rescomp} usually has the
suffix {\tt .res}.  The message compiler generates one or two files from a
source:
\begin{itemize}
\item{a binary file which is read at runtime by AS resp. its tools}
\item{optionally one further C header file assigning an index number to
      all messages. These index numbers in combination with an index
      table in the binary file allow a fast access to to individual
      messages at runtime.}
\end{itemize}

The source file for the message compiler is a pure ASCII file and can
therefore be modified with any editor.  It consists of a sequence of
control commands with parameters.  Empty lines and lines beginning with a
semicolon are ignored.  Inclusion of other files is possible via the {\tt
Include} statement:
\begin{verbatim}
Include <Datei>
\end{verbatim}

The first two statements in every source file must be two statements
describing the languages defined in the following.  The more important one
is {\tt Langs}, e.g.:
\begin{verbatim}
Langs DE(049) EN(001,061)
\end{verbatim}
describes that two languages will be defined in the rest of the file.  The
first one shall be used under Unix when the language has been set to {\tt
DE} via environment variable.  Similarly, It shall be used under DOS and
OS/2 when the country code was set to 049.  Similarly, the second set
shall be used for the settings {\tt DE} resp. 061 or 001.  While multiple
'telephone numbers' may point to a single language, the assignment to a
Unix language code is a one-to-one correspondence.  This is no problem in
practice since the {\tt LANG} variables Unix uses describe subversions via
appendices, e.g.:
\begin{verbatim}
de.de
de.ch
en.us
\end{verbatim}
AS only compares the beginning of the strings and therefore still comes to
the right decision.
The {\tt Default} statement defines the language that shall be used if
either no language has been set at all or a language is used that is not
mentioned in the asrgument list of {\tt Langs}.  This is typically the
english language:
\begin{verbatim}
Default EN
\end{verbatim}
These definitions are followed by an arbitrary number of {\tt Message}
statements, i.e. definitions of messages:
\begin{verbatim}
Message ErrName
 ": Fehler "
 ": error "
\end{verbatim}
In case {\em n} languages were announced via the {\tt Langs} statement,
the message compiler takes {\bf exactly} the following {\em n} as the
strings to be stored.  It is therefore impossible to leave out certain
languages for individual messages, and an empty line following the strings
should in no way be misunderstood as an end marker for the list; inserted
lines between statements only serve purposes of better readability.  It is
however allowed to split individual messages across multiple lines in the
source file; all lines except for the last one have to be ended with a
backslash as continuation character:
\begin{verbatim}
Message TestMessage2
 "Dies ist eine" \
 "zweizeilige Nachricht"
 "This is a" \
 "two-line message"
\end{verbatim}
As already mentioned, source files are pure ASCII files; national special
characters may be placed in message texts (and the compiler will correctly
pass them to the resulting file), a big disadvantage however is that such
a file is not fully portable any more: in case it is ported to another
system using a different coding for national special characters, the user
will probably be confronted with funny characters at runtime...special
characters should therefore always be written via special sequences
borrowed from HTML resp. SGML (see table \ref{TabSpecChars}).  Linefeeds
can be inserted into a line via \verb!\n!, similar to C.
\begin{table*}[htb]
\begin{center}\begin{tabular}{|l|l|}
\hline
Sequence... & results in... \\
\hline
\hline
\verb!&auml; &ouml; &uuml;! & "a "o "u (Umlauts)\\
\verb!&Auml; &Ouml; &Uuml;! & "A "O "U \\
\verb!&szlig;!              & "s (sharp s) \\
\verb!&agrave; &egrave; &igrave; &ograve;! & \'a \'e \'i \'o \\
\verb!&ugrave;! & \'u \\
\verb!&Agrave; &Egrave; &Igrave; &Ograve;! & \'A \'E \'I \'O \\
\verb!&Ugrave;! & \'U (Accent grave) \\
\verb!&aacute; &eacute; &iacute; &oacute;! & \`a \`e \`i \`o \\
\verb!&uacute;! & \`u \\
\verb!&Aacute; &Eacute; &Iacute; &Oacute;! & \`A \`E \`I \`O \\
\verb!&Uacute;! & \`U (Accent agiu) \\
\verb!&acirc; &ecirc; &icirc; &ocirc;! & \^a \^e \^i \^o \\
\verb!&ucirc;! & \^u \\
\verb!&Acirc; &Ecirc; &Icirc; &Ocirc;! & \^A \^E \^I \^O \\
\verb!&Ucirc;! & \^U (Accent circonflex) \\
\verb!&ccedil; &Ccedil;! & \c{c} \c{C}(Cedilla) \\
\verb!&ntilde; &Ntilde;! & \~n \~N \\
\verb!&aring; &Aring;! & \aa  \AA \\
\verb!&aelig; &Aelig;! & \ae  \AE \\
\verb!&iquest; &iexcl;! & inverted ! or ? \\
\hline
\end{tabular}\end{center}
\caption{Syntax for special character in {\em rescomp}\label{TabSpecChars}}
\end{table*}

%%---------------------------------------------------------------------------

\section{Creation of Documentation}

A source distribution of AS contains this documentation in LaTeX format
only.  Other formats are created from this one automatically  via tools
provided with AS.  One reason is to reduce the size of the source
distribution, another reason is that changes in the documentation only
have to be made once, avoiding inconsistencies.

LaTex was chosen as the master format because...because...because it's
been there all the time before.  Additionally, TeX is almost arbitrarily
portable and fits quite well to the demands of AS.  A standard
distribution therefore allows a nice printout on about any printer; for a
conversion to an ASCII file that used to be part of earlier distributions,
the converter {\em tex2doc} is included, additionally the converter {\em
tex2html} allowing to put the manual into the Web.

Generation of the documentation is started via a simple
\begin{verbatim}
make docs
\end{verbatim}
The two converters mentioned are be built first, then applied to the TeX
documentation and finally, LaTeX itself is called.  All this of course for
all languages...

%%---------------------------------------------------------------------------

\section{Test Suite}

Since AS deals with binary data of a precisely defined structure, it is
naturally sensitive for system and compiler dependencies.  To reach at
least a minimum amount of secureness that everything went right during
compilation, a set of test sources is provided in the subdirectory {\tt
tests} that allows to test the freshly built assembler.  These programs
are primarily trimmed to find faults in the translation of the machine
instruction set, which are commonplace when integer lenghts vary.
Target-independent features like the macro processors or conditional
assembly are only casually tested, since I assume that they work
everywhere when they work for me...

The test run is started via a simple {\em make test}.  Each test program
is assembled, converted to a binary file, and compared to a reference
image.  A test is considered to be passed if and only if the reference
image and the newly generated one are identical on a bit-by-bit basis.  At
the end of the test, the assembly time for every test is printed (those
who want may extend the file BENCHES with these results), accompanied with
a success or failure message.  Track down every error that occurs, even if
it occurs in a processor target you are never going to use!  It is always
possible that this points to an error that may also come up for other
targets, but by coincidence not in the test cases.

%%---------------------------------------------------------------------------

\section{Adding a New Target Processor}

The probably most common reason to modify the source code of AS is to add
a new target processor.   Apart from adding the new module to the
Makefile, there are few places in other modules that need a modification.
The new module will do the rest by registering itself in the list of code
generators.  I will describe the needed steps in a cookbook style in the
following sections:

\subsubsection{Choosing the Processor's Name}

The name chosen for the new processor has to fulfill two criterias:
\begin{enumerate}
\item{The name must not be already in use by another processor.  If one
      starts AS without any parameters, a list of the names already in
      use will be printed.}
\item{If the name shall appear completely in the symbol \tty{MOMCPU}, it may
      not contain other letters than A..F (except right at the
      beginning).  The variable \tty{MOMCPUNAME} however will always report
      the full name during assembly.  Special characters are generally
      disallowed, lowercase letters will be converted by the \tty{CPU}
      command to uppercase letters and are therefore senseless in the
      processor name.}
\end{enumerate}

The first step for registration is making an entry for the new processor
(family) in the file {\tt headids.c}.  As already mentioned, this file is
also used by the tools and specifies the code ID assigned to a processor
family, along with the default hex file format to be used.  I would like
to have some coordination before choosing the ID...

\subsubsection{Definition of the Code Generator Module}

The unit's name that shall be responsible for the new processor
should bear at least some similarity to the processor's name (just
for the sake of uniformity) and should be named in the style of
\tty{code....}.  The head with include statements is best taken from
another existing code generator.

Except for an initialization function that has to be called at the
begginning of the {\tt main()} function in module {\tt as.c}, the new
module neither has to export variables nor functions as the complete
communication is done at runtime via indirect calls.  They are simply done
by a call to the function
\tty{AddCPU} for each processor type that shall be treated by this unit:
\begin{verbatim}
   CPUxxxx:=AddCPU('XXXX',SwitchTo_xxxx);
\end{verbatim}
\tty{'XXXX'} is the name chosen for the processor which later must be used
in assembler programs to switch AS to this target processor.
\tty{SwitchTo\_xxxx} (abbreviated as the ''switcher'' in the following) is
a procedure without parameters that is called by AS when the switch to the
new processor actually takes place.  \tty{AddCPU} delivers an integer
value as result that serves as an internal ''handle'' for the new
processor.  The global variable \tty{MomCPU} always contains the handle of
the target processor that is currently set.  The value returned by
\tty{AddCPU} should be stored in a private variable of type \tty{CPUVar}
(called \tty{CPUxxxx} in the example above).  In case a code generator
module implements more than one processor (e.g. several processors of a
family), the module can find out which instruction subset is currently
allowed by comparing \tty{MomCPU} against the stored handles.

The switcher's task is to ''reorganize'' AS for the new target
processor.  This is done by changing the values of several global
variables:
\begin{itemize}
\item{\tty{ValidSegs}: Not all processors have all address spaces defined
      by AS.  This set defines which subset the \tty{SEGMENT} instruction
      will enable for the currently active target processor.  At least the
      \tty{CODE} segment has to be enabled.  The complete set of allowed
      segments can be looked up the file \tty{fileformat.h} (\tty{Seg....}
      constants).}
\item{\tty{SegInits}: This array stores the initial program counter values
      for the individual segments (i.e. the values the program counters
      will initially take when there is no \tty{ORG} statement).  There are
      only a few exceptions (like logically separated address spaces
      that physically overlap) which justify other initial values than
      0.}
\item{\tty{Grans}: This array specifies the size of the smallest addressable
      element in bytes for each segment, i.e. the size of an element
      that increases an address by 1.  Most processors need a value of
      1, even if they are 16- or 32-bit processors, but the PICs and
      signal processors are cases where higher values are required.}
\item{\tty{ListGrans}: This array specifies the size of byte groups that shall
      be shown in the assembly listing.  For example, instruction words
      of the 68000 are always 2 bytes long though the code segment's
      granularity is 1.  The \tty{ListGran} entry therefore has to be set to
      2.}
\item{\tty{SegLimits}: This array stores the highest possible address for
      each segment, e.g. 65535 for a 16-bit address space.  This array
      need not be filled in case the code generator takes over the
      {\tt ChkPC} method.}
\item{\tty{ConstMode}: This variable may take the values
      \tty{ConstModeIntel}, \tty{ConstModeMoto}, or \tty{ConstModeC}
      and rules which syntax has to be used to specify the base of
      integer constants.}
\item{\tty{PCSymbol}: This variable contains the string an assembler program
      may use to to get the current value of the program counter.
      Intel processors for example usually use a dollar sign.}
\item{\tty{TurnWords}: If the target processor uses big-endian addressing and
      one of the fields in \tty{ListGran} is larger than one, set this flag
      to true to get the correct byte order in the code output file.}
\item{\tty{SetIsOccupied}: Some processors have a \tty{SET} machine instruction.
      If this callback is set to a non-NULL value, the code generator
      may report back whether \tty{SET} shall not be interpreted as
      pseudo instruction.  The return value may be constant \tty{True} or
      or e.g. depend on the number of argument if a differentiation is
      possible.}
\item{\tty{HeaderID}: This variable contains the ID that is used to mark the
      current processor family in the the code output file (see the
      description of the code format described by AS).  I urge to
      contact me before selecting the value to avoid ambiguities.
      Values outside the range of \$01..\$7f should be avoided as they
      are reserved for special purposes (like a future extension to
      allow linkable code). Even though this value is still hard-coded
      in most code generators, the preferred method is now to fetch this
      value from {\tt headids.h} via {\tt FindFamilyByName}.}
\item{\tty{NOPCode}: There are some situations where AS has to fill unused
      code areas with NOP statements.  This variable contains the
      machine code of the NOP statement.}
\item{\tty{DivideChars}: This string contains the characters that are valid
      separation characters for instruction parameters.  Only extreme
      exotics like the DSP56 require something else than a single comma
      in this string.}
\item{\tty{HasAttrs}: Some processors like the 68k series additionally split
      an instruction into mnemonic and attribute.  If the new processor
      also does something like that, set this flag to true and AS will
      deliver the instructions' components readily split in the string
      variables \tty{OpPart} and \tty{AttrPart}.  If this flag is however set to
      false, no splitting will take place and the instruction will be
      delivered as a single piece in \tty{OpPart}.  \tty{AttrPart} will stay empty
      in this case.  One really should set this flag to false if the
      target processor does not have attributes as one otherwise looses
      the opportunity to use macros with a name containing dots (e.g.
      to emulate other assemblers).}
\item{\tty{AttrChars}: In case \tty{HasAttrs} is true, this string has to contain
      all characters that can separate mnemonic and attribute.  In most
      cases, this string only contains a single dot.}
\end{itemize}
Do not assume that any of these variables has a predefined value; set
them \bb{all}!!

Apart from these variables, some function pointers have to be set that
form the link form AS to the ''active'' parts of the code
generator:
\begin{itemize}
\item{\tty{MakeCode}: This routine is called after a source line has been
      split into mnemonic and parameters.  The mnemonic is stored into
      the variable \tty{OpPart}, and the parameters can be looked up in the
      array \tty{ArgStr}.  The number of arguments may be read from
      \tty{ArgCnt}.
      The binary code has to be stored into the array \tty{BAsmCode}, its
      length into \tty{CodeLen}.  In case the processor is word oriented
      like the 68000 (i.e. the \tty{ListGran} element corresponding to the
      currently active segment is 2), the field may be addressed
      wordwise via \tty{WAsmCode}.  There is also \tty{DAsmCode} for extreme
      cases... The code length has to be given in units corresponding
      to the current segment's granularity.}
\item{\tty{SwitchFrom}: This parameter-less procedure enables the code generator
      module to do ''cleanups'' when AS switches to another target processor.
      This hook allows e.g. to free memory that has been allocated in the
      generator and that is not needed as long as the generator is not
      active.  It may point to an empty procedure in the simplest case.
      One example for the usage of this hook is the module \tty{CODE370} that
      builds its instruction tables dynamically and frees them again after
      usage.}
\item{\tty{IsDef}: Some processors know additional instructions that impose
      a special meaning on a label in the first row like \tty{EQU} does.  One
      example is the \tty{BIT} instruction found in an 8051 environment.  This
      function has to return TRUE if such a special instruction is
      present.  In the simplest case (no such instructions), the routine
      may return a constant FALSE.}
\end{itemize}

Optionally, the code generator may additionally set the following function
pointers:
\begin{itemize}
\item{\tty{ChkPC} : Though AS internally treats all program counters as
      either 32 or 64 bits, most processors use an address space that is
      much smaller.  This function informs AS whether the current program
      counter has exceeded its allowed range.  This routine may of course
      be much more complicated in case the target processor has more than
      one address space.  One example is in module \tty{code16c8x.c}.  In
      case everything is fine, the function has to return TRUE, otherwise
      FALSE.  The code generator only has to implement this function if
      it did not set up the array {\tt SegLimits}.  This may e.g. become
      necessary when the allowed range of addresses in a segment is
      non-continuous.}
\item{\tty{InternSymbol} : Some processorcs, e.g. such with a register
      bank in their internal RAM, predefine such 'registers' as symbols,
      and it wouldn't make much sense to define them in a separate include
      file with 256 or maybe more {\tt EQU}s.  This hook allows access to
      the code generator of AS: It obtains an expression as an ASCII
      string and sets up the passed structure of type {\em TempResult}
      accordingly when one of these 'built-in' symbols is detected.  The
      element {\tt Typ} has to be set to {\tt TempNone} in case the check
      failed.  Errors messages from this routine should be avoided as
      unidentified names could signify ordinary symbols (the parser will
      check this afterwards).  Be extreme careful with this routine as
      it allows you to intervene into the parser's heart!}
\item{\tty{DissectBit} : In case the target platform supports bit objects,
      i.e. objects that pack both a register or memory address and a bit
      position into one integer number, this is the callback to dissect
      such a packed representation and transform it back into a source-code
      like, human-readable form.  This provides better readability of the
      listing.}
\item{\tty{DissectReg} : In case the target platform supports register
      symbols, this is the callback that translates register number and size
      back to a source-code like, human-readable form.  Again, this function
      is used for the listing.}
\item{\tty{QualifyQuote} : This optional callback allows to define on a
      per-platform base situations when a single quotation character does
      {\em not} lead in a character string.  An example for this is the
      Z80's alternate register bank, which is written as \tty{AF'}, or
      the hexadecimal constant syntax \tty{H'...} used on some Hitachi
      processors.}
\end{itemize}

By the way: People who want to become immortal may add a copyright
string.  This is done by adding a call to the procedure \tty{AddCopyright}
in the module's initialization part (right next to the \tty{AddCPU} calls):
\begin{verbatim}
   AddCopyright(
      "Intel 80986 code generator (C) 2010 Jim Bonehead");
\end{verbatim}
The string passed to \tty{AddCopyright} will be printed upon program start
in addition to the standard message.

If needed, the unit may also use its initialization part to hook into
a list of procedures that are called prior to each pass of assembly.
Such a need for example arises when the module's code generation
depends on certain flags that can be modified via pseudo
instructions.  An example is a processor that can operate in either
user or supervisor mode.  In user mode, some instructions are
disabled.  The flag that tells AS whether the following code executes
in user or supervisor mode might be set via a special pseudo
instruction.  But there must also be an initialization that assures
that all passes start with the same state.  The hook offered via
\tty{AddInitPassProc} offers a chance to do such initializations.  The
callback function passed to it is called before a new pass is
started.

The function chain built up via calls to \tty{AddCleanUpProc}
operates similar to \tty{AddInitPassProc}: It enables code
generators to do clean-ups after assembly (e.g.  freeing of
literal tables).  This makes sense when multiple files are
assembled with a single call of AS.  Otherwise, one would risk to
have 'junk' in tables from the previous run.  No module currently
uses this feature.

\subsubsection{Writing the Code Generator itself}

Now we finally reached the point where your creativity is challenged:
It is up to you how you manage to translate mnemonic and parameters
into a sequence of machine code.  The symbol tables are of course
accessible (via the formula parser) just like everything exported
from \tty{ASMSUB}.  Some general rules (take them as advises and not as
laws...):
\begin{itemize}
\item{Try to split the instruction set into groups of instructions that
      have the same operand syntax and that differ only in a few bits
      of their machine code.  For example, one can do all instructions
      without parameters in a single table this way.}
\item{Most processors have a fixed spectrum of addressing modes.  Place
      the parsing of an address expression in a separate routine so you
      an reuse the code.}
\item{The subroutine \tty{WrError} defines a lot of possible error codes and
      can be easily extended.  Use this!  It is no good to simply issue
      a ''syntax error'' on all error conditions!}
\end{itemize}
Studying other existing code generators should also prove to be
helpful.

\subsubsection{Modifications of Tools}

A microscopic change to the tolls' sources is still necessary, namely to
the routine {\tt Granularity()} in {\tt toolutils.c}: in case one of the
processor's address spaces has a granularity different to 1, the swich
statement in this place has to be adapted accordingly, otherwise PLIST,
P2BIN, and P2HEX start counting wrong...

\section{Localization to a New Language}

You are interested in this topic?  Wonderful!  This is an issue that is
often neglected by other programmers, especially when they come from the
country on the other side of the big lake...

The localization to a new language can be split into two parts: the
adaption of program messages and the translation of the manual.  The
latter one is definitely a work of gigantic size, however, the adaption of
program messages should be a work doable on two or three weekends, given
that one knows both the new and one of the already present messages.
Unfortunately, this translation cannot be done on a step-by-step basis
because the resource compiler currently cannot deal with a variable amount
of languages for different messages, so the slogan is 'all or nothing'.

The first oeration is to add the new language to {\tt header.res}.  The
two-letter-abbreviation used for this language is best fetched from the
nearest Unix system (in case you don't work on one anyway...), the
international telephone prefix from a DOS manual.

When this is complete, one can rebuild all necessary parts with a simple
{\em make} and obtains an assembler that supports one more language.  Do
not forget to forward the results to me.  This way, all users will benefit
from this with the next release :-)

%%===========================================================================

\cleardoublepage
\begin{thebibliography}{99}

\bibitem{Williams} Steve Williams: \\
                   {\em 68030 Assembly Language Reference.\/} \\
		   Addison-Wesley, Reading, Massachusetts, 1989

\bibitem{AMD29K} Advanced Micro Devices: \\
		 {\em AM29240, AM29245, and AM29243 RISC
                  Microcontrollers.\/} \\
		 1993

\bibitem{AtAVR} Atmel Corp.: \\
                {\em AVR Enhanced RISC Microcontroller Data Book.\/} \\ 
                May 1996

\bibitem{AVRObj} Atmel Corp.: \\
                 {\em 8-Bit AVR Assembler and Simulator Object File
                  Formats (Preliminary).\/} \\
                 (part of the AVR tools documentation)

\bibitem{CSG65CE02} Commodore Semiconductor Group: \\
                    {\em 65CE02 Microprocessor Preliminary Data Sheet.}

\bibitem{CMD816} CMD Microcircuits: \\
                {\em G65SC802/G65SC816 CMOS 8/16-Bit Microprocessor.\/} \\
		Family Data Sheet.

\bibitem{CFMAC} Freescale Semiconductor: \\
               {\em Digital Signal Processing Libraries Using the ColdFire
                eMAC and MAC User's Manual.\/}
               DSPLIBUM, Rev. 1.2, 03/2006

\bibitem{COP4} National Semiconductor: \\
               {\em COP410L/COP411L/COP310L/COP311L
               Single-Chip N-Channel Microcontrollers.\/}
               RRD-B30M105, March 1992

\bibitem{COPS} National Semiconductor: \\
               {\em COPS Family User's Guide.\/}                               

\bibitem{CPM68K} Digital Research: \\
                 {\em CP/M 68K Operating System User's Guide.\/} \\
        	 1983

\bibitem{Cyrix} Cyrix Corp.: \\
                {\em FasMath 83D87 User's Manual.\/} \\
		1990

\bibitem{Dallas320} Dallas Semiconductor: \\
		    {\em DS80C320 High-Speed Micro User's Guide.\/} \\
                    Version 1.30, 1/94

\bibitem{Fair1101} Fairchild Semiconductor: \\
                   {\em ACE1101 Data Sheet.\/} \\
                   Preliminary, May 1999

\bibitem{Fair1202} Fairchild Semiconductor: \\
                   {\em ACE1202 Data Sheet.\/} \\
                   Preliminary, May 1999

\bibitem{Fair8004} Fairchild Semiconductor: \\
                   {\em ACEx Guide to Developer Tools.\/}
                   AN-8004, Version 1.3 September 1998

\bibitem{FairF8UG} Fairchild Micro Systems: \\
                   {\em F8 User's Guide.\/}
                   67095665, 02-13-1976                   

\bibitem{FairF8GP} Fairchild Micro Systems: \\
                   {\em F8 Guide to Programming\/}
                   67095664, 1976

\bibitem{Free12X} Freescale Semiconductor: \\
                  {\em S12XCPUV1 Reference Manual.\/}
                  S12XCPUV1, v01.01, 03/2005

\bibitem{FreeRS08} Freescale Semiconductor: \\
                  {\em RS08 Core Reference Manual.\/}
                  RS08RM, Rev. 1.0, 04/2006     

\bibitem{FreeXGATE} Freescale Semiconductor: \\
                    {\em MC9S12XDP512 Data Sheet.\/}
                    MC9S12XDP512, Rev. 2.11, 5/2005

\bibitem{FujitsuCD} Fujitsu Limited: \\
                    {\em June 1998 Semiconductor Data Book.\/} \\
                    CD00-00981-1E

\bibitem{Fujitsu16} Fujitsu Semiconductor: \\
                    {\em F$^2$MC16LX 16-Bit Microcontroller MB90500 Series
                         Programming Manual.\/} \\
                    CM44-00201-1E, 1998

\bibitem{Hit180} Hitachi Ltd.: \\
                 {\em 8-/16-Bit Microprocessor Data Book.\/} \\
                 1986

\bibitem{Hit63} Trevor J.Terrel \& Robert J. Simpson: \\
                {\em Understanding HD6301X/03X CMOS Microprocessor
                 Systems.\/} \\
		published by Hitachi

\bibitem{HitH8_3} Hitachi Microcomputer: \\
                {\em H8/300H Series Programming Manual.\/} \\
                (21-032, no year of release given)

\bibitem{HitH16} Hitachi America, Ltd.: \\
                {\em HD641016 User's Manual.\/} \\
                ADE-602-003A, September 1989

\bibitem{HuC6280} {\em HuC6280 CMOS 8-bit Microprocessor Software Manual.\/} \\

\bibitem{Rock6519} Rockwell: \\
                {\em R65C19 Microcomputer Data Sheet.\/} \\
                Document Number 29400N10, January 1992

\bibitem{SH7000} Hitachi Semiconductor Design \& Development Center: \\
                 {\em SH Microcomputer Hardware Manual
                 (Preliminary).\/}

\bibitem{SH7700} Hitachi Semiconductor and IC Div.: \\
                 {\em SH7700 Series Programming Manual.\/} \\
                 1st Edition, September 1995

\bibitem{HMCS400} Hitachi America Ltd.:
                 {\em HMCS400 Series Handbook: Users Manual\/}
                 AD-E00078, March 1988

\bibitem{HitH8_5} Hitachi Semiconductor and IC Div.: \\
                  {\em H8/500 Series Programming Manual.\/} \\
        	  (21-20, 1st Edition Feb. 1989)

\bibitem{HitH8_532} Hitachi Ltd.: \\
                    {\em H8/532 Hardware Manual.\/} \\
		    (21-30, no year of release given)

\bibitem{HitH8_534} Hitachi Ltd.: \\
                    {\em H8/534,H8/536 Hardware Manual.\/} \\
		    (21-19A, no year of release given)

\bibitem{PPC403} IBM Corp.: \\
	         {\em PPC403GA Embedded Controller User's Manual.\/} \\
                 First Edition, September 1994

\bibitem{IntEmb} Intel Corp.: \\
                 {\em Embedded Controller Handbook.\/} \\
		 1987

\bibitem{IntMic} Intel Corp.: \\
                 {\em Microprocessor and Peripheral Handbook, Volume I
                 Microprocessor.\/} \\
                 1988

\bibitem{Int48}  Intel Corp. : \\
                 {\em MCS-48 Family of Single Chip Microcomputers User's Manual.\/} \\
                 September 1980

\bibitem{Int960} Intel Corp. : \\
                 {\em 80960SA/SB Reference Manual.\/} \\
                 1991

\bibitem{Int196} Intel Corp.: \\
                 {\em 8XC196NT Microcontroller User's Manual.\/} \\
                 June 1995

\bibitem{Int251} Intel Corp.: \\
                 {\em 8XC251SB High Performance CHMOS Single-Chip
                  Microcontroller.\/} \\
                 Sept. 1995, Order Number 272616-003

\bibitem{Int296} Intel Corp.: \\
                 {\em 80296SA Microcontroller User's Manual.\/} \\
                 Sept. 1996

\bibitem{Int4040} Intel Corp.: \\
                  {\em 4040: Single-Chip 4-Bit P-Channel
                   Microprocessor.\/} \\
                  (no year of release given)

\bibitem{Inters02} Intersil: \\
                 {\em CDP1802A, CDP1802AC, CDP1802BC CMOS 8-Bit
                  Microprocessors.\/} \\
                 March 1997

\bibitem{RCA1804}: RCA Inc.:
                 {\em CDP1804, CDP1804C Types Objective Data.}

\bibitem{Inters05} Intersil: \\
                 {\em CDP1805AC, CDP1806AC CMOS 8-Bit Microprocessor
                  with On-Chip RAM and Counter/Timer.\/} \\
                 March 1997

\bibitem{Kaku} Hirotsugu Kakugawa: \\
               {\em A memo on the secret features of 6309.\/} \\
    	       (available via World Wide Web: \\
      	       http://www.cs.umd.edu/users/fms/comp/CPUs/6309.txt)

\bibitem{LatMico8} Lattice Semiconductor Corporation: \\
                   {LatticeMico8 Microcontroller Users Guide.\/} \\
                   Reference Design RD1026, February 2008

\bibitem{MicroChip} Microchip Technology Inc.: \\
                    {\em Microchip Data Book.\/} \\
                    1993 Edition

\bibitem{Mil1750} US Department Of Defense: \\
                  {\em Military Standard Sixteen-Bit Computer
                  Instruction Set Architecture.\/} \\
                  MIL-STD-1750A (USAF), 2 July 1980

\bibitem{Mit41} Mitsubishi Electric: \\
                {\em Single-Chip 8-Bit Microcomputers.\/} \\
                Vol.2, 1987

\bibitem{Mit16} Mitsubishi Electric: \\
                {\em Single-Chip 16-Bit Microcomputers.\/} \\
                Enlarged edition, 1991

\bibitem{Mit8} Mitsubishi Electric: \\
               {\em Single-Chip 8 Bit Microcomputers.\/} \\
               Vol.2, 1992

\bibitem{Mit4500} Mitsubishi Electric: \\
                  {\em M34550Mx-XXXFP Users's Manual.\/} \\
                  Jan. 1994

\bibitem{Mit7751} Mitsubishi Electric: \\
                  { 7751 Series Software Manual.\/} \\
                  Rev. 1.01, 980731

\bibitem{MitM16} Mitsubishi Electric: \\
                 {\em M16 Family Software Manual.\/} \\
                 First Edition, Sept. 1994

\bibitem{MitM16C} Mitsubishi Electric: \\
      	          {\em M16C Software Manual.\/} \\
                  First Edition, Rev. C, 1996

\bibitem{Mit30600} Mitsubishi Electric: \\
                   {\em M30600-XXXFP Data Sheet.\/} \\
                   First Edition, April 1996

\bibitem{GreenM16} documentation about the M16/M32-developer's package
                   from Green Hills Software

\bibitem{Mostek3870} Mostek Corporation:
                     {\em 3870/F8 Microcomputer Data Book.\/} \\
                     1981, Publication Number MK79602

\bibitem{MotMic} Motorola Inc.: \\
                 {\em Microprocessor, Microcontroller and Peripheral
                 Data.\/} \\
		 Vol. I+II, 1988

\bibitem{Mot81} Motorola Inc.: \\
                {\em MC68881/882 Floating Point Coprocessor User's
                Manual.\/} \\
                Second Edition, Prentice-Hall, Englewood Cliffs 1989

\bibitem{Mot51} Motorola Inc.: \\
                {\em MC68851 Paged Memory Management Unit User's
                Manual.\/} \\
                Second Edition, Prentice-Hall, Englewood Cliffs 1989,1988

\bibitem{Mot32} Motorola Inc.: \\
                {\em CPU32 Reference Manual.\/} \\
                Rev. 1, 1990

\bibitem{Mot56} Motorola Inc.: \\
                {\em DSP56000/DSP56001 Digital Signal Processor User's
                Manual.\/} \\
                Rev. 2, 1990

\bibitem{Mot340} Motorola Inc.: \\
                 {\em MC68340 Technical Summary.\/} \\
                 Rev. 2, 1991

\bibitem{Mot16} Motorola Inc.: \\
                {\em CPU16 Reference Manual.\/} \\
                Rev. 1, 1991

\bibitem{Mot68K} Motorola Inc.: \\
                 {\em Motorola M68000 Family Programmer's
                  Reference Manual.\/} \\
                 1992

\bibitem{Mot332} Motorola Inc.: \\
                 {\em MC68332 Technical Summary.\/} \\
                 Rev. 2, 1993

\bibitem{Mot601} Motorola Inc.: \\
                 {\em PowerPC 601 RISC Microprocessor User's Manual.\/} \\
                 1993

\bibitem{Mot505} Motorola Inc.: \\
                 {\em PowerPC(tm) MPC505 RISC Microcontroller Technical
                 Summary.\/} \\
        	 1994

\bibitem{Mot821} Motorola Inc.: \\
                 {\em PowerPC(tm) MPC821 Portable Microprocessor User's Manual.\/} \\
                 1996

\bibitem{Mot12} Motorola Inc.: \\
                {\em CPU12 Reference Manual.\/} \\
                1st edition, 1996

\bibitem{Mot08} Motorola Inc.: \\
                {\em CPU08 Reference Manual.\/} \\
                Rev. 1 (no year of release given im PDF-File)

\bibitem{Mot360} Motorola Inc.: \\
                 {\em MC68360 User's Manual.\/}

\bibitem{MotCold} Motorola Inc.: \\
                  {\em MCF 5200 ColdFire Family Programmer's Reference
                   Manual.\/} \\
                  1995

\bibitem{MotMCore} Motorola Inc.: \\
                   {\em M*Core Programmer's Reference Manual.\/} \\
                   1997

\bibitem{Mot56300} Motorola Inc.: \\
                   {\em DSP56300 24-Bit Digital Signal Processor
                    Family Manual.\/} \\
                   Rev. 0 (no year of release given im PDF-File)

\bibitem{MotK4} Motorola Inc.: \\
                {\em MC68HC11K4 Technical Data.\/}
                1992

\bibitem{OKIMicro} OKI Semiconductor: \\
                {\em Microcontroller Data Book.\/}
                Second Edition, December 1986

\bibitem{SCMP} National Semiconductor: \\
               {\em SC/MP Program\-mier- und
                As\-sem\-bler-Hand\-buch.\/} \\
               Publication Number 4200094A, Aug. 1976

\bibitem{AsmCop} National Semiconductor: \\
                 {\em COP800 Assembler/Linker/Librarian User's
                 Manual.\/} \\
                 Customer Order Number COP8-ASMLNK-MAN \\
                 NSC Publication Number 424421632-001B \\
                 August 1993

\bibitem{Cop87L84} National Semiconductor: \\
                   {\em COP87L84BC microCMOS One-Time-Programmable (OTP)
                    Microcontroller.\/} \\
                   Preliminary, March 1996

\bibitem{Nat14xxx} National Semiconductor: \\
                   {\em SC14xxx DIP commands Reference guide.} \\
                   Application Note AN-D-031, Version 0.4, 12-28-1998

\bibitem{Nat8070} National Semiconductor: \\ 
                  {\em INS8070-Series Microprocessor Family.\/} 
                  October 1980 

\bibitem{NECV} NEC Corp.: \\
   	       {\em $\mu$pD70108/$\mu$pD70116/$\mu$pD70208/$\mu$pD70216/$\mu$pD72091 
               Data Book.\/} \\
               (no year of release given)

\bibitem{NEC78K0} NEC Electronics Europe GmbH: \\
                  {\em User's Manual $\mu$COM-87 AD Family.\/} \\
		  (no year of release given)

\bibitem{NEC75} NEC Corp.: \\
                {\em $\mu$COM-75x Family 4-bit CMOS Microcomputer User's
                 Manual.\/} \\
                Vol. I+II (no year of release given)

\bibitem{NEC78K2} NEC Corp.: \\
                {\em 78K/II Series 8-Bit Single-Chip Microcontroller
                 User's Manual - Instructions.\/} \\
                Document No. U10228EJ6V0UM00 (6th edition), December 1995

\bibitem{NEC78K3} NEC Corp.: \\
		{\em uPD78310/312CW/G 8 Bit CMOS Microcomputer Product
                 Description.\/}

\bibitem{NECSig} NEC Corp.: \\
                {\em Digital Signal Processor Product Description.\/} \\
                PDDSP.....067V20 (no year of release given)

\bibitem{NEC78} NEC Corp.: \\
                {\em $\mu$PD78070A, 78070AY 8-Bit Single-Chip Microcontroller
                 User's Manual.\/} \\
                Document No. U10200EJ1V0UM00 (1st edition), August 1995

\bibitem{NEC7814} NEC Corp.: \\
                  {\em Data Sheet $\mu$PD78014.\/}

\bibitem{NXPS12ZCore} NXP/Freescale: \\
                {\em CPU S12Z Reference Manual.} \\
                CPUS12ZRM, Rev. 1.01, 01/2013

\bibitem{NXPS12ZVC} NXP: \\
                {\em MC9S12ZVC-Family Reference Manual and Datasheet.} \\
                MC9S12ZVCRMV1, Rev. 1.9, 29-January-2018

\bibitem{ParaSX20} Parallax Inc. \\
                  { SX20AC/SX28AC Data Sheet.}
                  Revision 1.7, 4/23/2008

\bibitem{Phil84X1} Philips Semiconductor: \\
                 {\em MAB84X1, MAF84X1, MAF84AX1 Family Datasheet.} \\
                 August 1990

\bibitem{PhilXA} Philips Semiconductor: \\
                 {\em 16-bit 80C51XA Microcontrollers (eXtended
                  Architecture).\/} \\
                 Data Handbook IC25, 1996

\bibitem{SGS3870} SGS-Ates: \\
                 {\em M3870 Microcomputer Family Databookk.\/}\\
                 1st edition, issued January 1983

\bibitem{SGS04} SGS-Thomson Microelectronics: \\
                {\em 8 Bit MCU Families EF6801/04/05 Databook.\/}\\
                1st edition, 1989

\bibitem{SGS62} SGS-Thomson Microelectronics: \\
                {\em ST6210/ST6215/ST6220/ST6225 Databook.\/} \\
                1st edition, 1991

\bibitem{ST7Man} SGS-Thomson Microelectronics: \\
                 {\em ST7 Family Programming Manual.\/} \\
                 June 1995

\bibitem{SGS9} SGS-Thomson Microelectronics: \\
               {\em ST9 Programming Manual.\/} \\
               3rd edition, 1993

\bibitem{Siem166} Siemens AG: \\
                  {\em SAB80C166/83C166 User's Manual.\/} \\
                  Edition 6.90

\bibitem{Siem167} Siemens AG: \\
                  {\em SAB C167 Preliminary User's Manual.\/} \\
                  Revision 1.0, July 1992

\bibitem{Siem382} Siemens AG: \\
                  {\em Telephone Controller (Single-Chip 8-Bit \\
                  CMOS Microcontroller) SAB80C382/SAB80C482.\/}\\
                  May 1987

\bibitem{Siem502} Siemens AG: \\
                  {\em SAB-C502 8-Bit Single-Chip Microcontroller User's
                   Manual.\/} \\
                  Edition 8.94

\bibitem{Siem501} Siemens AG: \\
                  {\em SAB-C501 8-Bit Single-Chip Microcontroller User's
                   Manual.\/} \\
                  Edition 2.96

\bibitem{Siem504} Siemens AG: \\
                  {\em C504 8-Bit CMOS Microcontroller User's Manual.\/} \\
                  Edition 5.96

\bibitem{STM8ProgMan} ST Microelectronics: \\
                  {\em STM8 CPU Programming Manual.\/} \\
                  PM0044, Doc ID 13590 Rev 3, September 2011

\bibitem{STM8SRefMan} ST Microelectronics: \\
                  {\em STM8S Series and STM8AF Series 8-bit Microcontrollers Reference Manual.\/} \\
                  RM0016, DocID14587 Rev 14, October 2017

\bibitem{STM8S003F3Sheet} ST Microelectronics: \\
                  {\em STM8S003F3 STM8S003K3 Data Sheet.\/} \\
                  DS7147 Rev 10, August 2018

\bibitem{Syb68K} C.Vieillefond: \\
                 {\em Programmierung des 68000.\/} \\
                 Sybex-Verlag D\"usseldorf, 1985

\bibitem{Sym8xx} Symbios Logic Inc: \\
                 {\em Symbios Logic PCI-SCSI-I/O Processors Programming   
                 Guide.\/} \\
                 Version 2.0, 1995/96
                 
\bibitem{Ti990}  Texas Instruments: \\
                 {\em Model 990 Computer/TMS9900 Microprocessor
                      Assembly Language Programmer's Guide.\/} \\
                 1977, Manual No. 943441-9701

\bibitem{Ti9900} Texas Instruments: \\
                 {\em TMS9995 16-Bit Microcomputer.\/} \\
                 Preliminary Data Manual
                 1981

\bibitem{Ti99000} Texas Instruments: \\
                 {\em TMS99105 and TMS99110A 16-Bit Microprocessors.\/} \\
                 Preliminary Data Manual
                 1982

\bibitem{TiC10} Texas Instruments: \\
                {\em First-Generation TMS320 User's Guide.\/} \\
                1988, ISBN 2-86886-024-9

\bibitem{Ti7000} Texas Instruments: \\
                 {\em TMS7000 family Data Manual.\/} \\
                 1991, DB103

\bibitem{TiC30} Texas Instruments: \\
                {\em TMS320C3x User's Guide.\/} \\
                Revision E, 1991

\bibitem{TiC20} Texas Instruments: \\
                {\em TMS320C2x User's Guide.\/} \\
                Revision C, Jan. 1993

\bibitem{TiC40} Texas Instruments: \\
                   {\em TMS320C4x User's Guide.\/} \\
                   SPRU063C, May 1999

\bibitem{Ti370} Texas Instruments: \\
                {\em TMS370 Family Data Manual.\/} \\
                1994, SPNS014B

\bibitem{Ti430FamSoft} Texas Instruments: \\
                       {\em MSP430 Family Software User's Guide.\/} \\
                       1994, SLAUE11

\bibitem{Ti430Met} Texas Instruments: \\
                   {\em MSP430 Metering Application.\/} \\
                   1996, SLAAE10A

\bibitem{Ti430FamArch} Texas Instruments: \\
                       {\em MSP430 Family Architecture User's Guide.\/} \\
                       1995, SLAUE10A

\bibitem{Ti430X} Texas Instruments: \\
                 {\em MSP430 MSP430x5xx and MSP430x6xx Family
                  User's Guide.\/} \\
                 October 2016, SLAU208

\bibitem{TiC60} Texas Instruments: \\
                {\em TMS320C62xx CPU and Instruction Set Reference
                 Manual.\/} \\
                Jan. 1997, SPRU189A

\bibitem{TiC20x} Texas Instruments: \\
                 {\em TMS320C20x User's Guide.\/} \\
                 April 1999, SPRU127C

\bibitem{Tispru131g} Texas Instruments: \\
                     {\em TMS320C54x DSP Reference Set;
                     Volume 1: CPU and Peripherals.\/} \\
                     March 2001, SPRU172C

\bibitem{Tispru172c} Texas Instruments: \\
                     {\em TMS320C54x DSP; Volume 2: Mnemonic Instruction
                     Set.\/} \\
                     March 2001, SPRU172C

\bibitem{TMS1000PGMRef} Texas Instruments: \\
                     {\em TMS 1000 Series MOS/LSI One-Chip
                     Microcomputers Programmer's Reference
                     Manual.\/} \\
                     CM122-1 1275, 1975

\bibitem{Tosh90} Toshiba Corp.: \\
                 {\em 8-Bit Microcontroller TLCS-90 Development System
                 Manual.\/} \\
                 1990

\bibitem{Tosh870} Toshiba Corp.: \\
                  {\em 8-Bit Microcontroller TLCS-870 Series Data
                  Book.\/} \\
                  1992

\bibitem{Tosh900} Toshiba Corp.: \\
                  {\em 16-Bit Microcontroller TLCS-900 Series Users
                  Manual.\/} \\
                  1992

\bibitem{Tosh900L} Toshiba Corp.: \\
                   {\em 16-Bit Microcontroller TLCS-900 Series Data Book:
                    TMP93CM40F/ TMP93CM41F.\/} \\
                   1993

\bibitem{Tosh47} Toshiba Corp.: \\
                 {\em 4-Bit Microcontroller TLCS-47E/47/470/470A Development
                  System Manual.\/} \\
                 1993

\bibitem{Tosh9000} Toshiba Corp.: \\
                   {\em TLCS-9000/16 Instruction Set Manual Version
                   2.2.\/} \\
                   10. Feb 1994

\bibitem{Tosh9331} Toshiba Corp.: \\
                   {\em TC9331 Digital Audio Signal Processor
                   Application Information.}

\bibitem{Val8X} Valvo GmbH: \\
                {\em Bipolare Mikroprozessoren und bipolare
                LSI-Schaltungen.\/} \\
                Datenbuch, 1985, ISBN 3-87095-186-9

\bibitem{Xil213} Ken Chapman (Xilinx Inc.):\\
                 {\em PicoBlaze 8-Bit Microcontroller for Virtex-E
                 and Spartan-II/IIE Devices.\/} \\
                 Application Note XAPP213, Version 2.1, February 2003

\bibitem{Xil129} Xilinx Inc.: \\
                 {\em PicoBlaze 8-bit Embedded Microcontroller User Guide
                 for Spartan-3, Virtex-II, and Virtex-II Pro FPGAs.\/} \\
                 UG129 (v1.1) June 10, 2004
                          
\bibitem{XMOS} David May:: \\
               {\em The XMOS XS1 Architecture.\/} \\
               Publication Date: 2009/10/19, Copyright 2009 XMOS Ltd.

\bibitem{Zilog} data sheets from Zilog about the Z80 family

\bibitem{ZilZ8} Zilog Inc.: \\
                {\em Z8 Microcontrollers Databook.\/} \\
                1992

\bibitem{ZilZ8_2} Zilog Inc.: \\
                  {\em Discrete Z8 Microcontrollers Databook.\/} \\
                  (no year of release given)

\bibitem{ZilZ380} Zilog Inc.: \\
                  {\em Z380 CPU Central Processing Unit User's
                  Manual.\/} \\
                  (no year of release given)

\bibitem{ZileZ8} Zilog Inc.: \\
               {\em eZ8 CPU User Manual.\/} \\
               UM01285-0503

\bibitem{ZilSuper8} Zilog Inc.: \\
               {\em Z88C00 CMOS Super8 ROMless MCU Product Specification\/} \\
               2003, PS014602-0103

\bibitem{ZilF0830} Zilog Inc.: \\
               {\em Z8 Encore! F0830 Series Product Specification\/} \\
               2012, PS025113-1212


\end{thebibliography}

\cleardoublepage

\printindex

\end{document}
